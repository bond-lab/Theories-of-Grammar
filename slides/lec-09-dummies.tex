\documentclass[a4paper,landscape,headrule,footrule,dvips]{foils}
%%
%%% macros for Theories of Grammar
%%%
\usepackage{polyglossia}
\setdefaultlanguage{english}
%\setmainfont{TeX Gyre Pagella}


\newcommand{\logo}{~}
\newcommand{\header}[3]{%
\title{\vspace*{-2ex} \large HG4041 Theories of Grammar
\\[2ex] \Large  \emp{#2} \\ \emp{#3}}
\author{\blu{Francis Bond}   \\ 
\normalsize  \textbf{Division of Linguistics and Multilingual Studies}\\
\normalsize  \url{http://www3.ntu.edu.sg/home/fcbond/}\\
\normalsize  \texttt{bond@ieee.org}}
\MyLogo{HG4041 (2020)}
\renewcommand{\logo}{#2}
\hypersetup{
   pdfinfo={
     Author={Francis Bond},
     Title={#1: #2},
     Subject={HG4041: Theories of Grammar},
     Keywords={Syntax, Semantics, HPSG, Unification, Constructions},
     License={CC BY 4.0}
   }
 }

\date{#1
  \\ Location: LHN-TR+36}
}
\usepackage[hidelinks]{hyperref}




\usepackage{xcolor}
\usepackage{graphicx}
\newcommand{\blu}[1]{\textcolor{blue}{#1}}
\newcommand{\grn}[1]{\textcolor{green}{#1}}
\newcommand{\hide}[1]{\textcolor{white}{#1}}
\newcommand{\emp}[1]{\textcolor{red}{#1}}
\newcommand{\txx}[1]{\textbf{\textcolor{blue}{#1}}}
\newcommand{\lex}[1]{\textbf{\mtcitestyle{#1}}}

\usepackage{pifont}
\renewcommand{\labelitemi}{\textcolor{violet}{\ding{227}}}
\renewcommand{\labelitemii}{\textcolor{purple}{\ding{226}}}

\newcommand{\subhead}[1]{\noindent\textbf{#1}\\[5mm]}

\newcommand{\Bad}{\emp{\raisebox{0.15ex}{\ensuremath{\mathbf{\otimes}}}}}
\newcommand{\bad}[1]{*\eng{#1}}

\newcommand{\com}[1]{\hfill (\emp{#1})}%

\usepackage{relsize,xspace}
\newcommand{\into}{\ensuremath{\rightarrow}\xspace}
\newcommand{\tot}{\ensuremath{\leftrightarrow}\xspace}
\usepackage{url}
\newcommand{\lurl}[1]{\MyLogo{\url{#1}}}

\usepackage{mygb4e}
\newcommand{\lx}[1]{\textbf{\mtciteform{#1}}}
\newcommand{\ix}{\ex\slshape}
\let\eachwordone=\slshape



\newcommand{\ent}{\ensuremath{\Rightarrow}\xspace}
\newcommand{\ngv}{\ensuremath{\not\Rightarrow}\xspace}
%\usepackage{times}
%\usepackage{nttfoilhead}
\newcommand{\myslide}[1]{\foilhead[-25mm]{\raisebox{12mm}[0mm]{\emp{#1}}}\MyLogo{\logo}}
\newcommand{\myslider}[1]{\rotatefoilhead[-25mm]{\raisebox{12mm}[0mm]{\emp{#1}}}}
%\newcommand{\myslider}[1]{\rotatefoilhead{\raisebox{-8mm}{\emp{#1}}}}

\newcommand{\section}[1]{\myslide{}{\begin{center}\Huge \emp{#1}\end{center}}}



\usepackage[lyons,j,e,k]{mtg2e}
%\renewcommand{\mtcitestyle}[1]{\textcolor{-red!75!green!50}{\textsl{#1}}}
\renewcommand{\mtcitestyle}[1]{\textcolor{teal}{\textsl{#1}}}
\newcommand{\iz}[1]{\texttt{\textup{#1}}}
\newcommand{\gm}{\textsc}
\usepackage[normalem]{ulem}
\newcommand{\ul}{\uline}
\newcommand{\ull}{\uuline}
\newcommand{\wl}{\uwave}
\newcommand{\vs}{\ensuremath{\Leftrightarrow}~}
%%%
%%% Bibliography
%%%
\usepackage{natbib}
%\usepackage{url}
\usepackage{bibentry}


%%% From Tim
\newcommand{\WMngram}[1][]{$n$-gram#1\xspace}
\newcommand{\infers}{$\rightarrow$\xspace}

\usepackage[utf8]{inputenc}

\usepackage{rtrees,qtree}
\renewcommand{\lf}[1]{\br{#1}{}}
\usepackage{avm}
%\avmoptions{topleft,center}
\newcommand{\ft}[1]{\textsc{#1}}
\renewcommand{\val}[1]{\textit{#1}}
\newcommand{\typ}[1]{\textit{#1}}
\newcommand{\prd}[1]{\textbf{#1}}
\avmfont{\sc}
\avmvalfont{\it}
\avmsortfont{\smaller[2] \it}
\usepackage{multicol}
\newcommand{\blank}{\rule{3em}{1pt}\xspace}

% \usepackage{pst-node}
\newcommand{\OV}[1]{\ovalnode[linestyle=dotted,linecolor=red]{A}{#1}}
\newcommand{\OVB}[1]{\ovalnode[linestyle=dotted,linecolor=blue]{A}{#1}}

%%% From CSLI book
\newcommand{\mc}{\multicolumn}
\newcommand{\HD}{\textbf{H}\xspace}
\newcommand{\el}{\< \>}
\makeatother
\long\def\smalltree#1{\leavevmode{\def\\{\cr\noalign{\vskip12pt}}%
\def\mc##1##2{\multispan{##1}{\hfil##2\hfil}}%
\tabskip=1em%
\hbox{\vtop{\halign{&\hfil##\hfil\cr
#1\crcr}}}}}
\makeatletter

\newcommand{\A}{\noindent\textbf{A}: }
\newcommand{\Q}{\noindent\textbf{Q}: }
%\newcommand{\C}{\noindent\textbf{C}: }



\usepackage{times}
\usepackage{bchart}
%\usepackage{pgfplots}
\avmfont{\sc}
\begin{document}
\header{Lecture 9}{Non-referential NPs, Expletives, and Extraposition}{}
\maketitle




\myslide{Overview}
\MyLogo{Sag, Wasow and Bender (2003) --- Chapter 11}

\begin{itemize}
\item Existentials  (\eng{There is an X}, \ldots)
\item Extraposition (\eng{It worries me that X}, \ldots))
\item Idioms (\eng{X takes advantage of Y}, \ldots)
\end{itemize}

\myslide{Where We Are, and Where We’re Going}

\begin{itemize}
\item Last time, we met the passive \lex{be}.
\item Passive \lex{be} is just a special case -- that \lex{be}
generally introduces [PRED +] constituents 
%(next slide).
\item Today, we’ll start with another \lex{be}, which 
occurs in existential sentences starting with 
\lex{there}, e.g. \eng{There is a monster in Loch Ness.}
\item Then we’ll look at this use of \lex{there}.
\item Which will lead us to a more general 
examination of NPs that don’t refer, including 
some uses of \lex{it} and certain idiomatic uses of 
NPs.
\end{itemize}

\myslide{Chapter 10 entry for be}

\begin{avm}
  \< \textnormal{be},\ \[{\it be-lxm}\\
  arg-st & \< {\@1} , \avml\[syn & \[head & \[{\it verb}\\
  form &  pass \]\\
  val & \[spr & \q< {\@1} \q>\\
  comps & \el \]\]\\
  sem & \[index & $s$ \ \]\]\avmr \> \\ %{\@3}
  sem & \[index & \ $s$\\ %{\@3}\\
  restr & {\el} \] \] \>
\end{avm}


\myslide{Copula (generalized)}

\begin{avm}
  \< \textnormal{be},\ \[{\it be-lxm}\\
  arg-st & \< {\@1} , \avml\[syn & \[head & \[ pred \ $+$ \]\\
  val & \[spr & \q< {\@1} \q>\\
  comps & \el \]\]\\
  sem & \[index & $s$ \ \]\]\avmr \> \\ %{\@3}
  sem & \[index & \ $s$\\ %{\@3}\\
  restr & {\el} \] \] \>
\end{avm}

\myslide{Existentials}

\begin{itemize}
\item The \lex{be} in \eng{There is a page missing} cannot be the 
same \lex{be} that occurs in sentences like \eng{Pat is tall} or 
\eng{A cat was chased by a dog}.  Why not?
\item So we need a separate lexical entry for this \lex{be}, 
  stipulating:
  \begin{itemize}
  \item Its SPR must be \lex{there}
  \item It takes two complements, the first an NP and the 
    second an AP, PP, or (certain kind of) VP.
  \item The semantics should capture the relation between, e.g. 
    \eng{There is a page missing} and \eng{A page is missing}.  
  \end{itemize}
\end{itemize}

\myslide{Lexical Entry for the Existential \lex{be}}

\begin{avm}
  \< \textnormal{be},\ \[\makebox[0mm][l]{\textit{exist-be-lxm}}\\
  arg-st & \< \shortstack{NP\\\[ form
    & there \]}, {\@1},
  \avml\[syn & \[head & \[ pred \ $+$ \]\\
  val & \[spr \q< {\@1} \q> \\ comps \el \]\]\\
  sem & \[index & $s$ \ \]\]\avmr \> \\ %{\@3}
  sem & \[index & \ $s$\\ %{\@3}\\
  restr & {\el} \] \] \>
\end{avm}


\myslide{Questions About the Existential \lex{be}}

\begin{itemize}
\item What type of constituent is the third argument?
\item Why is the third argument [\ft{pred} +]?
\item Why is the second argument tagged as identical to the \ft{spr}
  of the third argument?
\end{itemize}

\begin{center}
  \begin{tiny}
    \begin{avm}
      \< \textnormal{be},\ \[\makebox[0mm][l]{\textit{exist-be-lxm}}\\
      arg-st & \< \shortstack{NP\\\[ form
          & there\]}, {\@1},
      \avml\[syn & \[head & \[ pred \ $+$ \]\\
      val & \[spr \q< {\@1} \q> \\ comps \el \]\]\\
      sem & \[index & $s$ \ \]\]\avmr \> \\ %{\@3}
      sem & \[index & \ $s$\\ %{\@3}\\
      restr & {\el} \] \] \>
    \end{avm}
  \end{tiny}
\end{center}

\myslide{There are questions left}

\begin{itemize}
\item What is the contribution of this \lex{be} to the semantics of the sentences 
it occurs in?
\item Can all [\ft{pred} +] predicates appear as the third argument in 
existentials?
\item How do we rule out *\lex{There was a greyhound a good runner}?  
\end{itemize}

\begin{center}
  \begin{tiny}
    \begin{avm}
      \< \textnormal{be},\ \[\makebox[0mm][l]{\textit{exist-be-lxm}}\\
      arg-st & \< \shortstack{NP\\\[ form
          & there \]}, {\@1},
      \avml\[syn & \[head & \[ pred \ $+$ \]\\
      val & \[spr \q< {\@1} \q> \\ comps \el \]\]\\
      sem & \[index & $s$ \ \]\]\avmr \> \\ %{\@3}
      sem & \[index & \ $s$\\ %{\@3}\\
      restr & {\el} \] \] \>
    \end{avm}
  \end{tiny}
\end{center}


\myslide{The Entry for Existential \lex{there}}

\begin{center}
  \begin{avm}
    \< \textnormal{there},\ \[\makebox[0mm][l]{\textit{prn-lxm}}\\
    syn & \[head & \[ form & there \\
    agr  & \[ per \ 3rd \]   \] \]\\
    sem & \[mode & none \\
    index & none \\
    rest & \el \]
    \] \>
  \end{avm}
\end{center}

\myslide{Questions About Existential there}

\begin{itemize}
\item Why do we call it a pronoun?
\item Why don't we give it a value for \ft{num}?
\item What does this entry claim is \lex{there}’s contribution to the
  semantics of the sentences it appears in?
\item Is this a correct claim?
\end{itemize}

\begin{center} \tiny
  \begin{avm}
    \< \textnormal{there},\ \[\makebox[0mm][l]{\textit{prn-lxm}}\\
    syn & \[head & \[ form & there \\
    agr  & \[ per \ 3rd \]   \] \]\\
    sem & \[mode & none \\
    index & none \\
    rest & \el \]
    \] \>
  \end{avm}
\end{center}


\myslide{Other NPs that don’t seem to refer}

{\large
\begin{exe}
\ex \eng{\ul{It} sucks that the Rockies lost the series.}
\ex \eng{\ul{It} is raining.}
\ex \eng{Andy took \ul{advantage} of the opportunity.}
\ex \eng{Lou kicked \ul{the bucket}.}
\end{exe}
}

\myslide{What about \eng{It follows that you are wrong}?}
\MyLogo{$^*$We need these anyway (independently motivated)}
\begin{itemize}
\item This is an example of \txx{extraposition}
\item To analyze it we need:
  \begin{itemize}
  \item An analysis of this use of \lex{that}$^*$
  \item Entries for verbs that take clausal subjects $^*$
    \begin{exe}
      \ex \label{s:that:follows} \eng{\ul{That you are wrong} follows}
    \end{exe}
  \item A lexical entry for dummy \lex{it}
  \item A rule to account for the relationship 
    between pairs like  (\ref{s:that:follows}) and (\ref{s:follows:that})
\begin{exe}
  \ex \label{s:follows:that}
  \eng{\ul{It} follows \ul{that you are wrong}.}
\end{exe}
\end{itemize}
\end{itemize}

\myslide{The Entry for Dummy \lex{it}}

\begin{center}
  \begin{avm}
    \< \textnormal{it},\ \[\makebox[0mm][l]{\textit{prn-lxm}}\\
    syn & \[head & \[ form & it \\
    agr  & 3sing   \] \]\\
    sem & \[mode & none \\
    index & none \\
    rest & \el \]
    \] \>
  \end{avm}
\end{center}

\myslide{Questions About Dummy \lex{it}}


\begin{itemize}
\item How does it differ from the entry for dummy \lex{there}? 
Why do they differ in this way?
\item Is this the only entry for \lex{it}?
\end{itemize}
\begin{center}\small
  \begin{avm}
    \< \textnormal{it},\ \[\makebox[0mm][l]{\textit{prn-lxm}}\\
    syn & \[head & \[ form & it \\
    agr  & 3sing   \] \]\\
    sem & \[mode & none \\
    index & none \\
    rest & \el \]
    \] \>
  \end{avm}
\end{center}

\myslide{A New Type of Lexeme:  Complementizers}

\begin{center}\large
  \begin{avm}
    \textit{comp-lxm}:  \[ \\
    syn & \[head & \[\makebox[0mm][l]{\textit{comp}} \\
    agr  & 3sing   \] \\
    val & \[ spr & \el \] \]\\
    arg-str & \< \shortstack{S \\ \[ index & s \]}\>\\
    sem & \[ index & s \\
    rest & \el \]
    \] 
  \end{avm}
\end{center}


\myslide{Questions About the Type \val{comp-lxm}}

\begin{itemize}
\item Why does it stipulate values for both \ft{spr} and \ft{arg-st}?
\item Why is its \ft{index} value the same as its argument’s?
\item What is its semantic contribution?
\end{itemize}
\begin{center}\small
  \begin{avm}
    \textit{comp-lxm}:  \[ \\
    syn & \[head & \[\makebox[0mm][l]{\textit{comp}} \\
    agr  & 3sing   \] \\
    val & \[ spr & \el \] \]\\
    arg-str & \< \shortstack{S \\ \[ index & s \]}\>\\
    sem & \[ index & s \\
    rest & \el \]
    \] 
  \end{avm}
\end{center}



\myslide{The Type comp}


\begin{center}
  \begin{avmtree}%\renewcommand{\lf}[1]{\br{#1}{}}\avmfont{\sc}
\it
\br{\shortstack{pos\\ \[ \textsc{form, pred} \]}}{\br{agr-pos \[ \textsc{agr} \] }{ 
    \br{verb \[ \textsc{aux} \]}{}
    \br{nominal \[ \textsc{case} \]}{
      \br{\shortstack{noun \\ \mbox{} }}{}
      \br{\shortstack{comp \\\[ \textsc{form} \ cform \]}}{}}
    \br{det \[ \textsc{count} \]}{} }
  \lf{prep}
  \lf{adj}
  \lf{conj}}
\end{avmtree}
\end{center}

\myslide{The Lexical Entry for Complementizer \lex{that}}
\bigskip
\begin{center}\large
  \begin{avm}
   \< \textnormal{that},\ \[\makebox[0mm][l]{\textit{comp-lxm}}\\
    arg-st  & \< \[ form & fin \]\>\\
    sem & \[mode & prop \]
    \] \>
  \end{avm}
\end{center}
\myslide{\ldots  with inherited information filled in}

\begin{center}\small
  \begin{avm}
   \< \textnormal{that},\ \[\makebox[0mm][l]{\textit{comp-lxm}}\\
   syn & \[head & \[\makebox[0mm][l]{\textit{comp}} \\
    form & \OV{cform} \\
    agr  & 3sing   \] \\
    val & \[ spr & \el \] \]\\
    arg-st  & \< \shortstack{S \\ \[ index & s \\ form & fin\] }\>\\
    sem & \[mode & prop \\
    index & s \\
    rest & \el \]
    \] \>
  \end{avm}
\end{center}

\begin{itemize}
\item Question: Where did [\ft{form} \val{cform}] come from?
\end{itemize}

\myslide{Structure of a Complementizer Phrase}

\begin{exe}
  \ex \eng{that the Giants lost}
\end{exe}

% \begin{center}\small
%   \begin{tree}%\renewcommand{\lf}[1]{\br{#1}{}}\avmfont{\sc}
% \br{\textnormal{CP} \begin{avm}\[ \ft{head} & \@1 \\
%           \ft{val} & \< \ft{spr} & \el \\ \ft{comps} & \el \> \]\end{avm}}%
% {\br{\textnormal{C} \begin{avm}\[ \ft{head} & \@1 \[\textit{comp} 
% \\  \ft{form} & cform \] \\
%           \ft{val} & \< \ft{spr} & \el \\ \ft{comps} & \< \@2 \> \> \]
% \end{avm}}{\lf{that}}
% \br{\@2 \textnormal{S}}{\tlf{the Giants lost}}}
% \end{tree}
% \end{center}
% \newpage
\begin{center}
  \begin{avmtree}%\renewcommand{\lf}[1]{\br{#1}{}}\avmfont{\sc}
\it
\br{\textnormal{CP} \[ \ft{head} & \@1 \\
          \ft{val} & \< \ft{spr} & \el \\ \ft{comps} & \el \> \]}%
{\br{\textnormal{C} \[ \ft{head} & \@1 \[\textit{comp} \\  \ft{form} & cform \] \\
          \ft{val} & \< \ft{spr} & \< \mbox{} \> \\ \ft{comps} & \< \@2 \> \> \]}{\lf{that}}
\br{\@2 \textnormal{S}}{\tlf{the Giants lost}}}
\end{avmtree}
\end{center}



\myslide{Sample Verb with a CP Subject}


\bigskip
\begin{center}\large
  \begin{avm}
   \< \textnormal{matter},\ \[\makebox[0mm][l]{\textit{siv-lxm}}\\
    arg-st  & \< \[ sem & \[ index & \@1 \] \]\>\\
    sem & \[index & s \\
    rstr & \<  reln & \prd{matter} \\
               sit & s \\
               mattering & \@1 
\> \]
    \] \>
  \end{avm}
\end{center}

Note:  the only constraint on the first argument is semantic

\myslide{A Problem}

\begin{itemize}
\item We constrained the subject of matter only semantically.  However\ldots{}
\item CP and S are semantically identical, but we get:
  \begin{exe} 
    \ex \eng{That Bush won matters}  vs. \eng{*Bush won matters}
  \end{exe}
\item Argument-marking PPs are semantically identical to their object 
  NPs, but we get:
  \begin{exe} 
    \ex \eng{The election mattered} vs. \eng{*Of the election mattered}
  \end{exe}
\item So we need to add a syntactic constraint.
\end{itemize}

\newpage

\begin{center}\small
  \begin{avm}
   \< \textnormal{matter},\ \[\makebox[0mm][l]{\textit{siv-lxm}}\\
    arg-st  & \< \[ syn & \[ head & \OV{nominal} \] \\
                    sem & \[ index & \@1 \] \]\>\\
    sem & \[index & s \\
    rstr & \<  \[ reln & \prd{matter} \\
                  sit & s \\
                  mattering & \@1 \] \> \]
    \] \>
  \end{avm}
\end{center}


\begin{itemize}
\item  S and PP subjects are generally impossible, so this constraint should
   probably be on \val{verb-lxm}.
 \end{itemize}
 

\myslide{Extraposition (at last)}
%\MyLogo{Examples adapted from wikipedia}
\begin{itemize}
\item Extraposition alters word order so that a relatively "heavy"
  constituent appears to the right of its canonical position.
\end{itemize}

\begin{exe}\large
  \ex 
  \begin{xlist}
    \ex  \eng{\ul{That you were wrong} follows.}
    \ex \eng{\ul{It} follows \ul{that you were wrong}.}
  \end{xlist}
  \ex 
  \begin{xlist}
    \ex  \eng{\ul{That I mistyped it} was frustrating.}
    \ex \eng{\ul{It} was frustrating \ul{that I mistyped it}.}
  \end{xlist}
  \ex 
  \begin{xlist}
   \ex \eng{Did \ul{that this happened} surprise you?}
   \ex  \eng{Did it surprise you \ul{that this happened}?}
  \end{xlist}
\end{exe}



% [1]
%   Extraposing a constituent results in a discontinuity and in this
%   regard, it is unlike shifting, which does not generate a
%   discontinuity. The extraposed constituent is separated from its
%   governor by one or more words that dominate its governor. Two types
%   of extraposition are acknowledged in theoretical syntax: standard
%   cases where extraposition is optional and it-extraposition where
%   extraposition is obligatory. Extraposition is motivated in part by a
%   desire to reduce center embedding by increasing right-branching and
%   thus easing processing, center-embedded structures being more
%   difficult to process. Extraposition occurs frequently in English and
%   related languages.

\myslide{The Extraposition Lexical Rule}

\begin{center}\small
  \begin{avm}\avmfont{\sc}
      \[ \asort{pi-rule}
      input & \< X, \[ syn & \[ val & \[ spr & \< \@1 CP \> \\ comps & \@{A} \] \]  \]  \>\\
      output & \<Y,  \[ syn & \[val & \[ spr & \< NP \[ form & it \] \> \\
      comps & \@{A} $\oplus$ \< \@1 \> \] \] \] \> \]

    \end{avm}
\end{center}

\begin{itemize}
\item Why is the type \val{pi-rule}?
\item Why doesn’t it say anything about the semantics?
\item  Why is the \ft{comps} value \avmbox{A} not $\langle \rangle$  ?
\end{itemize}

\myslide{Extraposition with Verbs whose \ft{comps} lists are Nonempty}


\begin{exe}\large
\ex \eng{It worries me that war is imminent.}
\ex \eng{It occurred to Pat that Chris knew the answer.}
\ex \eng{It endeared you to Andy that you wore a funny hat.}
\end{exe}

\myslide{Another Nonreferential Noun: \lex{advantage}}
\bigskip
\begin{center}
  \begin{avm}
    \< \textnormal{advantage},\ \[\makebox[0mm][l]{\textit{massn-lxm}}\\
    syn & \[head & \[ form & advantage \\
    agr  & 3sing   \] \]\\
    sem & \[mode & none \\
    index & none \\
    rest & \el \]
    \] \>
  \end{avm}
\end{center}


\myslide{The Verb that Selects \lex{advantage}}
\MyLogo{\textbf{take\_advantage} $\approx$ \textbf{exploit}}
\begin{center}
  \begin{avm}
   \< \textnormal{take},\ \[\makebox[0mm][l]{\textit{ptv-lxm}}\\
    arg-st  & \< NP$_i$, \[ form & advantage \], 
    \[ form & of  \\
       index & j \]\>\\
    sem & \[index & s \\
    rstr & \<  \[ reln & \prd{take\_advantage} \\
               sit & s \\
               exploiter & i \\
               exploited & j \] \> \]
    \] \>
  \end{avm}
\end{center}


\myslide{Our analyses of idioms and passives interact\ldots{}}


\begin{itemize}
\item We generate
\begin{exe} 
  \ex \eng{Advantage was taken of the situation by many people.}
  \ex \eng{Tabs are kept on foreign students.}
\end{exe}
\item But not:
  \begin{exe} 
    \ex \eng{Many people were taken advantage of.}
  \end{exe}
\item Why not?
\end{itemize}

\myslide{Overview}
\MyLogo{Sag, Wasow and Bender (2003) --- Chapter 11}

\begin{itemize}
\item Existentials (\lex{there}, \lex{be})
\item Extraposition (\lex{that}, \lex{it}, LR)
\item Idioms (\lex{take\_advantage}, \ldots)
\end{itemize}

\myslide{P1: \lex{there} and Agreement}
\MyLogo{Based on  Chapter 11, Problem 1, Sag, Wasow and Bender (2003)}

The analysis of existential \lex{there} sentences presented so far
says nothing about verb agreement.

\begin{itemize}
\item[A.] Consult your intuitions (and/or those of your friends, if
  you wish) to determine what the facts are regarding number
 agreement of the verb in \lex{there} sentences.
 Give an informal statement of a generalization covering these facts,
 and illustrate it with both grammatical and ungrammatical examples.
 [{\sl Note: Intuitions vary regarding this question, across both
    individuals and  dialects.  Hence there is more
    than one right answer to this question.}]

\item[B.] How would you elaborate or modify our analysis of the \lex{there}
  construction so as to capture the generalization you have discovered?
  Be as precise as you can.
\end{itemize}


\myslide{P2: Passing Up the Index}
\MyLogo{Based on  Chapter 11, Problem 3, Sag, Wasow and Bender (2003)}

\begin{itemize}
\item[A.] Give the \ft{restr} value that our grammar should assign to
the sentence in~(i).  Be sure that the \ft{sit} value of the \eng{smoke}
predication
is identified with the \ft{annoyance} value of the \prd{annoy}
predication.

\begin{itemize}
\item[(i)] \eng{That Dana is smoking annoys Leslie.}
\end{itemize}

\noindent [{\sl Hint: This sentence involves two of the phenomena
  analyzed in this chapter: predicative complements of \lex{be} and CP subjects.}]

\item[B.] Draw a tree for (i).  Use abbreviations for node labels, but
  show the index on each node.

\item[C.] Explain how the \ft{sit} value of the \prd{smoke} 
%relation
predication
gets identified with the \ft{annoyance} value of the \prd{annoy} 
%relation.
predication.
Be sure to make reference to lexical entries, phrase structure rules,
and principles, as appropriate.

\end{itemize}


%\vbox{


\myslide{P3: An Annoying Problem}
\MyLogo{Based on  Chapter 11, Problem 4, Sag, Wasow and Bender (2003)}

Assume that the lexical entry for the verb \lex{annoy} is the
following: 
%(Note that the first argument of this lexeme
%overrides the \index{defaults} default condition -- associated with the type {\it
%verb-lxm} -- requiring the first argument to be an NP.)

\begin{exe}
\ex\label{s:exi} {\begin{avm}

\< \textnormal{annoy} ,\  \[{\it stv-lxm}\\
              arg-st & \< \[sem\ [index\ {\@1}{\,}]\] ,
                               NP$_i$ \> \\
%			  \[sem\ [index $i$]\] \> \\
              sem    & \[index & $s$\\
                         restr  & \< \[reln & \prd{annoy}\\
                                       sit & \ \  $s$\\
                            	       annoyed & \ \ $i$\\
                            	       annoyance & \ \ {\@1} \] \> \]\] \>
\end{avm}}
\end{exe}

\newpage
\begin{itemize}
% \item[A.] Show the lexeme that this lexical entry licenses, being sure
% to include all of the constraints it inherits from its supertypes.
% Show a lexical entry for {\it annoys} that results from
% applying the 3rd-Singular Verb Lexical Rule to this entry.
\item[A.] What constraints are imposed on the lexical sequences that
  result from applying the \txx{3rd-Singular Verb Lexical Rule} to
  this entry (including those that involve inheritance of constraints
  from the entry's supertypes)?
% \item[B.] Show the lexical entry that results from applying the
% \index{Extraposition Lexical Rule} Extraposition Lexical Rule to
% your answer to part (A).

\item[B.] What constraints are imposed on lexical sequences that
  result from applying the \txx{Extraposition Lexical Rule} to your
  answer to part (A)?

\item[C.] Draw a tree structure for the sentence in (\ref{s:exii}).
  You should show the value of all \ft{sem} features on all of the nodes,
  as well as the \ft{spr} and \ft{comps} features for \eng{annoys}.

\begin{exe}
  \ex\label{s:exii} \eng{It annoys Lee that Fido barks}.
\end{exe}
\newpage
\item[D.] The lexical entry for \lex{annoy} allows NP subjects as
  well, as in (\ref{s:exiii}).  Why doesn't the grammar then also
  license (\ref{s:exiv})?

\begin{exe}
\ex\label{s:exiii} \eng{Sandy annoys me.}

\ex\label{s:exiv} \bad \eng{It annoys me Sandy.}
\end{exe}
\end{itemize}


\end{document}
