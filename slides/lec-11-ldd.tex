\documentclass[a4paper,landscape,headrule,footrule]{foils}
\usepackage{times}
%\usepackage{nttfoilhead}
%\newcommand{\myslide}[1]{\foilhead[-25mm]{\raisebox{12mm}[0mm]{\emp{#1}}}}
%\newcommand{\myslider}[1]{\rotatefoilhead[-25mm]{\raisebox{12mm}[0mm]{\emp{#1}}}}
%\newcommand{\myslider}[1]{\rotatefoilhead{\raisebox{-8mm}{\emp{#1}}}}

\input{headxx.tex}
\usepackage{tikz}
\usepackage{tikz-qtree}

\avmfont{\sc}
\begin{document}
\avmfont{\it}

\header{Lecture 11}{Long Distance Dependencies}{}
\maketitle


\myslide{Overview}
\MyLogo{Sag, Wasow and Bender (2003) --- Chapter 14}
\begin{itemize}
\item Some examples of long-distance dependencies
\item What is new and different about it
%\item Brief sketch of the TG approach
\item Broad outlines of our approach
\item Details of our approach
\item Subject extraction
\item Coordinate Structure Constraint
%\item Wrap up HPSG
\end{itemize}

% \section{Review of Binding}

% \myslide{Binding deals with pronouns}

% \begin{exe}
% \ix She likes herself
% \ix *She$_i$ likes her$_i$.
% \ix We gave presents to ourselves.
% \ix *We gave presents to us.
% \ix We gave ourselves presents
% \ix *We gave us presents.
% \ix *Leslie told us about us.
% \ix Leslie told us about ourselves.
% \ix *Leslie told ourselves about us.
% \ix *Leslie told ourselves about ourselves.
% \end{exe}

% \myslide{Some Terminology}
% \begin{itemize}
% \item \txx{Binding}: The association between a pronoun
% and an antecedent.
% \item \txx{Anaphoric}: A term to describe an element (e.g.
% a pronoun) that derives its interpretation from
% some other expression in the discourse.
% \item \txx{Antecedent}: The expression an anaphoric
% expression derives its interpretation from.
% \item \txx{Anaphora}: The relationship between an
% anaphoric expression and its antecedent.
% \end{itemize}

% \myslide{The Argument Realization Principle}
% \begin{itemize}
% \item We introduce a feature ARG-ST:\\
% \begin{avm}\avmfont{\sc} 
% \[ \asort{word}  
%  syn & \[ val & \[ spr &  \@{A} \\comps &  \@{B} \] \] \\
%  arg-st & \< \@{A} $\oplus$ \@{B} \> \]
% \end{avm}
% \item This is a constraint on the type \val{word}
% \begin{itemize}
% \item It’s neither in \ft{syn} nor \ft{sem}.
% \item It only appears on lexical heads (not
% appropriate for type phrase)
% \item No principle stipulates identity
% between \ft{arg-st}s.
% \end{itemize}
% \end{itemize}

% \myslide{The Binding Principles}
% \begin{itemize}
% \item \txx{Principle A}: A [\ft{mode} \val{ana}] element must be
% outranked by a coindexed element.
% \item \txx{Principle B}: A [\ft{mode} \val{ref}] element must not
% be outranked by a coindexed element.
% \item \txx{Anaphoric Agreement Principle} (AAP):
% Coindexed NPs agree.
% \end{itemize}

% \begin{description}
% \item [Formalization] ~\\[-2ex]
% \begin{itemize}
% \item Definition: If A precedes B on some \ft{arg-st} list,
% then A \txx{outranks} B.
% \item Elements that must be anaphoric --- that is, that
% require an antecedent --- are lexically marked
% [\ft{mode} \val{ana}]. These include reflexive pronouns
% and reciprocals.
% \end{itemize}

% \end{description}
% \myslide{Binding in PPs}
% \begin{itemize}
% \item The Binding Principles by themselves don’t block:
%   \begin{exe}
%     \ix I brought a book with me.
%     \ix *I brought a book with myself.
%     \ix *I mailed a book to me.
%     \ix I mailed a book to myself.
%   \end{exe}
% \item Two Types of Prepositions
% \begin{itemize}
% \item \txx{Argument-marking}: Function like casemarkers, indicating the
% roles of NP referents in the situation denoted by the verb.
%  \begin{itemize}
%   \item Share their objects' \ft{mode} and \ft{index} values.
%   \end{itemize}
% \item \txx{Predicative}: Introduce their own predication.
%   \begin{itemize}
%   \item Introduce their own \ft{mode} and \ft{index} values.
% \end{itemize}
% \end{itemize}
% \end{itemize}


% % \myslide{Redefining Rank}
% %  \begin{itemize}
% % \item If there is an \ft{arg-st} list on which A
% % precedes B, then A outranks B.
% % \item If a node is coindexed with its daughter, they
% % are of equal rank -- that is, they outrank the
% % same nodes and are outranked by the same
% % nodes.
% % \end{itemize}

% \myslide{\eng{I sent a letter to myself}}

% \begin{avmtree}\avmfont{\sc} 
% \br{S}{
%   \br{\@1 NP$_i$}{ \lf{\eng{I}} }
%   \br{VP \[ \q< \@1 \q> \]}{ 
%     \br{V \[ spr & \q< \@1 \q> \\
%              comps & \q< \@2, \@3 \q> \\ 
%              arg-st & \q< \@1, \@2, \@3 \q> \]}{
%       \lf{\eng{sent}}}
%     \br{\@2 NP$_j$}{
%       \br{D}{\lf{\eng{a}}}
%       \br{N$_j$}{\lf{\eng{letter}}}}
%     \br{\@3 PP$_i$}{
%       \br{P$_i$}{\lf{\eng{to}}}
%       \br{NP$_i$ \[ mode & ana \]}{\lf{\eng{myself}}}}
% }}
% \end{avmtree}



% \myslide{The Imperative Rule}

% \begin{avm}\avmfont{\sc} 
% \[ \asort{phrase}  
%    head & verb \\
%    val & \[ spr  & \<  \> \] \\
%    sem & \[ mode & dir \\ index & $s$ \] \]
%               \ $\rightarrow$\ \HD \ 
% \[ \asort{phrase}  
%    head & \[ \asort{verb} form & base \]\\
%    val & \[ spr  & \< NP \[ per 2nd \]  \> \] \\
%    sem & \[ index & $s$ \] \]
% \end{avm}


% \begin{itemize}\addtolength{\itemsep}{-2ex}
% \item Internal structure of a VP
% \item Directive function
% \item Base form
% \item Only 2nd person reflexives
% \end{itemize}

% \myslide{Imperative example}
% %\begin{multicols}{2}
% \begin{avmtree}\avmfont{\sc} 
% \br{S}{
%   \br{VP \[ spr & \q< \@1 NP$_h$ \[ per 2nd \\ num sg \] \]}{ 
%     \br{V \[ spr & \q< \@1 \q>  \\ arg-st & \q< \@1 NP$_h$,  \@2 PP$_i$ \q> \]}{
%       \lf{\eng{vote}}}
%     \br{\@2 PP$_i$}{
%       \br{P$_i$}{\lf{\eng{for}}}
%       \br{NP$_i$ \[ mode & ana \]}{\lf{\eng{yourself}}}}
% }}
% \end{avmtree}
% %\newpage

% % \begin{itemize}
% % \item What's the SPR value on S?
% % %Why?
% % \item What's the SPR value on VP?
% % %Why?
% % \item What's the SPR value on V?
% % %Why?
% % \item Which nodes have ARG-ST?
% % \item Which ARG-ST matters for
% % the licensing of \eng{yourself}?
% % \end{itemize}
% % %\newpage

\myslide{A Note on Adjectives}

\begin{itemize}
 \item Attributive adjectives are related to predicative adjectives by
   a lexical rule that co-indexes the first element of \ft{arg-st} with
   \ft{mod} and sets \ft{spr} to an empty list: 
   \eng{The dog is red}  \into \eng{The red dog}
   \\    \scalebox{0.8}{\begin{avm}\avmfont{\sc}
      \< \textnormal{red}, \ \[ \asort{word}
      head & {\it adj} \\
      % \[ \asort{agr-cat} agr & \[ per & 3 \\ num & sg \] \]
      val & \[ spr & \< \@1 NP \>\\
              comps & \<  \>\\
              mod & \<  \> \] \\
      arg-st &      \< \@1  \>   
      \]\> \ \
\into \ \ 
  \< \textnormal{red}, \ \[ \asort{word}
      head & {\it adj} \\
      % \[ \asort{agr-cat} agr & \[ per & 3 \\ num & sg \] \]
      val & \[ spr & \<  \>\\
              comps & \<  \>\\
              mod & \< \@1 NP \> \] \\
  arg-st &      \< \@1  \>   
     \]\>
     \end{avm}}
 \item We can't just co-index \ft{spr} and \ft{mod}.  Why?
 \item To form sentences, the \ft{spr} adjectives need a
   subject-raising \eng{be} to make the head of the sentence a verb.
 \end{itemize}


\section{Long Distance Dependencies}


\myslide{Examples}
\begin{multicols}{2}
\begin{itemize}
\item \emp{Grammatical}:
  \begin{exe}
    \ex \eng{Did you find something?}  
    \ex \eng{Tell me you talked to someone!}
  \end{exe}
\item \txx{wh-questions}:
  \begin{exe}
    \ex \eng{What did you find? } 
    \ex \eng{Tell me who you talked to}
  \end{exe}
\item  \txx{relative clauses}:
  \begin{exe}
    \ex \eng{the item that I found}
    \ex \eng{the guy who(m) I talked to}
  \end{exe}
\item \emp{Ungrammatical}:
  \begin{exe}
    \ex *\eng{did you find}
    \ex *\eng{you talked to}
  \end{exe}
\item  \txx{topicalization}:
  \begin{exe}
    \ex \eng{The manual, I can't find.}
    \ex \eng{Chris, you should talk to.}
  \end{exe}
\item  \lex{easy}-\txx{adjectives}:
  \begin{exe}
    \ex \eng{My house is easy to find.}
    \ex \eng{Pat is hard to talk to.}
  \end{exe}
\end{itemize}
\end{multicols}

\myslide{What these have in common}
\begin{itemize}
\item There is a \txx{gap}: nothing following \lex{find} and \lex{to},
even though both normally require objects.
\item Something that fills the role of the element
missing from the gap occurs at the beginning of
the clause.
\item We use topicalization and \lex{easy}-adjectives to
illustrate the phenomenon:
\begin{exe}
  \ex \eng{The manual, I can't find \blank}
  \ex \eng{Chris is easy to talk to \blank}
\end{exe}
\end{itemize}

\myslide{Gaps and their fillers can be far apart}
\begin{exe}
  \ex \eng{The solution to this problem, Pat said that
  someone claimed you thought I would never
  find \blank.}
  \ex \eng{Chris is easy to consider it impossible for anyone
  but a genius to try to talk to \blank.}
\end{exe}
\vspace*{-2ex}
\begin{itemize} 
\item Fillers often have syntactic properties associated with their
  gaps
\end{itemize}
\begin{exe}
\ex
\begin{xlist}
  \ex \eng{Him, I haven’t met \blank.}
  \ex \eng{*He, I haven’t met \blank.}
\end{xlist}
\ex
\begin{xlist}
  \ex \eng{The scissors, Pat told us  \blank were missing.}
  \ex \eng{*The scissors, Pat told us  \blank was missing.}
\end{xlist}
\ex
\begin{xlist}
  \ex \eng{On Pat, you can rely \blank}.
  \ex \eng{*To Pat, you can rely \blank.}
\end{xlist}
\end{exe}
\begin{itemize}
\item That’s why we call them \txx{long distance dependencies}
\end{itemize}


% \myslide{LDDs in TG}
% \begin{itemize}
% \item These were long thought to constitute the
% strongest evidence for transformations.
% \item They were handled in TG by moving the filler
% from the gap position.
% \item Case, agreement, preposition selection could
% apply before movement.
% \end{itemize}

% \myslide{A big debate about LDDs in TG}
% \begin{itemize}
% \item Does long-distance movement take place in one fell swoop
% or in lots of little steps?
% \end{itemize}
% % ↑

% % ↑

% % ↑
% % ↑

% Swooping

% Looping


\myslide{Other relevant facts}

\begin{itemize}
\item Various languages show morphological marking
on the verbs or complementizers of clauses
between the filler and the gap.
\item Psycholinguistic evidence indicates increased
processing load in the region between filler and
gap.
% \item This opens the door to non-transformational
% analyses, in which the filler-gap dependency is
% mediated by local information passing.
\end{itemize}

\myslide{A Rough Sketch of Our Approach}
\begin{itemize}
\item A feature \ft{gap} records information about a
missing constituent.
\item The \ft{gap} value is passed up the tree by a new
principle.
\item A new grammar rule expands S as a filler
followed by another S whose \ft{gap} value
matches the filler.
\item Caveat: Making the details of this general
idea work involves several complications.
\item The core idea comes from Gazdar (1981)
\end{itemize}



\myslide{The Feature \ft{gap}}
\begin{itemize}
\item Like valence features and \ft{arg-st}, \ft{gap}’s
value is a list of feature structures (often
empty).  You can have multiple gaps.
\item Subject gaps are introduced by a lexical rule.
\item Non-subject gaps are introduced by revising
the Argument Realization Principle.
\end{itemize}

\myslide{The Revised ARP}

\begin{avm}\avmfont{\sc} 
\[ \asort{word}  
 syn & \[ val & \[ spr &  \@{A} \\comps &  \@{B} $\ominus$ \@{C} \] \\
          gap & \@{C} \] \\
 arg-st & \< \@{A} $\oplus$ \@{B}\> \]
\end{avm}
\begin{itemize}
\item $\ominus$ is a kind of list subtraction 
  \begin{itemize}
  \item it's not always defined (the sublist must exist on the main list)
  \item when defined, it's not always unique 
  \end{itemize}
\item The ARP now says the non-\ft{spr} arguments are
distributed between \ft{comps} and \ft{gap}.
\end{itemize}

\myslide{A Word with a Non-Empty \ft{gap} Value}

    \begin{avm}\avmfont{\sc}
      \< \textnormal{hand}, \ \[ \asort{word}
      syn & \[ head & {\it verb} \\
      % \[ \asort{agr-cat} agr & \[ per & 3 \\ num & sg \] \]
      val & \[ spr & \< \@1 NP \[ case & nom \\ agr & non-3sing \] \> \\ 
      comps & \< \@3 PP:to \> \] \\
      gap &  \< \@2NP \[ acc \] \> \] \\
      arg-st & \< \@1, \@2, \@3 \>
      \]\>
    \end{avm}

% \myslide{How We Want \ft{gap} to Propagate}
% \MyLogo{Traditionally called \ft{slash}: VP/NP}
% \begin{avmtree}\avmfont{\sc}
% \br{S \[ gap & \el \]}{ \br{NP \[ gap & \el \]}{\lf{Kim}}
%   \br{S/NP \[ gap & \q< NP \q> \]}{\br{NP \[ gap & \el \]}{\lf{we}}
%     \br{VP/NP \[ gap & \q< NP \q> \]}{\br{V \[ gap & \el \]}{\lf{know}}
%       \br{S/NP \[ gap & \q< NP \q> \]}{\br{NP \[ gap & \el \]}{\lf{Bobby}}
%         \br{VP/NP \[ gap & \q< NP \q> \]}{\lf{hates}}}}}}
% \end{avmtree}


\myslide{How We Want \ft{gap} to Propagate}
\MyLogo{Traditionally called \ft{slash}: VP/NP}
% Kim we know bobby likes
\newcommand{\avn}[2]{#1\\\smaller\begin{avm}\avmfont{\sc}\[ #2 \]\end{avm}} %avm node
\newcommand{\avN}[2]{#1\,\smaller\begin{avm}\avmfont{\sc}\[ #2 \]\end{avm}} %avm node
\newcommand{\avX}[1]{\begin{avm}\avmfont{\sc}\[ #1 \]\end{avm}} %avm node

\scalebox{0.8}{\begin{tikzpicture}
\tikzset{level distance=8ex}
\tikzset{frontier/.style={distance from root=40ex}}
\tikzset{every tree node/.style={align=center,anchor=north}}
\Tree [.\avn{S}{gap & \el} 
         [.\avn{NP}{gap & \el} Kim ]
         \edge[style=double,very thick];
         [.\avn{S/NP}{gap & \q< NP \q>} 
           [.\avn{NP}{gap & \el} we ]
           \edge[style=double,very thick];
           [.\avn{VP/NP}{gap & \q< NP \q>} 
           \edge[style=double,very thick];
              [.\avn{V}{gap & \el} know ]
              [.\avn{S/NP}{gap & \q< NP \q>} 
                [.\avn{NP}{gap & \el} Bobby ]
                \edge[style=double,very thick];
                [.\avn{VP/NP}{gap & \q< NP \q>} likes ] ] ] ] ]
\end{tikzpicture}}
                              

\myslide{What \ft{gap} Propagation should doing}
\MyLogo{}
\begin{itemize}
\item Pass any \ft{gap} values from daughters up to their
mothers,
\item[\ldots] \emp{except} when the filler is found.
\item For topicalization, we can write the exception into
the grammar rule
\item For \lex{easy}-adjectives, the NP that corresponds to the
gap is the subject, which is introduced by the
Head-Specifier Rule.
\item Since specifiers are not generally gap fillers, we
can't write the gap-filling into the HSR.
\end{itemize}

\myslide{Our Solution to this Problem}
\begin{itemize}
\item For \lex{easy}-adjectives, we treat the adjective formally
as the filler, marking its \ft{spr} value as coindexed
with its \ft{gap} value.
\item We use a feature \ft{stop-gap} to trigger the
  emptying of the \ft{gap} list.
  \begin{itemize}
  \item \ft{stop-gap} stops gap propagation
  \item \lex{easy}-adjectives mark \ft{stop-gap} lexically
  \item a new grammar rule, the \txx{Head-Filler Rule}
    mentions \ft{stop-gap}
  \end{itemize}
\end{itemize}

\myslide{The \ft{gap} Principle}
A local subtree $\Phi$ satisfies the \ft{gap} Principle with respect to a
headed rule  if and only if $\Phi$ satisfies:

% \begin{avmtree}\avmfont{\sc}
%      \br{\[ gap & \( \@{A_1} \ $\oplus$\  \ldots \ $\oplus$\ \@{A_n} \)
%        \ $\ominus$\  \@{A_0}\]}{
%      \br{\[ gap & \@{A_1} \]}{\tlf{~}}
%       \lf{\ldots}
%       \br{\HD \[ gap & \@{A_i} \\ stop-gap & \@{A_0} \]}{\tlf{~}}
%       \lf{\ldots}
%       \br{\[ gap & \@{A_n} \]}{\tlf{~}}}
% \end{avmtree}

\begin{tikzpicture}
\tikzset{level distance=8ex}
   \Tree [.\avX{gap & \( \@{A_1} \ $\oplus$\  \ldots \ $\oplus$\
                         \@{A_n} \) $\ominus$\  \@{A_0}} 
            [.\avX{gap & \@{A_1}} ]
            [.{\ldots}  ]
            \edge[style=double,very thick];
            [.\avN{H}{gap & \@{A_i}\\ stop-gap & \@{A_0}}   ]
            [.{\ldots}  ]
            [.\avX{gap & \@{A_n}} ] ]
\end{tikzpicture}


\myslide{How does \ft{stop-gap} work?}
\begin{itemize}
\item \ft{stop-gap} is empty almost everywhere
\item When a gap is filled, \ft{stop-gap} is nonempty,
and its value is the same as the gap being filled.
\item This blocks propagation of that \ft{gap} value, so
gaps are only filled once.
\item The nonempty \ft{stop-gap} values come from two
  sources:
  \begin{itemize}
  \item a stipulation in the Head-Filler Rule
  \item lexical entries for \lex{easy}-adjectives
  \end{itemize}
\item No principle propagates \ft{stop-gap}
\end{itemize}

\myslide{The Head-Filler Rule}

\begin{avm}\avmfont{\sc}

\[{\it phrase}\] \ \ \into \ \ 
\ \ \@1 \[ gap & \el \]\ \  
\HD \[ head &  \val{verb}  \\
       val & \[ spr & \el\\
                comps & \el\] \\
       gap & \< \@1 \> \\
       stop-gap & \< \@1 \> \]
\end{avm}
\begin{itemize}\addtolength{\itemsep}{-2ex}
\item This only covers gap filling in Ss
\item The filler has to be identical to the \ft{gap} value
\item The \ft{stop-gap} value is also identical
\item The \ft{gap} Principle ensures that the mother's \ft{gap} value is the
empty list
\end{itemize}

\myslide{Gap Filling with \lex{easy}-Adjectives}
  \begin{exe}
    \ex \label{l:easy}
 \scalebox{0.8}{\begin{avm}\avmfont{\sc}
      \< \textnormal{easy}, \ \[ \asort{word}
      syn & \[ head & {\it adj} \\
      % \[ \asort{agr-cat} agr & \[ per & 3 \\ num & sg \] \]
      val & \[ spr & \< \@1 \>\\
              comps & \< \@3 \> \] \\
      stop-gap & \< \@2 \>\] \\
      arg-st &      \< \@1 NP$_i$,  \@3 VP \[ inf & + \\ gap & \< \@2 NP$_i$ \> \], \ldots  \>   
      \]\> \ \
    \end{avm}}

  \end{exe}
\begin{itemize}\addtolength{\itemsep}{-1ex}
\item Because \ft{stop-gap} and \ft{gap} have the same value, that
value will be subtracted from the mother's \ft{gap} value.
\item The first argument is coindexed with the \ft{gap} value,
accounting for the interpretation of the subject as the filler.
\end{itemize}

\myslide{A Tree for \eng{easy to talk to \blank}}
\begin{tikzpicture}
\tikzset{level distance=13ex}
%\tikzset{frontier/.style={distance from root=40ex}}
\tikzset{level 2/.style={level distance=20ex}}
\tikzset{every tree node/.style={align=center,anchor=north}}
\Tree [.\avn{AP}{val & \[ spr & \< \@2NP$_i$ \> \]\\ gap & \el}
\edge[style=double,very thick];
         [.\avn{A}{ val & \[ spr & \< \@2 \> \\ comps & \< \@3 \> \]\\ 
                    gap & \< ~ \> \\ 
                    stop-gap & \< \@1 \>} easy ]
         [.\avn{\bx{3} VP/NP}{val & \[ spr & \< NP \> \]\\ 
                           gap & \< \@1 NP$_i$ \>} \edge[roof]; {to talk to} ] ]

\end{tikzpicture}
% \begin{avmtree}\avmfont{\sc}

% \br{AP \[ val & \[ spr & \< \@2NP$_i$ \> \]\\ gap & \el \]}{
%   \br{A \[ val & \[ spr & \< \@2 \> \\ comps & \< \@3 \> \]\\ 
%     gap & \< ~ \> \\ stop-gap & \< \@1 \>\]}{\lf{easy}}
%   \br{\@3 VP\[ val & \[ spr & \< NP \> \]\\ 
%                gap & \< \@1 NP$_i$ \> \]}{\tlf{to talk to \blank}}}

% \end{avmtree}

\myslide{\ft{stop-gap} Housekeeping}
\begin{itemize}
\item Lexical entries with nonempty \ft{stop-gap}
values (like \lex{easy}) are rare, so \ft{stop-gap} is by
default empty in the lexicon.
\item Head-Specifier and Head-Modifier rules need to
say [\ft{stop-gap} $<$ $>$]
\item Lexical rules preserve \ft{stop-gap} values.
\end{itemize}

\myslide{\ft{gap} Housekeeping}
\begin{itemize}
\item[Q] The initial symbol must say [\ft{gap} $< >$ ]. Why?
  \begin{itemize}
  \item[A] To block \eng{*Pat found} and \eng{*Chris talked to} as
    stand-alone sentences.
  \end{itemize}
\item[Q] The Imperative Rule must propagate \ft{gap} values.
  Why?
  \begin{itemize}
  \item[A] It's not a headed rule, so the effect of the \ft{gap}
    Principle must be replicated
  \item [A] Imperatives can have gaps:
    \\ \eng{This book, put on the top shelf!}
  \end{itemize}
\end{itemize}

\myslide{Sentences with Multiple Gaps}
\begin{itemize}
\item Famous examples:
  \begin{exe}
    \ex \eng{This violin, sonatas are easy to play \blank on \blank.}
    \ex \eng{*Sonatas, this violin is easy to play \blank on \blank.}
  \end{exe}
\item Our analysis gets this:
  \begin{itemize}
  \item The subject of \lex{easy} is coindexed with the first
    element of the \ft{gap} list.
  \item The Head-Filler rule only allows one \ft{gap}
    remaining.
  \end{itemize}
\item There are languages that allow multiple gaps more
generally.
\end{itemize}

\myslide{Where We Are}

\begin{itemize}
\item \txx{filler-gap} structures:
  \begin{exe}
   \ex \eng{The solution to this problem, nobody
   understood \blank}
   \ex \eng{That problem is easy to understand \blank}
 \end{exe}
\item The feature \ft{gap} encodes information about
missing constituents
\item Modified ARP allows arguments that should be on
the \ft{comps} list to show up in the \ft{gap} list
\item \ft{gap} values are passed up the tree by the \ft{gap}
Principle
% \end{itemize}
\newpage
% \myslide{Where We Are (continued)}
\item The feature \ft{stop-gap} signals where \ft{gap} passing
should stop
\item The Head-Filler Rule matches a filler to a \ft{gap} and
(via \ft{stop-gap}) empties \ft{gap}
\item Lexical entries for \lex{easy}-adjectives require a gap in
the complement, coindex the subject with the gap,
and (via \ft{stop-gap}) empty \ft{gap} on the mother
\end{itemize}

\myslide{More Phenomena filler \ldots}

\begin{itemize}
\item Sentences with subject gaps
\item Gaps in coordinate constructions
\end{itemize}

\myslide{Subject Gaps}
\begin{itemize}
\item The ARP revision only allowed missing
complements.
\item But gaps occur in subject position, too:
  \begin{exe}
    \ex \eng{\ul{This problem}, everyone thought  \blank was too easy.}
  \end{exe}
\item We handle these via a lexical rule that, in effect,
moves the contents of the SPR list into the \ft{gap} list
\end{itemize}

\myslide{The Subject Extraction Lexical Rule}

\scalebox{0.9}{\begin{avm}\avmfont{\sc}
\[ \asort{pi-rule}
input & \< X, \[ syn & \[ head & verb \\
                          val & \[ spr & \< Z \> \] \] \\
                 arg-st & \@{A}         \]   \>\\
output & \<Y,  \[ syn & \[val & \[ spr & \< ~~  \>\\ 
                          gap & \< \@1 \> \] \] \\
                 arg-st & \@{A} \< \@1, \ldots \>        \]   \> \]
\end{avm}}

\begin{itemize}
\item NB: This says nothing about the phonology, because the
default for pi-rules is to leave the phonology unchanged.
\end{itemize}

\myslide{A Lexical Sequence This Licenses}

\begin{avm}\avmfont{\sc}
      \< \textnormal{likes}, \ \[ \asort{word}
      syn & \[
      head & {\it verb} \\
      % \[ \asort{agr-cat} agr & \[ per & 3 \\ num & sg \] \]
      val & \[ spr & \<  \>\\
              comps & \< \@2  \> \] \\
      gap & \< \@1 NP \[ case & nom \\ agr & 3sing \] \> \\
      stop-gap & \< ~~ \>\]\\
      arg-st &      \< \@1, \@2 NP:acc  \>   
      \]\>
    \end{avm}
    

\begin{itemize}
\item Note that the ARP is satisfied
\end{itemize}


\myslide{A Tree with a Subject Gap}
\MyLogo{\eng{Kim$_i$ we know  \blank$_i$ likes Bobby}}
\scalebox{0.8}{\begin{tikzpicture}
\tikzset{level distance=8ex}
\tikzset{frontier/.style={distance from root=40ex}}
\tikzset{every tree node/.style={align=center,anchor=north}}
\Tree [.\avn{S}{gap & \el} 
         [.\avn{NP}{gap & \el} Kim ]
         \edge[style=double,very thick];
         [.\avn{S/NP}{gap & \q< NP \q>} 
           [.\avn{NP}{gap & \el} we ]
           \edge[style=double,very thick];
           [.\avn{VP/NP}{gap & \q< NP \q>} 
           \edge[style=double,very thick];
              [.\avn{V}{gap & \el} know ]
              [.\avn{S/NP}{gap & \q< NP \q>} 
              \edge[style=double,very thick];
                [.\avn{V/NP}{gap & \q< NP \q>} likes ]
                [.\avn{NP}{gap & \el} Bobby ]  ] ] ] ]
\end{tikzpicture}}
                    

\myslide{Island Constraints}
\begin{itemize}
\item There are configurations that block filler-gap
dependencies, sometimes called \txx{islands}
\item Trying to explain them has been a central topic of
syntactic research since the mid 1960s
\item We'll look at just one, Ross's so-called
\txx{Coordinate Structure Constraint}
\item Loose statement of the constraint: a constituent
outside a coordinate structure cannot be the filler
for a gap inside the coordinate structure.
\end{itemize}

\myslide{Coordinate Structure Constraint Examples}

\begin{exe}
  \ex \eng{*This problem, nobody finished the extra credit and \blank}
  \ex \eng{*This problem, nobody finished \blank and the extra credit.}
  \ex \eng{*This problem, nobody finished  \blank and started the extra credit.}
  \ex \eng{*This problem, nobody started the extra credit and finished \blank}
  \ex \eng{This problem, everybody started \blank and nobody finished  \blank}
\end{exe}

\begin{itemize}
\item In a coordinate structure,
  \begin{itemize}
  \item no conjunct can be a gap (\txx{conjunct constraint})
  \item no gap can be contained in a conjunct if its filler is
    outside of that conjunct (element constraint)
  \item[\ldots] unless each conjunct has a gap that is paired
with the same filler (\txx{across-the-board exception})
\end{itemize}
\end{itemize}

\myslide{These observations cry out for explanation}
\begin{itemize}
\item In our analysis, the conjunct constraint is an immediate
consequence: individual conjuncts are not on the \ft{arg-st} list
of any word, so they can't be put on the \ft{gap} list
\item The element constraint and ATB exception suggest that \ft{gap}
is one of those features (along with \ft{val} and \ft{form}) that
must agree across conjuncts.
\item Note: There is no ATB exception to the conjunct constraint.
  \begin{exe}
    \ex \eng{*This problem, you can compare only \blank and \blank.}
  \end{exe}
\end{itemize}

\myslide{Our Coordination Rule, so far}

\scalebox{0.75}{\begin{avm}\avmfont{\sc}
    \[  val & \@0 \\ ind & $s_0$ \]
    \ \into \ 
  \[  val & \@0 \\ ind & $s_1$  \] \ \ldots \  
   \[  val & \@0  \\ ind & $s_{n-1}$ \]
    \[ head & conj \\
     ind  & $s_0$ \\ 
    restr & \q< \[ args & \q< $s_1, \ldots, s_{n-1}, s_n$ \q> \] \q>  \]\ \ 
  \[  val & \@0 \\ ind & $s_n$ \]
  \end{avm}}


\begin{itemize}
\item Recall that we have tinkered with what must agree across
conjuncts at various times.
\item Now we'll add \ft{gap} to the things that conjuncts must share
\end{itemize}


\myslide{Our Final Coordination Rule}

\scalebox{0.75}{\begin{avm}\avmfont{\sc}
    \[  val & \@0 \\ ind & $s_0$ \\ gap & \@{A} \]
    \ \into \ 
  \[  val & \@0 \\ ind & $s_1$  \\ gap & \@{A}\] \ \ldots \  
   \[  val & \@0  \\ ind & $s_{n-1}$ \\ gap & \@{A}\]
    \[ head & conj \\
     ind  & $s_0$ \\ 
    restr & \q< \[ args & \q< $s_1, \ldots, s_{n-1}, s_n$ \q> \] \q>  \]\ \ 
  \[  val & \@0 \\ ind & $s_n$ \\ gap & \@{A}\]
  \end{avm}}

\begin{itemize}
\item We've just added \ft{gap} to all the conjuncts and the mother.
\item This makes the conjuncts all have the same gap (if any)
\item Why do we need it on the mother?
\end{itemize}


\myslide{Closing Remarks on LDDs}
\begin{itemize}
\item This is a huge topic; we've only scratched the
surface
\item There are many more kinds of LDDs, which
would require additional grammar rules
\item There are also more island constraints, which also
need to be explained
\item Our account of the coordinate structure constraint
(based on ideas of Gazdar) is a step in the right
direction, but it would be even better to explain why certain
features must agree across conjuncts.
\end{itemize}


\myslide{Overview of LDD}
\begin{itemize}
\item Some examples of the phenomenon
\item What is new and different about it
\item Broad outlines of our approach
\item Details of our approach
\item Subject extraction
\item Coordinate Structure Constraint
\end{itemize}

\myslide{P0: Semantics are \lex{easy}}

Add the semantics to the lexeme \lex{easy} given on slide \pageref{l:easy}.

Then give the full rels list for the top node (i.e. the whole
sentence) for (\ref{s:easyi}) and (\ref{s:easyii}).  What is the deep
subject of \lex{easy} in each sentence?  

 \begin{exe}
    \ex \label{s:easyi}\eng{My house is easy to find.}
    \ex \label{s:easyii}\eng{Pat is easy to talk to.}
  \end{exe}

\myslide{P1: A Tree with a Gap}
\MyLogo{Based on  Chapter 14, Problem 1, Sag, Wasow and Bender (2003)}
Draw a tree for (\ref{s:toy-exxb}).  
%You should provide about the same
%level of formal detail as in (\ref{gapbb-tree}).
Use abbreviations for the node labels, and show the value of \ft{gap}
on all nodes.  Show the value of \ft{stop-gap} on any node where it is
non-empty.

\begin{exe}
  \ex \label{s:toy-exxb} \eng{This baby, I know that they handed a toy to {\blank}}
\end{exe}

\myslide{P2: Blocking Filled Gaps}
\MyLogo{Based on  Chapter 14, Problem 2, Sag, Wasow and Bender (2003)}

Examples (i) and (ii) are well-formed, but example (iii) is
ungrammatical:

\begin{exe}
\exi{(i)}  \eng{Pat thinks that I rely on some sort of trick.}
\exi{(ii)} \eng{This mnemonic, Pat thinks that I rely on.}
\exi{(iii)} *\eng{This mnemonic, Pat thinks that I rely on some sort of trick.}
\end{exe}
Explain in detail why the mechanisms that license (i) and (ii) do not
also permit (iii).

\myslide{P3: Subject Gaps} 
\MyLogo{Based on Chapter 14, Problem 3, Sag, Wasow and Bender (2003)} 

This problem is to make sure you understand how our
analysis accounts for examples like (\ref{se-s}).

\begin{exe}
\ex \label{se-s}
\begin{xlisti}
\ex \label{se-sa}\eng{Which candidates do you think like %raw oysters?
oysters on the half-shell?}
\ex \label{se-sb} \eng{That candidate, I think likes %raw oysters.
oysters on the half-shell.}
\end{xlisti}
\end{exe}

\begin{itemize}
\item [A.] Sketch the family of lexical sequences for \lex{likes} that is the input
to the Subject Extraction Lexical Rule.

\item [B.] Sketch the family of lexical sequences for \lex{likes} that is the
corresponding output of the Subject Extraction Lexical Rule.

%\item [(b)] Sketch the lexical entry for {\it likes} that is the output
%of the Subject Extraction Lexical Rule.

\item[C.] Sketch the %entire 
tree for the sentence in (\ref{se-sb}).
%Provide about the same level of formal detail as in (\ref{gapbb-tree}).
Use abbreviations for node labels, but show the value of \ft{gap} on
all nodes and the value of \ft{stop-gap} on any node where it is non-empty.
You may abbreviate the structure over the NP \eng{oysters on the
half-shell} with a triangle.
%Be sure to explain how our grammar ensures that all relevant feature
%specifications are as they are in your sketch.

\item[D.] Does our analysis correctly predict the contrast between
(\ref{se-sb}) and \ref{onew}?\\
\begin{exe}
  \ex \label{onew} *\eng{Those candidates, I think likes %raw oysters.
    oysters on the half-shell.}
\end{exe}
Explain why or why not.
\end{itemize}

% \myslide{Questions (2017)}
% \input{lec-11-ldd-q}

\myslide{Acknowledgments and References}

\begin{itemize}
\item Course design and slides borrow heavily from Emily Bender's course:
\textit{Linguistics 566: Introduction to Syntax for Computational Linguistics}
\\ \url{http://courses.washington.edu/ling566}
\end{itemize}



\end{document}
%%% Local Variables: 
%%% coding: utf-8
%%% mode: latex
%%% TeX-PDF-mode: t
%%% TeX-engine: xetex
%%% End: 

