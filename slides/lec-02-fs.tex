\documentclass[a4paper,landscape,headrule,footrule]{foils}
%\usepackage{nttfoilhead}
%\newcommand{\myslide}[1]{\foilhead[-25mm]{\raisebox{12mm}[0mm]{\emp{#1}}}}
%\newcommand{\myslider}[1]{\rotatefoilhead[-25mm]{\raisebox{12mm}[0mm]{\emp{#1}}}}
%\newcommand{\myslider}[1]{\rotatefoilhead{\raisebox{-8mm}{\emp{#1}}}}
%%
%% FIXME: not all of the features are smallcaps :-(
%%
%%
%%% macros for Theories of Grammar
%%%
\usepackage{polyglossia}
\setdefaultlanguage{english}
%\setmainfont{TeX Gyre Pagella}


\newcommand{\logo}{~}
\newcommand{\header}[3]{%
\title{\vspace*{-2ex} \large HG4041 Theories of Grammar
\\[2ex] \Large  \emp{#2} \\ \emp{#3}}
\author{\blu{Francis Bond}   \\ 
\normalsize  \textbf{Division of Linguistics and Multilingual Studies}\\
\normalsize  \url{http://www3.ntu.edu.sg/home/fcbond/}\\
\normalsize  \texttt{bond@ieee.org}}
\MyLogo{HG4041 (2020)}
\renewcommand{\logo}{#2}
\hypersetup{
   pdfinfo={
     Author={Francis Bond},
     Title={#1: #2},
     Subject={HG4041: Theories of Grammar},
     Keywords={Syntax, Semantics, HPSG, Unification, Constructions},
     License={CC BY 4.0}
   }
 }

\date{#1
  \\ Location: LHN-TR+36}
}
\usepackage[hidelinks]{hyperref}




\usepackage{xcolor}
\usepackage{graphicx}
\newcommand{\blu}[1]{\textcolor{blue}{#1}}
\newcommand{\grn}[1]{\textcolor{green}{#1}}
\newcommand{\hide}[1]{\textcolor{white}{#1}}
\newcommand{\emp}[1]{\textcolor{red}{#1}}
\newcommand{\txx}[1]{\textbf{\textcolor{blue}{#1}}}
\newcommand{\lex}[1]{\textbf{\mtcitestyle{#1}}}

\usepackage{pifont}
\renewcommand{\labelitemi}{\textcolor{violet}{\ding{227}}}
\renewcommand{\labelitemii}{\textcolor{purple}{\ding{226}}}

\newcommand{\subhead}[1]{\noindent\textbf{#1}\\[5mm]}

\newcommand{\Bad}{\emp{\raisebox{0.15ex}{\ensuremath{\mathbf{\otimes}}}}}
\newcommand{\bad}[1]{*\eng{#1}}

\newcommand{\com}[1]{\hfill (\emp{#1})}%

\usepackage{relsize,xspace}
\newcommand{\into}{\ensuremath{\rightarrow}\xspace}
\newcommand{\tot}{\ensuremath{\leftrightarrow}\xspace}
\usepackage{url}
\newcommand{\lurl}[1]{\MyLogo{\url{#1}}}

\usepackage{mygb4e}
\newcommand{\lx}[1]{\textbf{\mtciteform{#1}}}
\newcommand{\ix}{\ex\slshape}
\let\eachwordone=\slshape



\newcommand{\ent}{\ensuremath{\Rightarrow}\xspace}
\newcommand{\ngv}{\ensuremath{\not\Rightarrow}\xspace}
%\usepackage{times}
%\usepackage{nttfoilhead}
\newcommand{\myslide}[1]{\foilhead[-25mm]{\raisebox{12mm}[0mm]{\emp{#1}}}\MyLogo{\logo}}
\newcommand{\myslider}[1]{\rotatefoilhead[-25mm]{\raisebox{12mm}[0mm]{\emp{#1}}}}
%\newcommand{\myslider}[1]{\rotatefoilhead{\raisebox{-8mm}{\emp{#1}}}}

\newcommand{\section}[1]{\myslide{}{\begin{center}\Huge \emp{#1}\end{center}}}



\usepackage[lyons,j,e,k]{mtg2e}
%\renewcommand{\mtcitestyle}[1]{\textcolor{-red!75!green!50}{\textsl{#1}}}
\renewcommand{\mtcitestyle}[1]{\textcolor{teal}{\textsl{#1}}}
\newcommand{\iz}[1]{\texttt{\textup{#1}}}
\newcommand{\gm}{\textsc}
\usepackage[normalem]{ulem}
\newcommand{\ul}{\uline}
\newcommand{\ull}{\uuline}
\newcommand{\wl}{\uwave}
\newcommand{\vs}{\ensuremath{\Leftrightarrow}~}
%%%
%%% Bibliography
%%%
\usepackage{natbib}
%\usepackage{url}
\usepackage{bibentry}


%%% From Tim
\newcommand{\WMngram}[1][]{$n$-gram#1\xspace}
\newcommand{\infers}{$\rightarrow$\xspace}

\usepackage[utf8]{inputenc}

\usepackage{rtrees,qtree}
\renewcommand{\lf}[1]{\br{#1}{}}
\usepackage{avm}
%\avmoptions{topleft,center}
\newcommand{\ft}[1]{\textsc{#1}}
\renewcommand{\val}[1]{\textit{#1}}
\newcommand{\typ}[1]{\textit{#1}}
\newcommand{\prd}[1]{\textbf{#1}}
\avmfont{\sc}
\avmvalfont{\it}
\avmsortfont{\smaller[2] \it}
\usepackage{multicol}
\newcommand{\blank}{\rule{3em}{1pt}\xspace}

% \usepackage{pst-node}
\newcommand{\OV}[1]{\ovalnode[linestyle=dotted,linecolor=red]{A}{#1}}
\newcommand{\OVB}[1]{\ovalnode[linestyle=dotted,linecolor=blue]{A}{#1}}

%%% From CSLI book
\newcommand{\mc}{\multicolumn}
\newcommand{\HD}{\textbf{H}\xspace}
\newcommand{\el}{\< \>}
\makeatother
\long\def\smalltree#1{\leavevmode{\def\\{\cr\noalign{\vskip12pt}}%
\def\mc##1##2{\multispan{##1}{\hfil##2\hfil}}%
\tabskip=1em%
\hbox{\vtop{\halign{&\hfil##\hfil\cr
#1\crcr}}}}}
\makeatletter

\newcommand{\A}{\noindent\textbf{A}: }
\newcommand{\Q}{\noindent\textbf{Q}: }
%\newcommand{\C}{\noindent\textbf{C}: }


\usepackage{forest}
%\avmfont{\sc}
\begin{document}
%\avmfont{\it}

\header{Lecture 2}{Feature Structures}{Headed Rules, Trees}
\maketitle



\myslide{Overview}
\begin{itemize}
\item Review: problems with CFG
\item Modeling
\item Feature structures, unification (pizza)
\item Features for linguistic description
\item Reformulate grammar rules
\item Notion of head/headedness
\item Licensing of trees
\end{itemize}


\myslide{Our Goals}
\begin{itemize}
\item Descriptive, generative grammar
  \begin{itemize}
  \item Describing English (in this case)
  \item Generating all possible well-formed
    sentences (and no ill-formed ones)
  \item Assigning appropriate structures
  \end{itemize}
\item Design/discover an appropriate \txx{type} of
model (through incremental improvement)

\item Create a particular model (grammar
fragment) for English
\end{itemize}

\myslide{Problems with Context-Free Grammar}
\MyLogo{atomic node labels}

\begin{itemize}
\item Potentially arbitrary rules
\item Gets clunky quickly with cross-cutting properties
\item Not quite powerful enough for natural languages
\item \emp{Solution}: Replace atomic node labels with feature structures.
\end{itemize}

\myslide{Cross-cutting Grammatical Properties}
\MyLogo{}
\vfill
\begin{large}
  \begin{tabular}[t]{l|l|l|}
    \mc{1}{c}{} & \mc{1}{c}{3rd singular subject} & \mc{1}{c}{plural subject}\\ \cline{2-3}
    direct object NP & {\it denies} & {\it deny} \\ \cline{2-3}
    no direct object NP & {\it disappears} & {\it disappear} \\ \cline{2-3}
  \end{tabular}
\end{large}

\myslide{Language Models}

  \begin{itemize}
  \item Two Kinds of Language Models
    \begin{itemize}
    \item Speakers’ internalized knowledge (their grammar)
    \item Set of sentences in the language
    \end{itemize}
  \item Things Involved in Modeling Language
    \begin{itemize}
    \item Real world entities (utterance types)
    \item Models (fully specified trees)
    \item Descriptions of the models (rules, principles, lexical entries)
    \end{itemize}
  \end{itemize}
\myslide{Feature Structure Descriptions}
\vfill
\begin{Large}
  \begin{avm}
    \[ feature$_1$ &\ \ \  value$_1$\\
    feature$_2$ &\ \ \ value$_2$\\[-1ex]
    \hfil.\ .\ .\hfil\\
    feature$_n$ &\ \ \  value$_n$\]
  \end{avm}
\end{Large}

\bigskip
\bigskip
\bigskip

Like a dictionary in python (feature=key, value=value).

\myslide{A Pizza Type Hierarchy}
%\vspace*{-.5ex}
\begin{avmtree}%\avmfont{\sc}
\br{\val{pizza-thing}}{
    \br{\[ \asort{pizza} 
      crust & \{ \normalfont thick, thin, stuffed \}
     \\ toppings & topping-set \]}{ }
    \br{ \[ \asort{topping-set} olives & \{ +, -\} 
      \\ onions & \{ +, -\} \\ mushrooms& \{ +, -\}  \]}{
       \br{\val{vegetarian} }{ }
       \br{\[ \asort{non-vegetarian} sausage & \{ +, -\} 
         \\ pepperoni & \{ +, -\} \\ ham & \{ +, -\} \]}{}}}
\end{avmtree}
\vspace*{-2.5ex}

\myslide{Types}

\begin{tabular}{lll}
  Type & Features/Values & IST \\ \hline
  \smaller[2]\it pizza-thing & & NONE\\
  \smaller[2]\it  pizza & {\small  \begin{avm}%\avmfont{\sc}
    \[ crust & \{ \normalfont thick, thin, stuffed\}
    \\ toppings & \it topping-set \]
  \end{avm}}   & \smaller[2]\it  pizza-thing \\
  \smaller[2]\it topping-set &{\small  \begin{avm}%\avmfont{\sc}
      \[  olives & \{ +, -\} 
    \\ onions & \{ +, -\} \\ mushrooms& \{ +, -\}  \]
 \end{avm}}   & \smaller[2]\it pizza-thing \\
  \smaller[2]\it vegetarian &  & \smaller[2]\it topping-set \\
 \smaller[2]\it non-vegetarian &{\small  \begin{avm}%%%\avmfont{\sc}
     \[ sausage & \{ +, -\}  \\ pepperoni & \{ +, -\} 
     \\ ham & \{ +, -\} \]
 \end{avm}}   & \smaller[2]\it topping-set \\

\end{tabular}

\myslide{Type Hierarchies}

A type hierarchy \ldots
\begin{itemize}
\item[\ldots]  states what kinds of objects we claim exist (the
types)

\item[\ldots]  organizes the objects hierarchically into classes
with shared properties (the type hierarchy)

\item[\ldots]   states what general properties each kind of object
has (the feature and feature value declarations).
\end{itemize}

\myslide{Pizza Descriptions and Pizza Models}

\begin{center}
  \begin{avm}%\avmfont{\sc}
    \[ \asort{pizza}
    crust & thick \\
    toppings &
    \[ \asort{vegetarian} olives & + \\ onions & + \] \]
  \end{avm}
\end{center}
How many pizza models (by definition, fully resolved) satisfy this description?


\myslide{Answer: 2}


\begin{center}
  \begin{avm}%\avmfont{\sc}
  \[ \asort{pizza}
  crust & thick \\
  toppings &
  \[ \asort{vegetarian} olives & + \\ onions & + \\ mushrooms& $-$ \] \]
\end{avm}
\begin{avm}%\avmfont{\sc}
  \[ \asort{pizza}
  crust & thick \\
  toppings &
  \[ \asort{vegetarian} olives & + \\ onions & + \\ mushrooms & + \] \]
\end{avm}

\end{center}
\myslide{Pizza Descriptions and Pizza Models}
\begin{center}
  \begin{avm}%\avmfont{\sc}
    \[ \asort{pizza}
    crust & thick \\
    toppings &
    \[ \asort{vegetarian} olives & + \\ onions & + \] \]
  \end{avm}
\end{center}
\begin{itemize}
\item How many pizzas-in-the-world do the pizza
models correspond to?
\item [A]: A large, constantly-changing number.
\item the ‘type’/‘token’ distinction applies to sentences as well
\end{itemize}

\myslide{Combining Constraints}

\begin{center}
  \begin{avm}%\avmfont{\sc}
    \[ \asort{pizza}
    crust & thick \\
    toppings &
    \[ olives & + \\ ham & $-$ \] \]
    \& 
   \[ \asort{pizza}
    toppings &
    \[ olives & + \\ onions & + \] \]
  \end{avm} 
\end{center}

Unification is also written as: $\sqcap$.

\myslide{Combining Constraints}

\begin{center}
  \begin{avm}%\avmfont{\sc}
    \[ \asort{pizza}
    crust & thick \\
    toppings &
    \[ olives & + \\ ham & $-$ \\ onions & + \] \]
  \end{avm}
\end{center}



\myslide{Combining Constraints}
\begin{center}
  \begin{avm}%\avmfont{\sc}
    \[ \asort{pizza}
    crust & thick \\
    toppings &
    \[ olives & + \\ ham & $-$ \] \]
    \& 
   \[ \asort{pizza}
   crust & thin \\
    toppings &
    \[ olives & + \\ onions & + \] \]
  \end{avm} 
\begin{huge}
\\[2ex]   = $\varphi$
\end{huge}
\end{center}

NULL is also written as:  $\perp,  \emptyset, \phi$.

%\myslide{Combining Constraints}

\myslide{Combining Constraints}
\begin{center}
  \begin{avm}%\avmfont{\sc}
    \[ \asort{pizza}
    crust & thin \\
    toppings &
    \[ olives & + \\ ham & $-$ \] \]
    \& 
   \[ \asort{pizza}
   crust & thin \\
    toppings &
    \[ \asort{vegetarian} \] \]
  \end{avm} 
\begin{huge}
\\[2ex]   = $\varphi$
\end{huge}
\end{center}

\textit{vegetarian} has no feature \textsc{ham}.

\myslide{A New Theory of Pizzas}
\begin{center}
  \begin{avm}%\avmfont{\sc}
    \[ \asort{pizza}
    crust & \{ \normalfont thick, thin, stuffed \} \\
    one-half & \val{topping-set} \\
    other-half & \val{topping-set} \]
  \end{avm}
\end{center}


\myslide{Combining Constraints}
\begin{center}
  \begin{avm}%\avmfont{\sc}
    \[ \asort{pizza}
    one-half &
    \[ olives & + \\ ham & $-$ \] \]
    \& 
   \[ \asort{pizza}
    other-half &
    \[  olives & $-$ \\ 
         ham & + \] \] 
  \end{avm}
  \\ = 
  \begin{avm}%\avmfont{\sc}
 \[ \asort{pizza}
   one-half &
   \[ olives & + \\ ham & $-$ \] \\
   other-half &
   \[  olives & $-$ \\ ham & + \] \]
  \end{avm} 
\end{center}


\myslide{Identity Constraints (tags)}
\begin{center}
   \begin{avm}%\avmfont{\sc}
 \[ \asort{pizza}
   one-half &
   \[ olives & \@1 \\ ham & \@2 \] \\
   other-half &
   \[  olives & \@1 \\ ham & \@2 \] \]
  \end{avm} 
\end{center}

\myslide{Combining Constraints}
\begin{center}
  \begin{avm}%\avmfont{\sc}
    \[ \asort{pizza}
    one-half &
    \@1 \[ olives & + \\ ham & $-$ \]
    \\   other-half & \@1 \]
    \& 
   \[ \asort{pizza}
    other-half &
    \[  olives & + \\ mushroom & + \] \] 
  \end{avm}
  \\ = 
  \begin{avm}%\avmfont{\sc}
 \[ \asort{pizza}
   one-half &
    \@1 \[ olives & + \\ ham & $-$ \\ mushroom & + \] \\
   other-half & \@1 \]
 \end{avm} 
\end{center}


\myslide{Note}
\begin{center}
 \begin{avm}%\avmfont{\sc}
 \[ \asort{pizza}
   one-half &
    \@1 \[ olives & + \\ ham & $-$ \\ mushroom & + \] \\
   other-half & \@1 \]
 \end{avm} 
\\ =
  \begin{avm}%\avmfont{\sc}
 \[ \asort{pizza}
   one-half &
    \@1 \\
   other-half & \@1 \[ olives & + \\ ham & $-$ \\ mushroom & + \] \]
  \end{avm} 
\end{center}

\myslide{Combining Constraints}

\begin{center}
  \begin{avm}%\avmfont{\sc}
    \[ \asort{pizza}
    crust & thick \\
    one-half &
    \@1 \[ olives & + \\ onion & $-$ \] 
    \\ other-half & \@1  \textit{vegetarian} \]
    \& 
   \[ \asort{pizza}
   crust & thin \\
    one-half &
    \[ olives & + \\ pepperoni & + \] \]
  \end{avm} 
\begin{huge}
\\[2ex]   = $\varphi$
\end{huge}
\end{center}



\myslide{Why combine constraints?}
\begin{itemize}
\item The pizza example illustrates how
unification can be used to combine
information from different sources.

\item In our grammar, information will come
from lexical entries, grammar rules, and
general principles.
\end{itemize}

\myslide{Linguistic Application of Feature Structures:}
Making the Mnemonic Meaningful
\begin{itemize}
\item What do these CFG categories have in common?
  \begin{itemize}
  \item  NP \& VP:  are both phrases
  \item N \& V:  are both words
  \item NP \& N: are both ‘nouny’
  \item VP \& V:  are both ‘verby’
  \end{itemize}
\end{itemize}

\myslide{The Beginnings of Our Type Hierarchy}


\begin{center}
  \begin{tree}\renewcommand{\lf}[1]{\br{#1}{}}
\it
\br{feat-struc}{ \br{expression}{\lf{word}
				       	\lf{phrase}}
			\br{pos}{\lf{noun}
				 \lf{verb}
				 \lf{det}
				 \lf{prep}
				 \lf{adj}
				 \lf{conj}}}
\end{tree}
\end{center}

  \begin{tabular}{lclc}
NP &  \begin{avm}%\avmfont{\sc}
\[ \asort{phrase}
   head & {\it noun} \]
\end{avm} 
& VP & 
  \begin{avm}%\avmfont{\sc}
\[ \asort{phrase}
   head & {\it verb} \]
\end{avm} \\
N &
    \label{bird-ex}
\begin{avm}%\avmfont{\sc}
\< fly , \ \[ \asort{word}
               head & {\it noun} \]\>
\end{avm}
& V &
    \label{bird-ex}
\begin{avm}%\avmfont{\sc}
\< fly , \ \[ \asort{word}
               head & {\it verb} \]\>
\end{avm}
  \end{tabular}




\myslide{Type Hierarchy for Parts of Speech II}

\begin{center}
  \begin{avmtree}\renewcommand{\lf}[1]{\br{#1}{}}%%\avmfont{\sc}
\it
\br{feat-struc}{ \br{ expression  \[ \textsc{head} \] }{\lf{word}
                                  \lf{phrase}}
			\br{pos}{\br{agr-pos \[ \textsc{agr} \] }{ \lf{noun}
                            \lf{verb \[ \textsc{aux} \]}
				 \lf{det}}
				 \lf{prep}
				 \lf{adj}
				 \lf{conj}}}
\end{avmtree}
\end{center}


\myslide{Agreement}

We need more information to make words agree.

\begin{avm}%\avmfont{\sc}
\< \textnormal{fly}, \ \[ \asort{word}
               head & {\it noun} \[ \asort{agr-cat} agr 
               & \[ per & 3 \\ num & sg \] \] \]\>
\end{avm}

\begin{avm}%\avmfont{\sc}
\< \textnormal{flies}, \ \[ \asort{word}
               head & {\it verb}
               \[ \asort{agr-cat} agr & \[ per & 3 \\ num & sg \] \]
               \]\>
\end{avm}

\myslide{Agreement (S  $\rightarrow$  NP VP)}

\begin{avm} %\avmfont{\sc}
\[{\it phrase}\\
  head & {\@1}{\it verb}\\
  val & \[ comps & itr\\
           spr & $+$\]\] \ $\rightarrow$\ \ \avml\hfil NP\\[-1ex]
                              \[ head \ \[agr & {\@2}\]\]\avmr \ \ \[{\it phrase}\\
                                        head & {\@1}\[agr & {\@2}\]\\
                                        val & \[ comps & itr\\
                                                 spr & $-$\]\]
\end{avm}

The values on \ft{agr} for the subject NP and verb phrase must be identical.


\myslide{A Simple Feature for Valence}

\begin{tabular}[t]{cc}
\begin{avm}%\avmfont{\sc}
  iv =  \[{\it word}\\
  head & {\it verb}\\
  val & \[ {\it val-cat}\\
  comps & itr\]\]
\end{avm} &
\begin{avm}%\avmfont{\sc}
  tv  =  \[{\it word}\\
  head & {\it verb}\\
  val & \[ {\it val-cat}\\
  comps & str \]\]
\end{avm}\\
\begin{avm}%\avmfont{\sc}
  dtv =  \[{\it word}\\
  head & {\it verb}\\
  val & \[ {\it val-cat}\\
  comps & dtr \]\]
\end{avm}
\end{tabular}

\ft{comps} controls how many \txx{complements} are possible.


\myslide {Head-Complement Rules}
\begin{small}
1 \begin{avm}%\avmfont{\sc} 
\[{\it phrase}\\
  head & {\@1} \\
  val & \[ comps & itr\\
           spr & $-$\]\] \ $\rightarrow$\ \ \[{\it word}\\
                                               head & {\@1}\\
                                               val & \[ comps & itr\\
                                                        spr & $-$\]\]
\end{avm}

2 \begin{avm}%\avmfont{\sc} 
\[{\it phrase}\\
  head & {\@1}\\
  val & \[ comps & itr\\
           spr & $-$ \]\] \ $\rightarrow$\ \ \[{\it word}\\
                                               head & {\@1}\\
                                               val & \[ comps & str\\
                                                        spr & $-$\]\]\ \ NP
\end{avm}

3 \begin{avm}%\avmfont{\sc}
\[{\it phrase}\\
  head & {\@1}\\
  val & \[ comps & itr\\
           spr & $-$ \]\] \ $\rightarrow$\ \ \[{\it word}\\
                                               head & {\@1}\\
                                               val & \[ comps & dtr\\
                                                        spr & $-$\]\]\ \ NP\ \ NP
\end{avm}
\end{small}

\myslide{Underspecification}

\begin{avm}%\avmfont{\sc}
 V \ = \ \[{\it word}\\
              head & {\it verb}\]
            \end{avm}

\begin{avm}%\avmfont{\sc} 
            VP \ = \ \[{\it phrase}\\
              head & {\it verb}\]
            \end{avm}

\begin{avm}%\avmfont{\sc}
 \[ head & {\it verb}\]
\end{avm} 


\myslide{Another Valence Feature}
\begin{tabular}[t]{cc}
\begin{avm} \ \ \ \ NP  =  \ \[{\it phrase}\\
                              head & {\it noun} \\
			      val & \[ {\it val-cat}\\
				       comps & itr\\
				       spr & $+$\]\]
\end{avm}&
\begin{avm} NOM =  \ \[{\it phrase}\\
                head & {\it noun} \\
		val & \[ {\it val-cat}\\
			 comps & itr\\
                         spr & $-$\]\]
\end{avm}
\end{tabular}


\ft{spr} controls the \txx{specifier} (determiner and/or subject)

\myslide{spr and Verbs}

\begin{tabular}[t]{cc}
\begin{avm} \hskip 11pt\ \ \ S  =  \ \[{\it phrase}\\
                                       head & {\it verb} \\
				       val & \[{\it val-cat}\\
					       comps & itr\\
					       spr & $+$\]\]
\end{avm}&
\begin{avm}\ \ VP =  \ \[{\it phrase}\\
                head & {\it verb} \\  
	        val & \[ {\it val-cat}\\
			 comps & itr\\
                         spr & $-$\]\]
\end{avm}
\end{tabular}

\myslide{S and NP}

\begin{avm}
\[ val & \[ {\it val-cat}\\
	    comps & itr\\
            spr & $+$ \]\]
\end{avm}

\begin{itemize}
\item both are fully saturated: specified and no more complements
  \begin{exe}
    \ex \eng{We created a monster}
    \ex \eng{our creation of a monster}
  \end{exe}
\end{itemize}

\myslide{Type Hierarchy So Far}

\begin{center}
  \begin{avmtree}\renewcommand{\lf}[1]{\br{#1}{}}%\avmfont{\sc}
\it
\br{feat-struc}{ \br{ expression  \[ \textsc{head, val} \] }{\lf{word}
                                  \lf{phrase}}
			\br{pos}{\br{agr-pos \[ \textsc{agr} \] }{ \lf{noun}
                            \lf{verb \[ \textsc{aux} \]}
				 \lf{det}}
				 \lf{prep}
				 \lf{adj}
				 \lf{conj}}
                     \br{val-cat \[ \textsc{spr, comps} \] }{ }
                     \br{agr-cat \[ \textsc{per, num} \] }{ } 
                             }
\end{avmtree}
\end{center}


\myslide{Heads}
\begin{itemize}
\item Intuitive idea: A phrase typically contains a word that
  determines its most essential properties, including
  \begin{itemize}
  \item where it occurs in larger phrases
  \item what its internal structure is
  \end{itemize}
\item This is called the head
\item The term \txx{head} is used both for the head word in a
  phrase and for all the intermediate phrases containing
  that word
\item NB: Not all phrases have heads
\\ can you think of a phrase that doesn't?
\end{itemize}

\myslide{Formalizing the Notion of Head}
\begin{itemize}
\item Expressions have a feature \ft{head}
\item \ft{head}’s values are of type \val{pos} (part-of-speech)
\item For \ft{head} values of type \val{agr-cat}, \ft{head}’s
  value also includes the feature \ft{agr}
\item Well-formed trees are subject to the \txx{Head Feature Principle}
\end{itemize}

\myslide{The Head Feature Principle}
\begin{itemize}
\item Intuitive idea: Key properties of phrases are
shared with their heads
\item The \txx{HFP}:
  \begin{quote} \large    In any headed phrase, the \ft{head}
    value of the mother and the head daughter
    must be identical.    
  \end{quote}
\item Sometimes described in terms of properties
“percolating up” or “filtering down”, but this
is just metaphorical talk
\item  the \index{head daughter (H)} head daughter in a headed-rule will be labeled with  `{\bf H}'.

  \begin{quote}
    [ \typ{type} ]  \ \ \into  \ \ \ldots \ \ \HD [\ \  \ ]  \ \ \ldots
  \end{quote}


\end{itemize}

\myslide{A Tree is Well-Formed if \ldots }
\begin{itemize}
\item It and each subtree are licensed by a grammar rule
or lexical entry
\item All general principles (like the HFP) are satisfied.
\item NB: Trees are part of our model of the language,
so all their features have values (even though we
will often be economical and leave out the values
irrelevant to our current point).
\end{itemize}

\myslide{Question:}
Do phrases that are not headed have
\ft{head} features?

\myslide{Reformulating the Grammar Rules I}

Which simple phrase structure rules (Ch 2) do these correspond to?

\begin{itemize}
\item  Head-Complement Rule 1:

\begin{avm} 
\[{\it phrase}\\
  val & \[ comps & itr\\
           spr & $-$\]\] \ $\rightarrow$\ \HD \ \[{\it word}\\
                                               val & \[ comps & itr\\
                                                        spr & $-$\]\]
\end{avm}
\item  Head-Complement Rule 2:

\begin{avm} 
\[{\it phrase}\\
  val & \[ comps & itr\\
           spr & $-$ \]\] \ $\rightarrow$\ \HD \ \[{\it word}\\
                                               val & \[ comps & str\\
                                                        spr & $-$\]\]\ \ NP
\end{avm}
\newpage
\item  Head-Complement Rule 3:

\begin{avm} 
\[{\it phrase}\\
  val & \[ comps & itr\\
           spr & $-$ \]\] \ $\rightarrow$\ \HD \ \[{\it word}\\
                                               head & {\@1}\\
                                               val & \[ comps & dtr\\
                                                        spr & $-$\]\]\ \ NP\ \ NP
\end{avm}
\end{itemize}

\myslide{Reformulating the Grammar Rules II}
\MyLogo{Head-Specifier Rule 2 updated in the problems}

\begin{itemize}
\item Head-Specifier Rule 1:

\begin{avm} \[{\it phrase}\\
val & \[ comps & itr\\
         spr & $+$\]\] \ $\rightarrow$\ \ \avml\hfil NP\\
\[head \ \[agr \ {\@1}\]\]\avmr\ \ \ {\HD}\[{\it phrase}\\
                                        head & \[{\it verb}\\
agr & {\@1}\]\\
                                        val & \[ spr & $-$\]\]\end{avm}
\item Head-Specifier Rule 2:

\begin{avm} \[{\it phrase}\\
val & \[ comps & itr\\
         spr & $+$\]\] \ $\rightarrow$\ \ D \ \ {\HD}\[{\it phrase}\\
                                        head & {\it noun}\\
                                        val & \[ spr & $-$\]\]
\end{avm}
\end{itemize}

\myslide{Reformulating the Grammar Rules III}
\begin{itemize}
\item Non-Branching NP Rule

\begin{avm} \[{\it phrase}\\
%head & {\it noun}\\
val & \[ comps & itr\\
         spr & $+$\]\] \ $\rightarrow$\ \ {\HD}\[{\it word}\\
                                     head & {\it noun}\\
                                     val & \[ spr & $+$\]\]\end{avm}

\item Head-Modifier Rule

\begin{avm}\avml \[{\it phrase}\\
		    val & \[ comps & itr\\
		             spr & $-$ \]\]\avmr \ 
$\rightarrow$\ \ {\HD}\avml\[{\it phrase}\\
			     val & \[ spr & $-$\]\]\avmr \
                 PP
\end{avm}
\newpage
\item Coordination Rule

\begin{avm}\avml \[ head \@{1} \] $\rightarrow$\ \ \ %
\[ head \@{1} \]+\ \[ {\it word}\\
 head & {\it conj} \]\  \[ head \@{1} \]\avmr
\end{avm}
\\ Only coordinate like things!
\end{itemize}

\myslide{Advantages of the New Formulation}
\begin{itemize}
\item Subject-verb agreement is stipulated only
once (where?)
\item Common properties of verbs with different
valences are expressed by common features
(for example?)
\item Parallelisms across phrase types are captured
(for example?)
\end{itemize}

\myslide{Disadvantages of the New Formulation}
\begin{itemize}
\item We still have three head complement rules
\item We still have two head specifier rules
\item We only deal with three verb valences
\\ (Which ones? What are some others?)
\item The non-branching rule doesn’t really do any
empirical work
\item Anything else?
\end{itemize}



\myslide{Which rule licenses each node?}

\scalebox{0.5}{\begin{avmtree}%\avmfont{\sc}
 
\br{\[{\it phrase}\\
  head &\[{\it verb}\\
  agr &\[{\it agr-cat}\\
  per & 3rd\\ 
  num & pl\]\]\\
  val & \[{\it val-cat}\\
  comps & itr\\
  spr & $+$ \]\] }{
\br{ \[{\it phrase}\\
                                head & \[{\it noun}\\
                                        agr & \[{\it agr-cat}\\
   					       per & 3rd\\
                                                num & pl\]\]\\
 	 		       val & \[{\it val-cat}\\
 			               comps & itr\\
 			               spr & $+$ \]\]
} {
\br {\[{\it word}\\
  head &\[{\it noun}\\
  agr & \[{\it agr-cat}\\
  per & 3rd\\
  num & pl\]\]\\
  val & \[{\it val-cat}\\
  comps & itr\\
  spr & $+$ \]\]
}{\br{they}{}}
}
\br{\[{\it phrase}\\
                                head &\[{\it verb}\\
                                        agr &\[{\it agr-cat}\\
 					      per & 3rd\\ 
 					      num & pl\]\]\\
 	 		       val & \[{\it val-cat}\\
                                        comps & itr\\
                                        spr & $-$ \]\]
}{
\br{\[{\it word}\\
                             head  &\[{\it verb}\\
                                     agr &\[{\it agr-cat}\\
 					   per & 3rd\\
      				           num & pl\]\]\\
 			    val  & \[{\it val-cat}\\
                                     comps & itr\\
                                     spr  & $-$ \]\]
}{\br{swim}{}}}
}
\end{avmtree}}
\vspace*{-2ex}

\myslide{In abbreviated form}

\begin{avmtree}%\avmfont{\sc}
  \br{S \newline \[ \ft{head} \@1 [ \ft{agr} \@4 
    \[ \ft{per} & 3 \\ \ft{num} & pl \] \]}{
    \br{NP  \[ \ft{head} \@2 \[ \ft{agr} \@4 \] \] }{
      \br{ N  \[ \ft{head} \@2 \]} { \br{\textnormal{they}}{} }}
    \br{VP  \[ \ft{head} \@1 \] } {
      \br{ V  \[ \ft{head} \@1 \]} { \br{\textnormal{swim}}{ } }}}
\end{avmtree}
\vspace*{-2ex}
\begin{flushleft}
  % HFP $\Rightarrow$ head the same for mother and daughter; \\
  S $\Rightarrow$ \val{phrase}, \ft{head} \val{verb}, \ft{val}
  \val{itr}, \ft{spr} val{+};  \\
  VP $\Rightarrow$ \val{phrase}, \ft{head}
  \val{verb}, \ft{val} \val{itr}, \ft{spr} val{-};   \ldots
\end{flushleft}
\myslide{A Question:}
Since the lexical entry for swim below has only [NUM pl] as
the value of \ft{agr}, how did the tree on the previous slide get
[PER 3rd] in the \ft{agr} of swim?

\begin{avm}%\avmfont{\sc}
\< \textnormal{swim}, \ 
\[ \asort{word} 
  head & \[ \asort{verb}  agr &   \[  num & pl \] \] \\
            val &  \[ \asort{val-cat}  comps & itr \\ spr & $-$ \] \] 
 \>
\end{avm}


\myslide{Overview}
\begin{itemize}
\item Review: problems with CFG
\item Modeling
\item Feature structures, unification (pizza)
\item Features for linguistic description
\item Reformulate grammar rules
\item Notion of head/headedness
\item Licensing of trees
\item Next time: Valence and agreement: complex feature values
\end{itemize}




\myslide{3.1 Applying the grammar}
\begin{itemize}
\item[A.] Formulate \index{agreement|(} a \index{lexical entry} lexical entry for the word {\it defendants}.
\item[B.] Draw a tree for the sentence {\it The defendants walk}.  Show
the values for all of the features on every node and use tags to indicate
the effects of any identities that the %rules require.
grammar requires.
\item[C.] Explain how your lexical entry for {\it defendants} interacts
with the Chapter 3 grammar to rule out {\it *The defendants walks}.
Your explanation should make reference to grammar rules, lexical
entries and the \index{Head Feature Principle (HFP)}HFP. \index{agreement|)}
\end{itemize}

\myslide{Determiner-Noun Agreement}

The Chapter 3 grammar declares AGR to be a feature
appropriate for the types {\it noun}, {\it verb}, and {\it det}, but
so far we haven't discussed agreement involving determiners.  Unlike
the determiner {\it the}, most other English determiners do show
agreement with the nouns they combine with:

\begin{itemize}\addtolength{\itemsep}{-1ex}
\item[(i)] a bird/*a birds
\item[(ii)] this bird/*this birds
\item[(iii)] that bird/*that birds
\item[(iv)] these birds/*these bird
\item[(v)] those birds/*those bird
\item[(vi)] many birds/*many bird
\end{itemize}
\newpage

\begin{itemize}
\item[A.] Formulate lexical entries for {\it this} and {\it these}.
\item[B.] Modify Head-Specifier Rule 2 so that it enforces agreement
between the noun and the determiner just like Head-Specifier Rule 1
enforces agreement between the NP and the VP.
\item[C.] Draw a tree for the NP {\it these birds}. Show the
value for all features of every node and use tags to indicate the
effects of any identities that the %rules 
grammar 
(including your modified
HSR2 and the Head Feature Principle) requires.
\end{itemize}


\myslide{Types for English Agreement}

\begin{forest}
  /tikz/every node/.append style={font=\it},
  [agr-cat [sg,name=sg [3sg]] [, phantom] [, phantom]  [, phantom] [non-3sg
  [1sg, name=1sg] [non-1sg [2sg, name=2sg] [pl]]]]
  \draw (sg.south) -- (1sg.north);
  \draw (sg.south) -- (2sg.north);
\end{forest}

\myslide{Acknowledgments and References}
\begin{itemize}
\item Course design and slides borrow heavily from Emily Bender's course:
\textit{Linguistics 566: Introduction to Syntax for Computational Linguistics}
\\ \url{http://courses.washington.edu/ling566}
\item Problems from Sag, Wasow and Bender (2003)

\end{itemize}




\end{document}


%%% Local Variables: 
%%% coding: utf-8
%%% mode: latex
%%% TeX-PDF-mode: t
%%% TeX-engine: xetex
%%% End: 
