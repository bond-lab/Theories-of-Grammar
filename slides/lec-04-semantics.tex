\documentclass[a4paper,landscape,headrule,footrule,dvips]{foils}


%%
%%% macros for Theories of Grammar
%%%
\usepackage{polyglossia}
\setdefaultlanguage{english}
%\setmainfont{TeX Gyre Pagella}


\newcommand{\logo}{~}
\newcommand{\header}[3]{%
\title{\vspace*{-2ex} \large HG4041 Theories of Grammar
\\[2ex] \Large  \emp{#2} \\ \emp{#3}}
\author{\blu{Francis Bond}   \\ 
\normalsize  \textbf{Division of Linguistics and Multilingual Studies}\\
\normalsize  \url{http://www3.ntu.edu.sg/home/fcbond/}\\
\normalsize  \texttt{bond@ieee.org}}
\MyLogo{HG4041 (2020)}
\renewcommand{\logo}{#2}
\hypersetup{
   pdfinfo={
     Author={Francis Bond},
     Title={#1: #2},
     Subject={HG4041: Theories of Grammar},
     Keywords={Syntax, Semantics, HPSG, Unification, Constructions},
     License={CC BY 4.0}
   }
 }

\date{#1
  \\ Location: LHN-TR+36}
}
\usepackage[hidelinks]{hyperref}




\usepackage{xcolor}
\usepackage{graphicx}
\newcommand{\blu}[1]{\textcolor{blue}{#1}}
\newcommand{\grn}[1]{\textcolor{green}{#1}}
\newcommand{\hide}[1]{\textcolor{white}{#1}}
\newcommand{\emp}[1]{\textcolor{red}{#1}}
\newcommand{\txx}[1]{\textbf{\textcolor{blue}{#1}}}
\newcommand{\lex}[1]{\textbf{\mtcitestyle{#1}}}

\usepackage{pifont}
\renewcommand{\labelitemi}{\textcolor{violet}{\ding{227}}}
\renewcommand{\labelitemii}{\textcolor{purple}{\ding{226}}}

\newcommand{\subhead}[1]{\noindent\textbf{#1}\\[5mm]}

\newcommand{\Bad}{\emp{\raisebox{0.15ex}{\ensuremath{\mathbf{\otimes}}}}}
\newcommand{\bad}[1]{*\eng{#1}}

\newcommand{\com}[1]{\hfill (\emp{#1})}%

\usepackage{relsize,xspace}
\newcommand{\into}{\ensuremath{\rightarrow}\xspace}
\newcommand{\tot}{\ensuremath{\leftrightarrow}\xspace}
\usepackage{url}
\newcommand{\lurl}[1]{\MyLogo{\url{#1}}}

\usepackage{mygb4e}
\newcommand{\lx}[1]{\textbf{\mtciteform{#1}}}
\newcommand{\ix}{\ex\slshape}
\let\eachwordone=\slshape



\newcommand{\ent}{\ensuremath{\Rightarrow}\xspace}
\newcommand{\ngv}{\ensuremath{\not\Rightarrow}\xspace}
%\usepackage{times}
%\usepackage{nttfoilhead}
\newcommand{\myslide}[1]{\foilhead[-25mm]{\raisebox{12mm}[0mm]{\emp{#1}}}\MyLogo{\logo}}
\newcommand{\myslider}[1]{\rotatefoilhead[-25mm]{\raisebox{12mm}[0mm]{\emp{#1}}}}
%\newcommand{\myslider}[1]{\rotatefoilhead{\raisebox{-8mm}{\emp{#1}}}}

\newcommand{\section}[1]{\myslide{}{\begin{center}\Huge \emp{#1}\end{center}}}



\usepackage[lyons,j,e,k]{mtg2e}
%\renewcommand{\mtcitestyle}[1]{\textcolor{-red!75!green!50}{\textsl{#1}}}
\renewcommand{\mtcitestyle}[1]{\textcolor{teal}{\textsl{#1}}}
\newcommand{\iz}[1]{\texttt{\textup{#1}}}
\newcommand{\gm}{\textsc}
\usepackage[normalem]{ulem}
\newcommand{\ul}{\uline}
\newcommand{\ull}{\uuline}
\newcommand{\wl}{\uwave}
\newcommand{\vs}{\ensuremath{\Leftrightarrow}~}
%%%
%%% Bibliography
%%%
\usepackage{natbib}
%\usepackage{url}
\usepackage{bibentry}


%%% From Tim
\newcommand{\WMngram}[1][]{$n$-gram#1\xspace}
\newcommand{\infers}{$\rightarrow$\xspace}

\usepackage[utf8]{inputenc}

\usepackage{rtrees,qtree}
\renewcommand{\lf}[1]{\br{#1}{}}
\usepackage{avm}
%\avmoptions{topleft,center}
\newcommand{\ft}[1]{\textsc{#1}}
\renewcommand{\val}[1]{\textit{#1}}
\newcommand{\typ}[1]{\textit{#1}}
\newcommand{\prd}[1]{\textbf{#1}}
\avmfont{\sc}
\avmvalfont{\it}
\avmsortfont{\smaller[2] \it}
\usepackage{multicol}
\newcommand{\blank}{\rule{3em}{1pt}\xspace}

% \usepackage{pst-node}
\newcommand{\OV}[1]{\ovalnode[linestyle=dotted,linecolor=red]{A}{#1}}
\newcommand{\OVB}[1]{\ovalnode[linestyle=dotted,linecolor=blue]{A}{#1}}

%%% From CSLI book
\newcommand{\mc}{\multicolumn}
\newcommand{\HD}{\textbf{H}\xspace}
\newcommand{\el}{\< \>}
\makeatother
\long\def\smalltree#1{\leavevmode{\def\\{\cr\noalign{\vskip12pt}}%
\def\mc##1##2{\multispan{##1}{\hfil##2\hfil}}%
\tabskip=1em%
\hbox{\vtop{\halign{&\hfil##\hfil\cr
#1\crcr}}}}}
\makeatletter

\newcommand{\A}{\noindent\textbf{A}: }
\newcommand{\Q}{\noindent\textbf{Q}: }
%\newcommand{\C}{\noindent\textbf{C}: }



\avmfont{\sc}
\begin{document}
\avmfont{\it}

\header{Lecture 4}{Semantics}{}
\maketitle


\myslide{Overview}
\begin{itemize}
\item Last week: HC, HS, VP, SHAC
\item Some notes on the linguist's stance
\item Which aspects of semantics we’ll tackle
\item Our formalization; Semantics Principles
\item Building semantics of phrases
\item Modification, coordination
\item Structural ambiguity
\end{itemize}

\myslide{Overview}
%\MyLogo{Chapter 4}

\begin{itemize}
\item Complex Feature Structures allow us to
  \begin{itemize}
  \item Write more general rules
  \item Constrain them lexically
  \end{itemize}
\end{itemize}

\myslide{Head-Complement Rule}

\begin{avm}\avmfont{\sc} 
\[ \asort{phrase}  val & \[ comps & \<  \> \] \]
              \ $\rightarrow$\ \HD \ 
             \[ \asort{word}
             val & \[ comps & \< \@1, \ldots, \@{n} \> \] \]\ \ \@1, \ldots, \@{n}
\end{avm}
\begin{itemize}
\item The possible complements are specified lexically \\
\begin{small}\begin{tabular}{ll}
    \begin{avm}\avmfont{\sc}
      \< \textnormal{devour}, \ \[ \asort{word}
      head & {\it verb} \\
      % \[ \asort{agr-cat} agr & \[ per & 3 \\ num & sg \] \]
      val & \[ COMPS & \< NP \> \]
      \]\>
    \end{avm} &
    \begin{avm}\avmfont{\sc}
      \< \textnormal{put}, \ \[ \asort{word}
      head & {\it verb} \\
      % \[ \asort{agr-cat} agr & \[ per & 3 \\ num & sg \] \]
      val & \[ COMPS & \< NP PP \> \]
      \]\>
    \end{avm} 
  \end{tabular}
\end{small}
\end{itemize}

\myslide{Head-Specifier Rule}

\begin{avm}\[{\it phrase}\\
             VAL & \[ COMPS & \el\\
                      SPR & \el\]\]\ \ \
$\rightarrow$\ \ \ \ {\@2}\ \ \ \ \HD\[ %{\it phrase}\\
			                VAL & \[ COMPS & \el\\
                                                 SPR & \q< {\@2} \q> \]\] 
\end{avm}
\begin{itemize}
\item Combines the rules expanding S and NP (and other, \ldots).
\item Again, restrictions on specifiers come from the lexicon.
\end{itemize}
\begin{small}\begin{tabular}{ll}
    \begin{avm}\avmfont{\sc}
      \< \textnormal{dog}, \ \[ \asort{word}
      head & {\it noun} \\
      % \[ \asort{agr-cat} agr & \[ per & 3 \\ num & sg \] \]
      val & \[ spr & \< D \> \\ comps & \el \]
      \]\>
    \end{avm} &
    \begin{avm}\avmfont{\sc}
      \< \textnormal{eat}, \ \[ \asort{word}
      head & {\it verb} \\
      % \[ \asort{agr-cat} agr & \[ per & 3 \\ num & sg \] \]
      val & \[ spr & \< NP \> \\  comps & \< (NP) \> \]
      \]\>
    \end{avm} 
  \end{tabular}
\end{small}

\myslide{Two Principles}
\begin{itemize}
\item \txx{The Valence Principle}
  \begin{quote}
    Unless the rule says otherwise, the mother’s
    values for the VAL features (SPR and
    COMPS) are identical to those of the head
    daughter.
  \end{quote}
\item \txx{The Specifier-Head Agreement Constraint (SHAC)}
\\ Verbs and nouns must be specified as:
\begin{center}
  \begin{avm}\avmfont{\sc}
    \[ head & \[ agr \@1 \] \\
    val & \[ spr \< agr \@1 \> \] \]
  \end{avm}
\end{center}
Actually inherited from a lexical super-type
\end{itemize}

\section{Semantics}

\myslide{Overview}
\begin{itemize}
\item Some notes on the linguist's stance
\item Which aspects of semantics we’ll tackle
\item Our formalization; Semantics Principles
\item Building semantics of phrases
\item Modification, coordination
\item Structural ambiguity
\end{itemize}


\myslide{The Linguist's Stance: Building a precise model}

\begin{itemize}
\item Some of our statements are statements about how the model
works:
\begin{quote}
  “[\val{prep}] and [\ft{agr} \val{3sing}] can’t be combined because AGR is not a
  feature of the type prep.”
\end{quote}
\item Some of our statements are statements about how (we think)
English or language in general works.
\begin{quote}
  “The determiners \eng{a} and \eng{many} only occur with count nouns, the
  determiner \eng{much} only occurs with mass nouns, and the determiner \eng{the}
  occurs with either.”
\end{quote}
\item Some are statements about how we code a particular
linguistic fact within the model.
\begin{quote}
  “All count nouns are [\ft{spr}  $<$ [\ft{count} +]$>$].”
\end{quote}
\end{itemize}

\myslide{The Linguist's Stance:}

A Vista on the Set of Possible English Sentences
\begin{itemize}
\item \ldots  as a background against which linguistic
  elements (words, phrases) have a distribution
\item \ldots  as an arena in which linguistic elements
  “behave” in certain ways
\end{itemize}


\myslide{Semantics: Where's the Beef?}

So far, our grammar has no semantic representations. We
have, however, been relying on semantic intuitions in our
argumentation, and discussing semantic contrasts where
they line up (or don't) with syntactic ones.

\begin{itemize}

\item structural ambiguity
\item S/NP parallelism
\item count/mass distinction
\item complements vs. modifiers
\end{itemize}

\myslide{Our Slice of a World of Meanings}

Aspects of meaning we \emp{won't} account for (in this course)
\begin{itemize}
\item Pragmatics
\item Fine-grained lexical semantics
  \\ The meaning of \eng{life} is
  \begin{itemize}\addtolength{\itemsep}{2ex}
  \item \iz{life}  (or \textsf{life}$'$)
\item
  \begin{avm}\avmfont{\sc}%\avmvalfont{\it}
    \[ reln & life \\ inst & i \]
  \end{avm}
\item Not like wordnet: 
\iz{life$_1$}  $\subset$ \iz{being$_1$} $\subset$ \iz{state$_1$} \ldots

 % -- (a characteristic state or mode of living; "social life"; "city life"; "real life")
 %       => being, beingness, existence -- (the state or fact of existing; "a point of view gradually coming into being"; "laws in existence for centuries")
 %           => state -- (the way something is with respect to its main attributes; "the current state of knowledge"; "his state of health"; "in a weak financial state")
 %               => attribute -- (an abstraction belonging to or characteristic of an entity)
 %                   => abstraction, abstract entity -- (a general concept formed by extracting common features from specific examples)
 %                       => entity -- (that which is perceived or known or inferred to have its own distinct existence (living or nonliving))
 \end{itemize}

 \end{itemize}


% RELN life
% INST i

\myslide{Our Slice of a World of Meanings}
\begin{avm}\avmfont{\sc}
\[mode &  prop \\
 index &  $s$\\
 restr & \< \[ RELN & {\bf save}\\
               SIT & $s$\\
              SAVER &  $i$\\
              SAVED &  $j$ \] \ , 
              \[RELN & {\bf name}\\
              NAME &  Chris\\
              NAMED & \ $i$\]\ , 
              \[RELN & {\bf name}\\
              NAME &  Pat\\
              NAMED & \ $j$\] \> \]
\end{avm} 
\begin{quote}
  “\ldots the linguistic meaning of \eng{Chris saved Pat} is a proposition
  that will be true just in case there is an actual situation that
  involves the saving of someone named Pat by someone named Chris.”
  \begin{flushright}
    (Sag \textit{et al}, 2003, p. 140)
  \end{flushright}
\end{quote}

\myslide{Our Slice of a World of Meanings}

What we are accounting for is the \txx{compositionality} of
sentence meaning.

\begin{itemize}
\item How the pieces fit together
  \begin{quote}
    \txx{Semantic arguments} and \txx{indices}    
  \end{quote}
\item How the meanings of the parts add up to the meaning
of the whole.
\begin{quote}
  Appending \ft{restr} lists up the tree  
\end{quote}
\end{itemize}

The value of \txx{\ft{restr}iction} is the set of conditions that must
hold (in some possible world) for the expression to be applicable.

\myslide{Semantics in Constraint-Based Grammar}
\begin{itemize}
\item Constraints as (generalized) truth conditions
  \begin{itemize}
\item \txx{proposition}: what must be the case for a proposition to be true
\item \txx{directive}: what must happen for a directive to be fulfilled
\item \txx{question}: the kind of situation the asker is asking about
\item \txx{reference}: the kind of entity the speaker is referring to
\end{itemize}
\item \txx{Syntax/semantics interface}: 
  \begin{quote}
    Constraints on how syntactic arguments are related to semantic
    ones, and on how semantic information is compiled from different
    parts of the sentence.
  \end{quote}
\end{itemize}

\myslide{Feature Geometry}
\begin{avm}\avmfont{\sc}
\[\asort{expression}
syn & \[ \asort{syn-cat}
      head & \[ \asort{pos} ... \] \\
      val & \[ spr & \q< ... \q> \\
               comps & \q< ... \q> \]\]\\
      sem & \[ \asort{sem-cat}
      mode &  \q\{ prop, ques, dir, ref, none \q\} \\
      index &  \q\{ i, j, k, \ldots, s$_1$, s$_2$, \ldots \q\} \\
      restr & \q< ... \q> \]\]
 \end{avm}



\myslide{How the Pieces Fit Together}
\begin{avm}\avmfont{\sc}
  \< \eng{Kim}, \[syn & \[head & \[{\it noun}\\
  agr & {\it 3sing}\]\\
  val & \[spr & {\el}\\
  comps & {\el}\]\]\\
  sem & \[mode & ref\\
  index & \ \ $i$\\
  restr & \< \[reln & {\bf name}\\
  % sit & \ \ $s$\\
  name & Kim\\
  named & \ \ $i$\] \> \]\] \>  %
\end{avm}
\myslide{How the Pieces Fit Together}
\begin{avm}\avmfont{\sc}
\< \eng{sleep} ,\ \[syn & \[head & {\it verb}\\
                                     val & \[spr & \q< NP$_i$ \q>\\
                                     comps & \q<  \q>\]\]\\
                            sem & \[mode & prop\\
                                    index & $s$\\
                                    restr & \< \[reln & {\bf sleep}\\
                                                 sit & \ \ $s$\\
                                                 sleeper & \ \ $i$  \] \>\]\] \>
                                        \end{avm}
\myslide{The Pieces Together}

\begin{avmtree}\avmfont{\sc}
  \br{S}{
    \br{\@{1} NP  \[ sem & \[ index & $i$ \] \] }{\lf{\eng{Kim}}}
    \br{VP  \[ syn & \[ val & \[ spr & \< \@{1} \> \] \] \\
      sem & \[ mode & prop\\
      index & $s$\\
      restr & \< \[reln & {\bf sleep}\\
      sit & \ \ $s$\\
      sleeper & \ \ $i$  \] \> \] \]    }{\lf{\eng{slept}}}}
\end{avmtree}


\myslide{Another View of the Same Tree}
\scalebox{0.8}{\begin{avmtree}\avmfont{\sc}
  \br{S  \[ sem & \[ mode & prop \\ index & $s$\\
      restr & \< \[reln & {\bf name}\\
      name & \ \ Kim  \\ named & \ \ $i$ \] ,
      \[reln & {\bf sleep}\\
      sit & \ \ $s$\\
      sleeper & \ \ $i$  \]  \>  \] \] }{
    \br{\@{1} NP  \[ sem & \[ mode & ref \\ index & $i$\\
      restr & \< \[reln & {\bf name}\\
      name & \ \ Kim \\ named & \ \ $i$ \] \>  \] \] }{\lf{\eng{Kim}}}
    \br{\HD VP  \[ syn & \[ val & \[ spr & \< \@{1} \> \] \] \\
      sem & \[  mode & prop \\ index & $s$ \\ 
      restr & \< \[reln & {\bf sleep}\\
      sit & \ \ $s$\\
      sleeper & \ \ $i$  \] \> \] \]    }{\lf{\eng{slept}}}}
\end{avmtree}}


\myslide{How to Share Semantic Information}

We need the Semantics Principles
\begin{itemize}
\item \txx{The Semantic Inheritance Principle}
  \begin{quote}
    In any headed phrase, the mother's \ft{mode} and
    \ft{index} are identical to those of the head daughter.    
  \end{quote}
\item \txx{The Semantic Compositionality Principle}
  \begin{quote}
    In any well-formed phrase structure, the mother's
    \ft{restr} value is the sum of the \ft{restr} values of
    the daughter.
  \end{quote}
\end{itemize}

List summation: $\oplus$ (technically concatenation) 
$\langle A \rangle \oplus \langle B \rangle \ne 
\langle B \rangle \oplus \langle A \rangle$

$\langle a_1, a_2, \ldots, a_n \rangle \oplus \langle b_1, b_2, \ldots, b_m \rangle = 
\langle a_1, a_2, \ldots, a_n,  b_1, b_2, \ldots, b_m \rangle $

$ \langle b_1, b_2, \ldots, b_m \rangle \oplus \langle a_1, a_2, \ldots, a_n \rangle = 
\langle b_1, b_2, \ldots, b_m,   a_1, a_2, \ldots, a_n \rangle $

\myslide{What Identifies Indices?}

\scalebox{1.0}{\begin{avmtree}\avmfont{\sc}
  \br{S}{
    \br{\@{1} NP$_i$}{
      \br{D}{\lf{\eng{The}}}
      \br{Nom$_i$}{\lf{\eng{cat}}}
      }
      \br{VP \[ spr & \< \@{1} \> \]}{
        \br{VP   \[ spr & \< \@{1} \>  \\
          restr & \< \[reln & {\bf sleep}\\
          sit & \ \ $s$\\
          sleeper & \ \ $i$  \] \>  \]    }{\lf{\eng{slept}}}
        \br{PP}{\tlf{\eng{on the mat}}}}}
\end{avmtree}}




\myslide{Summary: Words \ldots }
\begin{itemize}
\item contribute predications
\item ‘expose’ one index in those predications, for use by words or phrases
\item relate syntactic arguments to semantic arguments
 \\ \scalebox{0.8}{\begin{avm}\avmfont{\sc}
\< \eng{sleep} ,\ \[syn & \[head & {\it verb}\\
                                     val & \[spr & \q< NP$_i$ \q>\\
                                     comps & \q<  \q>\]\]\\
                            sem & \[mode & prop\\
                                    index & $s$\\
                                    restr & \< \[reln & {\bf sleep}\\
                                                 sit & \ \ $s$\\
                                                 sleeper & \ \ $i$  \] \>\]\] \>
                                        \end{avm}}
\end{itemize}

\myslide{Summary: Grammar Rules \ldots }
\begin{itemize}
\item Identify feature structures (including the \ft{index} value) across daughters
  \begin{itemize}\addtolength{\itemsep}{1ex}
  \item Head Specifier Rule \\
\scalebox{0.9}{\begin{avm}\avmfont{\sc}
  \[ \asort{phrase} 
  val & \[ spr & \el \] \] \ \ \
  $\rightarrow$\ \ \ \ {\@1}\ \ \ \ \HD\[ %{\it phrase}\\
  val & \[ spr & \q< {\@1} \q>  \\ comps & \el \]\] 
\end{avm}}
\item Head Complement Rule \\
\scalebox{0.9}{\begin{avm}\avmfont{\sc}
    \[ \asort{phrase}
    val & \[ comps & \<  \> \] \]
    \ $\rightarrow$\ \HD \ 
    \[ \asort{word}
    val & \[ comps & \< \@1, \ldots, \@{n} \> \] \]\ \ \@1, \ldots, \@{n}
\end{avm}}
\item Head Modifier Rule \\
\scalebox{0.9}{\begin{avm}\avmfont{\sc}
    \[ \asort{phrase} \]
    \ $\rightarrow$\ \HD \ \@1  \[ val & \[ comps & \el \]  \] \ \ 
    \[ val & \[ comps & \el \\ mod & \< \@1 \> \] \]
\end{avm}}
\end{itemize}
\end{itemize}

\myslide{Summary: Grammar Rules \ldots }

\begin{itemize}
\item Identify feature structures (including the \ft{index} value) across daughters
\item License trees which are subject to the semantic principles
  \begin{itemize}
  \item SIP: ‘passes up’ \ft{mode} and \ft{index} from head daughter
  \item SCP: ‘gathers up’ predications (\ft{restr} list) from all daughters
  \end{itemize}
\item The semantics is strictly compositional --- all of the meaning
  comes from the words, rules and principles.
  \begin{itemize}
  \item We then enrich this with pragmatic inference --- but we need a
    base to infer from
  \end{itemize}
\end{itemize}

\myslide{Other Aspects of Semantics}
\begin{itemize}
\item Tense, Quantification (only touched on here)
\item Modification
\item Coordination
\item Structural Ambiguity
\end{itemize}

\myslide{Evolution of a Phrase Structure Rule}
\begin{itemize}
\item [C2] NOM \into NOM PP; VP \into VP PP
\item [C3] 
\scalebox{0.8}{\begin{avm}\avmfont{\sc}
    \[ \asort{phrase} val & \[ spr & - \\ comps & itr \] \] 
    \ $\rightarrow$\ \HD \  \[ \asort{phrase}  val & \[ spr & - \]   \] \ \ 
    PP 
\end{avm}}
\item [C4] 
\scalebox{0.8}{\begin{avm}\avmfont{\sc}
    \[ \asort{phrase}  \]
    \ $\rightarrow$\ \HD \ \[  val & \[ comps & \el \]  \] \ \ 
    PP
\end{avm}}
\item[C5]
\scalebox{0.8}{\begin{avm}\avmfont{\sc}
    \[ \asort{phrase} \]
    \ $\rightarrow$
    \HD \ \@1  \[ syn &  \[ val & \[ comps & \el \]  \] \] \ \ 
    \[ syn & \[ val & \[ comps & \el \\ mod & \< \@1 \> \] \] \]
  \end{avm}}

\item[=] 
\scalebox{0.8}{\begin{avm}\avmfont{\sc}
     \[ \asort{phrase} \]
     \ $\rightarrow$\ \HD \ \@1   \[ comps & \el  \] \ \ 
      \[ comps & \el \\ mod & \< \@1 \> \] 
 \end{avm}} 


\myslide{Evolution of Another Phrase Structure Rule}

\begin{itemize}
\item [C2] X \into X+ CONJ X; 
\item [C3] 
\scalebox{0.8}{\begin{avm}\avmfont{\sc}
    \@1 \ \into \ \@{1}+ \ \ 
    \[ \asort{word} head & conj \] \ \ 
    \@1
\end{avm}}
\item [C4] 
\scalebox{0.8}{\begin{avm}\avmfont{\sc}
    \[ val & \@1 \] \ \into \ \[ val & \@1 \]+ \ \ 
    \[ \asort{word} head & conj \] \ \ 
    \[ val & \@1 \]
\end{avm}}
\item[C5]
\scalebox{0.8}{\begin{avm}\avmfont{\sc}
    \[ syn & \[ val & \@0 \] \\ 
       sem & \[ ind & $s_0$ \]  \]
    \ \into \ 
  \end{avm}} \\
\scalebox{0.65}{\begin{avm}\avmfont{\sc}
  \[ syn & \[ val & \@0 \] \\ 
       sem & \[ ind & $s_1$ \]  \] \ \ldots \  
   \[ syn & \[ val & \@0 \] \\ 
     sem & \[ ind & $s_{n-1}$ \] \]
    \[ syn & \[ head & conj \] \\
    sem & \[ ind  & $s_0$ \\ 
    restr & \q< \[ args & \q< $s_1, \ldots, s_{n-1}, s_n$ \q> \] \q> \]  \]\ \ 
  \[ syn & \[ val & \@0 \] \\ 
     sem & \[ ind & $s_n$ \] \]
  \end{avm}}
\item[=]
\scalebox{0.75}{\begin{avm}\avmfont{\sc}
    \[  val & \@0 \\ ind & $s_0$ \]
    \ \into \ 
  \[  val & \@0 \\ ind & $s_1$  \] \ \ldots \  
   \[  val & \@0  \\ ind & $s_{n-1}$ \]
    \[ head & conj \\
     ind  & $s_0$ \\ 
    restr & \q< \[ args & \q< $s_1, \ldots, s_{n-1}, s_n$ \q> \] \q>  \]\ \ 
  \[  val & \@0 \\ ind & $s_n$ \]
  \end{avm}}
\end{itemize}
\end{itemize}
% \item[=] 
% \scalebox{0.8}{\begin{avm}\avmfont{\sc}
%      \[ \asort{phrase} \]
%      \ $\rightarrow$\ \HD \ \@1   \[ comps & \el  \] \ \ 
%       \[ comps & \el \\ mod & \< \@1 \> \] 
%  \end{avm}} 

\myslide{Combining Constraints and Coordination}

\begin{itemize}
\item  Coordination Rule \\
\scalebox{0.75}{\begin{avm}\avmfont{\sc}
    \[  val & \@0 \\ ind & $s_0$ \]
    \ \into \ 
  \[  val & \@0 \\ ind & $s_1$  \] \ \ldots \  
   \[  val & \@0  \\ ind & $s_{n-1}$ \]
    \[ head & conj \\
     ind  & $s_0$ \\ 
    restr & \q< \[ args & \q< $s_1, \ldots, s_{n-1}, s_n$ \q> \] \q>  \]\ \ 
  \[  val & \@0 \\ ind & $s_n$ \]
  \end{avm}}
\item Lexical Entry for \lex{and} \\
  \scalebox{0.9}{\begin{avm}\avmfont{\sc}
      \< \eng{and} ,\ \[ syn & \[head & {\it conj} \] \\
                          sem & \[mode & none\\
                                  ind & $s$\\
                                  restr & \< \[reln & {\bf and}\\
                                               sit & \ \ $s$\\ \] \>\]\] \>
                                        \end{avm}}
\end{itemize}

\myslide{Combining Constraints and Coordination}
\noindent\scalebox{0.9}{\begin{avmtree}\avmfont{\sc}
\br{S \[ ind & $s_0$ \\
restr & 
\< \[reln & {\bf name} \\ name & Joe \\ named & $j$ \],
 \[reln & {\bf joke} \\ sit & $s_1 $ \\ joker & $j$ \], 
\@0 \ \[reln & {\bf and}\\
sit & \ \ $s_0$\\ 
args & \q< $s_1, s_2$ \q> \], 
\[reln & {\bf name} \\ name & Kim \\ named & $k$\], 
\[reln & {\bf smile} \\ sit & $s_2 $ \\ smiler & $k$\] \> \]}{
  \br{S \[ ind & $s_1$ \]}{ \tlf{\eng{Joe jokes}}}
  \br{\[ head & {\it conj} \\
         ind & $s_0$\\
        restr & \< \@0 \[reln & {\bf and} \] \> \] }{\lf{\eng{and}}}
  \br{S \[ ind & $s_2$ \]}{ \tlf{\eng{Kim smiles}}}
}
\end{avmtree}}

\myslide{Ambiguity}
\noindent\scalebox{0.7}{\begin{avmtree}\avmfont{\sc}
\br{S \[ ind & $s_0$ \\
restr & 
\< \[reln & {\bf name} \\ name & Joe \\ named & $j$ \],
 \[reln & {\bf joke} \\ sit & $s_1 $ \\ joker & $j$ \], 
\@0 \ \[reln & {\bf and}\\
sit & \ \ $s_0$\\ 
args & \q< $s_1, s_2$ \q> \], 
\[reln & {\bf name} \\ name & Kim \\ named & $k$\], 
\[reln & {\bf smile} \\ sit & $s_2 $ \\ smiler & $k$\],
\[reln & {\bf often} \\ sit & $s_3 $ \\ arg & $s_0$ \]  \> \]}{
\br{\@1 S \[ ind & $s_0$ \]}{
  \br{S \[ ind & $s_1$ \]}{ \tlf{\eng{Joe jokes}}}
  \br{\[ head & {\it conj} \\
         ind & $s_0$\\
        restr & \< \@0 \[reln & {\bf and} \] \> \] }{\lf{\eng{and}}}
  \br{S \[ ind & $s_2$ \]}{ \tlf{\eng{Kim smiles}}}
}
\br{ADV \[ mod \q< \@1 \q> \]}{\tlf{\eng{often}}}}
\end{avmtree}}

\myslide{Ambiguity}
\noindent\scalebox{0.7}{\begin{avmtree}\avmfont{\sc}
\br{S \[ ind & $s_0$ \\
restr & 
\< \[reln & {\bf name} \\ name & Joe \\ named & $j$ \],
 \[reln & {\bf joke} \\ sit & $s_1 $ \\ joker & $j$ \], 
\@0 \ \[reln & {\bf and}\\
sit & \ \ $s_0$\\ 
args & \q< $s_1, s_2$ \q> \], 
\[reln & {\bf name} \\ name & Kim \\ named & $k$\], 
\[reln & {\bf smile} \\ sit & $s_2 $ \\ smiler & $k$\],
\[reln & {\bf often} \\ sit & $s_3 $ \\ arg & $s_2$ \]  \> \]}{
  \br{S \[ ind & $s_1$ \]}{ \tlf{\eng{Joe jokes}}}
  \br{\[ head & {\it conj} \\
         ind & $s_0$\\
        restr & \< \@0 \[reln & {\bf and} \] \> \] }{\lf{\eng{and}}}
\br{S \[ ind & $s_2$ \]}{
  \br{\@1 S \[ ind & $s_2$ \]}{ \tlf{\eng{Kim smiles}}}
  \br{ADV \[ mod \q< \@1 \q> \]}{\tlf{\eng{often}}}}}
\end{avmtree}}



\myslide{Question About Structural Ambiguity}
Why isn’t this a possible semantic representation for
the string \eng{Joe jokes and Kim smiles often}?

\noindent\scalebox{0.6}{\begin{avm}\avmfont{\sc}
\[ ind & $s_0$ \\
  mode & prop \\
  restr & 
  \< \[reln & {\bf name} \\ name & Joe \\ named & $j$ \],
  \[reln & {\bf joke} \\ sit & $s_1 $ \\ joker & $j$ \], 
  \@0 \ \[reln & {\bf and}\\
  sit & \ \ $s_0$\\ 
  args & \q< $s_1, s_2$ \q> \], 
  \[reln & {\bf name} \\ name & Kim \\ named & $k$\], 
  \[reln & {\bf smile} \\ sit & $s_2 $ \\ smiler & $k$\],
  \[reln & {\bf often} \\ sit & $s_3 $ \\ arg & $s_1$ \]  \> \]
\end{avm}}

\myslide{Some Standard Extensions}

\begin{itemize}
\item Quantification
  \begin{itemize}
  \item typically expressed as restrictions on scope
  \item Minimal Recursion Semantics goes further
  \end{itemize}
\item Pragmatics
  \begin{itemize}
  \item typically expressed as another feature: \ft{context}
  \item contains things like \val{speaker, hearer, audience}
  \item used for pronominal reference, politeness
\end{itemize}
\end{itemize}


\myslide{Problem: Two Kinds of Modifiers in English}
In English, modifiers of nouns can appear either
before or after the noun, although any given modifier is usually
restricted to one position or the other.

\begin{itemize}
\item[(i)] \textit{The red dog on the roof}
\item[(ii)]  \bad{\it The on the roof dog}
\item[(iii)]  \bad{\it The dog red}
\end{itemize}
%
Our current Head-Modifier Rule only licenses post-head modifiers
(like {\it on the roof} in (i)).

\newpage 
\begin{itemize}
\item[A.] Write a second Head-Modifier Rule that licenses
\index{pre head modifier@pre-head modifier}pre-head modifiers
(e.g., {\it red} in (i)).
\item[B.] Modify the Head-Modifier 1 and Head-Modifier 2 Rules so that
they are sensitive to which kind of modifier is present and don't
generate (ii) or (iii).  [\textsl{Hint: Use a feature
{\rm [}POST-HEAD \{+,$-$\}{\rm ]} to distinguish {\it red} and {\it on the roof}.}]
\item[C.] Is POST-HEAD a \index{HEAD}HEAD feature? Why or why not? 
\item[D.] Give lexical entries for {\it red} and {\it on} that show
the value of POST-HEAD. (You may omit the SEM features in these entries.)
\item[E.] Is (i) ambiguous according to your grammar (i.e.\ 
the Chapter 5 grammar modified to include the two Head-Modifier Rules,
instead of just one)? Explain your answer.  
\end{itemize}

This problem assumed that we don't want to make the two Head-Modifier
Rules sensitive to the part of speech of the modifier.  One reason for
this is that modifiers of the same part of speech can occur before and
after the head, even though individual modifiers might be restricted
to one position or the other.

\begin{itemize}
\item[F.] Provide three examples of English NPs with
\index{adjective}adjectives or APs after the noun. 
\item[G.] Provide three examples of \index{adverb (ADV)}adverbs that can come before 
the verbs they modify.
\item[H.] Provide three examples of adverbs that can come after
the verbs they modify.
\index{ambiguity}
\end{itemize}


\myslide{Problem: Semantics of Number Names}
In  Chapter 3, we considered the
syntax of English number names, and in particular how
to find the head of a number name expression. Based on the
results of that problem, the lexical entry for
{\it hundred} in a number name like {\it two hundred five}
should include the constraints in (i): (Here we are assuming
a new subtype of {\it pos}, {\it number}, which is appropriate
for number name words.)
\index{pos@{\it pos}}
\index{number@{\it number}}

\begin{itemize}
\item[(i)] \scalebox{0.8}{\begin{avm}
\< hundred , \[ SYN \[ HEAD & {\it number}\\
                       VAL & \[ SPR & \q< \[ HEAD & {\it number} \] \q>\\
                                COMPS & \q< \[ HEAD & {\it number} \] \q> \]\]\] \>
\end{avm}}
%\index{SPR}
%\index{COMPS}
\end{itemize}

\noindent
This lexical entry interacts with our ordinary Head-Complement
and Head-Specifier Rules to give us the phrase structure shown in
(ii):

\begin{itemize}
\item[(ii)] \scalebox{0.8}{\begin{tree}
\br{NumP}{\br{NumP}{\lf{two}}
	  \br{Num$'$}{\br{Num}{\lf{hundred}}
		      \br{NumP}{\lf{five}}}}
\end{tree}}
\end{itemize}

Smith (1999) provides a compositional semantics
of number names. The semantics of this NP should be (iii):

\begin{itemize}
\item[(iii)]
\scalebox{0.6}{\begin{avm}
\[ INDEX & $i$\\
   MODE & ref\\
   RESTR & {\Huge$\langle$}\  \[\avmspan{RELN \ {\bf constant}}\\
		 INST & \ \ \ $l$\\
		 VALUE &  \ \ \ 2\]\ ,
              \[\avmspan{RELN \ {\bf times}}\\
		 RESULT & \ \ $k$\\
		 FACTOR1 & \ \ $l$\\
                 FACTOR2 & \ \ $m$ \]\ , 
	     \[\avmspan{RELN \ {\bf constant}}\\
		 INST & \ \ \ $m$\\
		 VALUE & \ \ 100 \]\ , \hspace{10pt}\[ RELN & {\bf plus}\\
 		 RESULT & \  \ $i$\\
	         TERM1 & \ \ $j$\\
		 TERM2 & \ \ $k$ \]\ , 
              \[\avmspan{RELN \ {\bf constant}}\\
		 INST & \ \ \ $j$\\
                 VALUE & \ \ \ 5 \]\ {\Huge$\rangle$} \]
\end{avm}}
\end{itemize}

\noindent
This  expresses ``(two times one hundred) plus five" (i.e.\ 
205) as a FS.

\begin{itemize} 

\item[A.] Assume that the two constant 
% relations 
predications
with the values 2 and
5 are contributed by the lexical entries for {\it two} and {\it five}.
What predications must be on the \index{RESTR}RESTR list of the lexical entry for
{\it hundred} in order to build (iii) as the SEM value of {\it two
hundred five}?
\item[B.] The lexical entry for {\it hundred} will identify the
indices of its specifier and complement with the value of some feature
of a predication on its RESTR list.  Which feature of which predication
is the \index{index}index of the specifier identified with?  What about the index
of the complement?
\item[C.] The lexical entry for {\it hundred} will identify its
own INDEX with the value of some feature of some predication on
its RESTR list.  Which feature of which predication must this
be, in order for the grammar 
to build (iii) as the SEM value of {\it two hundred five}?

\newpage

\item[D.] Based on your answers in parts (A)--(C), 
give a lexical entry for {\it hundred} that includes
the constraints in (i) and a fully specified SEM value.  
[{\sl Note: Your lexical entry need only account for {\it hundred}
as it is used in {\it two hundred five}. Don't worry about other
valence possibilities, such as 
{\it two hundred}, {\it two hundred and five}, or {\it a hundred}.}]
\item[E.] The syntax and semantics of number names do
not line up neatly: In the syntax, {\it hundred} forms a constituent
with {\it five}, and {\it two} combines with {\it hundred five} to
give a larger constituent.  In the semantics, the constant
% relations 
predications
with the values 2 and 100 are related via the times 
% relation
predication.
The result of that is related to the constant 
% relation 
predication
with the value
5, via the plus 
predication
% relation.  
Why is this mismatch not a problem for the grammar?
\end{itemize}




\myslide{Overview}
\begin{itemize}
\item Some notes on the linguist's stance
\item Which aspects of semantics we’ll tackle
\item Our formalization; Semantics Principles
\item Building semantics of phrases
\item Modification, coordination
\item Structural ambiguity
\end{itemize}

\myslide{Acknowledgments and References}

\begin{itemize}
\item Course design and slides borrow heavily from Emily Bender's course:
\textit{Linguistics 566: Introduction to Syntax for Computational Linguistics}
\\ \url{http://courses.washington.edu/ling566}
\end{itemize}

\input{lq-04-semantics}

\end{document}


%%% Local Variables: 
%%% coding: utf-8
%%% mode: latex
%%% TeX-PDF-mode: t
%%% TeX-engine: xetex
%%% End: 

