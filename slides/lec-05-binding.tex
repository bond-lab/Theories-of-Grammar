\documentclass[a4paper,landscape,headrule,footrule]{foils}

%%
%%% macros for Theories of Grammar
%%%
\usepackage{polyglossia}
\setdefaultlanguage{english}
%\setmainfont{TeX Gyre Pagella}


\newcommand{\logo}{~}
\newcommand{\header}[3]{%
\title{\vspace*{-2ex} \large HG4041 Theories of Grammar
\\[2ex] \Large  \emp{#2} \\ \emp{#3}}
\author{\blu{Francis Bond}   \\ 
\normalsize  \textbf{Division of Linguistics and Multilingual Studies}\\
\normalsize  \url{http://www3.ntu.edu.sg/home/fcbond/}\\
\normalsize  \texttt{bond@ieee.org}}
\MyLogo{HG4041 (2020)}
\renewcommand{\logo}{#2}
\hypersetup{
   pdfinfo={
     Author={Francis Bond},
     Title={#1: #2},
     Subject={HG4041: Theories of Grammar},
     Keywords={Syntax, Semantics, HPSG, Unification, Constructions},
     License={CC BY 4.0}
   }
 }

\date{#1
  \\ Location: LHN-TR+36}
}
\usepackage[hidelinks]{hyperref}




\usepackage{xcolor}
\usepackage{graphicx}
\newcommand{\blu}[1]{\textcolor{blue}{#1}}
\newcommand{\grn}[1]{\textcolor{green}{#1}}
\newcommand{\hide}[1]{\textcolor{white}{#1}}
\newcommand{\emp}[1]{\textcolor{red}{#1}}
\newcommand{\txx}[1]{\textbf{\textcolor{blue}{#1}}}
\newcommand{\lex}[1]{\textbf{\mtcitestyle{#1}}}

\usepackage{pifont}
\renewcommand{\labelitemi}{\textcolor{violet}{\ding{227}}}
\renewcommand{\labelitemii}{\textcolor{purple}{\ding{226}}}

\newcommand{\subhead}[1]{\noindent\textbf{#1}\\[5mm]}

\newcommand{\Bad}{\emp{\raisebox{0.15ex}{\ensuremath{\mathbf{\otimes}}}}}
\newcommand{\bad}[1]{*\eng{#1}}

\newcommand{\com}[1]{\hfill (\emp{#1})}%

\usepackage{relsize,xspace}
\newcommand{\into}{\ensuremath{\rightarrow}\xspace}
\newcommand{\tot}{\ensuremath{\leftrightarrow}\xspace}
\usepackage{url}
\newcommand{\lurl}[1]{\MyLogo{\url{#1}}}

\usepackage{mygb4e}
\newcommand{\lx}[1]{\textbf{\mtciteform{#1}}}
\newcommand{\ix}{\ex\slshape}
\let\eachwordone=\slshape



\newcommand{\ent}{\ensuremath{\Rightarrow}\xspace}
\newcommand{\ngv}{\ensuremath{\not\Rightarrow}\xspace}
%\usepackage{times}
%\usepackage{nttfoilhead}
\newcommand{\myslide}[1]{\foilhead[-25mm]{\raisebox{12mm}[0mm]{\emp{#1}}}\MyLogo{\logo}}
\newcommand{\myslider}[1]{\rotatefoilhead[-25mm]{\raisebox{12mm}[0mm]{\emp{#1}}}}
%\newcommand{\myslider}[1]{\rotatefoilhead{\raisebox{-8mm}{\emp{#1}}}}

\newcommand{\section}[1]{\myslide{}{\begin{center}\Huge \emp{#1}\end{center}}}



\usepackage[lyons,j,e,k]{mtg2e}
%\renewcommand{\mtcitestyle}[1]{\textcolor{-red!75!green!50}{\textsl{#1}}}
\renewcommand{\mtcitestyle}[1]{\textcolor{teal}{\textsl{#1}}}
\newcommand{\iz}[1]{\texttt{\textup{#1}}}
\newcommand{\gm}{\textsc}
\usepackage[normalem]{ulem}
\newcommand{\ul}{\uline}
\newcommand{\ull}{\uuline}
\newcommand{\wl}{\uwave}
\newcommand{\vs}{\ensuremath{\Leftrightarrow}~}
%%%
%%% Bibliography
%%%
\usepackage{natbib}
%\usepackage{url}
\usepackage{bibentry}


%%% From Tim
\newcommand{\WMngram}[1][]{$n$-gram#1\xspace}
\newcommand{\infers}{$\rightarrow$\xspace}

\usepackage[utf8]{inputenc}

\usepackage{rtrees,qtree}
\renewcommand{\lf}[1]{\br{#1}{}}
\usepackage{avm}
%\avmoptions{topleft,center}
\newcommand{\ft}[1]{\textsc{#1}}
\renewcommand{\val}[1]{\textit{#1}}
\newcommand{\typ}[1]{\textit{#1}}
\newcommand{\prd}[1]{\textbf{#1}}
\avmfont{\sc}
\avmvalfont{\it}
\avmsortfont{\smaller[2] \it}
\usepackage{multicol}
\newcommand{\blank}{\rule{3em}{1pt}\xspace}

% \usepackage{pst-node}
\newcommand{\OV}[1]{\ovalnode[linestyle=dotted,linecolor=red]{A}{#1}}
\newcommand{\OVB}[1]{\ovalnode[linestyle=dotted,linecolor=blue]{A}{#1}}

%%% From CSLI book
\newcommand{\mc}{\multicolumn}
\newcommand{\HD}{\textbf{H}\xspace}
\newcommand{\el}{\< \>}
\makeatother
\long\def\smalltree#1{\leavevmode{\def\\{\cr\noalign{\vskip12pt}}%
\def\mc##1##2{\multispan{##1}{\hfil##2\hfil}}%
\tabskip=1em%
\hbox{\vtop{\halign{&\hfil##\hfil\cr
#1\crcr}}}}}
\makeatletter

\newcommand{\A}{\noindent\textbf{A}: }
\newcommand{\Q}{\noindent\textbf{Q}: }
%\newcommand{\C}{\noindent\textbf{C}: }



\avmfont{\sc}
\begin{document}
\avmfont{\it}

\header{Lecture 5}{Binding Theory}{}
\maketitle


\myslide{Overview}
\MyLogo{Sag, Wasow and Bender (2003) --- Chapter 7}
\begin{itemize}
\item What we are trying to do
\item Last week: Semantics
\item Review of Chapter 1's informal binding theory
\item What we already have that’s useful
\item What we add in Ch 7 (ARG-ST, ARP)
\item Formalized Binding Theory
\item Binding and PPs
\item Imperatives
\end{itemize}



\myslide{What We’re Trying To Do}
\begin{itemize}
\item Objectives
\begin{itemize}
\item Develop a theory of knowledge of language
\item Represent linguistic information explicitly enough to
distinguish well-formed from ill-formed expressions
\item Assign plausible semantic representations
\item Be parsimonious, capturing linguistically significant
generalizations.
\end{itemize}
\item Why Formalize?
\begin{itemize}
\item To formulate testable predictions
\item To check for consistency
\item To make it possible to get a computer to do it for us
\end{itemize}
\end{itemize}

\myslide{How We Construct Sentences}
\begin{itemize}
\item The Components of Our Grammar
\begin{itemize}
\item Grammar rules
\item Lexical entries
\item Principles
\item Type hierarchy (very preliminary, so far)
\item Initial symbol (S, for now)
\end{itemize}
\item We combine constraints from these components.
%\item Q: What says we have to combine them?
% \item More in \url{hpsg.stanford.edu/book/slides/Ch6a.pdf}, 
% \\ \url{hpsg.stanford.edu/book/slides/Ch6b.pdf}
\end{itemize}

\section{Review of Semantics}

\myslide{Overview}
\begin{itemize}
\item Which aspects of semantics we’ll tackle
\item Semantics Principles
\item Building semantics of phrases
\item Modification, coordination
\item Structural ambiguity
\end{itemize}


\myslide{Our Slice of a World of Meanings}

Aspects of meaning we \emp{won’t} account for (in this course)
\begin{itemize}
\item Pragmatics
\item Fine-grained lexical semantics
  \\ The meaning of \eng{life} is
  \begin{itemize}\addtolength{\itemsep}{2ex}
  \item \iz{life}  
or 
\textsf{life}$'$
or
  \begin{avm}\avmfont{\sc}%\avmvalfont{\it}
    \[ reln & life \\ inst & i \]
  \end{avm}
\item Not like wordnet: 
\iz{life$_1$}  $\subset$ \iz{being$_1$} $\subset$ \iz{state$_1$} \ldots

 % -- (a characteristic state or mode of living; "social life"; "city life"; "real life")
 %       => being, beingness, existence -- (the state or fact of existing; "a point of view gradually coming into being"; "laws in existence for centuries")
 %           => state -- (the way something is with respect to its main attributes; "the current state of knowledge"; "his state of health"; "in a weak financial state")
 %               => attribute -- (an abstraction belonging to or characteristic of an entity)
 %                   => abstraction, abstract entity -- (a general concept formed by extracting common features from specific examples)
 %                       => entity -- (that which is perceived or known or inferred to have its own distinct existence (living or nonliving))
 \end{itemize}
\item Quantification \hfill (covered lightly in the book)
\item Tense, Mood, Aspect \hfill (covered in the book)
 \end{itemize}


% RELN life
% INST i

\myslide{Our Slice of a World of Meanings}
\begin{avm}\avmfont{\sc}
\[mode &  prop \\
 index &  $s$\\
 restr & \< \[ RELN & {\bf save}\\
               SIT & $s$\\
              SAVER &  $i$\\
              SAVED &  $j$ \] \ , 
              \[RELN & {\bf name}\\ 
              NAME &  Chris\\
              NAMED & \ $i$\]\ , 
              \[RELN & {\bf name}\\
              NAME &  Pat\\
              NAMED & \ $j$\] \> \]
\end{avm} 
\begin{quote}
  “\ldots the linguistic meaning of \eng{Chris saved Pat} is a proposition
  that will be true just in case there is an actual situation that
  involves the saving of someone named Pat by someone named Chris.”
  \begin{flushright}
    (Sag \textit{et al}, 2003, p. 140)
  \end{flushright}
\end{quote}

\myslide{Semantics in Constraint-Based Grammar}
\begin{itemize}
\item Constraints as (generalized) truth conditions
  \begin{itemize}
\item \txx{proposition}: what must be the case for a proposition to be true
\item \txx{directive}: what must happen for a directive to be fulfilled
\item \txx{question}: the kind of situation the asker is asking about
\item \txx{reference}: the kind of entity the speaker is referring to
\end{itemize}
\item \txx{Syntax/semantics interface}: 
  \begin{quote}
    Constraints on how syntactic arguments are related to semantic
    ones, and on how semantic information is compiled from different
    parts of the sentence.
  \end{quote}
\end{itemize}

\myslide{Feature Geometry}
\begin{avm}\avmfont{\sc}
\[\asort{expression}
syn & \[ \asort{syn-cat}
      head & \[ \asort{pos} ... \] \\
      val & \[ spr & \q< ... \q> \\
               comps & \q< ... \q> \]\]\\
      sem & \[ \asort{sem-cat}
      mode &  \q\{ prop, ques, dir, ref, none \q\} \\
      index &  \q\{ i, j, k, \ldots, s$_1$, s$_2$, \ldots \q\} \\
      restr & \q< ... \q> \]\]
 \end{avm}



% \myslide{How the Pieces Fit Together}
% \begin{avm}\avmfont{\sc}
%   \< \eng{Kim}, \[syn & \[head & \[{\it noun}\\
%   agr & {\it 3sing}\]\\
%   val & \[spr & {\el}\\
%   comps & {\el}\]\]\\
%   sem & \[mode & ref\\
%   index & \ \ $i$\\
%   restr & \< \[reln & {\bf name}\\
%   % sit & \ \ $s$\\
%   name & Kim\\
%   named & \ \ $i$\] \> \]\] \>  %
% \end{avm}
% \myslide{How the Pieces Fit Together}
% \begin{avm}\avmfont{\sc}
% \< \eng{sleep} ,\ \[syn & \[head & {\it verb}\\
%                                      val & \[spr & \q< NP$_i$ \q>\\
%                                      comps & \q<  \q>\]\]\\
%                             sem & \[mode & prop\\
%                                     index & $s$\\
%                                     restr & \< \[reln & {\bf sleep}\\
%                                                  sit & \ \ $s$\\
%                                                  sleeper & \ \ $i$  \] \>\]\] \>
%                                         \end{avm}
% \myslide{The Pieces Together}

% \begin{avmtree}\avmfont{\sc}
%   \br{S}{
%     \br{\@{1} NP  \[ sem & \[ index & $i$ \] \] }{\lf{\eng{Kim}}}
%     \br{VP  \[ syn & \[ val & \[ spr & \< \@{1} \> \] \] \\
%       sem & \[ mode & prop\\
%       index & $s$\\
%       restr & \< \[reln & {\bf sleep}\\
%       sit & \ \ $s$\\
%       sleeper & \ \ $i$  \] \> \] \]    }{\lf{\eng{slept}}}}
% \end{avmtree}


\myslide{An Example}
\scalebox{0.8}{\begin{avmtree}\avmfont{\sc}
  \br{S  \[ sem & \[ mode & prop \\ index & $s$\\
      restr & \< \[reln & {\bf name}\\
      name & \ \ Kim \\
      named & \ \ $i$ \],
      \[reln & {\bf sleep}\\
      sit & \ \ $s$\\
      sleeper & \ \ $i$  \]  \>  \] \] }{
    \br{\@{1} NP  \[ sem & \[ mode & ref \\ index & $i$\\
      restr & \< \[reln & {\bf name}\\
      name & \ \ Kim \\
     named & \ \ $i$ \] \>  \] \] }{\lf{\eng{Kim}}}
    \br{\HD VP  \[ syn & \[ val & \[ spr & \< \@{1} \> \] \] \\
      sem & \[  mode & prop \\ index & $s$ \\ 
      restr & \< \[reln & {\bf sleep}\\
      sit & \ \ $s$\\
      sleeper & \ \ $i$  \] \> \] \]    }{\lf{\eng{slept}}}}
\end{avmtree}}


\myslide{How to Share Semantic Information}
\MyLogo{List summation: $\oplus$ (technically concatenation)} 
\begin{itemize}
\item \txx{The Semantic Inheritance Principle}
  \begin{quote}
    In any headed phrase, the mother's \ft{mode} and
    \ft{index} are identical to those of the head daughter.    
  \end{quote}
\item \txx{The Semantic Compositionality Principle}
  \begin{quote}
    In any well-formed phrase structure, the mother's
    \ft{restr} value is the sum of the \ft{restr} values of
    the daughters.
  \end{quote}
\end{itemize}

\myslide{Where is the information}
\MyLogo{}
\begin{itemize}
\item Words
  \begin{itemize}
  \item Contribute predications
  \item ‘expose’ one index in those predications, for use by words or phrases
  \item relate syntactic arguments to semantic arguments
  \end{itemize}
\item Rules
\begin{itemize}
\item Identify (link) feature structures across daughters
\item License trees which are subject to the semantic principles
  \begin{itemize}
  \item SIP: ‘passes up’ \ft{mode} and \ft{index} from head daughter
  \item SCP: ‘gathers up’ predications (\ft{restr} list) from all daughters
  \end{itemize}
\end{itemize}
\item The semantics is strictly compositional --- all of the meaning
  comes from the words, rules and principles.
\end{itemize}

\section{Binding}


\myslide{Some Examples from Chapter 1}

\begin{exe}
\ix She likes herself
\ix *She$_i$ likes her$_i$.
\ix We gave presents to ourselves.
\ix *We gave presents to us.
\ix We gave ourselves presents
\ix *We gave us presents.
\ix *Leslie told us about us.
\ix Leslie told us about ourselves.
\ix *Leslie told ourselves about us.
\ix *Leslie told ourselves about ourselves.
\end{exe}

\myslide{Some Terminology}
\begin{itemize}
\item \txx{Binding}: The association between a pronoun
and an antecedent.
\item \txx{Anaphoric}: A term to describe an element (e.g.
a pronoun) that derives its interpretation from
some other expression in the discourse.
\item \txx{Antecedent}: The expression an anaphoric
expression derives its interpretation from.
\item \txx{Anaphora}: The relationship between an
anaphoric expression and its antecedent.
\end{itemize}

\myslide{The Chapter 1 Binding Theory Reformulated}
\begin{itemize}
\item Old Formulation:
\begin{itemize}
\item A reflexive pronoun must be an argument of a verb that
has another preceding argument with the same reference.
\item A nonreflexive pronoun cannot appear as an argument of
a verb that has a preceding coreferential argument.
\end{itemize}
\item New Formulation(version I):
\begin{itemize}
\item \txx{Principle A}: A reflexive pronoun must be
bound by a preceding argument of the same verb.
\item \txx{Principle B}: A nonreflexive pronoun may not
be bound by a preceding argument of the same verb.
\end{itemize}
\item Opaque names come from Chomsky (1981)
\end{itemize}

\myslide{Some Challenges}
\begin{itemize}
\item Replace notions of \emp{bound} and \emp{preceding
argument of the same verb} by notions
definable in our theory.
\item Generalize the Binding Principles to get
better coverage.
\end{itemize}


\myslide{How can we do this?}
\begin{itemize}
\item[Q]  What would be a natural way to formalize
the notion of “bound” in our theory?
\item[A] Two expressions are bound if
they have the same \ft{index} value (“are
coindexed”).
\item[Q] Where in our theory do we have information
about a verb’s arguments?
\item[A] In the verb’s \ft{valence} features.
\item[Q] What determines the linear ordering of a
verb’s arguments in a sentence?
\item[A] The interaction of the grammar
rules and the ordering of elements in the
\ft{comps} list.
\end{itemize}

\myslide{The Argument Realization Principle}
\begin{itemize}
\item For Binding Theory, we need a single list with both subject
and complements.
\item We introduce a feature \ft{arg-st}, with the following
property: \\
\begin{avm}\avmfont{\sc} 
\[ \asort{word}  
 syn & \[ val & \[ spr &  \@{A} \\comps &  \@{B} \] \] \\
 arg-st & \< \@{A} $\oplus$ \@{B} \> \]
\end{avm}

\item This is a constraint on the type \val{word}
\end{itemize}

\myslide{Notes on ARG-ST}
\begin{itemize}
\item It’s neither in \ft{syn} nor \ft{sem}.
\item It only appears on lexical heads (not
appropriate for type phrase)
\item No principle stipulates identity
between \ft{arg-st}s.
\end{itemize}

\myslide{The Binding Principles}
\begin{itemize}
\item \txx{Principle A}: A [\ft{mode} \val{ana}] element must be
outranked by a coindexed element.
\item \txx{Principle B}: A [\ft{mode} \val{ref}] element must not
be outranked by a coindexed element.
\end{itemize}

\begin{description}
\item [Formalization] ~\\[-2ex]
\begin{itemize}
\item Definition: If A precedes B on some \ft{arg-st} list,
then A \txx{outranks} B.
\item Elements that must be \txx{anaphoric} --- that is, that
require an antecedent --- are lexically marked
[\ft{mode} \val{ana}]. These include reflexive pronouns
and reciprocals.  
\item Normal \txx{referential} NPs, including pronouns,  are marked [\ft{mode} \val{ref}].
\end{itemize}
\end{description}

\myslide{Pronoun-Antecedent Agreement}
\begin{itemize}
\item The Binding Principles by themselves don’t block:
  \begin{exe}
    \ix * I amused yourself.
    \ix * He amused themselves.
    \ix * She amused himself.
  \end{exe}
\item Coindexed NPs refer to the same entity, and \ft{agr} features
generally correlate with properties of the referent.
\item The \txx{Anaphoric Agreement Principle} (AAP):
Coindexed NPs agree.
\end{itemize}

\myslide{Binding in PPs}
\begin{itemize}
\item What do the Binding Principles predict about the
following?
\item The Binding Principles by themselves don’t block:
  \begin{exe}
    \ix I brought a book with me.
    \ix *I brought a book with myself.
    \ix *I mailed a book to me.
    \ix I mailed a book to myself.
  \end{exe}
\end{itemize}

\myslide{Two Types of Prepositions: the Intuition}
\begin{itemize}
\item \txx{Argument-marking}: Function like casemarkers in other languages, indicating the
roles of NP referents in the situation denoted by the verb.
\item \txx{Predicative}: Introduce their own predication.
\end{itemize}

\begin{description}
\item [Formalization] ~\\[-2ex]
  \begin{itemize}
  \item Argument-marking prepositions share their
    objects' \ft{mode} and \ft{index} values.
    \begin{itemize}
    \item This is done with tagging in the lexical
      entries of such prepositions.
    \item These features are also shared with the PP
      node, by the Semantic Inheritance Principle.
    \end{itemize}
  \item Predicative prepositions introduce their own
\ft{mode} and \ft{index} values.
\end{itemize}
\end{description}

\myslide{Argument-marking preposition \eng{to}}
\begin{avm}\avmfont{\sc}
  \< \textit{to}, \[ syn & \[ head & prep \\
                     val &  \[ spr \< \> \] \] \\

                   arg-st & \< NP \[syn & \[ head & \[ case & acc \]  \]\\
                                  sem & \[ mode & \@1 \\ index & \@2 \] \] \>  \\
                   sem &  \[ mode & \@1 \\ index & \@2 \] \]    \>
\end{avm}

\begin{avm}\avmfont{\sc} 
  \textsc{arg-st} =  \< spr $\oplus$ comps \> 
\end{avm}

Therfore comps here is a singleton list with the NP in it, \ldots

\myslide{Redefining Rank}
 \begin{itemize}
\item If there is an \ft{arg-st} list on which A
precedes B, then A outranks B.
\item If a node is coindexed with its daughter, they
are of equal rank -- that is, they outrank the
same nodes and are outranked by the same
nodes.
\end{itemize}

\myslide{\eng{I sent a letter to myself}}

\begin{avmtree}\avmfont{\sc} 
\br{S}{
  \br{\@1 NP$_i$}{ \lf{\eng{I}} }
  \br{VP \[ \q< \@1 \q> \]}{ 
    \br{V \[ spr & \q< \@1 \q> \\
             comps & \q< \@2, \@3 \q> \\ 
             arg-st & \q< \@1, \@2, \@3 \q> \]}{
      \lf{\eng{sent}}}
    \br{\@2 NP$_j$}{
      \br{D}{\lf{\eng{a}}}
      \br{N$_j$}{\lf{\eng{letter}}}}
    \br{\@3 PP$_i$}{
      \br{P$_i$}{\lf{\eng{to}}}
      \br{NP$_i$ \[ mode & ana \]}{\lf{\eng{myself}}}}
}}
\end{avmtree}




\myslide{The ARG-ST}
\begin{center}
  \begin{avm}\avmfont{\sc} 
\[  arg-st & \< 
NP$_i$ \[ mode & ref \],
NP$_j$ \[ mode & ref \],
PP$_i$ \[ mode & ana \]
\> \]
\end{avm}
\end{center}
\begin{itemize}
\item The PP is outranked by the first NP. (Why?)
\item \eng{myself} has the same rank as the PP. (Why?)
\item So, \eng{myself} is outranked by the first NP. (Why?)
\item Therefore, Principle A is satisfied:  A [\ft{mode} \val{ana}] element must be
outranked by a coindexed element.
\end{itemize}

\myslide{\eng{*I sent a letter to me}}
\begin{avmtree}\avmfont{\sc} 
  \br{S}{
  \br{\@1 NP$_i$}{ \lf{\eng{I}} }
  \br{VP \[ \q< \@1 \q> \]}{ 
    \br{V \[ spr & \q< \@1 \q> \\
             comps & \q< \@2, \@3 \q> \\ 
             arg-st & \q< \@1, \@2, \@3 \q> \]}{
      \lf{\eng{sent}}}
    \br{\@2 NP$_j$}{
      \br{D}{\lf{\eng{a}}}
      \br{N$_j$}{\lf{\eng{letter}}}}
    \br{\@3 PP$_i$}{
      \br{P$_i$}{\lf{\eng{to}}}
      \br{NP$_i$ \[ mode & ref \]}{\lf{\eng{me}}}}
}}
\end{avmtree}

\myslide{The ARG-ST}
\begin{center}
  \begin{avm}\avmfont{\sc} 
\[  arg-st & \< 
NP$_i$ \[ mode & ref \],
NP$_j$ \[ mode & ref \],
PP$_i$ \[ mode & ref \]
\> \]
\end{avm}
\end{center}
\begin{itemize}
\item The PP is outranked by the first NP.
\item \eng{me} has the same rank as the PP.
\item So, \eng{me} is outranked by the first NP.
\item Therefore, Principle B is violated: A [\ft{mode} \val{ref}] element must not
be outranked by a coindexed element.
\end{itemize} 

\myslide{\eng{I brought a pencil with me}}
\MyLogo{Here \eng{I} does not outrank \eng{me}, so Principle B is satisfied}

\begin{avmtree}\avmfont{\sc} 
\br{S}{
  \br{\@1 NP$_i$}{ \lf{\eng{I}} }
  \br{VP \[ \q< \@1 \q> \]}{ 
    \br{V \[ spr & \q< \@1 \q> \\
             comps & \q< \@2 \q> \\ 
             arg-st & \q< \@1, \@2 \q> \]}{
      \lf{\eng{brought}}}
    \br{\@2 NP$_j$}{
      \br{D}{\lf{\eng{a}}}
      \br{N$_j$}{\lf{\eng{pencil}}}}
    \br{\@3 PP$_k$}{
      \br{P$_k$ \[ comps & \q< \@1 \q> \\ 
                   mod & \q< \@2 \q> \\
                   arg-st & \q< \@1 \q> \] }{\lf{\eng{with}}}
      \br{NP$_l$ \[ mode & ref \]}{\lf{\eng{me}}}}
}}
\end{avmtree}

\myslide{\eng{*I brought a pencil with myself}}
\MyLogo{Here \eng{I} does not outrank \eng{myself}, so Principle A is violated}

\begin{avmtree}\avmfont{\sc} 
\br{S}{
  \br{\@1 NP$_i$}{ \lf{\eng{I}} }
  \br{VP \[ \q< \@1 \q> \]}{ 
    \br{V \[ spr & \q< \@1 \q> \\
             comps & \q< \@2 \q> \\ 
             arg-st & \q< \@1, \@2 \q> \]}{
      \lf{\eng{brought}}}
    \br{\@2 NP$_j$}{
      \br{D}{\lf{\eng{a}}}
      \br{N$_j$}{\lf{\eng{pencil}}}}
    \br{\@3 PP$_k$}{
      \br{P$_k$ \[ comps & \q< \@1 \q> \\ 
                   mod & \q< \@2 \q> \\
                   arg-st & \q< \@1 \q> \] }{\lf{\eng{with}}}
      \br{NP$_l$ \[ mode & ana \]}{\lf{\eng{myself}}}}
}}
\end{avmtree}

\myslide{Imperatives}
\MyLogo{}
\begin{itemize}
\item Have the internal structure of a VP
  \begin{exe}
    \ix Leave!
    \ix Read a book!
    \ix Give the dog a treat!
    \ix Put the ice cream in the freezer!
  \end{exe}
\item Function as \txx{directives} (commands or requests)
\item Have the verb in base form: \eng{Be careful!} not \eng{*Are careful!}
\begin{itemize}
\item Allow 2nd person reflexives, and no others
\begin{exe}
    \ix Defend yourself!
    \ix *Defend myself/himself!
  \end{exe}
\end{itemize}
\end{itemize}

\myslide{The Imperative Rule}

\begin{avm}\avmfont{\sc} 
\[ \asort{phrase}  
   head & verb \\
   val & \[ spr  & \<  \> \] \\
   sem & \[ mode & dir \\ index & $s$ \] \]
              \ $\rightarrow$\ \ 
\[ \asort{phrase}  
   head & \[ \asort{verb} form & base \]\\
   val & \[ spr  & \< NP \[ per 2nd \]   \> \\
            comps & \< \>  \] \\
   sem & \[ index & $s$ \] \]
\end{avm}


\begin{itemize}\addtolength{\itemsep}{-2ex}
\item Internal structure of a VP
\item Directive function
\item Base form
\item Only 2nd person reflexives
\item[Q] Note that this is not a headed rule. Why?
\item[A] It would violate the HFP and the SIP.
\end{itemize}

\myslide{Imperative example}
\begin{multicols}{2}
\begin{avmtree}\avmfont{\sc} 
\br{S}{
  \br{VP \[ spr & \q< \@1 NP \[ per 2nd \\ num sg \] \q> \]}{ 
    \br{V \[ spr & \q< \@1 \q> \]}{
      \lf{\eng{vote}}}
    \br{\@3 PP$_i$}{
      \br{P$_i$}{\lf{\eng{for}}}
      \br{NP$_i$ \[ mode & ana \]}{\lf{\eng{yourself}}}}
}}
\end{avmtree}
%\newpage

\begin{itemize}
\item What's the SPR value on S?
%Why?
\item What's the SPR value on VP?
%Why?
\item What's the SPR value on V?
%Why?
\item Which nodes have ARG-ST?
\item Which ARG-ST matters for
the licensing of \eng{yourself}?
\end{itemize}
%\newpage

\end{multicols}

\myslide{\ft{arg-st} on vote}
\begin{center}
  \begin{avm}\avmfont{\sc} 
\[  arg-st & \< 
NP$_i$ \[ mode & ref \\ per & 2nd \\ num & sg \],
PP$_i$ \[ mode & ana \]
\> \]
\end{avm}
\end{center}
\begin{itemize}
\item Is Principle A satisfied?
\item How?
\item Is Principle B satisfied?
\item How?
\end{itemize}

\myslide{Day 1 Revisited}
\begin{itemize}
\item Recall
  \begin{exe}
    \ix Screw yourself!
    \ix Go screw yourself!
    \ix screw you!
    \ix *Go screw you!
  \end{exe}
\item \eng{Screw NP!} has two analyses
  \begin{itemize}
  \item As an imperative
  \item As a truly subjectless fixed expression.
  \end{itemize}
\item \eng{Go screw NP!} can only be analyzed as an imperative.
\end{itemize}


\myslide{Overview}
\begin{itemize}\addtolength{\itemsep}{-2ex}
\item Review of Chapter 1 informal binding
theory
\item Formalized Binding Theory
  \begin{itemize}
  \item \txx{Argument Realization Principle} (ARP)
    \\  \ft{arg-st} = \ft{spr}  $\oplus$ \ft{comps}
  \item \txx{Anaphoric Agreement Principle} (AAP)
   \\ Coindexed NPs agree.
 \item \txx{Principle A}\ \  A [\ft{mode} \val{ana}] element must be
outranked by a coindexed element.
\item \txx{Principle B}\ \  A [\ft{mode} \val{ref}] element must not
be outranked by a coindexed element.
  \end{itemize}
\item Binding and PPs
\item Imperatives
\end{itemize}


\myslide{P1: English Possessives I}
\MyLogo{Based on  Chapter 6, Problem 3, Sag, Wasow and Bender (2003)}

English uses {\it 's} to express possession, as in the following examples:

\begin{exe}
\ex  \eng{Leslie's coffee spilled.}
\ex  \eng{Jesse met the president of the university's cousin.}\label{s:i}
\ex  \bad{Jesse met the president's of the university cousin.}
\ex  \eng{Don't touch that plant growing by the trail's leaves.}\label{s:iv}
\ex \bad{Don't touch that plant's growing by the trail leaves.}
\ex  \eng{The person you were talking to's pants are torn.}\label{s:vi}
\ex  \bad{The person's you were talking to pants are torn.}\label{s:vii}
\end{exe}
(While examples (\ref{s:iv}) and (\ref{s:vi}) are a bit awkward, people do use such
sentences, and there is certainly nowhere else that the \eng{'s} could
be placed to improve them).  
\begin{itemize}
\item[A.] What is the generalization about where the \eng{'s} of
possession appears in English?
\end{itemize}
\newpage

% Consider noun \index{possessive}
% phrases like \eng{Kim's brother} and \eng{the president's salary}. 
One
traditional treatment of the possessive marker (\eng{'s})
% in English 
is to
claim it is a \index{case} case marker.
%-- which in our terms would mean it is a value of the feature \index{CASE!possessive}CASE.
In our terms this means that it indicates 
a particular value for the feature CASE (say, `poss' for `possessive')
on the word it attaches to.
% The principal objection to a treatment of \eng{'s}
% as a case marker is that, in conjunction with our analysis of case (in
% particular our assumption that \index{CASE} CASE is a HEAD feature),
% it makes the wrong 
% predictions about examples like the following:
%\begin{itemize}
%\item[B.] 
If we tried to formalize this traditional treatment of \eng{'s}, we
% would  need 
might posit
a rule along the following lines, 
based on the fact that possessive NPs appear in the same position as determiners:
\begin{itemize}
\item[] D \ $\rightarrow$\ \ \begin{avm} NP \[CASE & poss \]\end{avm}
\end{itemize}
Taken together with our assumption that \index{CASE}CASE is
a \index{head feature}HEAD feature,  such an analysis of \eng{'s} 
%as a case marker 
makes predictions about the grammaticality of (\ref{s:i})--(\ref{s:vii}).  
\begin{itemize}
\item[B.] Which of these sentences does it
predict should be grammatical, and why?

\end{itemize}

\myslide{P2: English Possessives II}
\MyLogo{Based on  Chapter 6, Problem 4, Sag, Wasow and Bender (2003)}

An  alternative analysis of the possessive is to say that %it 
\eng{'s}
is a
\index{determiner (D)} determiner that builds a \index{determiner phrase
(DP)} determiner phrase (abbreviated DP),
via the \index{Head Specifier Rule@Head-Specifier Rule}Head-Specifier Rule. 
On this
analysis,  \eng{'s} selects for no complements, but it obligatorily
takes an NP \index{specifier} specifier. The word \eng{'s} thus has a lexical category
that is like an intransitive verb in valence.

This analysis is somewhat unintuitive, for two reasons:  first, it
requires that we have an independent lexical entry for \eng{'s}, which
seems more like a piece of a word, phonologically;  and second, it makes
the nonword \eng{'s} the head of its phrase!  However, this analysis
does a surprisingly good job of predicting the facts of English
 possessives, so we shall adopt it, at least for purposes of
this text.

\begin{itemize}
\item[A.] Ignoring semantics for the moment, give the lexical entry for
\eng{'s} assuming its analysis as a \index{determiner (D)} determiner, and draw 
%a phrasal SD (that
%is, 
a tree 
%description) 
for the NP \eng{Kim's brother}. 
%(Use abbreviations when
%necessary, for example: \hbox{NP[3rd, sg]} or
%\hbox{[SPR $\langle$NP[nom]$\rangle$]}.)
(The tree should show the value of \ft{head, spr} and \ft{comps} on every
node.  Use tags to show identities required by the grammar.
You may omit other features.)

\item[B.] Explain how your lexical entry gets the facts right in the 
% {\it Queen of England} examples above. 
following examples:
\begin{exe}
\ex \eng{The Queen of England's crown disappeared.}
\ex \bad{The Queen's of England crown disappeared.}
\end{itemize}

\item[C.] How does this analysis handle recursion in possessives, 
for example, \eng{Robin's brother's wife}, or \eng{Robin's brother's wife's
parents}?  Provide at least one tree %description 
fragment
to illustrate your
explanation.
% (again you can abbreviate complex feature structures
% in your trees, 
%if the omitted details are clearly irrelevant).
(You may use abbreviations for node labels in the tree.)
\end{itemize}



\myslide{P3: Prepositions}
\MyLogo{Based on  Chapter 7, Problem 1, Sag, Wasow and Bender (2003)}

For each of the following sentences, 
\begin{itemize}
\item[(a)] classify the underlined preposition into \txx{predicative}
  or \txx{argument marking} and
\item[(b)] justify by showing what reflexive and
  nonreflexive coreferential 
  pronouns can or cannot appear as the preposition's object.
\end{itemize}
\begin{itemize}\addtolength{\itemsep}{-1.5ex}
\item[(i)] \eng{The dealer dealt an ace \ul{to} Bo.}
\item[(ii)] \eng{The chemist held the sample away \ul{from} the flame.}
\item[(iii)] \eng{Hiromi kept a flashlight \ul{beside} the bed.}
\item[(iv)] \eng{We bought flowers \ul{for} you.}
\item[(v)] \eng{The car has a scratch \ul{on} the fender.}
\item[(vi)] \eng{The pronoun agrees \ul{with} its antecedent.}
\end{itemize}

\myslide{P4: Imperative `Subjects'}
\MyLogo{Based on  Chapter 7, Problem 2, Sag, Wasow and Bender (2003)}
\begin{itemize}
\item Some imperatives look like they have a subject:
  \begin{itemize}
  \item[(i)] \eng{\ul{You} get out of here!}
  \item[(ii)] \eng{\ul{Everybody} take out a sheet of paper!}
  \end{itemize}
\item But agreement is wierd:
  \begin{itemize}
  \item[(iii)] 
    \eng{Everybody found 
      $^?$himself/$^*$yourself/themselves/$^*$myself a seat.}
  \item[(iv)]
    \eng{Everybody find 
      $^?$himself/yourself/$^*$themselves/$^*$myself a seat.}
  \end{itemize}
\item What minimal modification of the Imperative Rule would account for
the indicated data? (don't worry about
the semantics)
\end{itemize}



\myslide{Acknowledgments and References}

\begin{itemize}
\item Course design and slides borrow heavily from Emily Bender's course:
\textit{Linguistics 566: Introduction to Syntax for Computational Linguistics}
\\ \url{http://courses.washington.edu/ling566}
\end{itemize}


%\input{lec-05-questions}

\end{document}

%%% Local Variables: 
%%% coding: utf-8
%%% mode: latex
%%% TeX-PDF-mode: t
%%% TeX-engine: xetex
%%% End: 
