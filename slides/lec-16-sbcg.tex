\documentclass[a4paper,landscape,headrule,footrule]{foils}
\usepackage{times}
%\usepackage{nttfoilhead}
%\newcommand{\myslide}[1]{\foilhead[-25mm]{\raisebox{12mm}[0mm]{\emp{#1}}}}
%\newcommand{\myslider}[1]{\rotatefoilhead[-25mm]{\raisebox{12mm}[0mm]{\emp{#1}}}}
%\newcommand{\myslider}[1]{\rotatefoilhead{\raisebox{-8mm}{\emp{#1}}}}

%%
%%% macros for Theories of Grammar
%%%
\usepackage{polyglossia}
\setdefaultlanguage{english}
%\setmainfont{TeX Gyre Pagella}


\newcommand{\logo}{~}
\newcommand{\header}[3]{%
\title{\vspace*{-2ex} \large HG4041 Theories of Grammar
\\[2ex] \Large  \emp{#2} \\ \emp{#3}}
\author{\blu{Francis Bond}   \\ 
\normalsize  \textbf{Division of Linguistics and Multilingual Studies}\\
\normalsize  \url{http://www3.ntu.edu.sg/home/fcbond/}\\
\normalsize  \texttt{bond@ieee.org}}
\MyLogo{HG4041 (2020)}
\renewcommand{\logo}{#2}
\hypersetup{
   pdfinfo={
     Author={Francis Bond},
     Title={#1: #2},
     Subject={HG4041: Theories of Grammar},
     Keywords={Syntax, Semantics, HPSG, Unification, Constructions},
     License={CC BY 4.0}
   }
 }

\date{#1
  \\ Location: LHN-TR+36}
}
\usepackage[hidelinks]{hyperref}




\usepackage{xcolor}
\usepackage{graphicx}
\newcommand{\blu}[1]{\textcolor{blue}{#1}}
\newcommand{\grn}[1]{\textcolor{green}{#1}}
\newcommand{\hide}[1]{\textcolor{white}{#1}}
\newcommand{\emp}[1]{\textcolor{red}{#1}}
\newcommand{\txx}[1]{\textbf{\textcolor{blue}{#1}}}
\newcommand{\lex}[1]{\textbf{\mtcitestyle{#1}}}

\usepackage{pifont}
\renewcommand{\labelitemi}{\textcolor{violet}{\ding{227}}}
\renewcommand{\labelitemii}{\textcolor{purple}{\ding{226}}}

\newcommand{\subhead}[1]{\noindent\textbf{#1}\\[5mm]}

\newcommand{\Bad}{\emp{\raisebox{0.15ex}{\ensuremath{\mathbf{\otimes}}}}}
\newcommand{\bad}[1]{*\eng{#1}}

\newcommand{\com}[1]{\hfill (\emp{#1})}%

\usepackage{relsize,xspace}
\newcommand{\into}{\ensuremath{\rightarrow}\xspace}
\newcommand{\tot}{\ensuremath{\leftrightarrow}\xspace}
\usepackage{url}
\newcommand{\lurl}[1]{\MyLogo{\url{#1}}}

\usepackage{mygb4e}
\newcommand{\lx}[1]{\textbf{\mtciteform{#1}}}
\newcommand{\ix}{\ex\slshape}
\let\eachwordone=\slshape



\newcommand{\ent}{\ensuremath{\Rightarrow}\xspace}
\newcommand{\ngv}{\ensuremath{\not\Rightarrow}\xspace}
%\usepackage{times}
%\usepackage{nttfoilhead}
\newcommand{\myslide}[1]{\foilhead[-25mm]{\raisebox{12mm}[0mm]{\emp{#1}}}\MyLogo{\logo}}
\newcommand{\myslider}[1]{\rotatefoilhead[-25mm]{\raisebox{12mm}[0mm]{\emp{#1}}}}
%\newcommand{\myslider}[1]{\rotatefoilhead{\raisebox{-8mm}{\emp{#1}}}}

\newcommand{\section}[1]{\myslide{}{\begin{center}\Huge \emp{#1}\end{center}}}



\usepackage[lyons,j,e,k]{mtg2e}
%\renewcommand{\mtcitestyle}[1]{\textcolor{-red!75!green!50}{\textsl{#1}}}
\renewcommand{\mtcitestyle}[1]{\textcolor{teal}{\textsl{#1}}}
\newcommand{\iz}[1]{\texttt{\textup{#1}}}
\newcommand{\gm}{\textsc}
\usepackage[normalem]{ulem}
\newcommand{\ul}{\uline}
\newcommand{\ull}{\uuline}
\newcommand{\wl}{\uwave}
\newcommand{\vs}{\ensuremath{\Leftrightarrow}~}
%%%
%%% Bibliography
%%%
\usepackage{natbib}
%\usepackage{url}
\usepackage{bibentry}


%%% From Tim
\newcommand{\WMngram}[1][]{$n$-gram#1\xspace}
\newcommand{\infers}{$\rightarrow$\xspace}

\usepackage[utf8]{inputenc}

\usepackage{rtrees,qtree}
\renewcommand{\lf}[1]{\br{#1}{}}
\usepackage{avm}
%\avmoptions{topleft,center}
\newcommand{\ft}[1]{\textsc{#1}}
\renewcommand{\val}[1]{\textit{#1}}
\newcommand{\typ}[1]{\textit{#1}}
\newcommand{\prd}[1]{\textbf{#1}}
\avmfont{\sc}
\avmvalfont{\it}
\avmsortfont{\smaller[2] \it}
\usepackage{multicol}
\newcommand{\blank}{\rule{3em}{1pt}\xspace}

% \usepackage{pst-node}
\newcommand{\OV}[1]{\ovalnode[linestyle=dotted,linecolor=red]{A}{#1}}
\newcommand{\OVB}[1]{\ovalnode[linestyle=dotted,linecolor=blue]{A}{#1}}

%%% From CSLI book
\newcommand{\mc}{\multicolumn}
\newcommand{\HD}{\textbf{H}\xspace}
\newcommand{\el}{\< \>}
\makeatother
\long\def\smalltree#1{\leavevmode{\def\\{\cr\noalign{\vskip12pt}}%
\def\mc##1##2{\multispan{##1}{\hfil##2\hfil}}%
\tabskip=1em%
\hbox{\vtop{\halign{&\hfil##\hfil\cr
#1\crcr}}}}}
\makeatletter

\newcommand{\A}{\noindent\textbf{A}: }
\newcommand{\Q}{\noindent\textbf{Q}: }
%\newcommand{\C}{\noindent\textbf{C}: }



\avmfont{\sc}
\begin{document}
\avmfont{\it}
\header{Lecture 16}{Sign-Based Construction Grammar}{}
\maketitle




\myslide{Overview}
\MyLogo{Sag, Wasow and Bender (2003) --- Chapter 16}

\begin{itemize}
\item Chapter 16 framework (same analyses, 
different underlying system)
\item General wrap up
\end{itemize}

\myslide{Construction Grammar (CxG)}

\begin{itemize}
\item A family of grammars based on the idea that
  \begin{itemize}
  \item Knowledge of a language comes from \emp{form and function pairings}
  \item \txx{function} includes meaning, content, or intent 
    (both semantics and pragmatics)
  \item \txx{form} includes phonology, syntax, orthography
  \end{itemize}
\item CxG grew out of generative semantics and cognitive linguistics,
  by researchers such as Charles Fillmore, Paul Kay and George Lakoff
\item Instead of language as a grammar+lexicon, think of it as a
  structured network of families of constructions
\end{itemize}


\myslide{Construction Grammars}
\begin{itemize}
\item \txx{Sign-based-Construction Grammar} (Berkeley Construction Grammar)
  \\  unification-based framework (with computational implementation)
\item \txx{Goldbergian/Lakovian Construction Grammar}
  \\ psychologically plausible
\item \txx{Radical Construction Grammar}
  \\ syntactic categories, roles, and relations are not universal: 
  \\ they are not only language-specific, but also construction specific 
\item \txx{Embodied Construction Grammar}
  \\ relates constructions to embodiment and sensorimotor experience
\item \txx{Fluid Construction Grammar}
  \\ learns grammars from the environment (with computational implementation)
\end{itemize}


\myslide{Overview of Differences (SBCG vs HPSG)}

\begin{itemize}
\item Multiple Inheritance
\item Signs
\item Grammar rules form a hierarchy --- many more rules
\item Every tree node has its own phonology
\item Many principles become constraints on 
grammar rules
\item The definition of well-formedness is 
simplified
\end{itemize}




\section{Wrap Up}

\myslide{Big picture: Our model}
\begin{center}
  \LARGE HPSG \\
  \Large Head-driven Phrase Structure Grammar  
\end{center}

\begin{itemize}
\item Describes a set of strings
\item Associates semantic representations (and
trees) with well-formed strings
\begin{itemize}
\item Is stated in terms of declarative constraints
\item[\ldots]  which are order-independent
\item Locates most constraints `in the lexicon'
\item Is stated in a precise fashion
\end{itemize}
\end{itemize}


\myslide{Parts of our model}
\begin{itemize}
\item Type hierarchy (lexical types, other types)
\item Phrase structure rules
\item Lexical rules
\item Lexical entries
\item Grammatical principles
\item Initial symbol
\end{itemize}


\myslide{Universals in our model}
\begin{itemize}
%\item Case constraint 
\item SHAC
\item Binding theory
\item Head-complement/-specifier/-modifier
\item Head Feature Principle
\item Valence Principle
\item Semantic Compositionality Principle
\item \ldots 
\end{itemize}


% \myslide{Key characteristics}
% \begin{itemize}
% \item Surface-oriented
% \item Constraint-based
% \item Lexicalist
% \end{itemize}
% \myslide{Psycholinguistic Evidence}
% \begin{itemize}
% \item Evidence for left-to-right effects
%   \begin{itemize}
%   \item Disfluencies are sensitive to structure
%   \item Repeat rate of \eng{the} varies with position and complexity of the NP it introduces
%   \item More complex NPs produce more disfluency
%   \end{itemize}
% \item Evidence for top-down planning
%   \begin{itemize}
%   \item Agreement errors are more common with PP complements
% than sentential complements: errors like (2) are significantly
% more common than errors like (1).
% \begin{exe}
%   \ex \eng{*The claim that the wolves had raised the babies were rejected.
%   \ex \eng{*The claim about the newborn babies were rejected.
% \end{exe}
% \end{itemize}
% \end{itemize}


% \myslide{So what?}
% \begin{itemize}
% \item Speculation: Clauses are their own agreement domains, so people don't
% mistake an NP in a lower clause as a trigger for agreement

% \item Grammar plays a role in production.
% \item Partial grammatical information should be accessible by
% the production mechanism as needed.
% % \begin{itemize}
% % \item This argues against grammatical theories that involve
% % sequential derivations with fixed ordering.
% % \begin{itemize}
% \item Our theory of grammar has the requisite flexibility.
% \end{itemize}


% \myslide{Speakers know a great deal about individual words}
% \begin{itemize}\addtolength{\itemsep}{-1ex}
% \item Individual lexical items have many idiosyncrasies in
% where they can occur, and in where they tend to
% occur.
% \begin{itemize}
% \item For example, the verb \textit{behoove} occurs only with the
% subject it (and only in certain verb forms), and the
% verb \textit{beware} has only the base form.
% \item Transitive use of \textit{walk} is much rarer than the intransitive.
% \end{itemize}
% \item Lexical biases influence processing
% \item Wasow et al ran a production experiment to test
% whether ambiguity avoidance would influence
% speakers' choice between the following:
% \begin{exe}
%   \ex \eng{They gave Grant's letters to Lincoln to a museum.
%   \ex \eng{ They gave a museum Grant's letters to Lincoln.
% \end{exe}
% \item 
% Lexical bias of the verbs turned out to be a significant
% predictor of which form speakers used (and ambiguity
% avoidance turned out not to be).
% \end{itemize}

% % \myslide{Experimental Method}

% % LISTENER

% % SPEAKER

% % 1. Speaker silently reads a sentence:
% % A museum in Philadelphia received Grant's
% % letters to Lincoln from the foundation.

% % \myslide{Experimental Method, continued}
% % What did the
% % foundation do?

% % LISTENER

% % SPEAKER

% % 2. The sentence disappears from the screen.
% % The listener reads the next question from a list.

% % \myslide{Experimental Method, continued}
% % The foundation gave \ldots . the
% % museum, um, Grant's letter's
% % to Lincoln.

% % LISTENER

% % SPEAKER

% % 3. The speaker answers the listener's question.
% % The listener chooses the correct response on
% % a list (from two choices).

% % \myslide{Experimental Results on Local Ambiguity}
% % V-NP-PP bias
% % V-NP-NP bias

% % 100%
% % 75%
% % 50%
% % 25%
% % 0%

% % % NP NP

% % No potential local ambiguity

% % Potential local ambiguity

% % \myslide{Reverse ambiguity effect}
% % \begin{itemize}
% % \item Arnold, Wasow, Asudeh \& Alrenga 2004
% % Journal of Memory \& Language
% % \begin{itemize}
% % \item Re-ran the experiment with slightly better

% % methodology and found a stronger reverse
% % ambiguity effect.


% \myslide{A psychologically real grammar should be lexicalist}
% \begin{itemize}
% %\item Early generative grammars downplayed the lexicon.
% \item Now the importance of the lexicon is widely
% recognized.
% \item This aspect of grammar has been developed in greater
% detail in HPSG than in any other.
% \end{itemize}

% \myslide{Conclusion}
% \begin{itemize}
% \item Grammatical theory should inform and be informed
% by psycholinguistic experimentation.
% \item This has happened less than it should have.
% \item Existing psycholinguistic evidence favors a
% constraint-based, lexicalist approach (like ours).
% \end{itemize}

\myslide{Design Goals of our Model}

\begin{itemize}
\item Precise
\item Robust
\item Psychologically Plausible
\item Computationally Tractable
\end{itemize}


\myslide{Course overview}

\begin{itemize}
\item Survey of some phenomena central to 
syntactic theory
\item Introduction to the HPSG framework
\item Process over product: How to build a 
grammar fragment
\item Value of precise formulation (and of getting 
a computer to do the tedious part for you!)
\end{itemize}

\myslide{Reflection}

\begin{itemize}
\item What was the most surprising thing in this 
class?
\item What do you think is most likely wrong?
\item What do you think is the coolest result?
\item What do you think you’re most likely to 
remember?
\item How do you think this course will influence 
your work as a (computational) linguist?
\end{itemize}


\myslide{Overview}
\MyLogo{Sag, Wasow and Bender (2003) --- Chapter 16}

\begin{itemize}
\item Chapter 16 framework (same analyses, 
different underlying system)
\item General wrap up
\end{itemize}


\end{document}
%%% Local Variables: 
%%% coding: utf-8
%%% mode: latex
%%% TeX-PDF-mode: t
%%% TeX-engine: xetex
%%% End: 
