\documentclass[a4paper,landscape,headrule,footrule]{foils}
%%
%%% macros for Theories of Grammar
%%%
\usepackage{polyglossia}
\setdefaultlanguage{english}
%\setmainfont{TeX Gyre Pagella}


\newcommand{\logo}{~}
\newcommand{\header}[3]{%
\title{\vspace*{-2ex} \large HG4041 Theories of Grammar
\\[2ex] \Large  \emp{#2} \\ \emp{#3}}
\author{\blu{Francis Bond}   \\ 
\normalsize  \textbf{Division of Linguistics and Multilingual Studies}\\
\normalsize  \url{http://www3.ntu.edu.sg/home/fcbond/}\\
\normalsize  \texttt{bond@ieee.org}}
\MyLogo{HG4041 (2020)}
\renewcommand{\logo}{#2}
\hypersetup{
   pdfinfo={
     Author={Francis Bond},
     Title={#1: #2},
     Subject={HG4041: Theories of Grammar},
     Keywords={Syntax, Semantics, HPSG, Unification, Constructions},
     License={CC BY 4.0}
   }
 }

\date{#1
  \\ Location: LHN-TR+36}
}
\usepackage[hidelinks]{hyperref}




\usepackage{xcolor}
\usepackage{graphicx}
\newcommand{\blu}[1]{\textcolor{blue}{#1}}
\newcommand{\grn}[1]{\textcolor{green}{#1}}
\newcommand{\hide}[1]{\textcolor{white}{#1}}
\newcommand{\emp}[1]{\textcolor{red}{#1}}
\newcommand{\txx}[1]{\textbf{\textcolor{blue}{#1}}}
\newcommand{\lex}[1]{\textbf{\mtcitestyle{#1}}}

\usepackage{pifont}
\renewcommand{\labelitemi}{\textcolor{violet}{\ding{227}}}
\renewcommand{\labelitemii}{\textcolor{purple}{\ding{226}}}

\newcommand{\subhead}[1]{\noindent\textbf{#1}\\[5mm]}

\newcommand{\Bad}{\emp{\raisebox{0.15ex}{\ensuremath{\mathbf{\otimes}}}}}
\newcommand{\bad}[1]{*\eng{#1}}

\newcommand{\com}[1]{\hfill (\emp{#1})}%

\usepackage{relsize,xspace}
\newcommand{\into}{\ensuremath{\rightarrow}\xspace}
\newcommand{\tot}{\ensuremath{\leftrightarrow}\xspace}
\usepackage{url}
\newcommand{\lurl}[1]{\MyLogo{\url{#1}}}

\usepackage{mygb4e}
\newcommand{\lx}[1]{\textbf{\mtciteform{#1}}}
\newcommand{\ix}{\ex\slshape}
\let\eachwordone=\slshape



\newcommand{\ent}{\ensuremath{\Rightarrow}\xspace}
\newcommand{\ngv}{\ensuremath{\not\Rightarrow}\xspace}
%\usepackage{times}
%\usepackage{nttfoilhead}
\newcommand{\myslide}[1]{\foilhead[-25mm]{\raisebox{12mm}[0mm]{\emp{#1}}}\MyLogo{\logo}}
\newcommand{\myslider}[1]{\rotatefoilhead[-25mm]{\raisebox{12mm}[0mm]{\emp{#1}}}}
%\newcommand{\myslider}[1]{\rotatefoilhead{\raisebox{-8mm}{\emp{#1}}}}

\newcommand{\section}[1]{\myslide{}{\begin{center}\Huge \emp{#1}\end{center}}}



\usepackage[lyons,j,e,k]{mtg2e}
%\renewcommand{\mtcitestyle}[1]{\textcolor{-red!75!green!50}{\textsl{#1}}}
\renewcommand{\mtcitestyle}[1]{\textcolor{teal}{\textsl{#1}}}
\newcommand{\iz}[1]{\texttt{\textup{#1}}}
\newcommand{\gm}{\textsc}
\usepackage[normalem]{ulem}
\newcommand{\ul}{\uline}
\newcommand{\ull}{\uuline}
\newcommand{\wl}{\uwave}
\newcommand{\vs}{\ensuremath{\Leftrightarrow}~}
%%%
%%% Bibliography
%%%
\usepackage{natbib}
%\usepackage{url}
\usepackage{bibentry}


%%% From Tim
\newcommand{\WMngram}[1][]{$n$-gram#1\xspace}
\newcommand{\infers}{$\rightarrow$\xspace}

\usepackage[utf8]{inputenc}

\usepackage{rtrees,qtree}
\renewcommand{\lf}[1]{\br{#1}{}}
\usepackage{avm}
%\avmoptions{topleft,center}
\newcommand{\ft}[1]{\textsc{#1}}
\renewcommand{\val}[1]{\textit{#1}}
\newcommand{\typ}[1]{\textit{#1}}
\newcommand{\prd}[1]{\textbf{#1}}
\avmfont{\sc}
\avmvalfont{\it}
\avmsortfont{\smaller[2] \it}
\usepackage{multicol}
\newcommand{\blank}{\rule{3em}{1pt}\xspace}

% \usepackage{pst-node}
\newcommand{\OV}[1]{\ovalnode[linestyle=dotted,linecolor=red]{A}{#1}}
\newcommand{\OVB}[1]{\ovalnode[linestyle=dotted,linecolor=blue]{A}{#1}}

%%% From CSLI book
\newcommand{\mc}{\multicolumn}
\newcommand{\HD}{\textbf{H}\xspace}
\newcommand{\el}{\< \>}
\makeatother
\long\def\smalltree#1{\leavevmode{\def\\{\cr\noalign{\vskip12pt}}%
\def\mc##1##2{\multispan{##1}{\hfil##2\hfil}}%
\tabskip=1em%
\hbox{\vtop{\halign{&\hfil##\hfil\cr
#1\crcr}}}}}
\makeatletter

\newcommand{\A}{\noindent\textbf{A}: }
\newcommand{\Q}{\noindent\textbf{Q}: }
%\newcommand{\C}{\noindent\textbf{C}: }


%%% FIXME should be slant
%\usepackage{mathptmx}

\usepackage{bchart}
\avmfont{\sc}
\begin{document}
\avmfont{\it}\header{Lecture 7}{Grammar and Processing}{}
\maketitle

%\


\myslide{Overview}
\MyLogo{Sag, Wasow and Bender (2003) --- Chapter 9}
\begin{itemize}
\item Psycholinguistics and grammar design
  \begin{itemize}
  \item What grammar has to say
  \item What psychological evidence has to say
    \begin{itemize}
    \item Acquisition
    \item Production
    \item Comprehension
    \end{itemize}
  \item Universals
  \end{itemize}
\end{itemize}

\myslide{What does grammar have to do with psychology?}

Three ways it could be relevant:

\begin{itemize}
\item It provides insight into how children 
acquire language.
\item It provides insight into how speakers 
produce utterances. 
\item It provides insight into how listeners 
understand utterances. 
\end{itemize}

\myslide{Our model: Key characteristics}

\vfill
\begin{center}\Large
 \txx{Surface-oriented} \\[3ex]
 \txx{Constraint-based} \\[3ex]
 \txx{Lexicalist}
\end{center}
\vfill
\myslide{Chomsky’s position:}


\begin{itemize}
\item Grammar represents knowledge of language 
(\txx{competence}).
\item This is distinct from use of language (\txx{performance}).
\item We can draw a strong conclusion about language 
acquisition, namely, most grammatical knowledge is 
innate and task-specific.
\item Serious study of language use (production and 
comprehension) depends on having a well-developed 
theory of competence.
\end{itemize}

\myslide{Brief remarks on language acquisition}

\begin{itemize}
\item Chomsky’s nativism is very controversial
  \begin{itemize}
  \item It is based on the \txx{poverty of the stimulus} argument, and a 
    model of learning as hypothesis testing.  
  \item The environment may be more informative than he assumes.
  \item There may be more powerful learning methods than he 
    assumes.
  \end{itemize}
\item There has not been much work on language acquisition 
  using constraint-based lexicalist theories like ours;  \emp{but}
  \begin{itemize}
  \item Explicit formulation is a prerequisite for testing learning models
  \item Our feature structures could model richer context information.
  \end{itemize}
\item We’re neutral with respect to this controversy.
\end{itemize}

\myslide{Production and Grammar}


\begin{itemize}
\item Evidence for left-to-right effects
\item Evidence for top-down planning
\end{itemize}

\myslide{Disfluencies are sensitive to structure}

Repeat rate of \eng{the} varies with position and complexity of the NP it
introduces:


\begin{itemize}
\item More common in front of complex NPs
\item More common with prominent NPs
  \\ Topic $>$ Subject $>$ Direct Object $>$ Preposition Object
\end{itemize}

\begin{exe}
  \ex \eng{\ul{The the} book I told the student about on the train}
  \ex $>$ \eng{The book I told \ul{the the} student about on the train}
  \ex $>$ \eng{The book I told the student about on \ul{the the} train}
\end{exe}

\myslide{Production errors are sensitive to syntactic structure}

\begin{itemize}
\item Agreement errors are more common with PP complements than
  sentential complements: errors like (\ref{s:wolfb}) are
  significantly more common than errors like~(\ref{s:wolfa}).
  \begin{exe}
    \ex\label{s:wolfa} *\eng{The claim that the wolves had raised the
      babies were rejected.}
    
    \ex\label{s:wolfb}  $>$ *\eng{The claim about the
      newborn babies were rejected.}
  \end{exe}
\item Why?
  \begin{itemize}
  \item Speculation: Clauses are their own 
    agreement domains, so people don’t 
    mistake an NP in a lower clause as a trigger 
    for agreement
  \item Original work: Kay Bock (1980s).
  \end{itemize}
\end{itemize}

\myslide{Some high-level sentence planning is necessary, too}

\begin{exe}
\item\gll Ich habe   \ul{dem}  Mann,  den    ich \uwave{gesehen} habe geholfen. \\
I   have  the-dat man who-acc I   seen      have   helped\\
\trans    “I helped the man I saw”
\ex \gll Ich habe   \ul{den}   Mann,  dem    ich \uwave{geholfen} habe gesehen.\\
    I   have the-acc man  who-dat I    helped    have   seen.\\
\trans    “I saw the man I helped ”
\end{exe}
\begin{itemize}
\item The choice between \eng{dem} and \eng{den} depends on the choice of 
  verbs several words later.
\end{itemize}

\myslide{Interaction of top-down and left-to-right information}


\begin{itemize}
\item Grammar plays a role in production.
\item Partial grammatical information should be accessible by 
the production mechanism as needed.
\item This argues against grammatical theories that involve 
sequential derivations with fixed ordering.
\item Our theory of grammar has the requisite flexibility.
\end{itemize}

\myslide{Comprehension}


\begin{itemize}
\item Early work tried to use transformational grammar in 
modeling comprehension
\item \txx{The Derivational Theory of Complexity}:  The 
psychological complexity of a sentence increases 
with the number of transformations involved in its 
derivation.
\item Initial results seemed promising, but later work 
falsified the DTC.
\end{itemize}

\myslide{Some relevant quotes}


\begin{itemize}
\item “The results show a remarkable correlation of 
amount of memory and number of transformations” 
\\ \mbox{} \hfill ---  Chomsky (1968)
\item “[I]nvestigations of DTC…have generally proved 
equivocal.  This argues against the occurrence of 
grammatical derivations in the computations 
involved in sentence recognition”                              
\\ \mbox{} \hfill --- Fodor, Bever, \& Garrett (1974)
\end{itemize}

% \myslide{Another quote}


\begin{itemize}
\item “Experimental investigations of the 
psychological reality of linguistic structural 
descriptions have \ldots proved quite successful.”                                        
\\ \mbox{} \hfill --- Fodor, Bever, \& Garrett (1974)
\item In particular, they concluded that \txx{deep 
structures} and \txx{surface structures} were 
psychologically real, but the transformations 
relating them weren’t.
\end{itemize}

\myslide{Evidence for the Psychological Reality of Deep Structures}


\begin{itemize}
\item The proposed Deep Structure for (\ref{s:g2}) had three
  occurrences of \lex{detective}, while the proposed DS for (\ref{s:g1})
  had only two:
\begin{exe}
  \ex \label{s:g1} \eng{The governor asked the detective to prevent drinking.}
  \ex \label{s:g2} \eng{The  governor asked the detective to cease drinking.}
\end{exe}
\item In a recall experiment, \lex{detective} was significantly more 
effective in prompting people to remember (\ref{s:g2}) than (\ref{s:g1}). 
\end{itemize}

\myslide{Typical Problem Cases for the DTC}
\begin{exe}
  \ex \label{s:g11} \eng{Pat swam faster than Chris swam.}
  \ex \label{s:g12} \eng{Pat swam faster than Chris did.}
  \ex \label{s:g13} \eng{Pat swam faster than Chris.}
\end{exe}
\begin{itemize}
\item The DTC predicts that (\ref{s:g11}) should be less complex than
  (\ref{s:g12}) or (\ref{s:g13}), because (\ref{s:g12}) and
  (\ref{s:g13}) involve an extra deletion transformation.
\item In fact, subjects responded more slowly to (\ref{s:g11}) than to 
either (\ref{s:g12}) or (\ref{s:g13}).
\end{itemize}

\myslide{What should a psychologically real theory of grammar be like?}

\begin{itemize}
\item The \txx{deep structure} distinctions that are not evident 
on the surface should be represented.
\item The transformational operations relating deep and 
surface structures should not be part of the theory.
\item Our information-rich trees include all of the essential 
information in the traditional deep structures, but 
without the transformations.
\end{itemize}

\myslide{Jerry Fodor claims the human mind is \txx{modular}}

\begin{quote}
  “A module is \ldots{} an informationally encapsulated computational system --
  an inference-making mechanism whose access to background information
  is constrained by general features of cognitive architecture.”
\end{quote}
\mbox{} \hfill --- Fodor, 1985  

\begin{itemize}
\item A central issue in psycholinguistics over the past 20 years has
  been whether language is processed in a modular fashion.
\end{itemize}

\myslide{Tanenhaus’s Eye-Tracking Experiments}


\begin{itemize}
\item Participants wear a device on their heads that makes 
a videotape showing exactly what they’re looking at.
\item They listen to spoken instructions and carry out 
various tasks. 
\item They eye-tracking provides evidence of the 
cognitive activity of participants that can be 
correlated with the linguistic input. 
\end{itemize}

\myslide{Non-linguistic visual information affects lexical access}


\begin{itemize}
\item Participants’ gaze settled on a referent before the 
word was completed, unless the initial syllable of the 
word was consistent with more than one object.  
\item For example, participants’ gaze rested on the pencil 
after hearing \eng{Pick up the pencil}
more slowly when both a \textbf{p}encil and a \textbf{p}enny were 
present.
\end{itemize}

\myslide{Non-linguistic visual information affects syntactic processing}


\begin{itemize}
\item Eye movements showed that people hearing (\ref{s:apple}) often 
temporarily misinterpreted on the towel as the 
destination.
\begin{exe}
  \ex \label{s:apple} \eng{Put the apple on the towel in the box.}
\end{exe}
\item When \eng{on the towel} helped them choose between two 
apples, such misparses were significantly less 
frequent than when there was only one apple.
\end{itemize}

\myslide{General Conclusion of Eye-Tracking Studies}


\begin{itemize}
\item People use whatever information is available as 
soon as it is useful in interpreting utterances.
\item This argues against Fodorian modularity.
\item It argues for a model of language in which 
information is represented in a uniform, order-
independent fashion.
\end{itemize}

\myslide{Speakers know a great deal about individual words}

\begin{itemize}
\item Individual lexical items have many idiosyncrasies in 
where they can occur, and in where they tend to 
occur.  
\item For example, the verb \lex{behoove} occurs only with the 
subject \lex{it} (and only in certain verb forms), and the 
verb \lex{beware} has only the base form.
\item We also know that the transitive use of \lex{walk} is much 
rarer than the intransitive. 
\end{itemize}

\myslide{Different verbs favor different COMPS lists}
\begin{center}
  \begin{bchart}[max=100,step=25,unit=\%,scale=2.2]
    \bcbar[plain, text=tell]{100} \bcbar[plain, text=give]{85}
    \bcbar[plain, text=show]{85} \bcbar[plain, text=hand]{68}
    \bcbar[plain, text=fax]{56} \bcbar[plain, text=bring]{38}
    \bcbar[plain, text=send]{20} \bcbar[plain, text=sell]{10}
    \bcxlabel{\% V-NP-NP vs V-NP-PP (in the NYT)}
  \end{bchart}
\end{center}

\myslide{Lexical biases influence processing}
\begin{itemize}
\item Wasow et al. ran a production experiment to test 
whether ambiguity avoidance would influence 
speakers’ choice between (\ref{s:g111}) and (\ref{s:g112}): 
\begin{exe}
\ex \label{s:g111} \eng{They gave Grant’s letters to Lincoln to a museum.} \hfill NP-PP
\ex \label{s:g112} \eng{They gave a museum Grant’s letters to Lincoln.} \hfill NP-NP
\end{exe}
\item Ambiguity avoidance predicts that you should prefer (\ref{s:g112})
\item Lexical bias of the verbs turned out to be a significant 
predictor of which form speakers used (and ambiguity 
avoidance turned out not to be).
\end{itemize}

\myslide{Experimental Method}

\begin{itemize}
\item Speaker and Listener sit next to each other.  Speaker can see a screen.
\item Speaker silently reads a sentence shown on the screen \\ \eng{A
    museum in Philadelphia received Grant's letters to Lincoln from
    the foundation.}
\item The sentence disappears from the screen.
\item Listener asks a question:
\\ \eng{What did the foundation do?}
\item The speaker answers the listener's question.
\\ \eng{The foundation gave \ldots{}. the 
museum, um, Grant's letter's 
to Lincoln.}
\item The listener records which kind of response on 
a list (from two choices).
\end{itemize}

\myslide{Experimental Results on Local Ambiguity}

\begin{center}
  \begin{bchart}[max=100,step=25,unit=\%,scale=2.2]
    \bcbar[plain, text=V-NP-PP~bias, color=red!20]{70}
    \bclabel{\shortstack{No potential \\ local ambiguity}}
    \bcbar[plain, text=V-NP-NP~bias]{78}
    \smallskip
    \bcbar[plain, text=V-NP-PP~bias, color=red!20]{48}
    \bclabel{\shortstack{Potential\\ local ambiguity}}
    \bcbar[plain, text=V-NP-NP~bias]{68}
    \bcxlabel{\% V-NP-NP vs V-NP-PP}
  \end{bchart}
\end{center}

\begin{itemize}
\item Arnold, Wasow, Asudeh \& Alrenga (2004:
\textit{Journal of Memory \& Language})
re-ran the experiment with slightly better 
methodology and found an even stronger reverse 
ambiguity effect.
\end{itemize}

\myslide{A psychologically real grammar should be lexicalist}


\begin{itemize}
\item Early generative grammars downplayed the lexicon.
\item Now, however, the importance of the lexicon is widely 
recognized.
\item This aspect of grammar has been developed in greater 
detail in HPSG than in any other theory.
\item It would be easy to add frequency information to the
lexicon, though there is debate over the wisdom of 
doing so.
\item Frequency is currently recorded as part of the \txx{parse
    ranking model}s which select the most plausible out of all
  possible interpretations.
\end{itemize}

\myslide{Conclusion}


\begin{itemize}
\item Grammatical theory should inform and be informed 
by psycholinguistic experimentation.
\item This has happened less than it should have.
\item Existing psycholinguistic evidence favors a 
constraint-based, lexicalist approach (like HPSG, lFG, construction grammar).
\end{itemize}

\myslide{Universals?}
\MyLogo{\url{http://www.delph-in.net/matrix/}}

\begin{itemize}
\item \txx{Principles and Parameters (P\&P)}: attempts to relate multiple 
typological properties to single parameters (\emp{top-down}).
\item \txx{Grammar Matrix}: attempts to 
describe many languages in a consistent 
framework and then takes stock of common 
constraints (\emp{bottom-up}).
\begin{itemize}
\item Until we know more detail about more languages, we cannot test theories properly
\item So describing languages \emp{in precise detail} is very important
\item HPSG is fully in the tradition of language documentation
\item The \href{http://depts.washington.edu/uwcl/aggregation/}{The
    AGGREGATION Project}\footnote{Automatic Generation of Grammars for
    Endangered Languages from Glosses and Typological Information}
  attempts to automate the construction of grammar fragments from
  language descriptions, building on interlinear glossed text (IGT)
  and using the Grammar Matrix
\end{itemize}
\end{itemize}

\myslide{Universals?}
\MyLogo{These are candidate universal facts about language}
\begin{itemize}
\item Case constraint
\item SHAC
\item Binding theory
\item Head-complement/-specifier/-modifier
\item Head Feature Principle
\item Valence Principle
\item Semantic Compositionality Principle
\item \ldots{}
\end{itemize}

\myslide{Overview}
\MyLogo{Sag, Wasow and Bender (2003) --- Chapter 9}
\begin{itemize}
\item Psycholinguistics and grammar design
  \begin{itemize}
  \item What grammar has to say
  \item What psychological evidence has to say
    \begin{itemize}
    \item Acquisition
    \item Production
    \item Comprehension
    \end{itemize}
  \item Universals
  \end{itemize}
\end{itemize}


\myslide{P1: Constant Rules}
\MyLogo{Based on  Chapter 9, Problem 1, Sag, Wasow and Bender (2003)}

The Singular Noun Lexical Rule, the Non-3rd-Singular Verb Lexical
Rule, and the Base Form Lexical Rule are all inflectional lexical
rules ({\it i-rule}) which have no effect on the shape (i.e.\ the
phonology) of the word.

\begin{itemize}
\item[A.] Explain why we need these rules anyway.  
\item[B.] Each of these rules have lexical exceptions, in the sense
that there are lexemes that idiosyncratically don't undergo them.
Thus, there are some nouns without singular forms, verbs without
non-third-person singular present tense forms, and verbs without base
forms.  List any you can think of. The verb part is \emp{HARD}
\end{itemize}

\color{white}{
\begin{itemize}\addtolength{\itemsep}{-1em}
\item[*] \textit{kudos,  outskirts, clothes, shenanigans, suds, \ldots, *scissors, *trousers, \ldots}
\item[*] \textit{behoove} (only \textit{it behooves}), \textit{beware} (only base) 
\item[*] modals: \textit{must, should, \ldots} \textit{* You must should go}
\end{itemize}}

%\input{lec-07-questions}

\end{document}


%%% Local Variables: 
%%% coding: utf-8
%%% mode: latex
%%% TeX-PDF-mode: t
%%% TeX-engine: xetex
%%% End: 
