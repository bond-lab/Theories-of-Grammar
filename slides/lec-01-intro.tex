\documentclass[a4paper,landscape,headrule,footrule]{foils}
%\usepackage{times}
%\usepackage{nttfoilhead}
%\newcommand{\myslide}[1]{\foilhead[-25mm]{\raisebox{12mm}[0mm]{\emp{#1}}}}
%\newcommand{\myslider}[1]{\rotatefoilhead[-25mm]{\raisebox{12mm}[0mm]{\emp{#1}}}}
%\newcommand{\myslider}[1]{\rotatefoilhead{\raisebox{-8mm}{\emp{#1}}}}

%%
%%% macros for Theories of Grammar
%%%
\usepackage{polyglossia}
\setdefaultlanguage{english}
%\setmainfont{TeX Gyre Pagella}


\newcommand{\logo}{~}
\newcommand{\header}[3]{%
\title{\vspace*{-2ex} \large HG4041 Theories of Grammar
\\[2ex] \Large  \emp{#2} \\ \emp{#3}}
\author{\blu{Francis Bond}   \\ 
\normalsize  \textbf{Division of Linguistics and Multilingual Studies}\\
\normalsize  \url{http://www3.ntu.edu.sg/home/fcbond/}\\
\normalsize  \texttt{bond@ieee.org}}
\MyLogo{HG4041 (2020)}
\renewcommand{\logo}{#2}
\hypersetup{
   pdfinfo={
     Author={Francis Bond},
     Title={#1: #2},
     Subject={HG4041: Theories of Grammar},
     Keywords={Syntax, Semantics, HPSG, Unification, Constructions},
     License={CC BY 4.0}
   }
 }

\date{#1
  \\ Location: LHN-TR+36}
}
\usepackage[hidelinks]{hyperref}




\usepackage{xcolor}
\usepackage{graphicx}
\newcommand{\blu}[1]{\textcolor{blue}{#1}}
\newcommand{\grn}[1]{\textcolor{green}{#1}}
\newcommand{\hide}[1]{\textcolor{white}{#1}}
\newcommand{\emp}[1]{\textcolor{red}{#1}}
\newcommand{\txx}[1]{\textbf{\textcolor{blue}{#1}}}
\newcommand{\lex}[1]{\textbf{\mtcitestyle{#1}}}

\usepackage{pifont}
\renewcommand{\labelitemi}{\textcolor{violet}{\ding{227}}}
\renewcommand{\labelitemii}{\textcolor{purple}{\ding{226}}}

\newcommand{\subhead}[1]{\noindent\textbf{#1}\\[5mm]}

\newcommand{\Bad}{\emp{\raisebox{0.15ex}{\ensuremath{\mathbf{\otimes}}}}}
\newcommand{\bad}[1]{*\eng{#1}}

\newcommand{\com}[1]{\hfill (\emp{#1})}%

\usepackage{relsize,xspace}
\newcommand{\into}{\ensuremath{\rightarrow}\xspace}
\newcommand{\tot}{\ensuremath{\leftrightarrow}\xspace}
\usepackage{url}
\newcommand{\lurl}[1]{\MyLogo{\url{#1}}}

\usepackage{mygb4e}
\newcommand{\lx}[1]{\textbf{\mtciteform{#1}}}
\newcommand{\ix}{\ex\slshape}
\let\eachwordone=\slshape



\newcommand{\ent}{\ensuremath{\Rightarrow}\xspace}
\newcommand{\ngv}{\ensuremath{\not\Rightarrow}\xspace}
%\usepackage{times}
%\usepackage{nttfoilhead}
\newcommand{\myslide}[1]{\foilhead[-25mm]{\raisebox{12mm}[0mm]{\emp{#1}}}\MyLogo{\logo}}
\newcommand{\myslider}[1]{\rotatefoilhead[-25mm]{\raisebox{12mm}[0mm]{\emp{#1}}}}
%\newcommand{\myslider}[1]{\rotatefoilhead{\raisebox{-8mm}{\emp{#1}}}}

\newcommand{\section}[1]{\myslide{}{\begin{center}\Huge \emp{#1}\end{center}}}



\usepackage[lyons,j,e,k]{mtg2e}
%\renewcommand{\mtcitestyle}[1]{\textcolor{-red!75!green!50}{\textsl{#1}}}
\renewcommand{\mtcitestyle}[1]{\textcolor{teal}{\textsl{#1}}}
\newcommand{\iz}[1]{\texttt{\textup{#1}}}
\newcommand{\gm}{\textsc}
\usepackage[normalem]{ulem}
\newcommand{\ul}{\uline}
\newcommand{\ull}{\uuline}
\newcommand{\wl}{\uwave}
\newcommand{\vs}{\ensuremath{\Leftrightarrow}~}
%%%
%%% Bibliography
%%%
\usepackage{natbib}
%\usepackage{url}
\usepackage{bibentry}


%%% From Tim
\newcommand{\WMngram}[1][]{$n$-gram#1\xspace}
\newcommand{\infers}{$\rightarrow$\xspace}

\usepackage[utf8]{inputenc}

\usepackage{rtrees,qtree}
\renewcommand{\lf}[1]{\br{#1}{}}
\usepackage{avm}
%\avmoptions{topleft,center}
\newcommand{\ft}[1]{\textsc{#1}}
\renewcommand{\val}[1]{\textit{#1}}
\newcommand{\typ}[1]{\textit{#1}}
\newcommand{\prd}[1]{\textbf{#1}}
\avmfont{\sc}
\avmvalfont{\it}
\avmsortfont{\smaller[2] \it}
\usepackage{multicol}
\newcommand{\blank}{\rule{3em}{1pt}\xspace}

% \usepackage{pst-node}
\newcommand{\OV}[1]{\ovalnode[linestyle=dotted,linecolor=red]{A}{#1}}
\newcommand{\OVB}[1]{\ovalnode[linestyle=dotted,linecolor=blue]{A}{#1}}

%%% From CSLI book
\newcommand{\mc}{\multicolumn}
\newcommand{\HD}{\textbf{H}\xspace}
\newcommand{\el}{\< \>}
\makeatother
\long\def\smalltree#1{\leavevmode{\def\\{\cr\noalign{\vskip12pt}}%
\def\mc##1##2{\multispan{##1}{\hfil##2\hfil}}%
\tabskip=1em%
\hbox{\vtop{\halign{&\hfil##\hfil\cr
#1\crcr}}}}}
\makeatletter

\newcommand{\A}{\noindent\textbf{A}: }
\newcommand{\Q}{\noindent\textbf{Q}: }
%\newcommand{\C}{\noindent\textbf{C}: }


\usepackage{fontspec}

\setmainfont{Noto Sans}
\newfontfamily\cjkfont{Noto Sans CJK SC}[Scale=0.9]
\newcommand{\zh}[1]{{\cjkfont #1}}


\begin{document}
\header{Lecture 1}{Introduction, Organization}{Morphology and Syntax}\maketitle

%\include{schedule}

\myslide{Overview}

\begin{itemize}
\item Syllabus; Administrivia
\item Prescriptive/descriptive grammar;
Competence/performance
\item Some history
\item Why study syntax?
\item Morphology
\end{itemize}

\myslide{Administrivia}
\begin{description}
\item [Coordinator]  Francis \ul{Bond} 
\url{<bond@ieee.org>}
\item All other details on the web page

% \item [Lecturer]  Joanna \ul{Sio} Ut Seong 
% {\small \url{<neosome@gmail.com>} !\url{<ussio@ntu.edu.sg>}}
%\item [Seminar] Tuesday 14:30--18:30 (HSS SR3)
% \item[Office hours] ~
%   \begin{tabular}[t]{llll}
% \multicolumn{2}{c}{Francis}  & \multicolumn{2}{c}{Joanna}\\ \hline
% Tuesday &  15:30--16:30   &  &\\
% Wednesday &  15:30--16:30   & Wednesday & 13:30--15:30 
%   \end{tabular}
% \\[3ex] Or by appointment: email us.
% \end{description}
\end{description}

\myslide{100\% Continuous Assessment}


\begin{itemize}
\item Weekly Problems (50\%)
\item Mid-term (20\%)
\item Final (30\%)
\end{itemize}



% \myslide{Course Content}

% This course introduces basic corpus skills for linguists:
% \begin{itemize}
% \item Marking up extra information
% \item Selecting text
% \item The range of existing corpora
% \item How to build your own corpus
% \item Using corpora to test linguistic hypotheses
% \item Using corpora to train language tools
% \end{itemize}

\myslide{What do you learn?}

On completion of this module, students should be able to:
\begin{itemize}
\item Understand the basics of morphology
\item Recognize certain classes of syntactic phenomena
\item Build analyses of those phenomena in a precise framework
\item Apply the process of building a formalized analysis to test
  linguistic hypotheses
\item Know a little about different approaches to the study of syntax
\item Be able to present linguistics using LaTeX
\end{itemize}
    
%%%
%%% this changes each year, so keep separate
%%%
%\include{schedule}


\myslide{Textbook and Readings}

\begin{itemize}
\item Textbooks
  \begin{itemize}
  \item Sag, Wasow and Bender 2003 \textit{Syntactic Theory: A Formal Introduction} 2nd ed. CSLI (\textbf{required})
  % \item Andrew Carnie 2006 \textit{Syntax: A Generative Introduction} 2nd ed. Blackwell (\textbf{recommended})
 \item Panocová, Renáta \href{https://unibook.upjs.sk/img/cms/2021/FF/basic-concepts-of-morphology-1.pdf}{\textit{Basic Concepts of Morphology I}} Univerzita Pavla Jozefa Šafárika  (\textbf{recommended})
  \end{itemize}

  
  \begin{itemize}
  \item You should read all chapters assigned before class.
  \item Ideas from the book will be pursued in parallel with the
    topics given above.
  \end{itemize}
\end{itemize}

% Other References

% Biber, D., S. Conrad \& R. Reppen, Corpus Linguistics: Investigating Language Structure and Use. Cambridge University Press, 1998.

% Kennedy, G. An Introduction to Corpus Linguistics. Longman, 1998.

% McEnery, Tony et al. Corpus-Based Language Studies: An Advanced Resource Book. Routledge, 2006.

% McEnery, Tony and Andrew Wilson Corpus Linguistics 2nd ed, Edinburgh UP, 2001

% Sinclair, John. Corpus Concordance Collocation. Oxford: Oxford UP, 1991


\myslide{Student Responsibilities}

By remaining in this class, the student agrees to:
\begin{enumerate}
\item  Make a good-faith effort to learn and enjoy the material.
\item  Read assigned texts and participate in class discussions and activities.
\item Submit assignments on time.
\item Attend class at all times, barring special circumstances (see below).
\item Get help early: approach us when you first have trouble understanding a concept or homework problem rather than complaining about a lack of understanding afterward.
\item Treat other students with respect in all class-related activities, including on-line discussions.
\end{enumerate}
\myslide{Attendance}
\begin{enumerate}
\item You are expected to attend all classes.
\item Be on time - lateness is disruptive to your own and others' learning.
\item Valid reasons for missing class include the following:
\begin{enumerate}
\item A medical emergency (including mental health emergencies)
\item A family emergency (death, birth, natural disaster,  \ldots{})
\item An important event (tournament, job interview, \ldots{})
\end{enumerate}
\item There will be significant material covered in class that is not in your readings.  You cannot expect to do well without coming to class.
\item If you miss a class, it is your responsibility to get the notes, any handouts you missed, schedule changes, etc. from a classmate.
\end{enumerate}

\myslide{Remediation and Academic Integrity}
\begin{enumerate}
\item No late work will be accepted, except in the case of a documented excuse.
\item For planned, justified, absences on class days or days on which assignments are due, advance notice must be provided.
\item Cheating will not be tolerated. Violations, including plagiarism, will be seriously dealt with, and could result in \textbf{a failing grade for the entire course}.
\item For all other issues of academic integrity, refer to the University Honour Code
\item As always, use your common sense and conscience.
\end{enumerate}


\myslide{The winning strategy}

\begin{itemize}
\item Read the books before class (and after again, if necessary)
\item Work together: make study groups
\item Homework: Discuss as much as you want, write up your own answers
\item Exams: No discussion
\item Ask questions \ldots  early and often!
\end{itemize}

\myslide{Resources}
\begin{itemize}
\item Glossary at back of textbook
\item Grammar summaries and Appendix A
\item Answers to exercises at back of book
\item Each other, grad-students,  office hours, \ldots
\item Online:
  \begin{itemize}
%  \item HPSG at Stanford: \url{http://hpsg.stanford.edu/}
  \item English Resource Grammar: \href{https://delph-in.github.io/delphin-viz/demo/}{delph-in.github.io/delphin-viz/demo/}
  \item Wikipedia has good summaries and many links
  \end{itemize}
\end{itemize}

\myslide{Layers of Linguistic Analysis}
\begin{enumerate}\addtolength{\itemsep}{-0.75ex}
\item Phonetics \& Phonology
\item \txx{Morphology}
\item \txx{Syntax} (Grammar)
\item Semantics 
\item Pragmatics
\item Stylistics
\end{enumerate}


\myslide{Two Conceptions of Grammar}
\begin{itemize}
\item
PRESCRIPTIVE
\begin{itemize}
\item Rules against certain
usages. Few if any
rules for what is
allowed
\item Proscribed forms
generally in use
\item Explicitly normative
enterprise
\end{itemize}
\item DESCRIPTIVE
\begin{itemize}
\item Rules characterizing
what people do say
\item Goal to characterize all
and only what speakers
find acceptable
\item Tries to be scientific
\end{itemize}
\end{itemize}

\myslide{Uses of Grammar}

\begin{itemize}
\item PRESCRIPTIVE
  \begin{itemize}
  \item Identify speaker’s
    socioeconomic class
    \& education level
  \item Identify level of
    formality of a
    particular usage
  \item Standardize language for smoother communication
  \end{itemize}
\item DESCRIPTIVE
  \begin{itemize}
  \item Understand how
    people produce \&
    understand language
  \item Identify similarities
    \& differences across
    languages
  \item Development of
    language technologies
  \end{itemize}
\end{itemize}

\myslide{Prescriptive grammar}
\begin{itemize}
\item Examples of silly prescriptive rules?
\\ \hide{split infinitive, that/which, singular they}
\item Examples of useful prescriptive rules?
\\ \hide{write clearly and legibly, use short sentences, no expletives}
\item Some applications which might need to encode prescriptive rules?
\\ \hide{grammar checkers, text generation}
\end{itemize}

\myslide{Fill in the blanks:}
\textit{he/his, they/their}, or something else?


\begin{exe}
\ex \eng{Everyone insisted that \blank record was unblemished.}
\ex  \eng{Everyone drives \blank own car to work.}
\ex  \eng{Everyone was happy because \blank   passed the test.}
\ex  \eng{Everyone left the room, didn’t \blank?}
\ex  \eng{Everyone left early. \blank seemed happy to get home.}
\end{exe}

\myslide{Descriptive Grammar: an example}
\MyLogo{in the Generative Semantics Tradition}
\begin{exe}
\ex  \eng{F\blank yourself!}
\ex  \eng{Go f\blank yourself!}
\ex  \eng{F\blank you!}
\ex  *\eng{Go f\blank you!}
\end{exe}
\begin{LARGE}
  \begin{itemize}
  \item Who taught you this?
  \item How did you learn it?
  \end{itemize}
\end{LARGE}
\myslide{Kinds of Things We’ll Worry About}
\MyLogo{}
\begin{itemize}
\item Where to use reflexives (e.g. \eng{myself}) vs. ordinary
pronouns (\eng{I, me})
\item Agreement (e.g. \eng{We sing} vs. *\eng{We sings})
\item Word order (e.g. *\eng{Sing we})
\item Case (e.g. *\eng{Us sing})
\item Coordinate conjunction (e.g. \eng{We sing and dance})
\item How to form questions, imperatives, negatives, \ldots 
\item[\ldots] and much more
\end{itemize}

\myslide{Competence vs. Performance}
\begin{itemize}
\item The Distinction
\begin{itemize}
\item Competence - knowledge of language
\item Performance - how the knowledge is used
\end{itemize}
\item Examples
\begin{exe}
\ex  \eng{That Sandy left bothered me.}
\ex  \eng{That that Sandy left bothered me bothered Kim.}
\ex  \eng{That that that Sandy left bothered me bothered Kim bothered Jo.}
\ex  \eng{The horse raced past the barn fell.}
\end{exe}
\end{itemize}

\myslide{Competence v. Performance}

\begin{exe}
\ex  \eng{You are what you eat}
\ex  \eng{You are what what you eat eats, too}
\ex  \eng{You are what what what you eat eats eats, too}
\end{exe}

\myslide{Acceptability vs. grammaticality}

\begin{itemize}
\item A sentence is \txx{acceptable} if native speakers
say it sounds good.

\item A sentence is \txx{grammatical} (with respect to
a particular grammar) if the grammar
licenses it.

\item Linguists are sometimes sloppy about the
difference.

\item Some people argue that it should be modeled probabilistically
  rather than as a binary distinction
  \begin{itemize}
  \item It depends on individual speakers
  \item But we often want to model groups of speakers
  \item It is good to combine judgments with attested data
    \\ but language is infinite, so we may not find the example we need attested
  \end{itemize}
\end{itemize}

\myslide{Some History}
\begin{itemize}
\item Writings on grammar go back at least 3000 years
\item Until 200 years ago, almost all of it was prescriptive
\item Until 70 years ago, most linguistic work concerned sound systems (phonology),
word structure (morphology), and the historical relationships among
languages
\end{itemize}

\myslide{The Generative Revolution}
\begin{itemize}
\item Noam Chomsky’s work in the 1950s
radically changed linguistics, making
syntax central.
\item Chomsky has been a dominant figure
in linguistics ever since.
\item The theory we will develop (HPSG) is in the
tradition started by Chomsky, but
diverges from his work in many ways.
\end{itemize}

\myslide{Main Tenets of Generative Grammar}
\begin{itemize}
\item Grammars should be formulated precisely
and explicitly.
\item Languages are infinite, so grammars must be
tested against invented data, not just attested
examples.
\item The theory of grammar is a theory of human
linguistic abilities.
\end{itemize}

\myslide{What does a theory do?}
\begin{itemize}
\item Monolingual
\begin{itemize}
\item Model grammaticality/acceptability
\item Model relationships between sentences
(internal structure)
\end{itemize}
\item Multilingual
\begin{itemize}
\item Model relationships between languages
\item Capture generalizations about possible
languages
\end{itemize}
\end{itemize}

\myslide{Some of Chomsky’s Controversial Claims}
\begin{itemize}
\item The superficial diversity of human languages
masks their underlying similarity.
\item All languages are fundamentally alike
because linguistic knowledge is largely
innate.
\item The central problem for linguistics is
explaining how children can learn language
so quickly and easily.
\end{itemize}

\myslide{Family Tree of Generative Syntactic Theories}

\newcommand{\gr}[2]{#1  (#2)}
%\newcommand{\gr}[2]{\parbox[t][12ex][t]{10em}{#1\\(#2)}}
%\begin{tabular}[t]{c}#1\\(#2)\\[2ex]\end{tabular}\rule[-12ex]{0ex}{12ex}}}

%\rule[lift]{width}{height}
%\rule{0ex}{12ex}
  \begin{tree}\tiny
    \br{Early Transformational Grammar (1955-1964)}%
    {\br{Standard Theory (1964-1967)}{%
        \br{EST (1967-1977)}{%
          \br{\gr{REST}{1977-1981}}{
            \br{\gr{GB}{1981-1993}}{\br{\gr{MP}{1993-present}}{ }}}
          \br{\gr{GPSG}{1979-1985}}{%
            \br{\rnode{HPSG}{\gr{HPSG}{1986-present}}}{
            \br{\gr{Sign-based CxG}{2001-present}}{ }}}
          \br{\gr{Realistic TG}{1978-1980}}{%
             \br{\rnode{LFG}{\gr{LFG}{1980-present}}}{
            }}}} {\br {\rnode{GS}{Generative Semantics (1966-1975)}}
        {\br{Relational Grammar (1974-present)}{\br{Arc Pair
              Grammar (1980)}{}}}}}
    \ncline[linestyle=dashed]{-}{LFG}{HPSG} \nccurve[linestyle=dashed,
    angleA=0, angleB=180]{-}{LFG}{GS}
  \end{tree}
\vspace*{-2em}
  \begin{small}
    \begin{itemize}
    \item Many Other Theories
      \begin{itemize}
      \item Dependency Grammar  (links words not phrases)
      \item Combinatory Categorical Grammar (allows multiple derivations)
      \item Tree Adjoining Grammar (links subtrees)
      \item Functional Grammar (considers function to be central)
        \begin{itemize}
        \item Systemic Functional Grammar
        \item Role and Reference Grammar
        \end{itemize}
      \item Biosemiotics (how living organisms produce, interpret, and exchange signs and meanings)
      \end{itemize}
    \end{itemize}
  \end{small}



\myslide{Which theory is best?}
\begin{itemize}
\item Most theories are better at explaining some things than others
  \begin{itemize}
  \item HPSG is good at modelling structure and (some) meaning
  \item Minimalism is good at modelling similarities
  \item Dependency grammar is easy to implement (and  good for case marked languages)
  \item Functional grammar is good at describing why we choose to use certain constructions
  \item Conversation Analysis is good at modelling turn taking and social interaction
  \end{itemize}
\item I teach HPSG because I know it well
  \begin{itemize}
  \item I know HPSG because I sat next to Tsunkeo Nakazawa at NTT
  \item I think it is a good model of syntax and semantics
  \item I don't think it is the only valid way of studying language
  \end{itemize}
\item The general approach to analysing language should be transferable to any theory
\end{itemize}


\myslide{What is Morphology}

\begin{itemize}
\item \txx{Morphology} is the study of form and structure.
\item In linguistics, it generally refers to the study of form and
  structure of words.\\[2ex]
  \begin{tabular}{ll}
    \eng{horses} & horse-s \\
    \eng{talked} & talk-ed \\
    \eng{unhappiness} &  un-happy-ness \\
    \eng{went} & go-ed \\
    \eng{yes} & yes \\
    \eng{talk}$_N$ &  \eng{talk}$_V$ \\
    \cs{psa}  & pes+I \\ 
  \end{tabular}
\end{itemize}

\myslide{Words and morphemes}
  There are two main usages of the term \txx{word}:
  \begin{enumerate}
    \item Surface form (spoken or written representation)
    \item Abstract form (lemma or dictionary entry, e.g. bare infinitives in English; nominative singular in Latin)
  \end{enumerate}
  \medskip
  The class of forms representing a word in different contexts is called a \txx{lexeme}.
  \begin{center}
    \lex{sing} $=$ \{\eng{sing}, \eng{sings}, \eng{sang}, \eng{sung}, \eng{singing}\}
  \end{center}  

% 6
\myslide{A definition of words?}
  \begin{itemize}
  \item \txx{Words} can be described as units of language (sequences of sounds or signs) that function as meaning bearers. But this is a fuzzy notion, e.g.:
    \begin{itemize}
    \item \eng{sang} expresses both ``singing'' and past tense.
    \item Is \eng{more or less} ``roughly'' one word, or are there three words?
    \end{itemize}
  \end{itemize}
  \medskip
  A structuralist solution: \txx{morphemes}.
%% end



% 8
\myslide{Morphemes and Morphological analysis}
\begin{itemize}
\item \txx{Morphemes} 
  \begin{itemize}
    \item \txx{Morphemes} are minimal meaning-bearing units.\\
          Example: \eng{talked} contains two morphemes: \eng{talk} and \eng{-ed} (past).
    \item Form–function pairs (sound/sign–meaning): basic units of morphology.
    \item The realisations of morphemes are called \txx{morphs}.\\
          Example: English plural morpheme \txx{[number pl]} has allomorphs \eng{-s}, \eng{-es}, \eng{-en}, \eng{$\emptyset$}: \eng{boy-s}, \eng{box-es}, \eng{ox-en}, \eng{sheep}.
    \item These different realisations of the same morpheme are called \txx{allomorphs}.
  \end{itemize}
\item \txx{Morphological analysis}\\
  \begin{itemize}
    \item Segmentation of expressions into basic units (mostly starting from word-level).
    \item Classification of these basic units according to function.
    \end{itemize}
\end{itemize}


\myslide{A language:}
\vfill
\begin{center}
  \begin{tabular}{c}
    11--112 \txx{phonemes} \\
    $\downarrow$ \\
    4{,}000--10{,}000 \txx{morphemes} \\
    $\downarrow$ \\
    An infinite number of \txx{sentences}
  \end{tabular}
\end{center}

\myslide{Types of morphemes}
  \begin{itemize}
  \item \txx{Free Morphemes}
  \begin{itemize}
    \item \txx{Free morphemes} can occur independently. Common in Chinese and English.\\
          Examples: \eng{boy}, \eng{sing}, \zh{狗} \eng[dog]{gǒu}.
  \end{itemize}
  \medskip
  \item \txx{Bound Morphemes}
  \begin{itemize}
    \item \txx{Bound morphemes} must be attached to another morpheme and cannot be used independently.\\
          Example: \txx{[number pl]} \eng{-s} $\rightarrow$ \eng{boys}.
    \item Typical bound morphemes are: \txx{affixes} (\eng{boy+s}, \eng{talk+ed}); \txx{clitics} (French: \eng{je ne sais pas}, \eng{je} and \eng{ne} cannot occur without a verb); \txx{roots} (Spanish \eng{habl-} needs an ending indicating person, number, mood, etc.).
    \end{itemize}
  \end{itemize}
%% end

% 10
  \myslide{Formatives and pseudo-morphemes}
  \txx{Morphemes} are form–meaning pairs, but not all segmentable forms have an identifiable meaning.
  \begin{itemize}
  \item \txx{Formatives}
    \begin{itemize}
    \item Forms without identifiable meaning.\\
      Example: linking elements in German compounds: \eng{Geburt+s+tag} (birthday), \eng{Schwan+en+hals} (swan neck).
    \end{itemize}
  \item \txx{Pseudo-morphemes / cranberry morphemes}
    \begin{itemize}
    \item Special cases of formatives: segmentable parts of a complex word without independent meaning.\\
      Examples: \eng{cran+berry}, \eng{rasp+berry}; \eng{re+ceive}, \eng{con+ceive}.
    \end{itemize}
  \end{itemize}


% 11
\myslide{What is morphology? (follow up)}
  \txx{Morphology} can refer to three different things:
  \begin{enumerate}
    \item Description of the behaviour of morphemes and how they are combined.
    \item Derivational, inflectional and compositional processes of word formation occurring in a specific language.\\
          Example: ``German has a richer morphology than English.''
    \item Description of such word-formation processes (i.e., the theory/grammar of a language’s morphology).
    \end{enumerate}

  %     We distinguish:
  % \begin{itemize}
  %   \item \txx{Word forming}:
  %     \begin{itemize}
  %       \item \txx{Derivational morphology}
  %       \item \txx{Compounding}
  %     \end{itemize}
  %   \item \txx{Inflection}
  % \end{itemize}
%% end

% % 12
% \myslide{Root, base and stem}

% \begin{description}[leftmargin=2.5em, style=nextline]
%     \item[Root:] an unanalysable form expressing the basic lexical content of a word; also ``what is left when all affixes are stripped.''
%     \item[Stem:] consists of at least a root; may contain derivational affix(es). In inflectional morphology, often \eng{root + thematic vowel}.
%     \item[Base:] a form to which an affix may be added; may be simplex (root) or complex (root + affixes).
%   \end{description}
% %% end

% % 13
% \myslide{Areas of morphology}
%   We distinguish:
%   \begin{itemize}
%     \item \txx{Word forming}:\\[0.25em]
%       \begin{itemize}
%         \item \txx{Derivational morphology}
%         \item \txx{Compounding}
%       \end{itemize}
%     \item \txx{Inflection}
%   \end{itemize}
% %% end

% 16
\myslide{Inflectional Morphology}
  \begin{itemize}
    \item \txx{Inflection} is required by syntactic criteria (e.g., an English verb must have tense).
    \item It marks grammatical (\txx{morphosyntactic}) distinctions:
      \begin{description}
        \item[Conjugation (verbal):] person, number, gender; tense, aspect, mood; agreement
        \item[Declension (nominal):] case, number, gender, degree, definiteness
      \end{description}
    \item Meaning of the general concept is generally not changed, though \eng{when}, \eng{who/what}, and sometimes \eng{where/how/whether} may be specified by inflectional morphemes.
    \item Some people consider bound and free inflectional morphemes:
      \begin{itemize}
      \item \eng{go [TENSE past] : went} 
      \item \eng{go [TENSE future] : will go}
      \end{itemize}

  \end{itemize}
%% end

% 17
\myslide{Inflection — paradigm}
  \begin{quote}\small
  ``A set of forms having the same root/stem, one of which must be selected in a certain syntactic environment'' (based on Crystal 1997:277; Payne 1997:26).
  \end{quote}
  \medskip
  For instance, German conjugation:
  \medskip
  \begin{center}
  \begin{tabular}{llllll}
     \textbf{present} &  &  & \textbf{past} &  & \\
 &     \textbf{singular} & \textbf{plural} & & \textbf{singular} & \textbf{plural} \\
    1. & \eng{dehn-e} & \eng{dehn-te}         &   1 &  \eng{dehn-en} & \eng{dehn-te-n} \\
    2. & \eng{dehn-st} & \eng{dehn-te-st}   &   2. & \eng{dehn-t} & \eng{dehn-te-t} \\
    3. & \eng{dehn-t} & \eng{dehn-te}       &   3. & \eng{dehn-en} & \eng{dehn-te-n} \\
  \end{tabular}
  \end{center}
%% end

% 18
\myslide{Paradigm — Czech noun first declension \eng[dog]{pes}}
\bigskip
\begin{center}
  \begin{tabular}{lll}
    \textbf{case} & \textbf{singular} & \textbf{plural} \\ \hline
    NOM & \eng{pes} & \eng{psi} \\
    GEN & \eng{psa} & \eng{psů} \\
    DAT & \eng{psoive, psu} & \eng{psům} \\
    ACC & \eng{psa} & \eng{psy} \\
    VOC & \eng{pse} & \eng{psi} \\
    LOC & \eng{psovi, psu} & \eng{psech} \\
    INST & \eng{psem} & \eng{psy} \\
  \end{tabular}
  \end{center}

  \begin{itemize}
  \item \txx{syncretism}: the same form expresses different feature combinations.\\
    \eng{psa} is both \txx{genitive} and \txx{accusative}
    \item \txx{exponence}: relation between form and function is $m\!:\!n$.
      \begin{itemize}
        \item \txx{multi-exponence (cumulation)}: one form expresses several functions.\\
              \eng{-ů} expresses both \txx{genitive} and \txx{plural}.
      \end{itemize}
  \end{itemize}

    
% 14
\myslide{Derivational Morphology}
  \begin{itemize}
    \item Builds complex words by combining bound and free morphemes.
    \item \txx{Derivational} operations are optional (not required by syntactic criteria).
    \item They may change:
      \begin{enumerate}
        \item \txx{semantics}, e.g. \eng{clear} $\rightarrow$ \eng{un+clear} = \eng{unclear}
        \item \txx{syntactic category}, e.g. \eng{derive}$_{\mathrm{V}}$ + \eng{ation}$_{\mathrm{N}}$ + \eng{al}$_{\mathrm{Adj}}$ = \eng{derivational}
        \item \txx{valency of a verb}, e.g. Havasupai \eng{qaw} ‘it breaks’ $\rightarrow$ \eng{t+qaw} ‘he breaks it’
        \item and more
      \end{enumerate}
  \end{itemize}
%% end

% 15
\myslide{Compounding}
  \begin{itemize}
    \item Builds complex words by juxtaposition of free morphemes.
    \item Examples: \eng{[sale]+s+[man]}, \eng{[dish]+[washer]}
    \item Productive compounding results in an infinite lexicon.
  \end{itemize}
  \medskip
  \[
    \left\{\begin{array}{l}
        English \\
        German \\
        Czech
      \end{array}
      \right\}
    \left\{\begin{array}{l}
        phonetics \\
        morphology \\
        syntax
      \end{array}
      \right\}
    \left\{\begin{array}{l}
        teacher \\
        researcher \\
        student
      \end{array}
      \right\}
  \]
 

\myslide{There are many other derivational patterns}

\begin{itemize}
\item \txx{Reduplication}: \eng{teeny-weeny}
\item \txx{Clipping}: \eng{ad[vertisment]}, \eng{[in]flu[enza]}
\item \txx{Acronym}: \eng{laser} ``Light Amplified by Stimulated Emission of Radiation'', \eng{hpsg} ``Head-Driven Phrase Structure Grammar'', \txx{UPOL}: ``U[niverzita] P[alackého] [v] Ol[omouci]'', \eng{東大} ``東京大学''
\item \txx{Blending}: \eng{brunch} ``breakfast + lunch'', \cs{srandista} ``sranda (fun, joke) + humorista (humorist)'' → ``joker, funny person''
\item \txx{Expletive infixing}: \eng{Abso-fucking-lutely} (but not \eng{Absolute-fucking-ly} or \eng{Abso-fucking-lute}!)
\end{itemize}
  
\myslide{Conclusions}

\begin{itemize}
\item Next week we will look into some simple theories of 
\end{itemize}
  
\myslide{Acknowledgments and References}

\begin{itemize}
\item Course design and slides borrow heavily from Emily Bender's course:
\textit{Linguistics 566: Introduction to Syntax for Computational Linguistics}
\\ \url{http://courses.washington.edu/ling566}
\item Morphology slides borrow from Antske Fokkens 
\item Thanks to Na-Rae Han for 
  inspiration for the student policies (from  \textit{LING 2050 Special Topics in Linguistics: Corpus linguistics}, U Penn; adapted).

\end{itemize}



\end{document}


%%% Local Variables: 
%%% coding: utf-8
%%% mode: latex
%%% TeX-PDF-mode: t
%%% TeX-engine: xetex
%%% End: 

