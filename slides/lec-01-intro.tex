\documentclass[a4paper,landscape,headrule,footrule]{foils}
%\usepackage{times}
%\usepackage{nttfoilhead}
%\newcommand{\myslide}[1]{\foilhead[-25mm]{\raisebox{12mm}[0mm]{\emp{#1}}}}
%\newcommand{\myslider}[1]{\rotatefoilhead[-25mm]{\raisebox{12mm}[0mm]{\emp{#1}}}}
%\newcommand{\myslider}[1]{\rotatefoilhead{\raisebox{-8mm}{\emp{#1}}}}

%%
%%% macros for Theories of Grammar
%%%
\usepackage{polyglossia}
\setdefaultlanguage{english}
%\setmainfont{TeX Gyre Pagella}


\newcommand{\logo}{~}
\newcommand{\header}[3]{%
\title{\vspace*{-2ex} \large HG4041 Theories of Grammar
\\[2ex] \Large  \emp{#2} \\ \emp{#3}}
\author{\blu{Francis Bond}   \\ 
\normalsize  \textbf{Division of Linguistics and Multilingual Studies}\\
\normalsize  \url{http://www3.ntu.edu.sg/home/fcbond/}\\
\normalsize  \texttt{bond@ieee.org}}
\MyLogo{HG4041 (2020)}
\renewcommand{\logo}{#2}
\hypersetup{
   pdfinfo={
     Author={Francis Bond},
     Title={#1: #2},
     Subject={HG4041: Theories of Grammar},
     Keywords={Syntax, Semantics, HPSG, Unification, Constructions},
     License={CC BY 4.0}
   }
 }

\date{#1
  \\ Location: LHN-TR+36}
}
\usepackage[hidelinks]{hyperref}




\usepackage{xcolor}
\usepackage{graphicx}
\newcommand{\blu}[1]{\textcolor{blue}{#1}}
\newcommand{\grn}[1]{\textcolor{green}{#1}}
\newcommand{\hide}[1]{\textcolor{white}{#1}}
\newcommand{\emp}[1]{\textcolor{red}{#1}}
\newcommand{\txx}[1]{\textbf{\textcolor{blue}{#1}}}
\newcommand{\lex}[1]{\textbf{\mtcitestyle{#1}}}

\usepackage{pifont}
\renewcommand{\labelitemi}{\textcolor{violet}{\ding{227}}}
\renewcommand{\labelitemii}{\textcolor{purple}{\ding{226}}}

\newcommand{\subhead}[1]{\noindent\textbf{#1}\\[5mm]}

\newcommand{\Bad}{\emp{\raisebox{0.15ex}{\ensuremath{\mathbf{\otimes}}}}}
\newcommand{\bad}[1]{*\eng{#1}}

\newcommand{\com}[1]{\hfill (\emp{#1})}%

\usepackage{relsize,xspace}
\newcommand{\into}{\ensuremath{\rightarrow}\xspace}
\newcommand{\tot}{\ensuremath{\leftrightarrow}\xspace}
\usepackage{url}
\newcommand{\lurl}[1]{\MyLogo{\url{#1}}}

\usepackage{mygb4e}
\newcommand{\lx}[1]{\textbf{\mtciteform{#1}}}
\newcommand{\ix}{\ex\slshape}
\let\eachwordone=\slshape



\newcommand{\ent}{\ensuremath{\Rightarrow}\xspace}
\newcommand{\ngv}{\ensuremath{\not\Rightarrow}\xspace}
%\usepackage{times}
%\usepackage{nttfoilhead}
\newcommand{\myslide}[1]{\foilhead[-25mm]{\raisebox{12mm}[0mm]{\emp{#1}}}\MyLogo{\logo}}
\newcommand{\myslider}[1]{\rotatefoilhead[-25mm]{\raisebox{12mm}[0mm]{\emp{#1}}}}
%\newcommand{\myslider}[1]{\rotatefoilhead{\raisebox{-8mm}{\emp{#1}}}}

\newcommand{\section}[1]{\myslide{}{\begin{center}\Huge \emp{#1}\end{center}}}



\usepackage[lyons,j,e,k]{mtg2e}
%\renewcommand{\mtcitestyle}[1]{\textcolor{-red!75!green!50}{\textsl{#1}}}
\renewcommand{\mtcitestyle}[1]{\textcolor{teal}{\textsl{#1}}}
\newcommand{\iz}[1]{\texttt{\textup{#1}}}
\newcommand{\gm}{\textsc}
\usepackage[normalem]{ulem}
\newcommand{\ul}{\uline}
\newcommand{\ull}{\uuline}
\newcommand{\wl}{\uwave}
\newcommand{\vs}{\ensuremath{\Leftrightarrow}~}
%%%
%%% Bibliography
%%%
\usepackage{natbib}
%\usepackage{url}
\usepackage{bibentry}


%%% From Tim
\newcommand{\WMngram}[1][]{$n$-gram#1\xspace}
\newcommand{\infers}{$\rightarrow$\xspace}

\usepackage[utf8]{inputenc}

\usepackage{rtrees,qtree}
\renewcommand{\lf}[1]{\br{#1}{}}
\usepackage{avm}
%\avmoptions{topleft,center}
\newcommand{\ft}[1]{\textsc{#1}}
\renewcommand{\val}[1]{\textit{#1}}
\newcommand{\typ}[1]{\textit{#1}}
\newcommand{\prd}[1]{\textbf{#1}}
\avmfont{\sc}
\avmvalfont{\it}
\avmsortfont{\smaller[2] \it}
\usepackage{multicol}
\newcommand{\blank}{\rule{3em}{1pt}\xspace}

% \usepackage{pst-node}
\newcommand{\OV}[1]{\ovalnode[linestyle=dotted,linecolor=red]{A}{#1}}
\newcommand{\OVB}[1]{\ovalnode[linestyle=dotted,linecolor=blue]{A}{#1}}

%%% From CSLI book
\newcommand{\mc}{\multicolumn}
\newcommand{\HD}{\textbf{H}\xspace}
\newcommand{\el}{\< \>}
\makeatother
\long\def\smalltree#1{\leavevmode{\def\\{\cr\noalign{\vskip12pt}}%
\def\mc##1##2{\multispan{##1}{\hfil##2\hfil}}%
\tabskip=1em%
\hbox{\vtop{\halign{&\hfil##\hfil\cr
#1\crcr}}}}}
\makeatletter

\newcommand{\A}{\noindent\textbf{A}: }
\newcommand{\Q}{\noindent\textbf{Q}: }
%\newcommand{\C}{\noindent\textbf{C}: }





\begin{document}
\header{Lecture 1}{Introduction, Organization}{First attempts at a theory of grammar}\maketitle

%\include{schedule}

\myslide{Overview}

\begin{itemize}
\item Syllabus; Administrivia
\item Prescriptive/descriptive grammar;
Competence/performance
\item Some history
\item Why study syntax?
\item Two theories that won’t work
\item Context Free Grammars
\item Central claims of CFG
\end{itemize}

\myslide{Administrivia}
\begin{description}
\item [Coordinator]  Francis \ul{Bond} 
{\small \url{<bond@ieee.org>} !\url{<fcbond@ntu.edu.sg>}}
\item All other details on the web page

% \item [Lecturer]  Joanna \ul{Sio} Ut Seong 
% {\small \url{<neosome@gmail.com>} !\url{<ussio@ntu.edu.sg>}}
%\item [Seminar] Tuesday 14:30--18:30 (HSS SR3)
% \item[Office hours] ~
%   \begin{tabular}[t]{llll}
% \multicolumn{2}{c}{Francis}  & \multicolumn{2}{c}{Joanna}\\ \hline
% Tuesday &  15:30--16:30   &  &\\
% Wednesday &  15:30--16:30   & Wednesday & 13:30--15:30 
%   \end{tabular}
% \\[3ex] Or by appointment: email us.
% \end{description}
\end{description}

\myslide{100\% Continuous Assessment}


\begin{itemize}
\item Mid-term (20\%)
\item Final (20\%)
\item Group Project: Presentation (20\%) 
  \begin{itemize}
  \item Give a precise and explicit model of some phenomenon not covered in class
  \item The talk must motivate the choice of phenomenon
  \item You need only cover  existing work
  \item In-class presentation with slides or handouts, not to exceed 17 minutes (12 presentation, 5 QA)
  \item You should choose something relevant to your final project if possible
  \end{itemize}
\newpage
\item Individual Project (40\%) 
  \begin{itemize}
  \item Give a precise and explicit model for some phenomenon not covered in class
    \begin{itemize}
    \item You should give attested and constructed examples
    \item You should clearly indicate what you can and can't explain
    \item It is expected that you can not explain everything perfectly
    \item Your model should make clear predictions
    \end{itemize}
  \item The paper must motivate the choice of phenomenon
  \item You should cover relevant existing work \emp{and add something new}
  \item LMS format, not to exceed 12 pages
  \end{itemize}
\end{itemize}


\myslide{Guidelines for Written Work in LMS}

\begin{itemize}
\item All assignments must follow the \textit{Guidelines to Submitting Written Work for the Division of Linguistics and Multilingual Studies}
  \begin{itemize}
  \item You can get it from:
    \url{http://www.soh.ntu.edu.sg/Programmes/linguistics/studentresources/Documents/Linguistics%20Assignment%20Guidelines.pdf}
      \begin{itemize}
      \item except: single spaced, double-sided
      \end{itemize}
  \item Useful advice on citation, transcription, formatting
  \item I also recommend my own \textit{(Computational) Linguistics Style Guide}:
 \url{www3.ntu.edu.sg/home/fcbond/data/ling-style.pdf}
  \item Proper citation is important 
    \\ --- failure to cite is plagiarism --- \textbf{fail subject}
 \\ See the NTU code of academic integrity 
 \\\url{http://www.ntu.edu.sg/ai/Pages/index.aspx}
  \end{itemize}
\end{itemize}


% \myslide{Course Content}

% This course introduces basic corpus skills for linguists:
% \begin{itemize}
% \item Marking up extra information
% \item Selecting text
% \item The range of existing corpora
% \item How to build your own corpus
% \item Using corpora to test linguistic hypotheses
% \item Using corpora to train language tools
% \end{itemize}

\myslide{What do you learn?}

On completion of this module, students should be able to:
\begin{itemize}
\item Recognize certain classes of syntactic phenomena
\item Build analyses of those phenomena in a precise framework
\item Apply the process of building a formalized analysis to test
  linguistic hypotheses
\item Know a little about different approaches to the study of syntax
\end{itemize}
    
%%%
%%% this changes each year, so keep separate
%%%
%\include{schedule}


\myslide{Textbook and Readings}

\begin{itemize}
\item Textbooks
  \begin{itemize}
  \item Sag, Wasow and Bender 2003 \textit{Syntactic Theory: A Formal Introduction} 2nd ed. CSLI (\textbf{required})
  % \item Andrew Carnie 2006 \textit{Syntax: A Generative Introduction} 2nd ed. Blackwell (\textbf{recommended})
  \end{itemize}
  \begin{itemize}
  \item You should read all chapters assigned before class.
  \item Ideas from the book will be pursued in parallel with the
    topics given above.
  \end{itemize}
\end{itemize}

% Other References

% Biber, D., S. Conrad \& R. Reppen, Corpus Linguistics: Investigating Language Structure and Use. Cambridge University Press, 1998.

% Kennedy, G. An Introduction to Corpus Linguistics. Longman, 1998.

% McEnery, Tony et al. Corpus-Based Language Studies: An Advanced Resource Book. Routledge, 2006.

% McEnery, Tony and Andrew Wilson Corpus Linguistics 2nd ed, Edinburgh UP, 2001

% Sinclair, John. Corpus Concordance Collocation. Oxford: Oxford UP, 1991


\myslide{Student Responsibilities}

By remaining in this class, the student agrees to:
\begin{enumerate}
\item  Make a good-faith effort to learn and enjoy the material.
\item  Read assigned texts and participate in class discussions and activities.
\item Submit assignments on time.
\item Attend class at all times, barring special circumstances (see below).
\item Get help early: approach us when you first have trouble understanding a concept or homework problem rather than complaining about a lack of understanding afterward.
\item Treat other students with respect in all class-related activities, including on-line discussions.
\end{enumerate}
\myslide{Attendance}
\begin{enumerate}
\item You are expected to attend all classes.
\item Be on time - lateness is disruptive to your own and others' learning.
\item Valid reasons for missing class include the following:
\begin{enumerate}
\item A medical emergency (including mental health emergencies)
\item A family emergency (death, birth, natural disaster, etc).
\end{enumerate}
You must provide documentation to me and the student office.
\item There will be significant material covered in class that is not in your readings.  You cannot expect to do well without coming to class.
\item If you miss a class, it is your responsibility to get the notes, any handouts you missed, schedule changes, etc. from a classmate.
\end{enumerate}

\myslide{Remediation and Academic Integrity}
\begin{enumerate}
\item No late work will be accepted, except in the case of a documented excuse.
\item For planned, justified, absences on class days or days on which assignments are due, advance notice must be provided.
\item Cheating will not be tolerated. Violations, including plagiarism, will be seriously dealt with, and could result in \textbf{a failing grade for the entire course}.
\item For all other issues of academic integrity, refer to the University Honour Code
\item As always, use your common sense and conscience.
\end{enumerate}


\myslide{The winning strategy}

\begin{itemize}
\item Read the books before class (and after again, if necessary)
\item Work together: make study groups
\item Homework: Discuss as much as you want, write up your own answers
\item Exams: No discussion
\item Ask questions \ldots  early and often!
\end{itemize}

\myslide{Resources}
\begin{itemize}
\item Glossary at back of textbook
\item Grammar summaries and Appendix A
\item Answers to exercises at back of book
\item Each other, grad-students,  office hours, \ldots
\item Online:
  \begin{itemize}
%  \item HPSG at Stanford: \url{http://hpsg.stanford.edu/}
  \item English Resource Grammar: \url{http://erg.delph-in.net/logon}
  \item Wikipedia page has lots of links
  \end{itemize}
\end{itemize}

\myslide{Two Conceptions of Grammar}
\begin{itemize}
\item
PRESCRIPTIVE
\begin{itemize}
\item Rules against certain
usages. Few if any
rules for what is
allowed
\item Proscribed forms
generally in use
\item Explicitly normative
enterprise
\end{itemize}
\item DESCRIPTIVE
\begin{itemize}
\item Rules characterizing
what people do say
\item Goal to characterize all
and only what speakers
find acceptable
\item Tries to be scientific
\end{itemize}
\end{itemize}

\myslide{Uses of Grammar}

\begin{itemize}
\item PRESCRIPTIVE
  \begin{itemize}
  \item Identify speaker’s
    socioeconomic class
    \& education level
  \item Identify level of
    formality of a
    particular usage
  \end{itemize}
\item DESCRIPTIVE
  \begin{itemize}
  \item Understand how
    people produce \&
    understand language
  \item Identify similarities
    \& differences across
    languages
  \item Development of
    language technologies
  \end{itemize}
\end{itemize}

\myslide{Prescriptive grammar}
\begin{itemize}
\item Examples of silly prescriptive rules?
\\ \hide{split infinitive, that/which, singular they}
\item Examples of useful prescriptive rules?
\\ \hide{write clearly and legibly, use short sentences, no expletives}
\item Some applications which might need to encode prescriptive rules?
\\ \hide{grammar checkers, text generation}
\end{itemize}

\myslide{Fill in the blanks:}
\textit{he/his, they/their}, or something else?


\begin{exe}
\ex \eng{Everyone insisted that \blank record was unblemished.}
\ex  \eng{Everyone drives \blank own car to work.}
\ex  \eng{Everyone was happy because \blank   passed the test.}
\ex  \eng{Everyone left the room, didn’t \blank?}
\ex  \eng{Everyone left early. \blank seemed happy to get home.}
\end{exe}

\myslide{Descriptive Grammar: an example}
\MyLogo{in the Generative Semantics Tradition}
\begin{exe}
\ex  \eng{F\blank yourself!}
\ex  \eng{Go f\blank yourself!}
\ex  \eng{F\blank you!}
\ex  *\eng{Go f\blank you!}
\end{exe}
\begin{LARGE}
  \begin{itemize}
  \item Who taught you this?
  \item How did you learn it?
  \end{itemize}
\end{LARGE}
\myslide{Kinds of Things We’ll Worry About}
\MyLogo{}
\begin{itemize}
\item Where to use reflexives (e.g. \eng{myself}) vs. ordinary
pronouns (\eng{I, me})
\item Agreement (e.g. \eng{We sing} vs. *\eng{We sings})
\item Word order (e.g. *\eng{Sing we})
\item Case (e.g. *\eng{Us sing})
\item Coordinate conjunction (e.g. \eng{We sing and dance})
\item How to form questions, imperatives, negatives, \ldots 
\item[\ldots] and much more
\end{itemize}

\myslide{Competence vs. Performance}
\begin{itemize}
\item The Distinction
\begin{itemize}
\item Competence - knowledge of language
\item Performance - how the knowledge is used
\end{itemize}
\item Examples
\begin{exe}
\ex  \eng{That Sandy left bothered me.}
\ex  \eng{That that Sandy left bothered me bothered Kim.}
\ex  \eng{That that that Sandy left bothered me bothered Kim bothered Jo.}
\ex  \eng{The horse raced past the barn fell.}
\end{exe}
\end{itemize}

\myslide{Competence v. Performance}

\begin{exe}
\ex  \eng{You are what you eat}
\ex  \eng{You are what what you eat eats, too}
\ex  \eng{You are what what what you eat eats eats, too}
\end{exe}

\myslide{Acceptability vs. grammaticality}

\begin{itemize}
\item A sentence is \txx{acceptable} if native speakers
say it sounds good.

\item A sentence is \txx{grammatical} (with respect to
a particular grammar) if the grammar
licenses it.

\item Linguists are sometimes sloppy about the
difference.

\item Some people argue that it should be modeled probabilistically
  rather than as a binary distinction
  \begin{itemize}
  \item It depends on individual speakers
  \item But we often want to model groups of speakers
  \item It is good to combine judgments with attested data
    \\ but language is infinite, so we may not find the example we need attested
  \end{itemize}
\end{itemize}

\myslide{Some History}
\begin{itemize}
\item Writings on grammar go back at least 3000 years
\item Until 200 years ago, almost all of it was prescriptive
\item Until 70 years ago, most linguistic work concerned sound systems (phonology),
word structure (morphology), and the historical relationships among
languages
\end{itemize}

\myslide{The Generative Revolution}
\begin{itemize}
\item Noam Chomsky’s work in the 1950s
radically changed linguistics, making
syntax central.
\item Chomsky has been the dominant figure
in linguistics ever since.
\item The theory we will develop (HPSG) is in the
tradition started by Chomsky, but
diverges from his work in many ways.
\end{itemize}

\myslide{Main Tenets of Generative Grammar}
\begin{itemize}
\item Grammars should be formulated precisely
and explicitly.
\item Languages are infinite, so grammars must be
tested against invented data, not just attested
examples.
\item The theory of grammar is a theory of human
linguistic abilities.
\end{itemize}

\myslide{What does a theory do?}
\begin{itemize}
\item Monolingual
\begin{itemize}
\item Model grammaticality/acceptability
\item Model relationships between sentences
(internal structure)
\end{itemize}
\item Multilingual
\begin{itemize}
\item Model relationships between languages
\item Capture generalizations about possible
languages
\end{itemize}
\end{itemize}

\myslide{Some of Chomsky’s Controversial Claims}
\begin{itemize}
\item The superficial diversity of human languages
masks their underlying similarity.
\item All languages are fundamentally alike
because linguistic knowledge is largely
innate.
\item The central problem for linguistics is
explaining how children can learn language
so quickly and easily.
\end{itemize}

\myslide{Family Tree of Generative Syntactic Theories}

\newcommand{\gr}[2]{#1  (#2)}
%\newcommand{\gr}[2]{\parbox[t][12ex][t]{10em}{#1\\(#2)}}
%\begin{tabular}[t]{c}#1\\(#2)\\[2ex]\end{tabular}\rule[-12ex]{0ex}{12ex}}}

%\rule[lift]{width}{height}
%\rule{0ex}{12ex}


  \begin{tree}\tiny
    \br{Early Transformational Grammar (1955-1964)}%
    {\br{Standard Theory (1964-1967)}{%
        \br{EST (1967-1977)}{%
          \br{\gr{REST}{1977-1981}}{
            \br{\gr{GB}{1981-1993}}{\br{\gr{MP}{1993-present}}{ }}}
          \br{\gr{GPSG}{1979-1985}}{%
            \br{\rnode{HPSG}{\gr{HPSG}{1986-present}}}{
            \br{\gr{Sign-based CxG}{2001-present}}{ }}}
          \br{\gr{Realistic TG}{1978-1980}}{%
             \br{\rnode{LFG}{\gr{LFG}{1980-present}}}{
            }}}} {\br {\rnode{GS}{Generative Semantics (1966-1975)}}
        {\br{Relational Grammar (1974-present)}{\br{Arc Pair
              Grammar (1980)}{}}}}}
    \ncline[linestyle=dashed]{-}{LFG}{HPSG} \nccurve[linestyle=dashed,
    angleA=0, angleB=180]{-}{LFG}{GS}
  \end{tree}
\vspace*{-2em}
  \begin{small}
    \begin{itemize}
    \item Many Other Theories
      \begin{itemize}
      \item Dependency Grammar  (links words not phrases)
      \item Combinatory Categorical Grammar (allows multiple derivations)
      \item Tree Adjoining Grammar (links subtrees)
      \item Functional Grammar (considers function to be central)
        \begin{itemize}
        \item Systemic Functional Grammar
        \item Role and Reference Grammar
        \end{itemize}
      \end{itemize}
    \end{itemize}
  \end{small}



\myslide{Why Study Syntax?}
\begin{itemize}
\item Why should linguists study syntax?
%\item Why should computational linguists study syntax?

\item Should anyone else study syntax? Why?
\item Why are you studying syntax?
\end{itemize}

\myslide{What makes a good model?}

\begin{itemize}
\item \txx{generative}: license all grammatical sentences and only them
\\ $\Rightarrow$ \txx{precise}
\item \txx{explanatory}: can explain generalizations
  \begin{itemize}
  \item  \eng{the cat chased the rat} $\sim$ \eng{the rat was chased by the cat} \com{semantics}
  \item phrases tend to act like one member of the phrase \com{headedness}
  \item new information tends to come first/last \com{information theory}
  \end{itemize}
\item \txx{concise}: the model is as simple as possible \com{elegant}
\\ $\Rightarrow$ \txx{universal} \com{minimal stipulations}
\item \txx{tractable}: the model can be modeled computationally
\end{itemize}

\begin{center}
  Our models are normally imperfect: \\
we aim for iteratively improved approximations
\end{center}

\myslide{Insufficient Theory \#1}
\begin{itemize}
\item A grammar is simply a list of sentences.
\item What’s wrong with this?
\end{itemize}

\myslide{Insufficient Theory \#2: Regular Expressions}
\begin{exe}
  \ex \gll the noisy dogs left \\
  D A N V \\
  \ex  \gll the noisy dogs chased the innocent cats \\
 D A N V D A N \\
\end{exe}
\begin{itemize}
\item (D) A* N V ((D) A* N)
\end{itemize}
\hrule
\txx{Regular expressions}: a formal language for matching things.
\\[2ex]
 \begin{tabular}{ll}
    Symbol & Matches \\ \hline
    . & any single character\\
%    {[ ]} & a single character that is contained within the brackets. \\
%    & {[a-z]} specifies a range which matches any  letter from "a" to "z".\\
%    {[\textasciicircum ~]} & 	a single character not in the brackets. \\
%    \textasciicircum 	& the starting position within the string/line. \\
%    \$ 	&  the ending position of the string/line. \\
    $*$ &	the preceding element zero or more times. \\
    ? &	 the preceding element zero or one time: OR just () = ()?. \\
    + &	 the preceding element one or more times. \\
    $|$ &  either the expression before or after the operator. \\
%    $\backslash$ & escapes the following character. \\
  \end{tabular}

% \myslide{A Finite State Machine}
% D

% N

% V

% D

% A

% N

% A
% V
% V


%\myslide{FSMs for Grammar, cont}

% \item Why are FSMs insufficient as a

% representation of natural language syntax?

% \item How might they be useful anyway?


% \myslide{Chomsky Hierarchy}
% Type 0 Languages
% Context-Sensitive Languages
% Context-Free Languages
% Regular Languages


\myslide{Context-Free Grammar}
\begin{itemize}
\item A quadruple: $\langle C, V, P, S \rangle$
\begin{itemize}
\item[$C$] set of categories ($\alpha, \beta, \ldots$)
\item[$V$] set of terminals (vocabulary)
\item[$P$] set of rewrite rules $\alpha \into \beta_1, \beta_2, \ldots, \beta_n$
\item[$S$] the start symbol $\mathbf{S} \in C$
\end{itemize}
\item For each rule $\alpha \into \beta_1, \beta_2, \ldots, \beta_n \in P$
  \begin{itemize}
  \item  $\alpha \in C$
  \item  $\beta_i \in C \cup V; 1 \le i \le n$ 
\end{itemize}
\end{itemize}


\myslide{A Toy Grammar}

\begin{itemize}
\item RULES
\\[2ex]  \begin{tabular}{lll}
\textbf{S}  & \into & NP VP \\
NP & \into & (D) A* N PP*\\
VP & \into & V (NP) (PP)\\
PP & \into & P NP\\
\end{tabular}

\item VOCABULARY
\begin{flushleft}
D: the, some\\
A: big, brown, old\\
N: birds, fleas, dog, hunter, I\\
V: attack, ate, watched\\
P: for, beside, with
\end{flushleft}
\end{itemize}

\myslide{Structural Ambiguity}
\begin{center} \large
  \eng{I saw the astronomer with the telescope.}
\end{center}

\myslide{Structure 1: PP under VP}
{%
 \leaf{\emph{I}}
 \branch{1}{N}
 \branch{1}{NP}
 \leaf{\emph{saw}}
 \branch{1}{V}
 \leaf{\emph{the}}
 \branch{1}{D}
 \leaf{\emph{astronomer}}
 \branch{1}{N}
 \branch{2}{NP}
 \leaf{\emph{with}}
 \branch{1}{P}
 \leaf{\emph{the}}
 \branch{1}{D}
 \leaf{\emph{telescope}}
 \branch{1}{N}
 \branch{2}{NP}
 \branch{2}{PP}
 \branch{3}{VP}
 \branch{2}{S}
 \qobitree}

\myslide{Structure 2: PP under NP}
\begin{small} {%
    \leaf{\emph{I}} \branch{1}{N} \branch{1}{NP} \leaf{\emph{saw}}
    \branch{1}{V} \leaf{\emph{the}} \branch{1}{D}
    \leaf{\emph{astronomer}} \branch{1}{N} \leaf{\emph{with}}
    \branch{1}{P} \leaf{\emph{the}} \branch{1}{D}
    \leaf{\emph{telescope}} \branch{1}{N} \branch{2}{NP}
    \branch{2}{PP} \branch{3}{NP} \branch{2}{VP}
    \branch{2}{S} \qobitree}
\end{small}

\myslide{Constituency Tests}
\begin{itemize}\addtolength{\itemsep}{-1ex}
\item Recurrent Patterns
\begin{exe}
\ex \eng{\ul{The quick brown fox with the bushy tail} jumped over \ul{the lazy brown dog
with one ear}.}\
\end{exe}

\item Coordination
\begin{exe}
\ex  \eng{\ul{The quick brown fox with the bushy tail} and \ul{the lazy brown dog with one
ear} are friends.}
\end{exe}
\item Sentence-initial position
\begin{exe}
\ex \eng{\ul{The election of 2000}, everyone will remember for a long time.}
\end{exe}
\item Cleft sentences
\begin{exe}
\ex \eng{It was \ul{a book about syntax} that they were reading.}
\end{exe}
\end{itemize}

\myslide{General Types of Constituency Tests}
\begin{itemize}
\item Distributional
\item Intonational
\item Semantic
\item Psycholinguistic
\item[\ldots] but they don’t always agree.
\end{itemize}

\myslide{Central claims implicit in CFG formalism:}
\begin{enumerate}
\item  Parts of sentences (larger than single words) are
linguistically significant units, i.e. phrases play a role in
determining meaning, pronunciation, and/or the
acceptability of sentences.
\item Phrases are contiguous portions of a sentence (no
discontinuous constituents).
\item Two phrases are either disjoint or one fully contains the
other (no partially overlapping constituents).
\item What a phrase can consist of depends only on what kind of
a phrase it is (that is, the label on its top node), not on what
appears around it.
\end{enumerate}

\newpage
\begin{itemize}
\item Claims 1-3 characterize what is called \txx{phrase structure grammar}

\item Claim 4 (that the internal structure of a phrase depends only on what type of phrase it is, not on where it appears) is what makes it \txx{Context-Free}.

\item \txx{Context-Sensitive Grammar} (CSG) gives up 4. 
That is, it allows the applicability of a
grammar rule to depend on what is in the
neighboring environment. So rules can have the
form: 
\\ $A \into X$ in the context of $\alpha \_\beta$ ($\alpha A\beta \into \alpha X\beta$)
\end{itemize}

\myslide{Possible Counterexamples}
\begin{itemize}
\item To Claim 2 (no discontinuous constituents):
\\ \eng{\ul{A technician} arrived \ul{who could solve the problem}.}

\item To Claim 3 (no overlapping constituents):
\\ \eng{I read \ul{what} was written about me.}

\item To Claim 4 (context independence):
  \begin{exe}
  \ex \eng{He arrives this morning.}
  \ex \eng{*He arrive this morning.}
  \ex \eng{*They arrives this morning.}
  \ex \eng{They arrive this morning.}
  \end{exe}
\end{itemize}


\myslide{Trees and Rules}
  {
 \leaf{$C_1$}
 \leaf{\ldots}
  \leaf{$C_2$}
 \branch{3}{$C_0$} \qobitree}is a well-formed nonlexical tree if (and only if)
\begin{itemize}
\item  $C_0, \ldots,  C_n$ are well-formed trees
\item $C_0$ \into $C_1 \ldots C_n$ is a grammar rule
\end{itemize}

\myslide{Bottom-up Tree Construction}

\begin{flushleft}
D: the \\
V: chased \\
N: dog, cat
\end{flushleft}

{ \leaf{the} \branch{1}{D} \qobitree}
{ \leaf{the} \branch{1}{D} \qobitree}
{ \leaf{chased} \branch{1}{V} \qobitree}
{ \leaf{dog} \branch{1}{N} \qobitree}
{ \leaf{cat} \branch{1}{N} \qobitree}

\begin{tabular}{ccc}
\multicolumn{2}{c}{NP \into D N} & VP \into V NP \\[2ex]

{ \leaf{the} \branch{1}{D}
  \leaf{dog} \branch{1}{N} 
\branch{2}{NP} \qobitree}
&
{ \leaf{the} \branch{1}{D}
  \leaf{cat} \branch{1}{N} 
\branch{2}{NP} \qobitree}
&
{
  \leaf{chased} \branch{1}{V}
    \leaf{the} \branch{1}{D}
    \leaf{cat} \branch{1}{N} 
  \branch{2}{NP}
\branch{2}{VP} \qobitree}
\end{tabular}

\newpage

\begin{tabular}{c}
 S \into NP VP \\[2ex]
{
    \leaf{the} \branch{1}{D}
    \leaf{dog} \branch{1}{N} 
  \branch{2}{NP}
    \leaf{chased} \branch{1}{V}
      \leaf{the} \branch{1}{D}
      \leaf{cat} \branch{1}{N} 
    \branch{2}{NP}
  \branch{2}{VP}
\branch{2}{S} \qobitree}
\end{tabular}

\myslide{Top-down Tree Construction}


\begin{tabular}{cccc}
 S \into NP VP  & VP \into V NP & NP \into D N & NP \into D N \\[2ex]
{ \leaf{NP}
  \leaf{VP}
\branch{2}{S} \qobitree}
&
{ \leaf{V}
  \leaf{NP}
\branch{2}{VP} \qobitree}
&
{ \leaf{D}
  \leaf{N}
\branch{2}{NP} \qobitree}
&
{ \leaf{D}
  \leaf{N}
\branch{2}{NP} \qobitree}
\end{tabular}

{ \leaf{D}
  \leaf{N}
\branch{2}{NP} 
   \leaf{V}
   \leaf{D}
   \leaf{N}
\branch{2}{NP}
\branch{2}{VP}
\branch{2}{S} \qobitree}
{ \leaf{the} \branch{1}{D} \qobitree}
{ \leaf{the} \branch{1}{D} \qobitree}
{ \leaf{chased} \branch{1}{V} \qobitree}
{ \leaf{dog} \branch{1}{N} \qobitree}
{ \leaf{cat} \branch{1}{N} \qobitree}

\begin{tabular}{c}
% Combine \\[2ex]
{
    \leaf{the} \branch{1}{D}
    \leaf{dog} \branch{1}{N} 
  \branch{2}{NP}
    \leaf{chased} \branch{1}{V}
      \leaf{the} \branch{1}{D}
      \leaf{cat} \branch{1}{N} 
    \branch{2}{NP}
  \branch{2}{VP}
\branch{2}{S} \qobitree}
\end{tabular}

\begin{itemize}
\item \txx{Bottom-up}: string \into tree
\item \txx{Top-down}: tree \into string
\item CFG is \txx{declarative} so it is independent of order
\end{itemize}

\myslide{Weaknesses of CFG (atomic node labels)}
\begin{itemize}
\item It doesn’t tell us what constitutes a linguistically
natural rule
\begin{itemize}
\item VP \into P NP
\item NP \into VP S
\end{itemize}
\item Rules get very cumbersome once we try to deal
with things like agreement and transitivity.
\item It has been argued that certain languages (notably
Swiss German and Bambara) contain constructions
that are provably beyond the descriptive capacity of
CFG.
\end{itemize}

\myslide{On the other hand \ldots}

\begin{itemize}
\item It’s a simple formalism that can generate
infinite languages and assign linguistically
plausible structures to them.

\item Linguistic constructions that are beyond the
descriptive power of CFG are rare.

\item It’s computationally tractable and
techniques for processing CFGs are well
understood.
\end{itemize}

\myslide{So \ldots}

\begin{itemize}
\item CFG is the starting point for most
types of generative grammar.

\item The theory we develop in this course is an
extension of CFG.
\end{itemize}

\myslide{Transitivity and Agreement}

\begin{itemize}
\item Consider the following transitivity examples
  \begin{exe}
    \ex \eng{The bird arrives}
    \ex \eng{The bird devours the worm}
    \ex *\eng{The bird arrives the worm}
    \ex *\eng{The bird devours}
  \end{exe}
\item Consider the following agreement examples
  \begin{exe}
    \ex \eng{The bird sings}
    \ex \eng{The birds sing}
    \ex *\eng{The bird sing}
    \ex *\eng{The birds sings}
  \end{exe}
\item Can we deal with them with a CFG?
\end{itemize}

\myslide{Summary}
\begin{enumerate}\addtolength{\itemsep}{-1ex}
\item Fundamentals
\item Investigate
\item Find out some stuff
\item Break our theory
\item Try to fix it.
\item Break it again.
\item Lather, rinse, repeat: we'll do that until we run out of time.
\end{enumerate}

Jorge Hankamer's outline of a syntax course, but it's pretty
applicable to everything we do.  More formally: \emp{Successive Approximation}.


\myslide{Chapter 2, Problem 1}
\begin{tabular}{cc}
  RULES & VOCABULARY \\[2ex]
  \begin{minipage}[t]{0.4\linewidth}
    \begin{tabular}[t]{lll}
      \textbf{S}  & \into & NP VP \\
      NP & \into & (D) NOM\\
      VP & \into & V (NP) (NP)\\
      NOM & \into & N\\
      NOM & \into & NOM PP \\
      VP  & \into & VP PP\\
      PP  & \into & P NP\\
      X   & \into & X+ CONJ X\\
    \end{tabular} 
  \end{minipage} &
  \begin{minipage}[t]{0.5\linewidth}
    \begin{flushleft}
  D: a, the\\
  N: cat, dog, hat, man, woman, roof\\
  V: admired, disappeared, put, relied\\
  P: in, on, with\\
  CONJ: and, or
\end{flushleft}
\end{minipage}
\end{tabular}

\myslide{Chapter 2, Problem 1}

\begin{itemize}
\item [A] Make a well-formed English sentence unambiguous
according to this grammar
\item  [B] Make a well-formed English sentence ambiguous
according to this grammar: draw trees
\item  [C] Make a well-formed English sentence not licensed by
this grammar (using $V$)
\item [D] Why is this (C) not licensed?

\newpage
\item [E] Make a string licensed by this grammar that is not a
  well-formed English sentence
\item [F] How can we stop licensing the string in E (stop over-generating)
\item [G] How many strings does this grammar license?
\item [H] How many strings does this grammar license without conjunctions?
\end{itemize}

\myslide{Shieber 1985}
\begin{itemize}
\item Swiss German example:
  \begin{exe}
    \ex \gll \ldots mer \uline{d’chind} \uuline{em Hans} es \uwave{huus} \uline{l\"ond} \uuline{h\"alfe} \uwave{aastriiche}\\
    \ldots we {the children-acc} Hans-dat the hous-acc let help paint \\
    \trans we let {the children} help Hans paint the house 
  \end{exe}
\item Cross-serial dependency:
\begin{itemize}
\item \eng[let]{l\"ond} governs case on \eng[children]{d’chind}
\item \eng[help]{h\"alfe} governs case on \eng[Hans]{Hans}
\item \eng[paint]{aastriiche} governs case on \eng[house]{huus}
\end{itemize}
\item This cannot be modeled in a context free language
\end{itemize}

\myslide{Strongly/weakly CF}
\begin{itemize}
\item A language is weakly \emp{context-free} if the set of
strings in the language can be generated by a CFG.
\item A language is \emp{strongly} context-free if the CFG
furthermore assigns the correct structures to the
strings.
\item Shieber’s argument is that SW is not \emp{weakly}
context-free and therefore not \emp{strongly} context-free.
\item Bresnan et al (1983) had already argued that Dutch
is \emp{strongly} not context-free, but the argument was
dependent on linguistic analyses.
\end{itemize}

\myslide{Overview}

\begin{itemize}
\item Prescriptive/descriptive grammar;
Competence/performance
\item Some history
\item Why study syntax?
\item Unsuccessful Attempts to model language
\item Formal definition of CFG
  \begin{itemize}
  \item Constituency, ambiguity, constituency tests
  \item Central claims of CFG
  \item Order independence
  \item Weaknesses of CFG
  \end{itemize}
\item Next Week: Feature structures
\end{itemize}


\myslide{Acknowledgments and References}

\begin{itemize}
\item Course design and slides borrow heavily from Emily Bender's course:
\textit{Linguistics 566: Introduction to Syntax for Computational Linguistics}
\\ \url{http://courses.washington.edu/ling566}
\item Thanks to Na-Rae Han for 
  inspiration for the student policies (from  \textit{LING 2050 Special Topics in Linguistics: Corpus linguistics}, U Penn; adapted).
\item Stuart M. Shieber. (1985) Evidence against the context-freeness of natural language. \textit{Linguistics and Philosophy}, 8:333-343
\end{itemize}



\end{document}


%%% Local Variables: 
%%% coding: utf-8
%%% mode: latex
%%% TeX-PDF-mode: t
%%% TeX-engine: xetex
%%% End: 

