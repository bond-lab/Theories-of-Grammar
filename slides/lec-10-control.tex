\documentclass[a4paper,landscape,headrule,footrule]{foils}
%%
%%% macros for Theories of Grammar
%%%
\usepackage{polyglossia}
\setdefaultlanguage{english}
%\setmainfont{TeX Gyre Pagella}


\newcommand{\logo}{~}
\newcommand{\header}[3]{%
\title{\vspace*{-2ex} \large HG4041 Theories of Grammar
\\[2ex] \Large  \emp{#2} \\ \emp{#3}}
\author{\blu{Francis Bond}   \\ 
\normalsize  \textbf{Division of Linguistics and Multilingual Studies}\\
\normalsize  \url{http://www3.ntu.edu.sg/home/fcbond/}\\
\normalsize  \texttt{bond@ieee.org}}
\MyLogo{HG4041 (2020)}
\renewcommand{\logo}{#2}
\hypersetup{
   pdfinfo={
     Author={Francis Bond},
     Title={#1: #2},
     Subject={HG4041: Theories of Grammar},
     Keywords={Syntax, Semantics, HPSG, Unification, Constructions},
     License={CC BY 4.0}
   }
 }

\date{#1
  \\ Location: LHN-TR+36}
}
\usepackage[hidelinks]{hyperref}




\usepackage{xcolor}
\usepackage{graphicx}
\newcommand{\blu}[1]{\textcolor{blue}{#1}}
\newcommand{\grn}[1]{\textcolor{green}{#1}}
\newcommand{\hide}[1]{\textcolor{white}{#1}}
\newcommand{\emp}[1]{\textcolor{red}{#1}}
\newcommand{\txx}[1]{\textbf{\textcolor{blue}{#1}}}
\newcommand{\lex}[1]{\textbf{\mtcitestyle{#1}}}

\usepackage{pifont}
\renewcommand{\labelitemi}{\textcolor{violet}{\ding{227}}}
\renewcommand{\labelitemii}{\textcolor{purple}{\ding{226}}}

\newcommand{\subhead}[1]{\noindent\textbf{#1}\\[5mm]}

\newcommand{\Bad}{\emp{\raisebox{0.15ex}{\ensuremath{\mathbf{\otimes}}}}}
\newcommand{\bad}[1]{*\eng{#1}}

\newcommand{\com}[1]{\hfill (\emp{#1})}%

\usepackage{relsize,xspace}
\newcommand{\into}{\ensuremath{\rightarrow}\xspace}
\newcommand{\tot}{\ensuremath{\leftrightarrow}\xspace}
\usepackage{url}
\newcommand{\lurl}[1]{\MyLogo{\url{#1}}}

\usepackage{mygb4e}
\newcommand{\lx}[1]{\textbf{\mtciteform{#1}}}
\newcommand{\ix}{\ex\slshape}
\let\eachwordone=\slshape



\newcommand{\ent}{\ensuremath{\Rightarrow}\xspace}
\newcommand{\ngv}{\ensuremath{\not\Rightarrow}\xspace}
%\usepackage{times}
%\usepackage{nttfoilhead}
\newcommand{\myslide}[1]{\foilhead[-25mm]{\raisebox{12mm}[0mm]{\emp{#1}}}\MyLogo{\logo}}
\newcommand{\myslider}[1]{\rotatefoilhead[-25mm]{\raisebox{12mm}[0mm]{\emp{#1}}}}
%\newcommand{\myslider}[1]{\rotatefoilhead{\raisebox{-8mm}{\emp{#1}}}}

\newcommand{\section}[1]{\myslide{}{\begin{center}\Huge \emp{#1}\end{center}}}



\usepackage[lyons,j,e,k]{mtg2e}
%\renewcommand{\mtcitestyle}[1]{\textcolor{-red!75!green!50}{\textsl{#1}}}
\renewcommand{\mtcitestyle}[1]{\textcolor{teal}{\textsl{#1}}}
\newcommand{\iz}[1]{\texttt{\textup{#1}}}
\newcommand{\gm}{\textsc}
\usepackage[normalem]{ulem}
\newcommand{\ul}{\uline}
\newcommand{\ull}{\uuline}
\newcommand{\wl}{\uwave}
\newcommand{\vs}{\ensuremath{\Leftrightarrow}~}
%%%
%%% Bibliography
%%%
\usepackage{natbib}
%\usepackage{url}
\usepackage{bibentry}


%%% From Tim
\newcommand{\WMngram}[1][]{$n$-gram#1\xspace}
\newcommand{\infers}{$\rightarrow$\xspace}

\usepackage[utf8]{inputenc}

\usepackage{rtrees,qtree}
\renewcommand{\lf}[1]{\br{#1}{}}
\usepackage{avm}
%\avmoptions{topleft,center}
\newcommand{\ft}[1]{\textsc{#1}}
\renewcommand{\val}[1]{\textit{#1}}
\newcommand{\typ}[1]{\textit{#1}}
\newcommand{\prd}[1]{\textbf{#1}}
\avmfont{\sc}
\avmvalfont{\it}
\avmsortfont{\smaller[2] \it}
\usepackage{multicol}
\newcommand{\blank}{\rule{3em}{1pt}\xspace}

% \usepackage{pst-node}
\newcommand{\OV}[1]{\ovalnode[linestyle=dotted,linecolor=red]{A}{#1}}
\newcommand{\OVB}[1]{\ovalnode[linestyle=dotted,linecolor=blue]{A}{#1}}

%%% From CSLI book
\newcommand{\mc}{\multicolumn}
\newcommand{\HD}{\textbf{H}\xspace}
\newcommand{\el}{\< \>}
\makeatother
\long\def\smalltree#1{\leavevmode{\def\\{\cr\noalign{\vskip12pt}}%
\def\mc##1##2{\multispan{##1}{\hfil##2\hfil}}%
\tabskip=1em%
\hbox{\vtop{\halign{&\hfil##\hfil\cr
#1\crcr}}}}}
\makeatletter

\newcommand{\A}{\noindent\textbf{A}: }
\newcommand{\Q}{\noindent\textbf{Q}: }
%\newcommand{\C}{\noindent\textbf{C}: }



\usepackage{times}
\usepackage{bchart}
%\usepackage{pgfplots}
\avmfont{\sc}
\begin{document}
\header{Lecture 10}{Raising and Control}{}
\maketitle

%


\myslide{Overview}
\MyLogo{Sag, Wasow and Bender (2003) --- Chapter 12}

\begin{itemize}
\item Intro to topic
\item Infinitival \lex{to}
\item (Subject) raising verbs
\item (Subject) control verbs
\item Raising/control in Transformational Grammar
\item Object raising and object control
%\item If time: Problem 12.4
\end{itemize}

\myslide{Where We Are \& Where We’re Going}

\begin{itemize}
\item In the last two lectures, we have seen a kind of 
subject sharing -- that is, cases where one NP 
served as the \ft{spr} for two different verbs.  
Examples?
\item Last time, we looked at \txx{dummy} NPs --  that is, 
non-referential NPs.  Examples?
\item Today, we’re going to look at the kind of subject 
sharing we saw with \lex{be} in more detail.
\item Then we’ll look at another kind of subject 
sharing, using dummy NPs in differentiating the 
two kinds.
\end{itemize}


\myslide{What Makes This Topic Different}
\begin{itemize}
\item The phenomena we have looked at so far (\txx{agreement},
  \txx{binding}, \txx{imperatives}, \txx{passives},
  \txx{existentials}, \txx{extraposition}) are easy to pick out on the
  basis of their form alone.
\item In this chapter, we look at constructions with the general form
  NP-V-(NP)-\lex{to}-VP.  It turns out that they divide into two
  kinds, differing in both syntactic and semantic properties.
\end{itemize}

\myslide{The Central Idea}

\begin{itemize}
\item  \eng{Pat continues to avoid conflict} and 
\eng{Pat tries to avoid conflict} 
both have the form NP-V-\lex{to}-VP
  \begin{itemize}
  \item But \lex{continue} is semantically a one-place 
    predicate, expressing a property of a situation 
    (namely, that it continues to be the case)
  \item Whereas \lex{try} is semantically a two-place 
    predicate, expressing a relation between someone 
    who tries and a situation s/he tries to bring about.
  \end{itemize}
\item This semantic difference has syntactic effects.
\end{itemize}

\myslide{The Status of Infinitival \lex{to}}


\begin{itemize}
\item It’s not obvious what part of speech to assign to \lex{to}.  
\item It’s not the same as the preposition to:
  \begin{exe}
    \ex \eng{Pat aspires to stardom}
    \ex \eng{Pat aspires to be a good actor}
    \ex *\eng{Pat aspires to stardom and to be a good actor}
  \end{exe}
\item We call it an \txx{auxiliary verb}, because this will make 
our analysis of auxiliaries a little simpler.
\end{itemize}

\myslide{The Lexical Entry for Infinitival \lex{to}}
\begin{center}
  \begin{small}
    \begin{avm}
      \< \textnormal{to}, \[
      syn &  \[ head & \[ form & base\\
      inf & $+$ \\
      aux & $+$ \] \] \\
      arg-st & \< \@1,
      \[ head & \[{\tiny \it verb} \\
      inf & $-$ \\
      form & base \] \\
      val & \[ spr & \< \@1 \> \\
      comps & \< ~ \> \] \\
      sem & \[ index & s \] \] \> \\
      sem & \[ index & s \\ restr & \< \> \] \] \>
    \end{avm}
  \end{small}
\end{center}
\myslide{The Syntax of Infinitival  \lex{to}}

\begin{center}
  \begin{avm}
    \[ syn &  \[ head & \[ form & base\\
    inf & $+$ \\
    aux & $+$ \] \] \]
  \end{avm}
\end{center}

\begin{itemize}\addtolength{\itemsep}{-1ex}
\item This makes it a verb, because \ft{aux} is declared on verb
\item {[\ft{inf}  $+$]} uniquely identifies the infinitival \lex{to}
\item Verbs select complements with different combinations 
  of \ft{form} and \ft{inf} values, e.g.
  \begin{itemize}
  \item complements of \lex{condescend} are [\ft{form} \val{base}] and [\ft{inf} $+$]
  \item complements of \lex{should} are [\ft{form} \val{base}] and [\ft{inf} $-$]
  \item complements of \lex{help} are [\ft{form} \val{base}]
  \end{itemize}
\item The meaning of [\ft{aux} $+$] becomes clear in Chapter 13.
\end{itemize}

\myslide{The Argument Structure}
\begin{center}
  \begin{avm}
     \[ arg-st & \< \@1,
      \[ head & \[{\tiny \it verb} \\
      inf & $-$ \\
      form & base \] \\
      val & \[ spr & \< \@1 \> \\
      comps & \< ~ \> \] \\
      sem & \[ index & s \] \] \> \] 
  \end{avm}
\end{center}
\begin{itemize}
\item What kind of constituent is the second argument?
\item The tagging of the first argument and the \ft{spr} of the second 
argument is exactly like \lex{be}.
\end{itemize}

\myslide{The Semantics of Infinitival \lex{to}}
\begin{center} \small
  \begin{avm}
     \[ arg-st & \< \@1,
      \[ head & \[{\tiny \it verb} \\
      inf & $-$ \\
      form & base \] \\
      val & \[ spr & \< \@1 \> \\
      comps & \< ~ \> \] \\
      sem & \[ index & \OV{s} \] \] \> \\
      sem & \[ index & \OV{s} \\ restr & \< \> \] \]
  \end{avm}
\end{center}

\begin{itemize}
\item The \ft{index} value is taken from the \ft{sem} of the second
  argument.
\item What is the semantic contribution of \lex{to}?
\end{itemize}

\myslide{Dummies and \lex{continue}}

\begin{itemize}
\item Some examples:
  \begin{exe}
\ex \eng{There continue to be seats available.}
\ex \eng{It continues to matter that we lost.}
\ex \eng{Advantage continues to be taken of the innocent.}
\ex *\eng{It continues to be seats available.}
\ex *\eng{There continues to matter that we lost.}
\ex *\eng{Advantage continues to be kept of the innocent.}
\end{exe}
\item Generalization:  Non-referential NPs can appear as the 
subject of \lex{continue} just in case they could be the subject 
of the complement of \lex{continue}.
\end{itemize}

\myslide{A New Type, for Verbs like \lex{continue}}

\begin{center} \small
\txx{Subject-Raising Verb Lexeme} (\val{srv-lxm}) \\
  \begin{avm}
     \[ arg-st & \< \@1,
      \[ 
      val & \[ spr & \< \@1 \> \\
      comps & \< ~ \> \] \\
      sem & \[ index & \OV{s} \] \] \> \\
      sem & \[ restr & \< \[ arg & s \] \> \] \]
  \end{avm}
\end{center}

\begin{itemize}\addtolength{\itemsep}{-1ex}
\item The subject sharing is just like for \lex{be} and \lex{to}: the
  subject of \lex{continue} is also the subject of its complement
\item  \lex{continue} imposes no other constraints on its subject
\item The index of the complement must be an argument of the 
predication introduced by the verb
\end{itemize}

\myslide{The Lexical Entry for \lex{continue}}

\bigskip
\begin{center}
    \begin{avm}
      \< \textnormal{continue}, \[{\tiny \val{srv-lxm}}\\[-3ex]
      syn &  \[ arg-st & \< X,
                  VP \[ inf & $+$  \]  \] \> \\
      sem & \[ index & s$_1$ \\ 
               restr & \< \[ reln & \textbf{continue} \\ 
                             sit & s$_1$ \] \> \] \] \>
    \end{avm}
\end{center}



\myslide{\lex{continue} with Inherited Information}

\begin{center}
  \begin{small}
    \begin{avm}
      \< \textnormal{continue}, \[{\tiny \val{srv-lxm}}\\[-3ex]
      syn &  \[ head & \[{\tiny \val{verb}} \\
      pred & $-$ \\
      inf & $-$ \\
      agr & \@2 \] \] \\
      arg-st & \< \@1 \[ head & \val{nominal}\\
                         val &  \[ spr \< \>  & comps \< \> \]  \],
                  VP \[ inf & $+$ \\ spr  & \< \@1 \> \\ index & s$_2$ \]
      \> \\
      sem & \[ mode & prop \\ index & s$_1$ \\ 
               restr & \< \[ reln & \textbf{continue} \\ 
                             sit & s$_1$ \\ arg & s$_2$  \] \> \] \] \>
    \end{avm}
  \end{small}
\end{center}


\myslide{Key Property of Subject-Raising Verbs}

The subject plays no semantic role in the predication introduced by
the \txx{SRV} itself.  Its semantic role (if any) is only in the
predication introduced in the complement.

\begin{center}
  \begin{small}
    \begin{avm}
      \< \textnormal{continue}, \[{\tiny \val{srv-lxm}}\\[-3ex]
      arg-st & \< \OV{\@1} \[ head & \val{nominal}\\
                         val &  \[ spr \< \>  & comps \< \> \]  \],
                  VP \[ inf & $+$ \\ spr  & \< \OV{\@1} \> \\ 
                        index & \OVB{s$_2$} \]
      \> \\
      sem & \[ mode & prop \\ index & s$_1$ \\ 
               restr & \< \[ reln & \textbf{continue} \\ 
                             sit & s$_1$ \\ arg & \OVB{s$_2$}  \] \> \] \] \>
    \end{avm}
  \end{small}
\end{center}


\myslide{Constraints on SRV's subjects are from their complements}

\begin{itemize}
\item SRVs take dummy subjects when and only when their 
complements do.
\item SRVs take idiom chunk subjects when and only when their 
complements do.
\item Passivizing the complement of an SRV doesn’t change the 
truth conditions of the whole sentence:
\begin{exe}
\ex \eng{Skeptics continue to question your hypothesis} 
\ex \eng{Your hypothesis continues to be questioned by skeptics}
\end{exe}
\end{itemize}

\myslide{\lex{continue} with active complement}

\begin{center} 
  \scalebox{0.7}{\begin{avmtree}%\renewcommand{\lf}[1]{\br{#1}{}}\avmfont{\sc}
    \br{S\avmbox{A}}{ \br{\avmbox{1} NP$_i$}{\lf{Skeptics}}
      \br{VP \[ \ft{spr} \< \avmbox{1} \> \]}{
        \br{V \[ \ft{spr} \< \avmbox{1} \> \]}{\lf{continue}}
        \br{VP \[ \ft{spr} \< \avmbox{1} \> \]}{\br{V \[ \ft{spr} \< \avmbox{1} \> \]}{\lf{to}}
                \br{VP \[ \ft{spr} \< \avmbox{1} \> \]}{
                  \br{V \[ \ft{spr} \< \avmbox{1} \> \]}{\lf{question}}
                  \br{NP$_j$}{\tlf{your hypothesis}}}}}}

\end{avmtree}}
\end{center}

\scalebox{0.7}{\raisebox{8ex}[0ex][0ex]{\small \begin{avm}
  \[ restr & \avmbox{A} \< \[ reln & \textbf{question} \\ doubter & i \\ doubted & j\], 
  \ldots \> \]
\end{avm}}}



\myslide{\lex{continue} with passive complement}
\begin{center} 
  \scalebox{0.55}{\begin{avmtree}%\renewcommand{\lf}[1]{\br{#1}{}}\avmfont{\sc}
    \br{S\avmbox{A}}{ \br{\avmbox{1} NP$_j$}{\tlf{Your hypothesis}}
      \br{VP \[ \ft{spr} \< \avmbox{1} \> \]}{
        \br{V \[ \ft{spr} \< \avmbox{1} \> \]}{\lf{continue}}
        \br{VP \[ \ft{spr} \< \avmbox{1} \> \]}{
          \br{V \[ \ft{spr} \< \avmbox{1} \> \]}{\lf{to}}
          \br{VP \[ \ft{spr} \< \avmbox{1} \> \]}{
            \br{V \[ \ft{spr} \< \avmbox{1} \> \]}{\lf{be}}
            \br{VP \[ \ft{spr} \< \avmbox{1} \> \]}{
              \br{V \[ \ft{spr} \< \avmbox{1} \> \]}{\lf{questioned}}
              \br{NP$_i$}{\br{P$_i$}{\lf{by}}
                \br{NP$_i$}{\br{NOM$_i$}{\lf{skeptics}}}}}}}}}

\end{avmtree}}
\end{center}
\scalebox{0.7}{\raisebox{8ex}[0ex][0ex]{\small \begin{avm}
  \[ restr & \avmbox{A} \< \[ reln & \textbf{question} \\ doubter & i \\ doubted & j\], 
  \ldots \> \]
\end{avm}}}


\myslide{Control Verbs}

\begin{itemize}
\item Control verbs, like \lex{try}, appear in contexts that 
look just like the contexts for raising verbs:
\begin{xlisti}
\ex \eng{Pat tried to stay calm}
\trans looks superficially like
\ex \eng{Pat continued to stay calm}
\end{xlisti}
\item Control verbs also share their subjects with their 
complements, but in a different way.
\item A control verb expresses a relation between the 
referent of its subject and the situation denoted by 
its complement.
\end{itemize}

\myslide{Control Verbs Are Not Transparent }


\begin{itemize}
\item They never take dummies or idiom chunks as 
  subjects.
  \begin{xlisti}
    \ex *\eng{There try to be bugs in my program}
    \ex *\eng{It tries to upset me that the Giants lost}
    \ex *\eng{Advantage tries to be taken of tourists}
  \end{xlisti}
\item Passivizing the complement’s verb changes the truth 
conditions.
\begin{xlisti}
    \ex \eng{The police tried to arrest disruptive demonstrators}
    \trans $\not=$
    \ex \eng{Disruptive demonstrators tried to be arrested by the police}
  \end{xlisti}
\end{itemize}

\myslide{A New Type}
\begin{center} \small
\txx{Subject-Control Verb Lexeme} (\val{scv-lxm})\\
  \begin{avm}
     \[ arg-st & \< NP$_i$,
      \[ 
      val & \[ spr & \< NP$_i$ \> \\
      comps & \< ~ \> \] \\
      sem & \[ index & \OV{s} \] \] \> \\
      sem & \[ restr & \< \[ arg & s \] \> \] \]
  \end{avm}
\end{center}

\begin{itemize}\addtolength{\itemsep}{-2ex}
\item This differs from \val{srv-lxm} in that the first argument and the
   \ft{spr} of the second argument are coindexed, not tagged (re-entrant). 
\item This means that they only need to share \ft{index} values, but may
   differ on other features
\item And the first argument -- the subject -- must have an \ft{index}
  value, so it cannot be non-referential
\end{itemize}

\myslide{The lexical entry for \lex{try}}

\bigskip
\begin{center}
    \begin{avm}
      \< \textnormal{try}, \[{\tiny \val{scv-lxm}}\\[-3ex]
      syn &  \[ arg-st & \< NP$_i$,
                  VP \[ inf & $+$  \]  \] \> \\
      sem & \[ index & s$_1$ \\ 
               restr & \< \[ reln & \textbf{try} \\ 
                             sit & s$_1$ \\
                             trier & i \] \> \] \] \>
    \end{avm}
\end{center}


Note that the subject (NP$_i$) plays a semantic role with 
respect to the verb, namely the \ft{trier}.

\myslide{\lex{try} with Inherited Information}

\begin{tabular}[t]{cc}
\begin{minipage}{0.5\linewidth}
  \footnotesize
    \begin{avm}
      \< \textnormal{try}, \[{\tiny \val{scv-lxm}}\\[-3ex]
      syn &  \[ head & \[{\tiny \val{verb}} \\
      pred & $-$ \\
      inf & $-$ \\a
      agr & \@1 \] \] \\
      arg-st & \< NP$_i$,
                  VP \[ inf & $+$ \\ spr  & \< NP$_i$ \> \\ index & s$_2$ \]
      \> \\
      sem & \[ mode & prop \\ index & s$_1$ \\ 
               restr & \< \[ reln & \textbf{try} \\ 
                             sit & s$_1$ \\ 
                             arg & s$_2$\\
                             trier & i  \] \> \] \] \>
    \end{avm}
  \end{minipage}
  &
\begin{minipage}{0.4\linewidth}
\small
Things to Note:
\begin{itemize}
\item The first argument has an index
\item The first argument is coindexed with \ft{spr} of the second argument
\item Both the first and  second arguments play 
semantic roles in the \prd{try} relation
\item Very little had to be stipulated in the entry 
\end{itemize}
\end{minipage}
\end{tabular}

\myslide{Questions}
\begin{itemize}
\item What rules out dummies and idiom chunks as 
subjects of \lex{try}?
\item What accounts for the semantic non-equivalence of 
pairs like the following?
\begin{xlisti}
  \ex \eng{Reporters tried to interview the candidate}
  \ex \eng{The candidate tried to be interviewed by reporters}
\end{xlisti}
\item Why does \lex{continue} behave differently in these 
respects?
\end{itemize}

\myslide{\lex{try} with an active complement}

\begin{center} 
  \scalebox{0.7}{\begin{avmtree}%\renewcommand{\lf}[1]{\br{#1}{}}\avmfont{\sc}
    \br{S\avmbox{A}}{ \br{\avmbox{1} NP$_i$}{\tlf{The police}}
      \br{VP \[ \ft{spr} \< \avmbox{1} \> \]}{
        \br{V \[ \ft{spr} \< \avmbox{1}$_i$ \> \]}{\lf{tried}}
        \br{VP \[ \ft{spr} \< \avmbox{2}$_i$ \> \]}{\br{V \[ \ft{spr} \< \avmbox{2}$_i$ \> \]}{\lf{to}}
                \br{VP \[ \ft{spr} \< \avmbox{2}$_i$ \> \]}{
                  \br{V \[ \ft{spr} \< \avmbox{2}$_i$ \> \]}{\lf{arrest}}
                  \br{NP$_j$}{\tlf{the suspects}}}}}}
\end{avmtree}}
\end{center}

\scalebox{0.7}{\raisebox{12ex}[0ex][0ex]{\small \begin{avm}
  \[ restr & \avmbox{A} \< 
  \[ reln & \textbf{try} \\ sit  & s$_2$ \\ trier & i \\ tried & s$_2$ \], \\
  \[ reln & \textbf{arrest} \\ sit  & s$_1$ \\ arrester & i \\ arrested & j\]
  \ldots \> \]
\end{avm}}}



\myslide{\lex{try} with passive complement}
\begin{center} 
  \scalebox{0.55}{\begin{avmtree}%\renewcommand{\lf}[1]{\br{#1}{}}\avmfont{\sc}
    \br{S\avmbox{A}}{ \br{\avmbox{1} NP$_j$}{\tlf{The suspects}}
      \br{VP \[ \ft{spr} \< \avmbox{1} \> \]}{
        \br{V \[ \ft{spr} \< \avmbox{1}$_j$ \> \]}{\lf{tried}}
        \br{VP \[ \ft{spr} \< \avmbox{2}$_j$ \> \]}{
          \br{V \[ \ft{spr} \< \avmbox{2}$_j$ \> \]}{\lf{to}}
          \br{VP \[ \ft{spr} \< \avmbox{2}$_j$ \> \]}{
            \br{V \[ \ft{spr} \< \avmbox{2}$_j$ \> \]}{\lf{be}}
            \br{VP \[ \ft{spr} \< \avmbox{2}$_j$ \> \]}{
              \br{V \[ \ft{spr} \< \avmbox{2}$_j$ \> \]}{\lf{arrested}}
              \br{NP$_i$}{\br{P$_i$}{\lf{by}}
                \br{NP$_i$}{\br{NOM$_i$}{\tlf{the police}}}}}}}}}

\end{avmtree}}
\end{center}
\scalebox{0.7}{\raisebox{22ex}[0ex][0ex]{\small \begin{avm}
  \[ restr & \avmbox{A} \< 
  \[ reln & \textbf{try} \\ sit  & s$_2$ \\ trier & \OV{j} \\ tried & s$_2$ \], \\
  \[ reln & \textbf{arrest} \\ sit  & s$_1$ \\ arrester & i \\ arrested & j\]
  \ldots \> \]
\end{avm}}}


\myslide{ARG-ST of raising vs control verbs}
\bigskip

\begin{center}
\begin{tabular}{cc}
  \begin{avm}
    \< NP$_i$,\ \ 
    VP \[ inf & $+$ \\ spr  & \< NP$_i$ \> \\ index & s$_2$ \]
    \> 
  \end{avm}& 
  \begin{avm}
    \< \@1 NP,\ \ 
    VP \[ inf & $+$ \\ spr  & \< \@1 \> \\ index & s$_2$ \]
    \> 
  \end{avm} \\
  \\[2ex]
  \txx{Control} & \txx{Raising}
\end{tabular}
\end{center}



\myslide{Raising \& Control in Transformational Grammar}


\begin{itemize}
\item Raising
  \begin{exe}
    \ex    \eng{\rnode{A}{\rule{4em}{1pt}} continue [\rnode{B}{\ul{the dogs}} to bark]}
    
     \ncbar[offsetB=4pt,angleA=-90,nodesep=3pt]{<-}{A}{B}
  \end{exe}
\item Control
  \begin{exe}
    \ex \eng{[the dogs]$_i$ try [NP$_i$  to bark]}
  \end{exe}
  \begin{itemize}
  \item In early TG, the NP got deleted.
  \item In more recent TG, it’s a silent pronoun.
  \end{itemize}
\end{itemize}

\myslide{Problems with the TG Accounts}


\begin{itemize}
\item Details never fully worked out (e.g. where does \lex{to}
come from?)

\item What blocks 
  \begin{xlisti}
    \ex *\eng{The cat continued (for) the dog to bark}
    \ex *\lex{The cat tried (for) the dog to bark}?

  \end{xlisti}

\item Failure of experimental attempts to find evidence for 
psychological reality of these transformations.
\end{itemize}

\myslide{We make another raising/control distinction}

\begin{exe}
  \ex 
  \begin{xlist}
    \ex \eng{I \ul{expected} Leslie to be aggressive.} \hfill \val{orv}
    \ex \eng{I \ul{persuaded} Leslie to be aggressive.} \hfill \val{ocv}
  \end{xlist}
\end{exe}

\begin{center} \small
  \begin{tabular}{cc}
\txx{Object-Raising Verb Lexeme} (\val{orv-lxm}) &
\txx{Object-Control Verb Lexeme} (\val{ocv-lxm}) \\
  \begin{avm}
     \[ arg-st & \< NP, \@1,
                   \[ spr   & \< \@1 \> \\
                      comps & \< ~ \> \\
                      index & \OV{s}  \] \> \\
      sem & \[ restr & \< \[ arg & s \] \> \] \]
  \end{avm}
&    \begin{avm}
      \[ arg-st & \< NP, NP$_i$,
                     \[  spr & \< NP$_i$ \> \\
                        comps & \< ~ \> \\
                        index & \OV{s}  \] \> \\
       sem & \[ restr & \< \[ arg & s \] \> \] \]
   \end{avm}
\end{tabular}
\end{center}
\begin{itemize}
\item The formal distinction is again between tagging and coindexing
\item This time it’s the second argument and the \ft{spr} of the third
  argument.
\end{itemize}

\myslide{Example \typ{orv-lxm} and \typ{ocv-lxm Entries}}

\begin{center} \small
  \begin{tabular}{cc}
\txx{Object-Raising Verb Lexeme} (\val{orv-lxm}) &
\txx{Object-Control Verb Lexeme} (\val{ocv-lxm}) \\
  \begin{avm} \tiny
     \< \textnormal{expect}, \ 
     \[\val{\tiny orv-lxm} \\
     arg-st & \< NP$_j$, X, VP \[ inf & $+$ \] \> \\
      sem & \[ index & s \\
               restr & \< \[ reln & \prd{expect} \\
                             sit & s \\
                             expecter & j  \] \> \] \] \>
  \end{avm}
& \begin{avm} \tiny
     \< \textnormal{persuade}, \ 
     \[\val{\tiny orv-lxm} \\
     arg-st & \< NP$_j$, NP$_i$, VP \[ inf & $+$ \] \> \\
      sem & \[ index & s \\
               restr & \< \[ reln & \prd{persuade} \\
                             sit & s \\
                             persuader & j\\
                             persuadee & i \] \> \] \] \>
  \end{avm} 
\end{tabular}
\end{center}



\begin{itemize}
\item Note that the \val{orv-lxm} \prd{persuade} relation has three arguments,
  but the \prd{expect} relation has only two
\item And the object’s \ft{index} plays a role in the \prd{persuade} relation,
  but not in the \prd{expect} relation
\end{itemize}

% \myslide{Justifying the difference between expect and persuade (Prob. 12.4)}

% Construct examples of each of the following four 
% types which show a contrast between expect and 
% persuade:
% \begin{enumerate}
% \item Examples with dummy \lex{there}
% \item Examples with dummy \lex{it}
% \item Examples with idiom chunks
% \item Examples of relevant pairs of sentences containing 
% active and passive complements.  Indicate whether 
% they are or are not paraphrases of each other. 
% \end{enumerate}

\myslide{P1: Classifying Verbs}
\MyLogo{Based on  Chapter 12, Problem 1, Sag, Wasow and Bender (2003)}

\noindent Classify the following verbs as raising or control:
%Give as many distinct tests as you can to justify your
%classifications, constructing three or four interestingly different
%examples for each verb.

\begin{itemize}
\item \lex{tend, decide, manage, fail, happen}
\end{itemize}

\noindent
Justify your classification by applying each of the following four
tests to each verb.  Show your work by providing relevant examples
and indicating their grammaticality. 

\begin{itemize} 
\item[(i)] Can the verb take a  dummy \lex{there} subject if and only if its complement selects for a dummy \lex{there} subject?

\item[(ii)] Can the verb take a dummy \lex{it} subject if and only if
  its complement selects for a dummy \lex{it} subject?

\item[(iii)] Can the verb take an idiom chunk subject if and only
if the rest of the idiom is in its complement?

\item[(iv)] Do pairs of sentences containing active and 
passive complements to the verb end up being  paraphrases 
of each other?
\end{itemize}

\noindent
Make sure to restrict your attention to cases of the form: \hbox{NP V
  {\it to} VP}. That is, ignore cases like \eng{Kim manages a store},
\eng{Alex failed physics}, and any other valence that doesn't resemble
the \lex{continue} vs.\ \lex{try} pattern.

\myslide{P2: Classifying \index{adjective}Adjectives}
\MyLogo{Based on  Chapter 12, Problem 2, Sag, Wasow and Bender (2003)}
Classify the following adjectives as \index{raising!adjective}raising or 
\index{control!adjective}control:
%Justify your classifications, constructing three or four
%interestingly different relevant examples for each adjective.

\begin{itemize}
\item \lex{ anxious, apt, bound, certain, lucky}
\end{itemize}

\noindent
Justify your classification by providing each of the four types
of data discussed in the previous problem for each adjective.

Make sure to restrict your attention to cases of the form: \hbox{NP
  \lex{be} Adj \lex{to} VP}. That is, ignore cases like \eng{Kim is
  anxious about the exam}, \eng{Carrie is certain of the answer}, and
any other valence that doesn't resemble the \lex{likely} vs.\
\lex{eager} pattern.

\myslide{P3: \lex{expect} vs.\ \lex{persuade}}
\MyLogo{Based on  Chapter 12, Problem 4, Sag, Wasow and Bender (2003)}

Construct the arguments that underlie the proposed distinction between
\typ{orv-lxm} and \typ{ocv-lxm}.

Construct examples of each of the following four types which 
show a contrast between \lex{expect} and \lex{persuade}.  Explain how
the contrasts are accounted for by the differences in the types
\typ{orv-lxm} and \typ{ocv-lxm} and/or the lexical entries for \lex{expect}
and \lex{persuade}.

\begin{itemize} \addtolength{\itemsep}{-1ex}
\item[(i)] Examples with dummy \lex{there}.

\item[(ii)] Examples with dummy \lex{it}.

\item[(iii)] Examples with idiom chunks.

\item[(iv)] Examples of relevant pairs of sentences containing 
active and passive complements.  Indicate whether they are or are
not paraphrases of each other.
\end{itemize}



% \noindent Again, make sure you ignore all irrelevant uses of
% these verbs, including cases of 
% %finite 
% CP complements, e.g.\ \eng{persuade NP that ...} or
% \eng{expect that ...} and anything else not directly relevant (\eng{I
%   expect to go}, \eng{I am expecting Kim}, \eng{She is expecting}, and
% so forth).

\myslide{P4: A Type for Existential \lex{be}}
\MyLogo{Based on  Chapter 12, Problem 6, Sag, Wasow and Bender (2003)}

The \lex{be} that takes \lex{there} as its subject wasn't given a true
lexical type in Chapter 11 (Sag, Wasow and Bender, 2003), because no
suitable type had been introduced.  One of the types in this chapter
will do, if we make some of its constraints defeasible.

\begin{itemize}
\item[A.] Which of the types introduced in this chapter comes
closest to being consistent with the constraints on \lex{there}-taking
\lex{be}?
\item[B.] Rewrite that type indicating which constraints must
be made defeasible.
\item[C.] Give a stream-lined lexical entry for the \lex{there}-taking
\lex{be} which stipulates only those constraints which are truly
idiosyncratic to the lexeme.
\end{itemize}




\myslide{Overview}
\MyLogo{Sag, Wasow and Bender (2003) --- Chapter 12}
\begin{itemize}
\item Intro to topic
\item Infinitival \lex{to}
\item (Subject) raising verbs
\item (Subject) control verbs
\item Raising/control in TG
\item Object raising and object control
\end{itemize}



%\input{lec-10-control-q}

\end{document}

%%% Local Variables: 
%%% coding: utf-8
%%% mode: latex
%%% TeX-PDF-mode: t
%%% TeX-engine: xetex
%%% End: 
