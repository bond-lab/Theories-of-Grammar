\documentclass[a4paper,landscape,headrule,footrule]{foils}
%%
%%% macros for Theories of Grammar
%%%
\usepackage{polyglossia}
\setdefaultlanguage{english}
%\setmainfont{TeX Gyre Pagella}


\newcommand{\logo}{~}
\newcommand{\header}[3]{%
\title{\vspace*{-2ex} \large HG4041 Theories of Grammar
\\[2ex] \Large  \emp{#2} \\ \emp{#3}}
\author{\blu{Francis Bond}   \\ 
\normalsize  \textbf{Division of Linguistics and Multilingual Studies}\\
\normalsize  \url{http://www3.ntu.edu.sg/home/fcbond/}\\
\normalsize  \texttt{bond@ieee.org}}
\MyLogo{HG4041 (2020)}
\renewcommand{\logo}{#2}
\hypersetup{
   pdfinfo={
     Author={Francis Bond},
     Title={#1: #2},
     Subject={HG4041: Theories of Grammar},
     Keywords={Syntax, Semantics, HPSG, Unification, Constructions},
     License={CC BY 4.0}
   }
 }

\date{#1
  \\ Location: LHN-TR+36}
}
\usepackage[hidelinks]{hyperref}




\usepackage{xcolor}
\usepackage{graphicx}
\newcommand{\blu}[1]{\textcolor{blue}{#1}}
\newcommand{\grn}[1]{\textcolor{green}{#1}}
\newcommand{\hide}[1]{\textcolor{white}{#1}}
\newcommand{\emp}[1]{\textcolor{red}{#1}}
\newcommand{\txx}[1]{\textbf{\textcolor{blue}{#1}}}
\newcommand{\lex}[1]{\textbf{\mtcitestyle{#1}}}

\usepackage{pifont}
\renewcommand{\labelitemi}{\textcolor{violet}{\ding{227}}}
\renewcommand{\labelitemii}{\textcolor{purple}{\ding{226}}}

\newcommand{\subhead}[1]{\noindent\textbf{#1}\\[5mm]}

\newcommand{\Bad}{\emp{\raisebox{0.15ex}{\ensuremath{\mathbf{\otimes}}}}}
\newcommand{\bad}[1]{*\eng{#1}}

\newcommand{\com}[1]{\hfill (\emp{#1})}%

\usepackage{relsize,xspace}
\newcommand{\into}{\ensuremath{\rightarrow}\xspace}
\newcommand{\tot}{\ensuremath{\leftrightarrow}\xspace}
\usepackage{url}
\newcommand{\lurl}[1]{\MyLogo{\url{#1}}}

\usepackage{mygb4e}
\newcommand{\lx}[1]{\textbf{\mtciteform{#1}}}
\newcommand{\ix}{\ex\slshape}
\let\eachwordone=\slshape



\newcommand{\ent}{\ensuremath{\Rightarrow}\xspace}
\newcommand{\ngv}{\ensuremath{\not\Rightarrow}\xspace}
%\usepackage{times}
%\usepackage{nttfoilhead}
\newcommand{\myslide}[1]{\foilhead[-25mm]{\raisebox{12mm}[0mm]{\emp{#1}}}\MyLogo{\logo}}
\newcommand{\myslider}[1]{\rotatefoilhead[-25mm]{\raisebox{12mm}[0mm]{\emp{#1}}}}
%\newcommand{\myslider}[1]{\rotatefoilhead{\raisebox{-8mm}{\emp{#1}}}}

\newcommand{\section}[1]{\myslide{}{\begin{center}\Huge \emp{#1}\end{center}}}



\usepackage[lyons,j,e,k]{mtg2e}
%\renewcommand{\mtcitestyle}[1]{\textcolor{-red!75!green!50}{\textsl{#1}}}
\renewcommand{\mtcitestyle}[1]{\textcolor{teal}{\textsl{#1}}}
\newcommand{\iz}[1]{\texttt{\textup{#1}}}
\newcommand{\gm}{\textsc}
\usepackage[normalem]{ulem}
\newcommand{\ul}{\uline}
\newcommand{\ull}{\uuline}
\newcommand{\wl}{\uwave}
\newcommand{\vs}{\ensuremath{\Leftrightarrow}~}
%%%
%%% Bibliography
%%%
\usepackage{natbib}
%\usepackage{url}
\usepackage{bibentry}


%%% From Tim
\newcommand{\WMngram}[1][]{$n$-gram#1\xspace}
\newcommand{\infers}{$\rightarrow$\xspace}

\usepackage[utf8]{inputenc}

\usepackage{rtrees,qtree}
\renewcommand{\lf}[1]{\br{#1}{}}
\usepackage{avm}
%\avmoptions{topleft,center}
\newcommand{\ft}[1]{\textsc{#1}}
\renewcommand{\val}[1]{\textit{#1}}
\newcommand{\typ}[1]{\textit{#1}}
\newcommand{\prd}[1]{\textbf{#1}}
\avmfont{\sc}
\avmvalfont{\it}
\avmsortfont{\smaller[2] \it}
\usepackage{multicol}
\newcommand{\blank}{\rule{3em}{1pt}\xspace}

% \usepackage{pst-node}
\newcommand{\OV}[1]{\ovalnode[linestyle=dotted,linecolor=red]{A}{#1}}
\newcommand{\OVB}[1]{\ovalnode[linestyle=dotted,linecolor=blue]{A}{#1}}

%%% From CSLI book
\newcommand{\mc}{\multicolumn}
\newcommand{\HD}{\textbf{H}\xspace}
\newcommand{\el}{\< \>}
\makeatother
\long\def\smalltree#1{\leavevmode{\def\\{\cr\noalign{\vskip12pt}}%
\def\mc##1##2{\multispan{##1}{\hfil##2\hfil}}%
\tabskip=1em%
\hbox{\vtop{\halign{&\hfil##\hfil\cr
#1\crcr}}}}}
\makeatletter

\newcommand{\A}{\noindent\textbf{A}: }
\newcommand{\Q}{\noindent\textbf{Q}: }
%\newcommand{\C}{\noindent\textbf{C}: }



\usepackage{times}
\usepackage{bchart}
%\usepackage{pgfplots}
\avmfont{\sc}
\begin{document}
\header{Lecture 8}{The Passive Construction}{}
\maketitle




\myslide{Overview}
\MyLogo{Sag, Wasow and Bender (2003) --- Chapter 10}

\begin{itemize}
\item Passive
  \begin{itemize}
  \item Arguments for lexicalist account
  \item Details of our analysis
  \end{itemize}
\item Questions
\end{itemize}

\myslide{The Passive in Transformational Grammar}

\begin{itemize}
\item Passive was the paradigmatic transformation in early TG.
\item Motivations
  \begin{itemize}
  \item Near paraphrase of active/passive pairs.
  \item Simplified statement of cooccurrence restrictions.
    \begin{itemize}
    \item E.g. \lex{devour} must be followed by an NP, \lex{put} by NP-PP
    \item Such restrictions refer to pre-transformational (\txx{deep}) structure.
    \end{itemize}
  \item Intuition that active forms were more basic, in some sense. 
  \end{itemize}
\item Its formulation was complex:  
  \begin{itemize}
  \item Promote object
  \item Demote subject, inserting \lex{by}
  \item Insert appropriate form of \lex{be}, changing main verb to a participle.
  \end{itemize}
\end{itemize}

\myslide{But transforming whole sentences is overkill}


\begin{itemize}
\item Passive sentences look an awful lot like some actives:  
  \begin{exe}
    \ex \eng{The cat was chased by the dog}
    \ex \eng{The cat was lying by the door}   
  \end{exe}
\item Passives occur without \lex{be} and without the \lex{by} phrase:
  \begin{exe}
    \ex \eng{Cats chased by dogs usually get away.}
    \ex \eng{My cat was attacked.}
\end{exe}
\end{itemize}


\myslide{So a lexical analysis seems called for}


\begin{itemize}
\item What really changes are the verb’s form and its 
cooccurrence restrictions (that is, its valence).
\item There are lexical exceptions
  \begin{itemize}
  \item Negative:  
    \begin{exe}
      \ex \eng{Pat resembles Bo} 
      \ex  *\eng{Bo is resembled by Pat}
      \ex \eng{That look suits you}
      \ex *\eng{You are suited  by that look}
    \end{exe}
  \item Positive
    \begin{exe}
      \ex \eng{Chris is rumored to be a spy}
      \ex *\eng{They rumor Chris to be a spy}
    \end{exe}
  \end{itemize}
\end{itemize}  

\myslide{We posit a lexical rule}

\begin{itemize}
\item Why not just list passive participles individually?
  \begin{itemize}
  \item To avoid redundancy
  \item To capture productivity (for example?)
  \end{itemize}
\item We make it a derivational (lexeme-to-lexeme) rule.  
  \\ Why?
  \begin{itemize}
  \item Our constraints on lexeme-to-word rules wouldn’t allow 
    us to make Passive one: we change the syntax between input and output.
  \end{itemize}
\end{itemize}  

\myslide{The Passive Lexical Rule}
\begin{avm}\avml
\[{\it d-rule}\\
INPUT & \< {\@1} , \[{\it tv-lxm}\\
ARG-ST & \q< \[ INDEX & $i$ \] \q>\ \ $\oplus$\ \ \@{A}\]\>\\
OUTPUT & \< F$_{PSP}$({\@1}) , 
\[{\it part-lxm}\\
SYN & \[HEAD & \[ FORM &  pass \] \] \\
% PRED &  +
ARG-ST &  \@{A}\ \ 
$\oplus$\ \ \< \(\avml\hfil PP\\[-1ex]
\[ FORM & by\\
INDEX & $i$\]\avmr \) \> \] \>\]\avmr\end{avm}


\myslide{Questions}

\begin{footnotesize}
  \begin{avm}\avml
    \[{\it d-rule}\\
    INPUT & \< {\@1} , \[{\it tv-lxm}\\
    ARG-ST & \q< \[ INDEX & $i$ \] \q>\ \ $\oplus$\ \ \@{A}\]\>\\
    OUTPUT & \< F$_{PSP}$({\@1}) ,
    \[{\it part-lxm}\\
    SYN & \[HEAD & \[ FORM &  pass \] \] \\
    % PRED & +
    ARG-ST & \@{A}\ \
    $\oplus$\ \ \< \(\avml\hfil PP\\[-1ex]
    \[ FORM & by\\
    INDEX & $i$\]\avmr \) \> \] \>\]\avmr\end{avm}
\end{footnotesize}
\begin{itemize}\addtolength{\itemsep}{-1ex}
\item Why is the morphological function F$_{PSP}$?
\item Why do we have a separate FORM value {\it pass}?  
  Why not just  {\it psp}?
\item Is the \lex{by}-phrase argument-marking  or predicational?
\end{itemize}

\myslide{More Questions}

\begin{footnotesize}
  \begin{avm}\avml
    \[{\it d-rule}\\
    INPUT & \< {\@1} , \[{\it tv-lxm}\\
    ARG-ST & \q< \[ INDEX & $i$ \] \q>\ \ $\oplus$\ \ \@{A}\]\>\\
    OUTPUT & \< F$_{PSP}$({\@1}) ,
    \[{\it part-lxm}\\
    SYN & \[HEAD & \[ FORM &  pass \] \] \\
    % PRED & +
    ARG-ST & \@{A}\ \
    $\oplus$\ \ \< \(\avml\hfil PP\\[-1ex]
    \[ FORM & by\\
    INDEX & $i$\]\avmr \) \> \] \>\]\avmr\end{avm}
\end{footnotesize}
\begin{itemize}
\item What makes the object turn into the subject? 
\item Why is the type of the input \val{tv-lxm}?  
\item What would happen if it were just \val{verb-lxm}?
\end{itemize}

\myslide{Intransitives have passives in German}

\begin{exe}
\ex \gll In der Küche  wird nicht getanzt. \\
in the kitchen   is     not   danced \\
\trans ‘There is no dancing in the kitchen.’
\end{exe}

The exact analysis for such examples is debatable, but German, like
many other languages, allows passives of intransitives, as would be
allowed by our analysis if the input type in the Passive LR is
\val{verb-lxm} (although the linking needs more work to get right).

\myslide{Intransitives have passives in Japanese}

Japanese also allows passives of intransitives, although with very
different properties.

\begin{exe}
\ex \gll otoosan-ga shin-da  \\
         father-\textsc{nom} died \\
\trans ‘My father died.’
\ex \gll watashi-ha otoosan-ni shin-areta  \\
         me-\textsc{top} father-\textsc{dat} died \\
\trans `My father died on me.' lit: `As for me, died by my father'
\end{exe}

We need a separate (but related) rule for this.


\myslide{Passive Input and Output}

\begin{tabular}{cc}
  This entry & also gets you this \\
\begin{tiny}
\begin{avm} 
  \< \textnormal{love},
  \[{\it stv-lxm}\\
  SYN & \[HEAD & \[{\it verb}\\
  AGR& {\@1}\]\\
  VAL & \[SPR & \q< \[AGR\ {\@1}\]\ \q>\]\]\\
  ARG-ST & \q< NP$_i$ , NP$_j$ \q>\\
  SEM & \[INDEX & $s$\\
  RESTR & \< \[RELN & {\bf love}\\
  SIT & \ \ $s$\\
  LOVER & \ \ $i$ \\
  LOVED & \ \ $j$ \] \>\] \]\ \>
\end{avm} 
\end{tiny}
&
\begin{tiny}
\begin{avm} 
  \< \textnormal{loved}, \[{\it part-lxm}\\
  SYN & \[HEAD & \[{\it verb}\\
  AGR& {\@1}\\                                
  FORM & pass\]\\
  VAL & \[SPR & \q< \[AGR& {\@1}\]\ \q>\]\]\\
  ARG-ST & \< NP$_j$ \( ,\ \avml\hfil PP\\[-1ex]
  \[FORM & by\\
  INDEX & $i$\]\avmr\ \) \>\\
  SEM & \[INDEX & $s$\\
  RESTR & \< \[RELN & {\bf love}\\
  SIT & \ \ $s$\\
  LOVER & \ \ $i$ \\
  LOVED & \ \ $j$ \] \>\] \]\ \>
\end{avm} 
\end{tiny}
\end{tabular}

Through the magic of the Passive-Lexical rule!

\myslide{And also this}

\begin{tiny}
   \begin{avm}      
     \< \textnormal{loved}, 
     \[{\it word}\\
     SYN & \[HEAD & \[{\it verb}\\
     AGR& {\@1}\\                                
     FORM & pass\]\\
     VAL & \[SPR & \q< {\@2}[AGR\ {\@1}{\,}] \q>\\
     COMPS & {\@{B}}\]\]\\
     ARG-ST & \< {\@2}NP$_j$ \>\ $\oplus$\ {\@{B}} \< \( ,\ \avml\hfil PP\\[-1ex]
     \[FORM & by\\
     INDEX & $i$\]\avmr\ \) \>\\
     SEM & \[INDEX & $s$\\
     RESTR & \< \[RELN & {\bf love}\\
     SIT & \ \ $s$\\
     LOVER & \ \ $i$ \\
     LOVED & \ \ $j$ \] \>\] \]\ \>
   \end{avm} 
\end{tiny}

Through the magic of the Constant Lexeme Lexical Rule!

\myslide{The \lex{be} that Occurs with Most Passives}

\begin{avm}
  \< \textnormal{be},\ \[{\it be-lxm}\\
  ARG-ST & \< {\@1} , \avml\[SYN & \[HEAD & \[{\it verb}\\
  FORM \ pass \]\\
  VAL & \[SPR & \q< {\@1} \q>\\
  COMPS & \el \]\]\\
  SEM & \[INDEX & $s$ \ \]\]\avmr \> \\ %{\@3}
  SEM & \[INDEX & \ $s$\\ %{\@3}\\
  RESTR & {\el} \] \] \>
\end{avm}


\myslide{Questions About the Entry for \lex{be}}

\begin{tiny}
  \begin{avm}
  \< \textnormal{be},\ \[{\it be-lxm}\\
  ARG-ST & \< {\@1} , \avml\[SYN & \[HEAD & \[{\it verb}\\
  FORM \ pass \]\\
  VAL & \[SPR & \q< {\@1} \q>\\
  COMPS & \el \]\]\\
  SEM & \[INDEX & $s$ \ \]\]\avmr \> \\ %{\@3}
  SEM & \[INDEX & \ $s$\\ %{\@3}\\
  RESTR & {\el} \] \] \>
\end{avm}
\end{tiny}

\begin{itemize}\addtolength{\itemsep}{-1ex}
\item Why doesn’t it include valence features?
\item What is the category of its complement (i.e. its 2nd argument)?
\item What is its contribution to the semantics of the sentences it 
appears in?
\item Why is the first argument tagged as identical to the second 
argument’s SPR value?
\end{itemize}

\myslide{Passive tree}
\begin{tree}
  \br{S}{
    \br{\avmbox{1}NP}{Kim}
    \br{VP\makebox[0mm][l]{$[$ SPR $\langle \avmbox{1} \rangle ]$}}{
      \br{V\tiny \makebox[0mm][l]{$[$ SPR $\langle \avmbox{1} \rangle ]$}}{is}
      \br{VP\makebox[0mm][l]{$[$ SPR $\langle \avmbox{1} \rangle ]$}}{
        \br{V}{loved}
        \br{PP}{
          \br{P}{by}
          \br{NP}{everyone}}}}}
\end{tree}

\begin{itemize}
\item Which rule licenses each node?
\item What is the SPR value of the upper VP?
\item What is the SPR value of the  lower VP?
\item What is the SPR value of \eng{is}?
\end{itemize}

\myslide{More Questions}
\begin{itemize}
\item Why do we get this?
  \begin{exe}
    \ex \eng{They are noticed by everyone}
  \end{exe}
\item Why don't we get this?
 \begin{exe}
    \ex *\eng{Them are noticed by everyone?}
  \end{exe}
\item Why don’t we get this? 
\begin{exe}
    \ex *\eng{They is noticed by everyone}
  \end{exe}
\item What would facts like these entail for a transformational 
analysis?
\end{itemize}

\myslide{Overview}
\MyLogo{Sag, Wasow and Bender (2003) --- Chapter 10}

\begin{itemize}
\item Passive
\item Arguments for lexicalist account
\item Details of our analysis
\item Questions
\end{itemize}

\myslide{P1: Passives and Binding}
\MyLogo{Based on  Chapter 10, Problem 1, Sag, Wasow and Bender (2003)}

The analysis of passive makes some predictions about binding possibilities
in passive sentences. Consider the following data:

\begin{exe}
\exi{(i)} \eng{She$_i$ was introduced to herself$_i$ (by the doctor).}
\exi{(ii)} \bad \eng{She$_i$ was introduced to her$_i$ (by the doctor). }
\exi{(iii)} \eng{The barber$_i$ was shaved (only) by himself$_i$.}
\exi{(iv)} \bad \eng{The barber$_i$ was shaved (only) by him$_i$. }
\exi{(v)} \eng{The students$_i$ were introduced to each other$_i$ (by Leslie).}
\exi{(vi)} \bad \eng{The students$_i$ were introduced to them$_i$ (by Leslie). }
\exi{(vii)} \eng{Kim was introduced to Larry$_i$ by himself$_i$. }
\exi{(viii)} \bad \eng{Kim was introduced to himself$_i$ by Larry$_i$. }
\end{exe}

\noindent
Assuming that \lex{to} and \lex{by} in these examples are uniformly
treated as  argument-marking prepositions, 
% is the data 
does the treatment of passives sketched in the text
correctly predict the judgements in (i)--(viii)?
If so, explain why; if not, discuss the inadequacy of the analysis in
precise terms. 

An ideal answer should examine each one of the eight sentences and determine
if it follows the binding principles. That is, the analysis of passive
presented in this chapter associates a particular ARG-ST list with the passive
verb form in each example and these lists interact with the binding principles
of Chapter 7 to make predictions. Check to see if the predictions made by our
Binding Theory match the grammaticality judgements given. 
%Note where our theory accurately predicts or fails to predict the data given.

\myslide{P3: The Dative Alternation} 
\MyLogo{Based on Chapter 10,  Problem 3, Sag, Wasow and Bender (2003)} 

The \txx{dative alternation} could also be described by a lexical
rule: that is, a rule that produces entries for verbs that appear in
both of the valence patterns exemplified in (i) and (ii):

\begin{exe}
\exi{(i)} \eng{Dale \{gave/handed/sold/loaned/mailed\} Merle a book.}
\exi{(ii)} \eng{Dale \{gave/handed/sold/loaned/mailed\}  a book to Merle.}
\end{exe}

\begin{itemize}
\item[A.] Is this alternation productive?  Justify your answer with at
  least two examples.
\item[B.] Formulate a \index{lexical rule} lexical rule for the dative
  alternation. 
\newpage
\item[C.] Show how your rule interacts with the Passive Lexical Rule
  to make possible the generation of both (iii) and (iv).  Your answer
  should include ARG-ST values showing the effect of applying the
  rules. 
\begin{exe}
\exi{(iii)} \eng{Merle was handed a book by Dale.}
\exi{(iv)}  \eng{A book was handed to Merle by Dale.}
\end{exe}
\item[D.] Explain why your rule correctly fails to license (v) (or,
more precisely, fails to license~(v) with the sensible meaning
that the book was the thing handed to Merle).
\begin{exe}
\exi{(v)} ?*\eng{A book was handed Merle by Dale.}
\end{exe}
\end{itemize}

\end{document}

%%% Local Variables: 
%%% coding: utf-8
%%% mode: latex
%%% TeX-PDF-mode: t
%%% TeX-engine: xetex
%%% End: 
