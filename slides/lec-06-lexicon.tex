\documentclass[a4paper,landscape,headrule,footrule]{foils}

%%
%%% macros for Theories of Grammar
%%%
\usepackage{polyglossia}
\setdefaultlanguage{english}
%\setmainfont{TeX Gyre Pagella}


\newcommand{\logo}{~}
\newcommand{\header}[3]{%
\title{\vspace*{-2ex} \large HG4041 Theories of Grammar
\\[2ex] \Large  \emp{#2} \\ \emp{#3}}
\author{\blu{Francis Bond}   \\ 
\normalsize  \textbf{Division of Linguistics and Multilingual Studies}\\
\normalsize  \url{http://www3.ntu.edu.sg/home/fcbond/}\\
\normalsize  \texttt{bond@ieee.org}}
\MyLogo{HG4041 (2020)}
\renewcommand{\logo}{#2}
\hypersetup{
   pdfinfo={
     Author={Francis Bond},
     Title={#1: #2},
     Subject={HG4041: Theories of Grammar},
     Keywords={Syntax, Semantics, HPSG, Unification, Constructions},
     License={CC BY 4.0}
   }
 }

\date{#1
  \\ Location: LHN-TR+36}
}
\usepackage[hidelinks]{hyperref}




\usepackage{xcolor}
\usepackage{graphicx}
\newcommand{\blu}[1]{\textcolor{blue}{#1}}
\newcommand{\grn}[1]{\textcolor{green}{#1}}
\newcommand{\hide}[1]{\textcolor{white}{#1}}
\newcommand{\emp}[1]{\textcolor{red}{#1}}
\newcommand{\txx}[1]{\textbf{\textcolor{blue}{#1}}}
\newcommand{\lex}[1]{\textbf{\mtcitestyle{#1}}}

\usepackage{pifont}
\renewcommand{\labelitemi}{\textcolor{violet}{\ding{227}}}
\renewcommand{\labelitemii}{\textcolor{purple}{\ding{226}}}

\newcommand{\subhead}[1]{\noindent\textbf{#1}\\[5mm]}

\newcommand{\Bad}{\emp{\raisebox{0.15ex}{\ensuremath{\mathbf{\otimes}}}}}
\newcommand{\bad}[1]{*\eng{#1}}

\newcommand{\com}[1]{\hfill (\emp{#1})}%

\usepackage{relsize,xspace}
\newcommand{\into}{\ensuremath{\rightarrow}\xspace}
\newcommand{\tot}{\ensuremath{\leftrightarrow}\xspace}
\usepackage{url}
\newcommand{\lurl}[1]{\MyLogo{\url{#1}}}

\usepackage{mygb4e}
\newcommand{\lx}[1]{\textbf{\mtciteform{#1}}}
\newcommand{\ix}{\ex\slshape}
\let\eachwordone=\slshape



\newcommand{\ent}{\ensuremath{\Rightarrow}\xspace}
\newcommand{\ngv}{\ensuremath{\not\Rightarrow}\xspace}
%\usepackage{times}
%\usepackage{nttfoilhead}
\newcommand{\myslide}[1]{\foilhead[-25mm]{\raisebox{12mm}[0mm]{\emp{#1}}}\MyLogo{\logo}}
\newcommand{\myslider}[1]{\rotatefoilhead[-25mm]{\raisebox{12mm}[0mm]{\emp{#1}}}}
%\newcommand{\myslider}[1]{\rotatefoilhead{\raisebox{-8mm}{\emp{#1}}}}

\newcommand{\section}[1]{\myslide{}{\begin{center}\Huge \emp{#1}\end{center}}}



\usepackage[lyons,j,e,k]{mtg2e}
%\renewcommand{\mtcitestyle}[1]{\textcolor{-red!75!green!50}{\textsl{#1}}}
\renewcommand{\mtcitestyle}[1]{\textcolor{teal}{\textsl{#1}}}
\newcommand{\iz}[1]{\texttt{\textup{#1}}}
\newcommand{\gm}{\textsc}
\usepackage[normalem]{ulem}
\newcommand{\ul}{\uline}
\newcommand{\ull}{\uuline}
\newcommand{\wl}{\uwave}
\newcommand{\vs}{\ensuremath{\Leftrightarrow}~}
%%%
%%% Bibliography
%%%
\usepackage{natbib}
%\usepackage{url}
\usepackage{bibentry}


%%% From Tim
\newcommand{\WMngram}[1][]{$n$-gram#1\xspace}
\newcommand{\infers}{$\rightarrow$\xspace}

\usepackage[utf8]{inputenc}

\usepackage{rtrees,qtree}
\renewcommand{\lf}[1]{\br{#1}{}}
\usepackage{avm}
%\avmoptions{topleft,center}
\newcommand{\ft}[1]{\textsc{#1}}
\renewcommand{\val}[1]{\textit{#1}}
\newcommand{\typ}[1]{\textit{#1}}
\newcommand{\prd}[1]{\textbf{#1}}
\avmfont{\sc}
\avmvalfont{\it}
\avmsortfont{\smaller[2] \it}
\usepackage{multicol}
\newcommand{\blank}{\rule{3em}{1pt}\xspace}

% \usepackage{pst-node}
\newcommand{\OV}[1]{\ovalnode[linestyle=dotted,linecolor=red]{A}{#1}}
\newcommand{\OVB}[1]{\ovalnode[linestyle=dotted,linecolor=blue]{A}{#1}}

%%% From CSLI book
\newcommand{\mc}{\multicolumn}
\newcommand{\HD}{\textbf{H}\xspace}
\newcommand{\el}{\< \>}
\makeatother
\long\def\smalltree#1{\leavevmode{\def\\{\cr\noalign{\vskip12pt}}%
\def\mc##1##2{\multispan{##1}{\hfil##2\hfil}}%
\tabskip=1em%
\hbox{\vtop{\halign{&\hfil##\hfil\cr
#1\crcr}}}}}
\makeatletter

\newcommand{\A}{\noindent\textbf{A}: }
\newcommand{\Q}{\noindent\textbf{Q}: }
%\newcommand{\C}{\noindent\textbf{C}: }



\begin{document}
\avmfont{\it}

\header{Lecture 6}{Structure of the lexicon}{}
\maketitle


\myslide{Lexical Types and Rules}

We will use slides from Emily Bender:


\begin{itemize}
\item \href{ch08a-uw.pdf}{Lexical Types}
\item \href{ch08a-uw.pdf}{Lexical Rules}
\end{itemize}


\myslide{P1: \eng{'s} and the SHAC}
\MyLogo{Based on  Chapter 8 Problem 1, Sag, Wasow and Bender (2003)}

The name `Specifier-Head Agreement Constraint' suggests that heads
always agree with their specifiers.  Examples like \eng{Pat's parents}
and \eng{the children's game} look like counterexamples:  in both
cases, the possessive NP in the DP that functions as the specifier of
the noun differs in number from that noun.

Explain why these are not really counterexamples, given our
formulation of SHAC as a type constraint, together with the analysis
of possessives developed in Problem 4 of Chapter 6.  {\sl [Hint: The
  fact that \eng{'s} is the head of the DP is crucial.]}

\myslide{P2: Plural and Mass NPs Without Specifiers}
\MyLogo{Based on  Chapter 8, Problem 2, Sag, Wasow and Bender (2003)}

There is a problem with our treatment of  common nouns.  
The type  \val{cn-lxm} requires common nouns to have nonempty SPR lists, and this
requirement is preserved in the  Plural Noun Lexical Rule.
Similarly, the type \val{massn-lxm} inherits the constraint on the SPR,
and this constraint is preserved when these nouns undergo the inflectional
rules.
This treatment makes the wrong predictions:
%But
specifiers are optional for plural nouns and mass nouns.
\begin{itemize}
\item[A.] 
%Give at least three examples showing the \index{optionality of specifier} optionality of 
%specifiers for \index{noun (N)!plural noun} plural and \index{noun (N)!mass noun} mass nouns.
Give examples showing, for one plural noun 
and one \index{noun (N)!mass noun} mass noun, that
the specifier is optional (i.e.\ permitted but not obligatory).
\end{itemize}

\newpage
Two \index{noun phrase (NP)!without determiner|(} obvious approaches to this problem are the following:
%\pagebreak
\begin{itemize}
\item[(i)] allow empty SPR lists in the lexical entries for plural
and mass nouns; or 
\item[(ii)] introduce a new  \index{phrase structure!rule (PS rule)}grammar rule to account for NPs with
plural or mass heads and no specifiers.
\end{itemize}

\noindent
Alternative (i) would involve modifying the Plural Noun Lexical Rule,
%as well as introducing a new subtype of \val{cn-lxm} for mass nouns.
as well as the type \val{massn-lxm} to make the first member of
the ARG-ST list optional.\footnote{This would require making the
constraint on the ARG-ST of \val{cn-lxm} defeasible.}
\index{noun phrase (NP)!without determiner|)}  

The rule in alternative (ii) is analogous to the \index{Imperative
Rule} Imperative Rule given in Chapter 7, in that it would have only
one constituent on the right hand side, and its function would be to
license a constituent without a specifier, although its daughter has a
nonempty SPR list.

It turns out that alternative (i) makes incorrect predictions about
prenominal modifiers (see Problem~1 of Chapter~5).  We want adjectives
like \eng{cute} to modify plural nouns even when they don't have
specifiers:

\begin{itemize}
\item[(iii)] \eng{Cute puppies make people happy.}
\end{itemize}

\noindent
Under alternative (i), in order to generate (iii), we would have
to allow adjectives like \eng{cute} to modify NPs (i.e.\ expressions
that are \mbox{[SPR $\langle$  $\rangle$]}).  If we do that, however, we have no
way to block (iv):\footnote{There are also technical problems with
making alternative (i) work with the ARP.}

\begin{itemize}
\item[(iv)] \bad{Cute the puppies make people happy.}
\end{itemize}

\newpage
Alternative (ii), on the other hand, would allow \eng{cute}
to always modify a \index{NOM}NOM \mbox{([SPR $\langle$ DP $\rangle$])}
constituent.  A NOM, modified or otherwise, could either be
the daughter of the non-branching rule, or the head daughter of
the \index{Head Specifier Rule@Head-Specifier Rule}Head-Specifier Rule.


\begin{itemize}
%\item[B.]  The ARP does not manage to sufficiently constrain what
%alternative (i) would allow.  Explain why.\smallskip
%
%\noindent
%[{\sl Hint: Assume that alternative
%(i) involves making the \index{determiner (D)} determiner an optional member of the noun's
%ARG-ST. Consider all the ways that an \index{ARGUMENT STRUCTURE (ARG ST)@ARGUMENT-STRUCTURE (ARG-ST)} ARG-ST element could be realized
%on the SPR or COMPS list.  You may also want to construct examples of
%common nouns with complements but no specifiers.}]

\item[B.]  Formulate the rule required for alternative (ii).  \smallskip
%Be as precise as you can.\smallskip%\hfill\break

\noindent
{[{\sl Hint:  The trickiest part is formulating the rule so that it
applies to both plural \index{noun (N)!count noun}count nouns and \index{noun (N)!mass noun}mass nouns, while not applying
to singular count nouns.  You will need to include a disjunction in 
the rule.  The 
%\index{SPR}
SPR list of the \index{head daughter (H)}head daughter is a good place to state
it, since the three types of nouns differ in the requirements they
place on their specifiers.}]}
\index{noun (N)!singular noun}
\index{specifier!optional|)}

%disjunction of plural or mass nouns.  The place to do this is in the
%SPR list of the feature structure description 
%on the right hand side of the rule.}]
\end{itemize}


\myslide{P3: Arguments in Japanese}
\MyLogo{Based on  Chapter 8, Problem 6, Sag, Wasow and Bender (2003)}
As noted \index{Japanese|(}
in Chapter 2, Japanese word order differs from English in a number of ways,
including the fact that it is a \index{Subject Object Verb (SOV)@Subject-Object-Verb (SOV)} `Subject-Object-Verb' (SOV) language.  Here
are a few relevant examples. In the glosses, `{\sc nom}', `{\sc acc}', and 
`{\sc dat}' stand for \index{nominative case} nominative,
\index{accusative case} accusative, and \index{dative case} dative 
case, respectively.  (Note that Japanese has one more case -- dative -- than
English does.  This doesn't have any important effects on the analysis; it
merely requires that we posit one more possible value of \index{CASE}CASE for Japanese
than for English).\footnote{The examples
marked with `*' here
are unacceptable with the indicated meanings.  Some of these might be
well-formed with some other meaning of no direct relevance; others might be
well-formed with special \index{intonation} intonation that we will ignore for present
purposes.}

\begin{exe}
  \ex %一人 の 男 が その 本 を 読んだ 。
  \gll Hitorino  otoko-ga sono  hon-o yonda. \\
 one  man-{\sc nom} that  book-{\sc acc} read.{\sc past} \\
\trans {`One man read that book.'}

[cf.\ *Yonda hitorino otoko-ga sono hon-o. \\ {*}Hitorino otoko-ga yonda
sono hon-o. \\ {*}Otoko-ga hitorino sono hon-o yonda. \\ {*}Hitorino
otoko-ga hon-o sono yonda. \\ {*}Hitorino otoko-ni/-o sono hon-o
yonda. \\ {*}Hitorino otoko-ga sono hon-ga/-ni yonda.]  

\ex 
\gll  Hanako-ga  hon-o  yonda \\
 Hanako-{\sc nom}  book-{\sc acc}  read.{\sc past} \\
\trans {`Hanako read the book(s)'}

[cf.\ *Yonda Hanako-ga hon-o. \\ {*}Hanako-ga yonda hon-o. \\
{*}Hanako-ni/-o hon-o yonda. \\ {*}Hanako-ga hon-ni/-ga yonda.]

\ex \label{5}
\gll   sensei-ga  Taroo-ni  sono   hon-o  ageta \\
 teacher-{\sc nom} Taroo-{\sc dat} that   book-{\sc acc} gave.{\sc past} \\
 \trans {`The teacher(s) gave that book to Taroo'}

[cf.\ *Ageta sensei-ga Taroo-ni sono hon-o. \\ {*}Sensei-ga ageta Taroo-ni
sono hon-o. \\ {*}Sensei-ga Taroo-ni ageta sono hon-o. \\ {*}Sensei-o/-ni
Taroo-ni sono hon-o ageta. \\ {*}Sensei-ga Taroo-ga/-o sono hon-o ageta.\\
{*}Sensei-ga Taroo-ni sono hon-ga/-ni ageta.]

%\item[(iv)]{\shortex{2}
%\eng{Sono hon-ga \it akai}
%{that book-nom  be.red}
%{`That book is red'}
%
%[cf. \ *Akai sono hon-ga.]}

\ex\label{2}
\gll Hanako-ga   kita \\
 Hanako-{\sc nom}   arrive.{\sc past} \\
\trans {`Hanako arrived.'}

[cf. \ *Kita Hanako-ga.]

\end{exe}


									
\noindent
As the contrasting ungrammatical examples show, the verb must appear
in final position in \index{Japanese} Japanese. In addition, we see
that \index{verb (V)} verbs select for NPs of a particular
\index{case} case, much as in English. In the following tasks, assume
that the nouns and verbs of Japanese are \index{inflection!inflected word} inflected words, derived by lexical rule from the appropriate
lexemes.

\begin{itemize}\addtolength{\itemsep}{1ex}

\item[A.] Write \index{Head Specifier Rule@Head-Specifier Rule}Head-Specifier and 
\index{Head Complement Rule@Head-Complement Rule}Head-Complement Rules for
Japanese that account for the data illustrated here.  How are
they different (if at all) from the Head-Specifier and Head-Complement 
Rules for English?
%Given the data illustrated here, how could the 
%Head-Specifier and Head-Complement rules be revised to deal
%with Japanese? Explain the effect of the difference(s) you have posited.

\item[B.] Give the lexical entry for each of the verbs
illustrated in (i)--(iv). 

[{\sl Make sure your entries interact with the rules you
formulated in part (A) to account for the above data.  The data given
permit you to specify only some features; leave others unspecified.
Assume that there is a Past-Tense Verb Lexical Rule (an \val{i-rule})
that relates your lexical entries to the words shown in (i)--(iv).  We
have not provided a hierarchy of lexeme types for Japanese.  You may
either give all relevant constraints directly on the lexical entries,
or posit and use subtypes of \val{lexeme}.  In the latter case, you
must also provide those types.}]
\newpage

\item[C.] Give the lexical entries for the nouns \eng{Taroo} and
\eng{hon}. [{\sl See notes on part (B).}]

\item[D.] Formulate the lexical rule for deriving the inflected
forms ending in \eng{-o} from
the nominal lexemes.

% EB -- they've been doing this over and over again.  Enough already.
%\item[E.] Explain the role of the \index{Head Feature Principle (HFP)} Head Feature Principle 
%in accounting for the case-selecting properties of verbs in
%(i)--(iv).

\end{itemize}


\myslide{Acknowledgments and References}

\begin{itemize}
\item Course design and slides borrow heavily from Emily Bender's course:
\textit{Linguistics 566: Introduction to Syntax for Computational Linguistics}
\\ \url{http://courses.washington.edu/ling566}
\end{itemize}


%\input{lec-05-questions}

\end{document}

%%% Local Variables: 
%%% coding: utf-8
%%% mode: latex
%%% TeX-PDF-mode: t
%%% TeX-engine: xetex
%%% End: 
