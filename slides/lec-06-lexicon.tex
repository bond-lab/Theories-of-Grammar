\documentclass[a4paper,landscape,headrule,footrule,dvips]{foils}

%%
%%% macros for Theories of Grammar
%%%
\usepackage{polyglossia}
\setdefaultlanguage{english}
%\setmainfont{TeX Gyre Pagella}


\newcommand{\logo}{~}
\newcommand{\header}[3]{%
\title{\vspace*{-2ex} \large HG4041 Theories of Grammar
\\[2ex] \Large  \emp{#2} \\ \emp{#3}}
\author{\blu{Francis Bond}   \\ 
\normalsize  \textbf{Division of Linguistics and Multilingual Studies}\\
\normalsize  \url{http://www3.ntu.edu.sg/home/fcbond/}\\
\normalsize  \texttt{bond@ieee.org}}
\MyLogo{HG4041 (2020)}
\renewcommand{\logo}{#2}
\hypersetup{
   pdfinfo={
     Author={Francis Bond},
     Title={#1: #2},
     Subject={HG4041: Theories of Grammar},
     Keywords={Syntax, Semantics, HPSG, Unification, Constructions},
     License={CC BY 4.0}
   }
 }

\date{#1
  \\ Location: LHN-TR+36}
}
\usepackage[hidelinks]{hyperref}




\usepackage{xcolor}
\usepackage{graphicx}
\newcommand{\blu}[1]{\textcolor{blue}{#1}}
\newcommand{\grn}[1]{\textcolor{green}{#1}}
\newcommand{\hide}[1]{\textcolor{white}{#1}}
\newcommand{\emp}[1]{\textcolor{red}{#1}}
\newcommand{\txx}[1]{\textbf{\textcolor{blue}{#1}}}
\newcommand{\lex}[1]{\textbf{\mtcitestyle{#1}}}

\usepackage{pifont}
\renewcommand{\labelitemi}{\textcolor{violet}{\ding{227}}}
\renewcommand{\labelitemii}{\textcolor{purple}{\ding{226}}}

\newcommand{\subhead}[1]{\noindent\textbf{#1}\\[5mm]}

\newcommand{\Bad}{\emp{\raisebox{0.15ex}{\ensuremath{\mathbf{\otimes}}}}}
\newcommand{\bad}[1]{*\eng{#1}}

\newcommand{\com}[1]{\hfill (\emp{#1})}%

\usepackage{relsize,xspace}
\newcommand{\into}{\ensuremath{\rightarrow}\xspace}
\newcommand{\tot}{\ensuremath{\leftrightarrow}\xspace}
\usepackage{url}
\newcommand{\lurl}[1]{\MyLogo{\url{#1}}}

\usepackage{mygb4e}
\newcommand{\lx}[1]{\textbf{\mtciteform{#1}}}
\newcommand{\ix}{\ex\slshape}
\let\eachwordone=\slshape



\newcommand{\ent}{\ensuremath{\Rightarrow}\xspace}
\newcommand{\ngv}{\ensuremath{\not\Rightarrow}\xspace}
%\usepackage{times}
%\usepackage{nttfoilhead}
\newcommand{\myslide}[1]{\foilhead[-25mm]{\raisebox{12mm}[0mm]{\emp{#1}}}\MyLogo{\logo}}
\newcommand{\myslider}[1]{\rotatefoilhead[-25mm]{\raisebox{12mm}[0mm]{\emp{#1}}}}
%\newcommand{\myslider}[1]{\rotatefoilhead{\raisebox{-8mm}{\emp{#1}}}}

\newcommand{\section}[1]{\myslide{}{\begin{center}\Huge \emp{#1}\end{center}}}



\usepackage[lyons,j,e,k]{mtg2e}
%\renewcommand{\mtcitestyle}[1]{\textcolor{-red!75!green!50}{\textsl{#1}}}
\renewcommand{\mtcitestyle}[1]{\textcolor{teal}{\textsl{#1}}}
\newcommand{\iz}[1]{\texttt{\textup{#1}}}
\newcommand{\gm}{\textsc}
\usepackage[normalem]{ulem}
\newcommand{\ul}{\uline}
\newcommand{\ull}{\uuline}
\newcommand{\wl}{\uwave}
\newcommand{\vs}{\ensuremath{\Leftrightarrow}~}
%%%
%%% Bibliography
%%%
\usepackage{natbib}
%\usepackage{url}
\usepackage{bibentry}


%%% From Tim
\newcommand{\WMngram}[1][]{$n$-gram#1\xspace}
\newcommand{\infers}{$\rightarrow$\xspace}

\usepackage[utf8]{inputenc}

\usepackage{rtrees,qtree}
\renewcommand{\lf}[1]{\br{#1}{}}
\usepackage{avm}
%\avmoptions{topleft,center}
\newcommand{\ft}[1]{\textsc{#1}}
\renewcommand{\val}[1]{\textit{#1}}
\newcommand{\typ}[1]{\textit{#1}}
\newcommand{\prd}[1]{\textbf{#1}}
\avmfont{\sc}
\avmvalfont{\it}
\avmsortfont{\smaller[2] \it}
\usepackage{multicol}
\newcommand{\blank}{\rule{3em}{1pt}\xspace}

% \usepackage{pst-node}
\newcommand{\OV}[1]{\ovalnode[linestyle=dotted,linecolor=red]{A}{#1}}
\newcommand{\OVB}[1]{\ovalnode[linestyle=dotted,linecolor=blue]{A}{#1}}

%%% From CSLI book
\newcommand{\mc}{\multicolumn}
\newcommand{\HD}{\textbf{H}\xspace}
\newcommand{\el}{\< \>}
\makeatother
\long\def\smalltree#1{\leavevmode{\def\\{\cr\noalign{\vskip12pt}}%
\def\mc##1##2{\multispan{##1}{\hfil##2\hfil}}%
\tabskip=1em%
\hbox{\vtop{\halign{&\hfil##\hfil\cr
#1\crcr}}}}}
\makeatletter

\newcommand{\A}{\noindent\textbf{A}: }
\newcommand{\Q}{\noindent\textbf{Q}: }
%\newcommand{\C}{\noindent\textbf{C}: }



\begin{document}
\avmfont{\it}

\header{Lecture 6}{Structure of the lexicon}{}
\maketitle


\myslide{Questions}
\MyLogo{Sag, Wasow and Bender (2003) --- Chapter 8}


\Q can you please explain or give examples on the lexical rule instantiation?

\A There are many in the slides

\Q Why is there a need for default?

\A So that we can express things more compactly

\Q What is an example of a constraint that will cause contradiction to the default constraint inheritance?

\A \ft{spr} \val{$< DP >$} in proper noun \eng{Miami Heat}

\Q Why is ARG-ST included under adj-lxm \& conj-lxm, but not adv-lxm \& det-lxm?

\A I think it is everywhere, maybe just not shown for space

\newpage

\Q In a nutshell, would it be accurate to say that inflectional rules affect the SYN feature while derivational rules affect the SEM feature?

\A I think that describes their main effect, yes (although of course
both get changed)

\Q In this week's readings it said that lexeme and expression are both direct subtypes of synsem. Looking at the tree given on pg 229, I have the understanding that the features that fall under expressions cannot be in the same synsem as the features that fall under lexeme. However, I do not really understand what this means, how it is possible or if my understanding is even correct.

\A Yes, that is correct.  So, for example, $phrase$ does not have \ft{arg-st}

\newpage

\Q Last-week's topic:
For the imperative rule, why is the COMPS list empty?

\A Because we want the verb to have all of its compliments before we
use it.  \eng{*Put!} vs \eng{Put it there!}

\Q Is the defeasible symbol '/' only used for lexeme?

\A No it can be used elsewhere (such as in rules)

\Q
What is the difference between an empty list $ < >$ and a defeasible
list $/< >$? (E.g. ARG-ST $<DP> + < >$ and ARG-ST $<DP> + /< >$)

\A One can be overuled by more specific types the other cannot.

\newpage

\Q Gerneral qns
Is ARG-ST $<DP$ \\
  $[COUNT +]>$ ([...] on the 2nd line)
and ARG-ST DP [COUNT +] the same?

\A No: only the first one is correct (ARG-ST must be a list)

\Q Could you explain what the function of INPUT and OUTPUT is? Why do we need them?

\A Input is daughter, output is mother.   The rule applies to
something (the input) to make something different (the output), declaratively.

\newpage

\Q Apart from the exceptions of some proper nouns like mountains and team names, could this exception (that proper nouns are usually 3sing and ARG-ST list must be empty) be applied to brand names as well?

For example,
\\(1) The Blackberrys/-ies we saw today were stunning!
\\(2) *The Blackberrys/ies we saw today is stunning! 

\A Yes!



\Q In the beginning of the chapter, they were talking about the concept of "words with spaces". What are they and how does it relate to the chapter's discussion of lexemes?

\A Things that look like two words, but act as one: \eng{ad hoc}

\newpage

\Q How do we determine whether or not a constraint is defeasible? Is everything non-defeasible till there is a counterexample that shows that it is? 

\A In the model, it is only defeasible if we say it is.  We make
things defeasible if they are true for most types, but not all.

\Q Why does the book say that only S with FORM finite can be stand-alone sentences (51), when it gave examples in its list (48) of values for FORM that are also acceptable sentences?

\A in examples like \eng{Kim is \ul{eating} lunch}, \eng{eating} is
FORM prp, but the sentence is headed by \eng{is} which is FORM fin.

\newpage

% \Q Around pages 243/244, we get introduced to a Two-Argument taking preposition for 'around' in (42) "I wrapped the blanket around me";  besides mentioning that it's the first time we see the same constituent being arguments of two things st the same time, the book seems to not motivate why would a preposition need a specifier (if it's selected by the verb); the same (or similar) seems to be happening in (45), where adjectives now have an NP specifier...

% \A See chapter~12




\myslide{Acknowledgments and References}

\begin{itemize}
\item Course design and slides borrow heavily from Emily Bender's course:
\textit{Linguistics 566: Introduction to Syntax for Computational Linguistics}
\\ \url{http://courses.washington.edu/ling566}
\end{itemize}


\end{document}

\avmfont{\sc}
\begin{document}
\avmfont{\it}

\header{Lecture 5}{Binding Theory}{}
\maketitle


\myslide{Overview}
\MyLogo{Sag, Wasow and Bender (2003) --- Chapter 7}
\begin{itemize}
\item What we are trying to do
\item Last week: Semantics
\item Review of Chapter 1's informal binding theory
\item What we already have that’s useful
\item What we add in Ch 7 (ARG-ST, ARP)
\item Formalized Binding Theory
\item Binding and PPs
\item Imperatives
\end{itemize}



\myslide{What We’re Trying To Do}
\begin{itemize}
\item Objectives
\begin{itemize}
\item Develop a theory of knowledge of language
\item Represent linguistic information explicitly enough to
distinguish well-formed from ill-formed expressions
\item Be parsimonious, capturing linguistically significant
generalizations.
\end{itemize}
\item Why Formalize?
\begin{itemize}
\item To formulate testable predictions
\item To check for consistency
\item To make it possible to get a computer to do it for us
\end{itemize}
\end{itemize}

\myslide{How We Construct Sentences}
\begin{itemize}
\item The Components of Our Grammar
\begin{itemize}
\item Grammar rules
\item Lexical entries
\item Principles
\item Type hierarchy (very preliminary, so far)
\item Initial symbol (S, for now)
\end{itemize}
\item We combine constraints from these components.
%\item Q: What says we have to combine them?
\item More in \url{hpsg.stanford.edu/book/slides/Ch6a.pdf}, 
\\ \url{hpsg.stanford.edu/book/slides/Ch6b.pdf}
\end{itemize}

\section{Review of Semantics}

\myslide{Overview}
\begin{itemize}
\item Which aspects of semantics we’ll tackle
\item Semantics Principles
\item Building semantics of phrases
\item Modification, coordination
\item Structural ambiguity
\end{itemize}


\myslide{Our Slice of a World of Meanings}

Aspects of meaning we \emp{won’t} account for (in this course)
\begin{itemize}
\item Pragmatics
\item Fine-grained lexical semantics
  \\ The meaning of \eng{life} is
  \begin{itemize}\addtolength{\itemsep}{2ex}
  \item \iz{life}  
or 
\textsf{life}$'$
or
  \begin{avm}\avmfont{\sc}%\avmvalfont{\it}
    \[ reln & life \\ inst & i \]
  \end{avm}
\item Not like wordnet: 
\iz{life$_1$}  $\subset$ \iz{being$_1$} $\subset$ \iz{state$_1$} \ldots

 % -- (a characteristic state or mode of living; "social life"; "city life"; "real life")
 %       => being, beingness, existence -- (the state or fact of existing; "a point of view gradually coming into being"; "laws in existence for centuries")
 %           => state -- (the way something is with respect to its main attributes; "the current state of knowledge"; "his state of health"; "in a weak financial state")
 %               => attribute -- (an abstraction belonging to or characteristic of an entity)
 %                   => abstraction, abstract entity -- (a general concept formed by extracting common features from specific examples)
 %                       => entity -- (that which is perceived or known or inferred to have its own distinct existence (living or nonliving))
 \end{itemize}
\item Quantification \hfill (covered lightly in the book)
\item Tense, Mood, Aspect \hfill (covered in the book)
 \end{itemize}


% RELN life
% INST i

\myslide{Our Slice of a World of Meanings}
\begin{avm}\avmfont{\sc}
\[mode &  prop \\
 index &  $s$\\
 restr & \< \[ RELN & {\bf save}\\
               SIT & $s$\\
              SAVER &  $i$\\
              SAVED &  $j$ \] \ , 
              \[RELN & {\bf name}\\ 
              NAME &  Chris\\
              NAMED & \ $i$\]\ , 
              \[RELN & {\bf name}\\
              NAME &  Pat\\
              NAMED & \ $j$\] \> \]
\end{avm} 
\begin{quote}
  “\ldots the linguistic meaning of \eng{Chris saved Pat} is a proposition
  that will be true just in case there is an actual situation that
  involves the saving of someone named Pat by someone named Chris.”
  \begin{flushright}
    (Sag \textit{et al}, 2003, p. 140)
  \end{flushright}
\end{quote}

\myslide{Semantics in Constraint-Based Grammar}
\begin{itemize}
\item Constraints as (generalized) truth conditions
  \begin{itemize}
\item \txx{proposition}: what must be the case for a proposition to be true
\item \txx{directive}: what must happen for a directive to be fulfilled
\item \txx{question}: the kind of situation the asker is asking about
\item \txx{reference}: the kind of entity the speaker is referring to
\end{itemize}
\item \txx{Syntax/semantics interface}: 
  \begin{quote}
    Constraints on how syntactic arguments are related to semantic
    ones, and on how semantic information is compiled from different
    parts of the sentence.
  \end{quote}
\end{itemize}

\myslide{Feature Geometry}
\begin{avm}\avmfont{\sc}
\[\asort{expression}
syn & \[ \asort{syn-cat}
      head & \[ \asort{pos} ... \] \\
      val & \[ spr & \q< ... \q> \\
               comps & \q< ... \q> \]\]\\
      sem & \[ \asort{sem-cat}
      mode &  \q\{ prop, ques, dir, ref, none \q\} \\
      index &  \q\{ i, j, k, \ldots, s$_1$, s$_2$, \ldots \q\} \\
      restr & \q< ... \q> \]\]
 \end{avm}



% \myslide{How the Pieces Fit Together}
% \begin{avm}\avmfont{\sc}
%   \< \eng{Kim}, \[syn & \[head & \[{\it noun}\\
%   agr & {\it 3sing}\]\\
%   val & \[spr & {\el}\\
%   comps & {\el}\]\]\\
%   sem & \[mode & ref\\
%   index & \ \ $i$\\
%   restr & \< \[reln & {\bf name}\\
%   % sit & \ \ $s$\\
%   name & Kim\\
%   named & \ \ $i$\] \> \]\] \>  %
% \end{avm}
% \myslide{How the Pieces Fit Together}
% \begin{avm}\avmfont{\sc}
% \< \eng{sleep} ,\ \[syn & \[head & {\it verb}\\
%                                      val & \[spr & \q< NP$_i$ \q>\\
%                                      comps & \q<  \q>\]\]\\
%                             sem & \[mode & prop\\
%                                     index & $s$\\
%                                     restr & \< \[reln & {\bf sleep}\\
%                                                  sit & \ \ $s$\\
%                                                  sleeper & \ \ $i$  \] \>\]\] \>
%                                         \end{avm}
% \myslide{The Pieces Together}

% \begin{avmtree}\avmfont{\sc}
%   \br{S}{
%     \br{\@{1} NP  \[ sem & \[ index & $i$ \] \] }{\lf{\eng{Kim}}}
%     \br{VP  \[ syn & \[ val & \[ spr & \< \@{1} \> \] \] \\
%       sem & \[ mode & prop\\
%       index & $s$\\
%       restr & \< \[reln & {\bf sleep}\\
%       sit & \ \ $s$\\
%       sleeper & \ \ $i$  \] \> \] \]    }{\lf{\eng{slept}}}}
% \end{avmtree}


\myslide{An Example}
\scalebox{0.8}{\begin{avmtree}\avmfont{\sc}
  \br{S  \[ sem & \[ mode & prop \\ index & $s$\\
      restr & \< \[reln & {\bf name}\\
      name & \ \ Kim \\
      named & \ \ $i$ \],
      \[reln & {\bf sleep}\\
      sit & \ \ $s$\\
      sleeper & \ \ $i$  \]  \>  \] \] }{
    \br{\@{1} NP  \[ sem & \[ mode & ref \\ index & $i$\\
      restr & \< \[reln & {\bf name}\\
      name & \ \ Kim \\
     named & \ \ $i$ \] \>  \] \] }{\lf{\eng{Kim}}}
    \br{\HD VP  \[ syn & \[ val & \[ spr & \< \@{1} \> \] \] \\
      sem & \[  mode & prop \\ index & $s$ \\ 
      restr & \< \[reln & {\bf sleep}\\
      sit & \ \ $s$\\
      sleeper & \ \ $i$  \] \> \] \]    }{\lf{\eng{slept}}}}
\end{avmtree}}


\myslide{How to Share Semantic Information}
\MyLogo{List summation: $\oplus$ (technically concatenation)} 
\begin{itemize}
\item \txx{The Semantic Inheritance Principle}
  \begin{quote}
    In any headed phrase, the mother's \ft{mode} and
    \ft{index} are identical to those of the head daughter.    
  \end{quote}
\item \txx{The Semantic Compositionality Principle}
  \begin{quote}
    In any well-formed phrase structure, the mother's
    \ft{restr} value is the sum of the \ft{restr} values of
    the daughters.
  \end{quote}
\end{itemize}

\myslide{Where is the information}
\MyLogo{}
\begin{itemize}
\item Words
  \begin{itemize}
  \item Contribute predications
  \item ‘expose’ one index in those predications, for use by words or phrases
  \item relate syntactic arguments to semantic arguments
  \end{itemize}
\item Rules
\begin{itemize}
\item Identify (link) feature structures across daughters
\item License trees which are subject to the semantic principles
  \begin{itemize}
  \item SIP: ‘passes up’ \ft{mode} and \ft{index} from head daughter
  \item SCP: ‘gathers up’ predications (\ft{restr} list) from all daughters
  \end{itemize}
\end{itemize}
\item The semantics is strictly compositional --- all of the meaning
  comes from the words, rules and principles.
\end{itemize}

\section{Binding}


\myslide{Some Examples from Chapter 1}

\begin{exe}
\ix She likes herself
\ix *She$_i$ likes her$_i$.
\ix We gave presents to ourselves.
\ix *We gave presents to us.
\ix We gave ourselves presents
\ix *We gave us presents.
\ix *Leslie told us about us.
\ix Leslie told us about ourselves.
\ix *Leslie told ourselves about us.
\ix *Leslie told ourselves about ourselves.
\end{exe}

\myslide{Some Terminology}
\begin{itemize}
\item \txx{Binding}: The association between a pronoun
and an antecedent.
\item \txx{Anaphoric}: A term to describe an element (e.g.
a pronoun) that derives its interpretation from
some other expression in the discourse.
\item \txx{Antecedent}: The expression an anaphoric
expression derives its interpretation from.
\item \txx{Anaphora}: The relationship between an
anaphoric expression and its antecedent.
\end{itemize}

\myslide{The Chapter 1 Binding Theory Reformulated}
\begin{itemize}
\item Old Formulation:
\begin{itemize}
\item A reflexive pronoun must be an argument of a verb that
has another preceding argument with the same reference.
\item A nonreflexive pronoun cannot appear as an argument of
a verb that has a preceding coreferential argument.
\end{itemize}
\item New Formulation(version I):
\begin{itemize}
\item \txx{Principle A}: A reflexive pronoun must be
bound by a preceding argument of the same verb.
\item \txx{Principle B}: A nonreflexive pronoun may not
be bound by a preceding argument of the same verb.
\end{itemize}
\item Opaque names come from Chomsky (1981)
\end{itemize}

\myslide{Some Challenges}
\begin{itemize}
\item Replace notions of \emp{bound} and \emp{preceding
argument of the same verb} by notions
definable in our theory.
\item Generalize the Binding Principles to get
better coverage.
\end{itemize}


\myslide{How can we do this?}
\begin{itemize}
\item[Q]  What would be a natural way to formalize
the notion of “bound” in our theory?
\item[A] Two expressions are bound if
they have the same \ft{index} value (“are
coindexed”).
\item[Q] Where in our theory do we have information
about a verb’s arguments?
\item[A] In the verb’s \ft{valence} features.
\item[Q] What determines the linear ordering of a
verb’s arguments in a sentence?
\item[A] The interaction of the grammar
rules and the ordering of elements in the
\ft{comps} list.
\end{itemize}

\myslide{The Argument Realization Principle}
\begin{itemize}
\item For Binding Theory, we need a single list with both subject
and complements.
\item We introduce a feature ARG-ST, with the following
property: \\
\begin{avm}\avmfont{\sc} 
\[ \asort{word}  
 syn & \[ val & \[ spr &  \@{A} \\comps &  \@{B} \] \] \\
 arg-st & \< \@{A} $\oplus$ \@{B} \> \]
\end{avm}

\item This is a constraint on the type \val{word}
\end{itemize}

\myslide{Notes on \ft{arg-st}}
\begin{itemize}
\item It’s neither in \ft{syn} nor \ft{sem}.
\item It only appears on lexical heads (not
appropriate for type phrase)
\item No principle stipulates identity
between \ft{arg-st}s.
\end{itemize}

\myslide{The Binding Principles}
\begin{itemize}
\item \txx{Principle A}: A [\ft{mode} \val{ana}] element must be
outranked by a coindexed element.
\item \txx{Principle B}: A [\ft{mode} \val{ref}] element must not
be outranked by a coindexed element.
\end{itemize}

\begin{description}
\item [Formalization] ~\\[-2ex]
\begin{itemize}
\item Definition: If A precedes B on some \ft{arg-st} list,
then A \txx{outranks} B.
\item Elements that must be anaphoric --- that is, that
require an antecedent --- are lexically marked
[\ft{mode} \val{ana}]. These include reflexive pronouns
and reciprocals.
\end{itemize}
\end{description}

\myslide{Pronoun-Antecedent Agreement}
\begin{itemize}
\item The Binding Principles by themselves don’t block:
  \begin{exe}
    \ix * I amused yourself.
    \ix * He amused themselves.
    \ix * She amused himself.
  \end{exe}
\item Coindexed NPs refer to the same entity, and \ft{agr} features
generally correlate with properties of the referent.
\item The \txx{Anaphoric Agreement Principle} (AAP):
Coindexed NPs agree.
\end{itemize}

\myslide{Binding in PPs}
\begin{itemize}
\item What do the Binding Principles predict about the
following?
\item The Binding Principles by themselves don’t block:
  \begin{exe}
    \ix I brought a book with me.
    \ix *I brought a book with myself.
    \ix *I mailed a book to me.
    \ix I mailed a book to myself.
  \end{exe}
\end{itemize}

\myslide{Two Types of Prepositions: the Intuition}
\begin{itemize}
\item \txx{Argument-marking}: Function like casemarkers in other languages, indicating the
roles of NP referents in the situation denoted by the verb.
\item \txx{Predicative}: Introduce their own predication.
\end{itemize}

\begin{description}
\item [Formalization] ~\\[-2ex]
  \begin{itemize}
  \item Argument-marking prepositions share their
    objects' \ft{mode} and \ft{index} values.
    \begin{itemize}
    \item This is done with tagging in the lexical
      entries of such prepositions.
    \item These features are also shared with the PP
      node, by the Semantic Inheritance Principle.
    \end{itemize}
  \item Predicative prepositions introduce their own
\ft{mode} and \ft{index} values.
\end{itemize}
\end{description}

\myslide{Redefining Rank}
 \begin{itemize}
\item If there is an \ft{arg-st} list on which A
precedes B, then A outranks B.
\item If a node is coindexed with its daughter, they
are of equal rank -- that is, they outrank the
same nodes and are outranked by the same
nodes.
\end{itemize}

\myslide{\eng{I sent a letter to myself}}

\begin{avmtree}\avmfont{\sc} 
\br{S}{
  \br{\@1 NP$_i$}{ \lf{\eng{I}} }
  \br{VP \[ \q< \@1 \q> \]}{ 
    \br{V \[ spr & \q< \@1 \q> \\
             comps & \q< \@2, \@3 \q> \\ 
             arg-st & \q< \@1, \@2, \@3 \q> \]}{
      \lf{\eng{sent}}}
    \br{\@2 NP$_j$}{
      \br{D}{\lf{\eng{a}}}
      \br{N$_j$}{\lf{\eng{letter}}}}
    \br{\@3 PP$_i$}{
      \br{P$_i$}{\lf{\eng{to}}}
      \br{NP$_i$ \[ mode & ana \]}{\lf{\eng{myself}}}}
}}
\end{avmtree}




\myslide{The ARG-ST}
\begin{center}
  \begin{avm}\avmfont{\sc} 
\[  arg-st & \< 
NP$_i$ \[ mode & ref \],
NP$_j$ \[ mode & ref \],
PP$_i$ \[ mode & ana \]
\> \]
\end{avm}
\end{center}
\begin{itemize}
\item The PP is outranked by the first NP. (Why?)
\item \eng{myself} has the same rank as the PP. (Why?)
\item So, \eng{myself} is outranked by the first NP. (Why?)
\item Therefore, Principle A is satisfied.
\end{itemize}

\myslide{\eng{*I sent a letter to me}}
\begin{avmtree}\avmfont{\sc} 
  \br{S}{
  \br{\@1 NP$_i$}{ \lf{\eng{I}} }
  \br{VP \[ \q< \@1 \q> \]}{ 
    \br{V \[ spr & \q< \@1 \q> \\
             comps & \q< \@2, \@3 \q> \\ 
             arg-st & \q< \@1, \@2, \@3 \q> \]}{
      \lf{\eng{sent}}}
    \br{\@2 NP$_j$}{
      \br{D}{\lf{\eng{a}}}
      \br{N$_j$}{\lf{\eng{letter}}}}
    \br{\@3 PP$_i$}{
      \br{P$_i$}{\lf{\eng{to}}}
      \br{NP$_i$ \[ mode & ref \]}{\lf{\eng{me}}}}
}}
\end{avmtree}

\myslide{The ARG-ST}
\begin{center}
  \begin{avm}\avmfont{\sc} 
\[  arg-st & \< 
NP$_i$ \[ mode & ref \],
NP$_j$ \[ mode & ref \],
PP$_i$ \[ mode & ref \]
\> \]
\end{avm}
\end{center}
\begin{itemize}
\item The PP is outranked by the first NP.
\item \eng{me} has the same rank as the PP.
\item So, \eng{me} is outranked by the first NP.
\item Therefore, Principle B is violated.
\end{itemize}

\myslide{\eng{I brought a pencil with me}}
\MyLogo{Here \eng{I} does not outrank \eng{me}, so Principle B is satisfied}

\begin{avmtree}\avmfont{\sc} 
\br{S}{
  \br{\@1 NP$_i$}{ \lf{\eng{I}} }
  \br{VP \[ \q< \@1 \q> \]}{ 
    \br{V \[ spr & \q< \@1 \q> \\
             comps & \q< \@2 \q> \\ 
             arg-st & \q< \@1, \@2 \q> \]}{
      \lf{\eng{brought}}}
    \br{\@2 NP$_j$}{
      \br{D}{\lf{\eng{a}}}
      \br{N$_j$}{\lf{\eng{pencil}}}}
    \br{\@3 PP$_k$}{
      \br{P$_k$ \[ comps & \q< \@1 \q> \\ 
                   mod & \q< \@2 \q> \\
                   arg-st & \q< \@1 \q> \] }{\lf{\eng{with}}}
      \br{NP$_l$ \[ mode & ref \]}{\lf{\eng{me}}}}
}}
\end{avmtree}

\myslide{\eng{*I brought a pencil with myself}}
\MyLogo{Here \eng{I} does not outrank \eng{myself}, so Principle A is violated}

\begin{avmtree}\avmfont{\sc} 
\br{S}{
  \br{\@1 NP$_i$}{ \lf{\eng{I}} }
  \br{VP \[ \q< \@1 \q> \]}{ 
    \br{V \[ spr & \q< \@1 \q> \\
             comps & \q< \@2 \q> \\ 
             arg-st & \q< \@1, \@2 \q> \]}{
      \lf{\eng{brought}}}
    \br{\@2 NP$_j$}{
      \br{D}{\lf{\eng{a}}}
      \br{N$_j$}{\lf{\eng{pencil}}}}
    \br{\@3 PP$_k$}{
      \br{P$_k$ \[ comps & \q< \@1 \q> \\ 
                   mod & \q< \@2 \q> \\
                   arg-st & \q< \@1 \q> \] }{\lf{\eng{with}}}
      \br{NP$_l$ \[ mode & ana \]}{\lf{\eng{myself}}}}
}}
\end{avmtree}

\myslide{Imperatives}
\MyLogo{}
\begin{itemize}
\item Have the internal structure of a VP
  \begin{exe}
    \ix Leave!
    \ix Read a book!
    \ix Give the dog a treat!
    \ix Put the ice cream in the freezer!
  \end{exe}
\item Function as \txx{directives} (commands or requests)
\item Have the verb in base form: \eng{Be careful!} not \eng{*Are careful!}
\begin{itemize}
\item Allow 2nd person reflexives, and no others
\begin{exe}
    \ix Defend yourself!
    \ix *Defend myself/himself!
  \end{exe}
\end{itemize}
\end{itemize}

\myslide{The Imperative Rule}

\begin{avm}\avmfont{\sc} 
\[ \asort{phrase}  
   head & verb \\
   val & \[ spr  & \<  \> \] \\
   sem & \[ mode & dir \\ index & $s$ \] \]
              \ $\rightarrow$\ \ 
\[ \asort{phrase}  
   head & \[ \asort{verb} form & base \]\\
   val & \[ spr  & \< NP \[ per 2nd \]  \> \] \\
   sem & \[ index & $s$ \] \]
\end{avm}


\begin{itemize}\addtolength{\itemsep}{-2ex}
\item Internal structure of a VP
\item Directive function
\item Base form
\item Only 2nd person reflexives
\item[Q] Note that this is not a headed rule. Why?
\item[A] It would violate the HFP and the SIP.
\end{itemize}

\myslide{Imperative example}
\begin{multicols}{2}
\begin{avmtree}\avmfont{\sc} 
\br{S}{
  \br{VP \[ spr & \q< \@1 NP \[ per 2nd \\ num sg \] \q> \]}{ 
    \br{V \[ spr & \q< \@1 \q> \]}{
      \lf{\eng{vote}}}
    \br{\@3 PP$_i$}{
      \br{P$_i$}{\lf{\eng{for}}}
      \br{NP$_i$ \[ mode & ana \]}{\lf{\eng{yourself}}}}
}}
\end{avmtree}
%\newpage

\begin{itemize}
\item What's the SPR value on S?
%Why?
\item What's the SPR value on VP?
%Why?
\item What's the SPR value on V?
%Why?
\item Which nodes have ARG-ST?
\item Which ARG-ST matters for
the licensing of \eng{yourself}?
\end{itemize}
%\newpage

\end{multicols}

\myslide{\ft{arg-st} on vote}
\begin{center}
  \begin{avm}\avmfont{\sc} 
\[  arg-st & \< 
NP$_i$ \[ mode & ref \\ per & 2nd \\ num & sg \],
PP$_i$ \[ mode & ana \]
\> \]
\end{avm}
\end{center}
\begin{itemize}
\item Is Principle A satisfied?
\item How?
\item Is Principle B satisfied?
\item How?
\end{itemize}

\myslide{Day 1 Revisited}
\begin{itemize}
\item Recall
  \begin{exe}
    \ix Screw yourself!
    \ix Go screw yourself!
    \ix screw you!
    \ix *Go screw you!
  \end{exe}
\item \eng{Screw NP!} has two analyses
  \begin{itemize}
  \item As an imperative
  \item As a truly subjectless fixed expression.
  \end{itemize}
\item \eng{Go screw NP!} can only be analyzed as an imperative.
\end{itemize}


\myslide{Overview}
\begin{itemize}\addtolength{\itemsep}{-2ex}
\item Review of Chapter 1 informal binding
theory
\item Formalized Binding Theory
  \begin{itemize}
  \item \txx{Argument Realization Principle} (ARP)
    \\  \ft{arg-st} = \ft{spr}  $\oplus$ \ft{comps}
  \item \txx{Anaphoric Agreement Principle} (AAP)
   \\ Coindexed NPs agree.
 \item \txx{Principle A}\ \  A [\ft{mode} \val{ana}] element must be
outranked by a coindexed element.
\item \txx{Principle B}\ \  A [\ft{mode} \val{ref}] element must not
be outranked by a coindexed element.
  \end{itemize}
\item Binding and PPs
\item Imperatives
\end{itemize}


\myslide{P1: Prepositions}
\MyLogo{Based on  Chapter 7, Problem 1, Sag, Wasow and Bender (2003)}

For each of the following sentences, 
\begin{itemize}
\item[(a)] classify the underlined preposition into \txx{predicative}
  or \txx{argument marking} and
\item[(b)] justify by showing what reflexive and
  nonreflexive coreferential 
  pronouns can or cannot appear as the preposition's object.
\end{itemize}
\begin{itemize}\addtolength{\itemsep}{-1.5ex}
\item[(i)] \eng{The dealer dealt an ace \ul{to} Bo.}
\item[(ii)] \eng{The chemist held the sample away \ul{from} the flame.}
\item[(iii)] \eng{Hiromi kept a flashlight \ul{beside} the bed.}
\item[(iv)] \eng{We bought flowers \ul{for} you.}
\item[(v)] \eng{The car has a scratch \ul{on} the fender.}
\item[(vi)] \eng{The pronoun agrees \ul{with} its antecedent.}
\end{itemize}

\myslide{P2: Imperative `Subjects'}
\MyLogo{Based on  Chapter 7, Problem 2, Sag, Wasow and Bender (2003)}
\begin{itemize}
\item Some imperatives look like they have a subject:
  \begin{itemize}
  \item[(i)] \eng{\ul{You} get out of here!}
  \item[(ii)] \eng{\ul{Everybody} take out a sheet of paper!}
  \end{itemize}
\item But agreement is wierd:
  \begin{itemize}
  \item[(iii)] 
    \eng{Everybody found 
      $^?$himself/$^*$yourself/themselves/$^*$myself a seat.}
  \item[(iv)]
    \eng{Everybody find 
      $^?$himself/yourself/$^*$themselves/$^*$myself a seat.}
  \end{itemize}
\item What minimal modification of the Imperative Rule would account for
the indicated data? (don't worry about
the semantics)
\end{itemize}



\myslide{Acknowledgments and References}

\begin{itemize}
\item Course design and slides borrow heavily from Emily Bender's course:
\textit{Linguistics 566: Introduction to Syntax for Computational Linguistics}
\\ \url{http://courses.washington.edu/ling566}
\end{itemize}


\input{lec-05-questions}

\end{document}

%%% Local Variables: 
%%% coding: utf-8
%%% mode: latex
%%% TeX-PDF-mode: t
%%% TeX-engine: xetex
%%% End: 
