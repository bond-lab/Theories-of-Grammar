\documentclass[a4paper,landscape,headrule,footrule]{foils}
%\usepackage{times}
%\usepackage{nttfoilhead}
%\newcommand{\myslide}[1]{\foilhead[-25mm]{\raisebox{12mm}[0mm]{\emp{#1}}}}
%\newcommand{\myslider}[1]{\rotatefoilhead[-25mm]{\raisebox{12mm}[0mm]{\emp{#1}}}}
%\newcommand{\myslider}[1]{\rotatefoilhead{\raisebox{-8mm}{\emp{#1}}}}

%%
%%% macros for Theories of Grammar
%%%
\usepackage{polyglossia}
\setdefaultlanguage{english}
%\setmainfont{TeX Gyre Pagella}


\newcommand{\logo}{~}
\newcommand{\header}[3]{%
\title{\vspace*{-2ex} \large HG4041 Theories of Grammar
\\[2ex] \Large  \emp{#2} \\ \emp{#3}}
\author{\blu{Francis Bond}   \\ 
\normalsize  \textbf{Division of Linguistics and Multilingual Studies}\\
\normalsize  \url{http://www3.ntu.edu.sg/home/fcbond/}\\
\normalsize  \texttt{bond@ieee.org}}
\MyLogo{HG4041 (2020)}
\renewcommand{\logo}{#2}
\hypersetup{
   pdfinfo={
     Author={Francis Bond},
     Title={#1: #2},
     Subject={HG4041: Theories of Grammar},
     Keywords={Syntax, Semantics, HPSG, Unification, Constructions},
     License={CC BY 4.0}
   }
 }

\date{#1
  \\ Location: LHN-TR+36}
}
\usepackage[hidelinks]{hyperref}




\usepackage{xcolor}
\usepackage{graphicx}
\newcommand{\blu}[1]{\textcolor{blue}{#1}}
\newcommand{\grn}[1]{\textcolor{green}{#1}}
\newcommand{\hide}[1]{\textcolor{white}{#1}}
\newcommand{\emp}[1]{\textcolor{red}{#1}}
\newcommand{\txx}[1]{\textbf{\textcolor{blue}{#1}}}
\newcommand{\lex}[1]{\textbf{\mtcitestyle{#1}}}

\usepackage{pifont}
\renewcommand{\labelitemi}{\textcolor{violet}{\ding{227}}}
\renewcommand{\labelitemii}{\textcolor{purple}{\ding{226}}}

\newcommand{\subhead}[1]{\noindent\textbf{#1}\\[5mm]}

\newcommand{\Bad}{\emp{\raisebox{0.15ex}{\ensuremath{\mathbf{\otimes}}}}}
\newcommand{\bad}[1]{*\eng{#1}}

\newcommand{\com}[1]{\hfill (\emp{#1})}%

\usepackage{relsize,xspace}
\newcommand{\into}{\ensuremath{\rightarrow}\xspace}
\newcommand{\tot}{\ensuremath{\leftrightarrow}\xspace}
\usepackage{url}
\newcommand{\lurl}[1]{\MyLogo{\url{#1}}}

\usepackage{mygb4e}
\newcommand{\lx}[1]{\textbf{\mtciteform{#1}}}
\newcommand{\ix}{\ex\slshape}
\let\eachwordone=\slshape



\newcommand{\ent}{\ensuremath{\Rightarrow}\xspace}
\newcommand{\ngv}{\ensuremath{\not\Rightarrow}\xspace}
%\usepackage{times}
%\usepackage{nttfoilhead}
\newcommand{\myslide}[1]{\foilhead[-25mm]{\raisebox{12mm}[0mm]{\emp{#1}}}\MyLogo{\logo}}
\newcommand{\myslider}[1]{\rotatefoilhead[-25mm]{\raisebox{12mm}[0mm]{\emp{#1}}}}
%\newcommand{\myslider}[1]{\rotatefoilhead{\raisebox{-8mm}{\emp{#1}}}}

\newcommand{\section}[1]{\myslide{}{\begin{center}\Huge \emp{#1}\end{center}}}



\usepackage[lyons,j,e,k]{mtg2e}
%\renewcommand{\mtcitestyle}[1]{\textcolor{-red!75!green!50}{\textsl{#1}}}
\renewcommand{\mtcitestyle}[1]{\textcolor{teal}{\textsl{#1}}}
\newcommand{\iz}[1]{\texttt{\textup{#1}}}
\newcommand{\gm}{\textsc}
\usepackage[normalem]{ulem}
\newcommand{\ul}{\uline}
\newcommand{\ull}{\uuline}
\newcommand{\wl}{\uwave}
\newcommand{\vs}{\ensuremath{\Leftrightarrow}~}
%%%
%%% Bibliography
%%%
\usepackage{natbib}
%\usepackage{url}
\usepackage{bibentry}


%%% From Tim
\newcommand{\WMngram}[1][]{$n$-gram#1\xspace}
\newcommand{\infers}{$\rightarrow$\xspace}

\usepackage[utf8]{inputenc}

\usepackage{rtrees,qtree}
\renewcommand{\lf}[1]{\br{#1}{}}
\usepackage{avm}
%\avmoptions{topleft,center}
\newcommand{\ft}[1]{\textsc{#1}}
\renewcommand{\val}[1]{\textit{#1}}
\newcommand{\typ}[1]{\textit{#1}}
\newcommand{\prd}[1]{\textbf{#1}}
\avmfont{\sc}
\avmvalfont{\it}
\avmsortfont{\smaller[2] \it}
\usepackage{multicol}
\newcommand{\blank}{\rule{3em}{1pt}\xspace}

% \usepackage{pst-node}
\newcommand{\OV}[1]{\ovalnode[linestyle=dotted,linecolor=red]{A}{#1}}
\newcommand{\OVB}[1]{\ovalnode[linestyle=dotted,linecolor=blue]{A}{#1}}

%%% From CSLI book
\newcommand{\mc}{\multicolumn}
\newcommand{\HD}{\textbf{H}\xspace}
\newcommand{\el}{\< \>}
\makeatother
\long\def\smalltree#1{\leavevmode{\def\\{\cr\noalign{\vskip12pt}}%
\def\mc##1##2{\multispan{##1}{\hfil##2\hfil}}%
\tabskip=1em%
\hbox{\vtop{\halign{&\hfil##\hfil\cr
#1\crcr}}}}}
\makeatletter

\newcommand{\A}{\noindent\textbf{A}: }
\newcommand{\Q}{\noindent\textbf{Q}: }
%\newcommand{\C}{\noindent\textbf{C}: }


\newenvironment{block}[1]%
  {%
    \par\medskip
    \noindent\textbf{#1}\par}%
  {%
    \medskip
  }


\begin{document}
\header{Lecture 1b}{First attempts at a theory of grammar}{}\maketitle

%\include{schedule}

\myslide{Overview}

\begin{itemize}
\item Some initial attempts to model grammar
  \begin{itemize}
  \item Context Free Grammars
  \item Central claims of CFG
  \end{itemize}
\item Three approaches to morphology
  \begin{itemize}
  \item \txx{Item-and-Arrangement (IA)} 
  \item \txx{Item-and-Process (IP)}
  \item \txx{Word-and-Paradigm (WP)}
  \end{itemize}
\end{itemize}


\myslide{What makes a good model?}

\begin{itemize}
\item \txx{generative}: license all grammatical sentences and only them
\\ $\Rightarrow$ \txx{precise}
\item \txx{explanatory}: can explain generalizations
  \begin{itemize}
  \item  \eng{the cat chased the rat} $\sim$ \eng{the rat was chased by the cat} \com{semantics}
  \item phrases tend to act like one member of the phrase \com{headedness}
  \item new information tends to come first/last \com{information theory}
  \end{itemize}
\item \txx{concise}: the model is as simple as possible \com{elegant}
\\ $\Rightarrow$ \txx{universal} \com{minimal stipulations}
\item \txx{tractable}: the model can be modeled computationally
\end{itemize}

\begin{center}
  Our models are normally imperfect: \\
we aim for iteratively improved approximations
\end{center}


\myslide{Identifying constituents}
\begin{block}{}
  \begin{itemize}
  \item
    Language structure is \textbf{hierarchical}, and is made up of identifiable constituents
  \item
    Some words go together more closely to form a constituent, e.g. noun phrases and postpositional phrases.  
  \item
   Such strings form \textbf{immediate constituents} of units higher up in the hierarchy.
\item
    Constituents are identified on the basis of \textbf{formal criteria ...}
  \end{itemize}
\end{block}

%------------------------------------------------------------------------------------------
\myslide{Formal criteria for identifying constituents}
\begin{block}{Three things to consider:}
  \begin{itemize}
  \item
    \textbf{Distribution}: If the same sequence of constituents occurs repeatedly, this sequence might be a constituent.  
  \item
    \textbf{Substitution}: If we can substitute a sequence of words by a single word, keeping the reference more or less the same, then this sequence is probably a constituent.
  \item
    \textbf{Mobility}: If we can move a sequence of constituents around in a sentence, and they have to move together, then this is probably a constituent
  \end{itemize}
\end{block}






\myslide{Insufficient Theory \#1}
\begin{itemize}
\item A grammar is simply a list of sentences.
\item What’s wrong with this?
\end{itemize}

\myslide{Insufficient Theory \#2: Regular Expressions}
\begin{exe}
  \ex \gll the noisy dogs left \\
  D A N V \\ 
  \ex  \gll the noisy dogs chased the innocent cats \\
 D A N V D A N \\
\end{exe}
\begin{itemize}
\item (D) A* N V ((D) A* N)
\end{itemize}
\hrule
\txx{Regular expressions}: a formal language for matching things.
\\[2ex]
 \begin{tabular}{ll}
    Symbol & Matches \\ \hline
    . & any single character\\
%    {[ ]} & a single character that is contained within the brackets. \\
%    & {[a-z]} specifies a range which matches any  letter from "a" to "z".\\
%    {[\textasciicircum ~]} & 	a single character not in the brackets. \\
%    \textasciicircum 	& the starting position within the string/line. \\
%    \$ 	&  the ending position of the string/line. \\
    $*$ &	the preceding element zero or more times. \\
    ? &	 the preceding element zero or one time: OR just () = ()?. \\
    + &	 the preceding element one or more times. \\
    $|$ &  either the expression before or after the operator. \\
%    $\backslash$ & escapes the following character. \\
  \end{tabular}

% \myslide{A Finite State Machine}
% D

% N

% V

% D

% A

% N

% A
% V
% V


%\myslide{FSMs for Grammar, cont}

% \item Why are FSMs insufficient as a

% representation of natural language syntax?

% \item How might they be useful anyway?


% \myslide{Chomsky Hierarchy}
% Type 0 Languages
% Context-Sensitive Languages
% Context-Free Languages
% Regular Languages


\myslide{Context-Free Grammar}
\begin{itemize}
\item A quadruple: $\langle C, V, P, S \rangle$
\begin{itemize}
\item[$C$] set of categories ($\alpha, \beta, \ldots$)
\item[$V$] set of terminals (vocabulary)
\item[$P$] set of rewrite rules $\alpha \into \beta_1, \beta_2, \ldots, \beta_n$
\item[$S$] the start symbol $\mathbf{S} \in C$
\end{itemize}
\item For each rule $\alpha \into \beta_1, \beta_2, \ldots, \beta_n \in P$
  \begin{itemize}
  \item  $\alpha \in C$
  \item  $\beta_i \in C \cup V; 1 \le i \le n$ 
\end{itemize}
\end{itemize}


\myslide{A Toy Grammar}

\begin{itemize}
\item RULES
\\[2ex]  \begin{tabular}{lll}
\textbf{S}  & \into & NP VP \\
NP & \into & (D) A* N PP*\\
VP & \into & V (NP) (PP)\\
PP & \into & P NP\\
\end{tabular}

\item VOCABULARY
\begin{flushleft}
D: the, some\\
A: big, brown, old\\
N: birds, fleas, dog, hunter, I\\
V: attack, ate, watched\\
P: for, beside, with
\end{flushleft}
\end{itemize}

\myslide{Structural Ambiguity}
\begin{center} \large
  \eng{I saw the astronomer with the telescope.}
\end{center}

\myslide{Structure 1: PP under VP}
{%
 \leaf{\emph{I}}
 \branch{1}{N}
 \branch{1}{NP}
 \leaf{\emph{saw}}
 \branch{1}{V}
 \leaf{\emph{the}}
 \branch{1}{D}
 \leaf{\emph{astronomer}}
 \branch{1}{N}
 \branch{2}{NP}
 \leaf{\emph{with}}
 \branch{1}{P}
 \leaf{\emph{the}}
 \branch{1}{D}
 \leaf{\emph{telescope}}
 \branch{1}{N}
 \branch{2}{NP}
 \branch{2}{PP}
 \branch{3}{VP}
 \branch{2}{S}
 \qobitree}

\myslide{Structure 2: PP under NP}
\begin{small} {%
    \leaf{\emph{I}} \branch{1}{N} \branch{1}{NP} \leaf{\emph{saw}}
    \branch{1}{V} \leaf{\emph{the}} \branch{1}{D}
    \leaf{\emph{astronomer}} \branch{1}{N} \leaf{\emph{with}}
    \branch{1}{P} \leaf{\emph{the}} \branch{1}{D}
    \leaf{\emph{telescope}} \branch{1}{N} \branch{2}{NP}
    \branch{2}{PP} \branch{3}{NP} \branch{2}{VP}
    \branch{2}{S} \qobitree}
\end{small}

\myslide{Constituency Tests}
\begin{itemize}\addtolength{\itemsep}{-1ex}
\item Recurrent Patterns
\begin{exe}
\ex \eng{\ul{The quick brown fox with the bushy tail} jumped over \ul{the lazy brown dog
with one ear}.}\
\end{exe}

\item Coordination
\begin{exe}
\ex  \eng{\ul{The quick brown fox with the bushy tail} and \ul{the lazy brown dog with one
ear} are friends.}
\end{exe}
\item Sentence-initial position
\begin{exe}
\ex \eng{\ul{The election of 2000}, everyone will remember for a long time.}
\end{exe}
\item Cleft sentences
\begin{exe}
\ex \eng{It was \ul{a book about syntax} that they were reading.}
\end{exe}
\end{itemize}

\myslide{General Types of Constituency Tests}
\begin{itemize}
\item Distributional
\item Intonational
\item Semantic
\item Psycholinguistic
\item[\ldots] but they don’t always agree.
\end{itemize}

\myslide{Central claims implicit in CFG formalism:}
\begin{enumerate}
\item  Parts of sentences (larger than single words) are
linguistically significant units, i.e. phrases play a role in
determining meaning, pronunciation, and/or the
acceptability of sentences.
\item Phrases are contiguous portions of a sentence (no
discontinuous constituents).
\item Two phrases are either disjoint or one fully contains the
other (no partially overlapping constituents).
\item What a phrase can consist of depends only on what kind of
a phrase it is (that is, the label on its top node), not on what
appears around it.
\end{enumerate}

\newpage
\begin{itemize}
\item Claims 1-3 characterize what is called \txx{phrase structure grammar}

\item Claim 4 (that the internal structure of a phrase depends only on what type of phrase it is, not on where it appears) is what makes it \txx{Context-Free}.

\item \txx{Context-Sensitive Grammar} (CSG) gives up 4. 
That is, it allows the applicability of a
grammar rule to depend on what is in the
neighboring environment. So rules can have the
form: 
\\ $A \into X$ in the context of $\alpha \_\beta$ ($\alpha A\beta \into \alpha X\beta$)
\end{itemize}

\myslide{Possible Counterexamples}
\begin{itemize}
\item To Claim 2 (no discontinuous constituents):
\\ \eng{\ul{A technician} arrived \ul{who could solve the problem}.}

\item To Claim 3 (no overlapping constituents):
\\ \eng{I read \ul{what} was written about me.}

\item To Claim 4 (context independence):
  \begin{exe}
  \ex \eng{He arrives this morning.}
  \ex \eng{*He arrive this morning.}
  \ex \eng{*They arrives this morning.}
  \ex \eng{They arrive this morning.}
  \end{exe}
\end{itemize}


\myslide{Trees and Rules}
  {
 \leaf{$C_1$}
 \leaf{\ldots}
  \leaf{$C_2$}
 \branch{3}{$C_0$} \qobitree}is a well-formed nonlexical tree if (and only if)
\begin{itemize}
\item  $C_0, \ldots,  C_n$ are well-formed trees
\item $C_0$ \into $C_1 \ldots C_n$ is a grammar rule
\end{itemize}

\myslide{Bottom-up Tree Construction}

\begin{flushleft}
D: the \\
V: chased \\
N: dog, cat
\end{flushleft}

{ \leaf{the} \branch{1}{D} \qobitree}
{ \leaf{the} \branch{1}{D} \qobitree}
{ \leaf{chased} \branch{1}{V} \qobitree}
{ \leaf{dog} \branch{1}{N} \qobitree}
{ \leaf{cat} \branch{1}{N} \qobitree}

\begin{tabular}{ccc}
\multicolumn{2}{c}{NP \into D N} & VP \into V NP \\[2ex]

{ \leaf{the} \branch{1}{D}
  \leaf{dog} \branch{1}{N} 
\branch{2}{NP} \qobitree}
&
{ \leaf{the} \branch{1}{D}
  \leaf{cat} \branch{1}{N} 
\branch{2}{NP} \qobitree}
&
{
  \leaf{chased} \branch{1}{V}
    \leaf{the} \branch{1}{D}
    \leaf{cat} \branch{1}{N} 
  \branch{2}{NP}
\branch{2}{VP} \qobitree}
\end{tabular}

\newpage

\begin{tabular}{c}
 S \into NP VP \\[2ex]
{
    \leaf{the} \branch{1}{D}
    \leaf{dog} \branch{1}{N} 
  \branch{2}{NP}
    \leaf{chased} \branch{1}{V}
      \leaf{the} \branch{1}{D}
      \leaf{cat} \branch{1}{N} 
    \branch{2}{NP}
  \branch{2}{VP}
\branch{2}{S} \qobitree}
\end{tabular}

\myslide{Top-down Tree Construction}


\begin{tabular}{cccc}
 S \into NP VP  & VP \into V NP & NP \into D N & NP \into D N \\[2ex]
{ \leaf{NP}
  \leaf{VP}
\branch{2}{S} \qobitree}
&
{ \leaf{V}
  \leaf{NP}
\branch{2}{VP} \qobitree}
&
{ \leaf{D}
  \leaf{N}
\branch{2}{NP} \qobitree}
&
{ \leaf{D}
  \leaf{N}
\branch{2}{NP} \qobitree}
\end{tabular}

{ \leaf{D}
  \leaf{N}
\branch{2}{NP} 
   \leaf{V}
   \leaf{D}
   \leaf{N}
\branch{2}{NP}
\branch{2}{VP}
\branch{2}{S} \qobitree}
{ \leaf{the} \branch{1}{D} \qobitree}
{ \leaf{the} \branch{1}{D} \qobitree}
{ \leaf{chased} \branch{1}{V} \qobitree}
{ \leaf{dog} \branch{1}{N} \qobitree}
{ \leaf{cat} \branch{1}{N} \qobitree}

\begin{tabular}{c}
% Combine \\[2ex]
{
    \leaf{the} \branch{1}{D}
    \leaf{dog} \branch{1}{N} 
  \branch{2}{NP}
    \leaf{chased} \branch{1}{V}
      \leaf{the} \branch{1}{D}
      \leaf{cat} \branch{1}{N} 
    \branch{2}{NP}
  \branch{2}{VP}
\branch{2}{S} \qobitree}
\end{tabular}

\begin{itemize}
\item \txx{Bottom-up}: string \into tree
\item \txx{Top-down}: tree \into string
\item CFG is \txx{declarative} so it is independent of order
\end{itemize}

\myslide{Weaknesses of CFG (atomic node labels)}
\begin{itemize}
\item It doesn’t tell us what constitutes a linguistically
natural rule
\begin{itemize}
\item VP \into P NP
\item NP \into VP S
\end{itemize}
\item Rules get very cumbersome once we try to deal
with things like agreement and transitivity.
\item It has been argued that certain languages (notably
Swiss German and Bambara) contain constructions
that are provably beyond the descriptive capacity of
CFG.
\end{itemize}

\myslide{On the other hand \ldots}

\begin{itemize}
\item It’s a simple formalism that can generate
infinite languages and assign linguistically
plausible structures to them.

\item Linguistic constructions that are beyond the
descriptive power of CFG are rare.

\item It’s computationally tractable and
techniques for processing CFGs are well
understood.
\end{itemize}

\myslide{So \ldots}

\begin{itemize}
\item CFG is the starting point for most
types of generative grammar.

\item The theory we develop in this course is an
extension of CFG.
\end{itemize}

\myslide{Transitivity and Agreement}

\begin{itemize}
\item Consider the following transitivity examples
  \begin{exe}
    \ex \eng{The bird arrives}
    \ex \eng{The bird devours the worm}
    \ex *\eng{The bird arrives the worm}
    \ex *\eng{The bird devours}
  \end{exe}
\item Consider the following agreement examples
  \begin{exe}
    \ex \eng{The bird sings}
    \ex \eng{The birds sing}
    \ex *\eng{The bird sing}
    \ex *\eng{The birds sings}
  \end{exe}
\item Can we deal with them with a CFG?
\end{itemize}



\myslide{Basic Concepts of Morphology: Morphological Models}
\begin{itemize}
\item \txx{Item-and-Arrangement (IA)}: list morphemes + specify their linear arrangement.
  \\ \eng{spiral + -ize + -er}. 
\item \txx{Item-and-Process (IP)}: items undergo processes/rules
  \\ (\eng{make-verb}, \eng{instrument-noun}). 
\item \txx{Word-and-Paradigm (WP)}: words as \txx{word-forms} in \txx{paradigms};
  \\ realizational rules fill paradigm slots. 
\item Each model fits different language properties, \ldots
 \end{itemize}

Based on Panocová (2021, Chapter 3), Czech examples added

% 2
\myslide{IA: The Morpheme List + Arrangement}
\begin{itemize}
  \item Hockett: utterances = minimal meaningful elements (\txx{morphemes}) arranged linearly.
  \item Focus: \txx{free} vs.\ \txx{bound} forms; segment-and-label.
  \item IA Worked Example
    \begin{itemize}
    \item \eng{spiralizer} $\Rightarrow$ \eng{spiral} + \eng{-iz(e)} + \eng{-er}.
    \item Inventory entries: \eng{spiral} (free), \eng{-ize} (V-forming), \eng{-er} (N-forming: instrument/agent).
    \end{itemize}
  \end{itemize}
  
\myslide{IA Strengths and Weaknesses}    
\begin{itemize}
\item IA Strength: Transparent Affixation
    \begin{itemize}
    \item Agglutinative-style sequences: \txx{one function ⟷ one form}.
    \item Hungarian: suffix allomorphs by stem vowels (back vs.\ front). \\
      	\begin{tabular}{llll}
	\textbf{olvas-} & `read'	& \textbf{yl-} & `sit'\\
	\textbf{olvas-unk}	& `we read'	& \textbf{yl-ynk	} & `we sit'\\
	\textbf{olvas-tok}	& `you (pl) read'	& \textbf{yl-tyk} & 	`you (pl) sit'\\
	\textbf{olvas-nak}	& `they read'	& \textbf{yl-nek}	& `they sit'\\
	\end{tabular}

    \end{itemize}
  \item IA Challenges: When Segments Don’t Behave
    \begin{itemize}
    \item \txx{Ablaut} pasts: \eng{take} $\rightarrow$ \eng{took}—no neat “past” morpheme.
    \item \txx{Cumulative exponence}: Czech \eng{-u} in \eng{ženu} = [accusative singular feminine].
    \item Such patterns motivate IP/WP analyses.
    \end{itemize}
  \end{itemize}

\myslide{IP: Items Undergo Processes}
\begin{itemize}
  \item Lexicon stores bases; morphology applies \txx{processes} to derive outputs
  \item \eng{spiral} → \eng{spiralize} (\texttt{make-verb})
    \\\eng{spiralize}  → \eng{spiralizer} (\texttt{instrument-noun})
  \item IA vs.\ IP: What Counts as an “Item”?
    \begin{itemize}
    \item \txx{IA}: morphemes + concatenation.
    \item \txx{IP}: bases + processes (rules for alternations, allomorphy)
    \end{itemize}
\end{itemize}

% 8
\myslide{WP: Words and Paradigms}
\begin{itemize}
  \item Central unit: \txx{word-form} (member of a lexeme’s paradigm).
  \item Goal: \txx{realizational rules} filling paradigm slots.
  \item Great for \txx{syncretism}, rich inflection.
  \item WP in Action: Czech Masculine Nouns
\begin{itemize}
  \item Genitive plural is \eng{-ů} across many declension types. \\
\begin{tabular}{l|l}
Lexeme & Gen.\ pl. \\
\hline
\eng{pán} `lord' & \eng{pánů} \\
\eng{hrad} `castle' & \eng{hradů} \\
\eng{muž} `man' & \eng{mužů} \\
\eng{stroj} `machine' & \eng{strojů} \\
\end{tabular}
\end{itemize}
\medskip
\begin{itemize}
  \item Rule (simplified): $x_{N:Mc}$[+pl, +gen] $\Rightarrow$  \eng{$x$-ů}$_{N}$.
\end{itemize}
\end{itemize}

% 10
\myslide{Mini-paradigm Tables (Czech)}
\begin{itemize}
  \item \eng{pán} `lord' (paradigm: nominative–accusative–genitive plural):
\end{itemize}
\begin{tabular}{l|l}
Nom.\ sg. & \eng{pán} \\
Acc.\ sg. & \eng{pána} \\
Gen.\ pl. & \eng{pánů} \\
\end{tabular}

\medskip
\begin{itemize}
  \item \eng{hrad} `castle' (hard-type masculine inanimate):
\end{itemize}
\begin{tabular}{l|l}
Nom.\ sg. & \eng{hrad} \\
Acc.\ sg. & \eng{hrad} \\
Gen.\ pl. & \eng{hradů} \\
\end{tabular}

There can be many sub rules and irregularities


\myslide{Choosing the Right Lens}
\begin{itemize}
  \item \txx{IA}: clean concatenation, transparent affixation, listable allomorphy (also influential in \txx{Distributed Morphology}). 
  \item \txx{IP}: rules/processes for alternations, non-linear morphology, construction-based generalizations. 
  \item \txx{WP}: paradigm-centered patterns, syncretism, realizational statements over lexemes. 
  \item Use the model that best fits the language property you’re analyzing. %
\end{itemize}


\myslide{How Humans Handle Morphology}
\begin{itemize}
  \item We combine \txx{rules} with \txx{memory}.
  \item \txx{Rules / Productivity}: apply general patterns 
    (\eng{talk $\rightarrow$ talked}, Cz.\ \eng{hrad $\rightarrow$ hradů}).
  \item \txx{Lexical Memory}: store irregulars 
    (\eng{go $\rightarrow$ went}, Cz.\ \eng{oko $\rightarrow$ oči})  and regular high-frequency forms! 
  \item \txx{Dual-route model}: both operate in parallel.
\end{itemize}

% Optional second slide
\myslide{Evidence: Rules + Exceptions Together}
\begin{itemize}
  \item \txx{Analogy}: children overgeneralize (\eng{goed}, Cz.\ \eng{?hradové}).
  \item \txx{Frequency effects}: common forms often stored whole, retrieved faster.
\begin{itemize}
  \item Even \txx{regular} forms can show whole-word frequency effects at high frequency.
  \item \txx{English} \eng{V+ed}: very frequent pasts (e.g., \eng{looked}, \eng{called}) behave like stored units rather than rule-built on the fly. \citep{Alegre:Gordon:1999}.
  \item \txx{Dutch} \eng{N+en}: high-frequency regular plurals show evidence of storage alongside parsing. \citep{Baayen:Dijkstra:Schreuder:1997}.
\end{itemize}
    
  \item \txx{Statistical sensitivity}: humans track distribution of endings; 
    in Czech, \eng{-ů} becomes the \txx{default} genitive plural despite exceptions.
  \item Result: humans learn to generalize where possible, memorize where needed,
    and adjust as experience grows.
\end{itemize}

\myslide{Classification of grammatical categories}
Languages do not differ so much in what you can say, but rather in \textit{how} you must say it.
	
  \begin{itemize}
  \item
    \textbf{Inherent categories}: show a property related to the word class they are attached to.  For example, plural in English.
  \item
    \textbf{Agreement categories}: show a property related to another word in the sentence.  For example in English, an \emph{-s} on the verb shows that the subject of the verb is 3rd person singular.
   \item
  \textbf{Relational categories}: show how a word fits into a larger structure.  For example English  \emph{I} for nominative case or \emph{me} for accusative case. The  form is determined by the grammatical relation (subject or object) of the argument.
  \end{itemize}

%-------------------------------------------------------------------------------------------
\myslide{Different languages have different grammatical categories}
  There is lots of variation across the languages of the world.\\
\vspace{3mm}
Examples of nominal grammatical categories:
  \begin{itemize}
  \item
    person: 1st/2nd/3rd; inclusive~exclusive distinctions
  \item
  number: singular/dual/plural
  \item
 noun class or gender
  \item
case: core versus oblique
  \item
definiteness/specificity
  \end{itemize}

\myslide{Verbal grammatical categories}

  \begin{itemize}
  \item
    person
  \item
  number and gender of arguments as agreement categories
  \item
 temporal deixis
\begin{enumerate}
	\item tripartite systems: past/present/future
    	\item binary systems: past/non-past; non-future/future
    	\item metrical tense systems
        \end{enumerate}
        \newpage
  \item
Aktionsart (lexical aspect): static, telic, punctual
  \item
Aspect:
\begin{enumerate}
	\item perfective and imperfective
	\item progressive
	\item perfect
\end{enumerate}
\item
Mood and modality:  Realis versus irrealis; indicative, subjunctive, interrogative, imperative;
\item
Evidentiality
  \end{itemize}

\myslide{Czech verb derivational \& inflectional morphology}


\begin{tabular}{lll}
 & Czech & English gloss\\
 \hline
a. &  dát & `to give'\\
b. & dávat & `to keep giving'\\
c. & dám & `I will give'\\
d. & předáš & `you will hand over'\\
e. & prodáme & `we will sell (hand in exchange)'\\
f. & vyprodá & `she will sell off'\\
g. & vydám & `I will hand out, give out'\\
h. & povyprodáme & `we will sell out gradually'\\
i. & dáváš & `you keep giving'\\
j. & dopovyprodávášmi & `you will gradually finish selling (it) out for me'\\
k. & dopovyprodávámeti & `we will gradually finish selling (it) out for you'\\
\end{tabular}

What kind of information is encoded in the morphology?



\myslide{Position-class diagrams: Czech verbs}

\begin{center}%\small
\begin{tabular}{llllllllll}
    &	p4	& p3		& p2		& p1		& root 	& s1 		& s2	& s3	& s4\\
 \hline
a. &  		&		&		&		& dá		&		& -t & 	&\\
b. & 		&		&		&		& dá		& -va		& -t & 	&\\
c. & 		&		&		&		& dá		&		& -m & 	&\\
d. & 		& 		&		& pře-	& dá 		& 		& -š 	& 	&\\
e. & 		&		&		& pro-	& dá		&		& -m & -e & \\
f. & 		&		& vy- 	& pro-	& dá 		& 		&	&	&\\
g. & 		&		&		& vy- 	& dá		&		& -m & 	&\\
h. & 		& po-	& vy-		& pro-	& dá		& 		& -m & -e & \\
i. & 		&		&		&		& dá 		& -vá 	& -š 	& 	& \\
j. & do- 	& po- 	& vy- 	& pro- 	& dá 		& -vá 	& -š 	& 	& -mi\\
k. & do- 	& po- 	& vy- 	& pro- 	& dá 		& -vá 	& -m	& -e 	& -ti\\
\hline
 & \textsc{asp} 	& \textsc{asp} 	& \textsc{asp} 	& \textsc{dir} 	& root 	& \textsc{asp} 	& \textsc{s}	& \textsc{num} 	& \textsc{o}\\

\end{tabular}
\end{center}

\myslide{Position-class diagrams: Czech verbs}

\bigskip

\hspace*{-4em}
{\small
\begin{tabular}{|l|l|l|l|l|l|l|l|l|}
\hline
prefix.4	& prefix.3		& prefix.2		& prefix.1		& root 	& suffix.1 		& suffix.2	& suffix.3	& suffix.4\\
\hline
\textsc{asp} 	& \textsc{asp} 	& \textsc{asp} 	& \textsc{direction} 	& root 	& \textsc{iterative} 	& \textsc{subj}	& \textsc{number} 	& \textsc{obj}\\
 do- 	& po- & vy- & pro- `for' & dá 	& va-$\sim$-vá & -m (1) & -e (\textsc{pl}) & =mi (\textsc{1sg.dat}) \\
`to'	& `along' 	& `out'  & pře- `over' & `give' 	& 			& -š (2sg) & 			 & =ti (\textsc{2sg.dat}) \\
	&  	&  	& vy- `out' &  	& 			& -0 (3sg) & 			 & 					 \\	
\hline
\end{tabular}}

Morphology is full of special cases

\begin{itemize}
\item Not every root can take every affix
\item The meaning changes are unpredictable
\item There are often sound/spelling changes
\item But you can see patterns everywhere!
\end{itemize}

  
\myslide{How do we do linguistic analysis?}
\begin{enumerate}\addtolength{\itemsep}{-1ex}
\item Learn the Fundamentals
\item Investigate
\item Find out some stuff
\item Break our theory
\item Try to fix it.
\item Break it again.
\item Lather, rinse, repeat: we'll do that until we run out of time.
\end{enumerate}

Jorge Hankamer's outline of a syntax course, but it's pretty
applicable to everything we do.  More formally: \emp{Successive Approximation}.


\myslide{Chapter 2, Problem 1}
\begin{tabular}{cc}
  RULES & VOCABULARY \\[2ex]
  \begin{minipage}[t]{0.4\linewidth}
    \begin{tabular}[t]{lll}
      \textbf{S}  & \into & NP VP \\
      NP & \into & (D) NOM\\
      VP & \into & V (NP) (NP)\\
      NOM & \into & N\\
      NOM & \into & NOM PP \\
      VP  & \into & VP PP\\
      PP  & \into & P NP\\
      X   & \into & X+ CONJ X\\
    \end{tabular} 
  \end{minipage} &
  \begin{minipage}[t]{0.5\linewidth}
    \begin{flushleft}
  D: a, the\\
  N: cat, dog, hat, man, woman, roof\\
  V: admired, disappeared, put, relied\\
  P: in, on, with\\
  CONJ: and, or
\end{flushleft}
\end{minipage}
\end{tabular}



\myslide{Chapter 2, Problem 1}

\begin{itemize}
\item [A] Make a well-formed English sentence unambiguous
according to this grammar
\item  [B] Make a well-formed English sentence ambiguous
according to this grammar: draw trees
\item  [C] Make a well-formed English sentence not licensed by
this grammar (using $V$)
\item [D] Why is this (C) not licensed?

\newpage
\item [E] Make a string licensed by this grammar that is not a
  well-formed English sentence
\item [F] How can we stop licensing the string in E (stop over-generating)
\item [G] How many strings does this grammar license?
\item [H] How many strings does this grammar license without conjunctions?
\end{itemize}

\myslide{Shieber 1985}
\begin{itemize}
\item Swiss German example:
  \begin{exe}
    \ex \gll \ldots mer \uline{d’chind} \uuline{em Hans} es \uwave{huus} \uline{l\"ond} \uuline{h\"alfe} \uwave{aastriiche}\\
    \ldots we {the children-acc} Hans-dat the hous-acc let help paint \\
    \trans we let {the children} help Hans paint the house 
  \end{exe}
\item Cross-serial dependency:
\begin{itemize}
\item \eng[let]{l\"ond} governs case on \eng[children]{d’chind}
\item \eng[help]{h\"alfe} governs case on \eng[Hans]{Hans}
\item \eng[paint]{aastriiche} governs case on \eng[house]{huus}
\end{itemize}
\item This cannot be modeled in a context free language
\end{itemize}

\myslide{Strongly/weakly CF}
\begin{itemize}
\item A language is weakly \emp{context-free} if the set of
strings in the language can be generated by a CFG.
\item A language is \emp{strongly} context-free if the CFG
furthermore assigns the correct structures to the
strings.
\item Shieber’s argument is that SW is not \emp{weakly}
context-free and therefore not \emp{strongly} context-free.
\item Bresnan et al (1983) had already argued that Dutch
is \emp{strongly} not context-free, but the argument was
dependent on linguistic analyses.
\end{itemize}





\myslide{Overview}

\begin{itemize}
\item Formal definition of CFG
  \begin{itemize}
  \item Constituency, ambiguity, constituency tests
  \item Central claims of CFG
  \item Order independence
  \item Weaknesses of CFG
  \end{itemize}
\item Three approaches to morphology
\item Next Week: Feature structures
\end{itemize}


\myslide{Acknowledgments and References}

\begin{itemize}
\item Course design and slides borrow heavily from Emily Bender's course:
\textit{Linguistics 566: Introduction to Syntax for Computational Linguistics}
\\ \url{http://courses.washington.edu/ling566}
\item Thanks to Alex Coupe for some inspirational slides
\item Stuart M. Shieber. (1985) Evidence against the context-freeness of natural language. \textit{Linguistics and Philosophy}, 8:333-343
\end{itemize}

\bibliographystyle{aclnat}
\bibliography{abb,mtg,nlp,ling}


\end{document}


%%% Local Variables: 
%%% coding: utf-8
%%% mode: latex
%%% TeX-PDF-mode: t
%%% TeX-engine: xetex
%%% End: 

