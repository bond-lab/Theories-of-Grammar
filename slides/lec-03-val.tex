\RequirePackage{expl3}
\documentclass[a4paper,landscape,headrule,footrule]{foils}
\PassOptionsToPackage{xetex}{expl3}
%%
%%% macros for Theories of Grammar
%%%
\usepackage{polyglossia}
\setdefaultlanguage{english}
%\setmainfont{TeX Gyre Pagella}


\newcommand{\logo}{~}
\newcommand{\header}[3]{%
\title{\vspace*{-2ex} \large HG4041 Theories of Grammar
\\[2ex] \Large  \emp{#2} \\ \emp{#3}}
\author{\blu{Francis Bond}   \\ 
\normalsize  \textbf{Division of Linguistics and Multilingual Studies}\\
\normalsize  \url{http://www3.ntu.edu.sg/home/fcbond/}\\
\normalsize  \texttt{bond@ieee.org}}
\MyLogo{HG4041 (2020)}
\renewcommand{\logo}{#2}
\hypersetup{
   pdfinfo={
     Author={Francis Bond},
     Title={#1: #2},
     Subject={HG4041: Theories of Grammar},
     Keywords={Syntax, Semantics, HPSG, Unification, Constructions},
     License={CC BY 4.0}
   }
 }

\date{#1
  \\ Location: LHN-TR+36}
}
\usepackage[hidelinks]{hyperref}




\usepackage{xcolor}
\usepackage{graphicx}
\newcommand{\blu}[1]{\textcolor{blue}{#1}}
\newcommand{\grn}[1]{\textcolor{green}{#1}}
\newcommand{\hide}[1]{\textcolor{white}{#1}}
\newcommand{\emp}[1]{\textcolor{red}{#1}}
\newcommand{\txx}[1]{\textbf{\textcolor{blue}{#1}}}
\newcommand{\lex}[1]{\textbf{\mtcitestyle{#1}}}

\usepackage{pifont}
\renewcommand{\labelitemi}{\textcolor{violet}{\ding{227}}}
\renewcommand{\labelitemii}{\textcolor{purple}{\ding{226}}}

\newcommand{\subhead}[1]{\noindent\textbf{#1}\\[5mm]}

\newcommand{\Bad}{\emp{\raisebox{0.15ex}{\ensuremath{\mathbf{\otimes}}}}}
\newcommand{\bad}[1]{*\eng{#1}}

\newcommand{\com}[1]{\hfill (\emp{#1})}%

\usepackage{relsize,xspace}
\newcommand{\into}{\ensuremath{\rightarrow}\xspace}
\newcommand{\tot}{\ensuremath{\leftrightarrow}\xspace}
\usepackage{url}
\newcommand{\lurl}[1]{\MyLogo{\url{#1}}}

\usepackage{mygb4e}
\newcommand{\lx}[1]{\textbf{\mtciteform{#1}}}
\newcommand{\ix}{\ex\slshape}
\let\eachwordone=\slshape



\newcommand{\ent}{\ensuremath{\Rightarrow}\xspace}
\newcommand{\ngv}{\ensuremath{\not\Rightarrow}\xspace}
%\usepackage{times}
%\usepackage{nttfoilhead}
\newcommand{\myslide}[1]{\foilhead[-25mm]{\raisebox{12mm}[0mm]{\emp{#1}}}\MyLogo{\logo}}
\newcommand{\myslider}[1]{\rotatefoilhead[-25mm]{\raisebox{12mm}[0mm]{\emp{#1}}}}
%\newcommand{\myslider}[1]{\rotatefoilhead{\raisebox{-8mm}{\emp{#1}}}}

\newcommand{\section}[1]{\myslide{}{\begin{center}\Huge \emp{#1}\end{center}}}



\usepackage[lyons,j,e,k]{mtg2e}
%\renewcommand{\mtcitestyle}[1]{\textcolor{-red!75!green!50}{\textsl{#1}}}
\renewcommand{\mtcitestyle}[1]{\textcolor{teal}{\textsl{#1}}}
\newcommand{\iz}[1]{\texttt{\textup{#1}}}
\newcommand{\gm}{\textsc}
\usepackage[normalem]{ulem}
\newcommand{\ul}{\uline}
\newcommand{\ull}{\uuline}
\newcommand{\wl}{\uwave}
\newcommand{\vs}{\ensuremath{\Leftrightarrow}~}
%%%
%%% Bibliography
%%%
\usepackage{natbib}
%\usepackage{url}
\usepackage{bibentry}


%%% From Tim
\newcommand{\WMngram}[1][]{$n$-gram#1\xspace}
\newcommand{\infers}{$\rightarrow$\xspace}

\usepackage[utf8]{inputenc}

\usepackage{rtrees,qtree}
\renewcommand{\lf}[1]{\br{#1}{}}
\usepackage{avm}
%\avmoptions{topleft,center}
\newcommand{\ft}[1]{\textsc{#1}}
\renewcommand{\val}[1]{\textit{#1}}
\newcommand{\typ}[1]{\textit{#1}}
\newcommand{\prd}[1]{\textbf{#1}}
\avmfont{\sc}
\avmvalfont{\it}
\avmsortfont{\smaller[2] \it}
\usepackage{multicol}
\newcommand{\blank}{\rule{3em}{1pt}\xspace}

% \usepackage{pst-node}
\newcommand{\OV}[1]{\ovalnode[linestyle=dotted,linecolor=red]{A}{#1}}
\newcommand{\OVB}[1]{\ovalnode[linestyle=dotted,linecolor=blue]{A}{#1}}

%%% From CSLI book
\newcommand{\mc}{\multicolumn}
\newcommand{\HD}{\textbf{H}\xspace}
\newcommand{\el}{\< \>}
\makeatother
\long\def\smalltree#1{\leavevmode{\def\\{\cr\noalign{\vskip12pt}}%
\def\mc##1##2{\multispan{##1}{\hfil##2\hfil}}%
\tabskip=1em%
\hbox{\vtop{\halign{&\hfil##\hfil\cr
#1\crcr}}}}}
\makeatletter

\newcommand{\A}{\noindent\textbf{A}: }
\newcommand{\Q}{\noindent\textbf{Q}: }
%\newcommand{\C}{\noindent\textbf{C}: }



\avmfont{\sc}
\begin{document}
\avmfont{\it}

\header{Lecture 3}{Complex Feature Values}{Valence, Agreement, Case}
\maketitle


\myslide{Overview}
\MyLogo{Chapter 4}
\begin{itemize}
\item Last week
\item A problem with the Chapter 3 grammar
\item Generalize COMPS and SPR
\item The Valence Principle
\item Agreement
\item The SHAC
\item Work through problems 4.1, 4.5, 4.6
\end{itemize}

\myslide{Pizza review}
\MyLogo{}
\begin{itemize}
\item Unification is an operation for combing
constraints from different sources.

\item What are those sources in the pizza
example?

\item Why do we need to combine information
from different sources in our grammars?
\end{itemize}

\myslide{Reminder: Where We Are}
\begin{itemize}
\item Attempting to model English with CFG led to problems
  with the granularity of categories, e.g.
  \begin{itemize}
  \item Need to distinguish various subtypes of verbs
  \item Need to identify properties common to all verbs
  \end{itemize}
\begin{itemize}
\item So we broke categories down into feature structures and
  began constructing a hierarchy of types of feature
  structures.
\item This allows us to schematize rules and state crosscategorial
  generalizations, while still making fine distinctions.
\end{itemize}
\end{itemize}

\myslide{Heads}
\begin{itemize}
\item Intuitive idea: A phrase typically contains a word that
determines its most essential properties, including
\begin{itemize}
\item where it occurs in larger phrases
\item what its internal structure is
\end{itemize}
\item This is called the \txx{head}
\item The term \txx{head} is used both for the head word in a
phrase and for all the intermediate phrases containing
that word
\item NB: Not all phrases have heads
\end{itemize}

\myslide{Formalizing the Notion of Head}
\begin{itemize}
\item Expressions have a feature \ft{head}
\item \ft{head}’s values are of type \val{pos}
\item For \ft{head} values of type \val{agr-cat}, \ft{head}’s
value also includes the feature \ft{agr}
\item Well-formed trees are subject to the \txx{Head
Feature Principle}:
\begin{quote}
  In any headed phrase, the \ft{head}
value of the mother and the head daughter
must be identical.
\end{quote}
\end{itemize}

\myslide{A Tree is Well-Formed if \ldots }
\begin{itemize}
\item It and each subtree are licensed by a grammar rule
or lexical entry
\item All general principles (like the HFP) are satisfied.
\item NB: Trees are part of our model of the language,
so all their features have values (even though we
will often be \sout{lazy} compact and leave out the values
irrelevant to our current point).
\end{itemize}


\myslide{But it’s still not quite right \ldots }
There’s still too much redundancy: the rules and features encode the same information in different ways.

\begin{small}
\begin{avm} 
\[{\it phrase}\\
  VAL & \[ COMPS & itr\\
           SPR & $-$\]\] \ $\rightarrow$\ \HD \ \[{\it word}\\
                                               VAL & \[ COMPS & itr\\
                                                        SPR & $-$\]\]
\end{avm}

\begin{avm} 
\[{\it phrase}\\
  VAL & \[ COMPS & itr\\
           SPR & $-$ \]\] \ $\rightarrow$\ \HD \ \[{\it word}\\
                                               VAL & \[ COMPS & str\\
                                                        SPR & $-$\]\]\ \ NP
\end{avm}

\begin{avm} 
\[{\it phrase}\\
  VAL & \[ COMPS & itr\\
           SPR & $-$ \]\] \ $\rightarrow$\ \HD \ \[{\it word}\\
                                               VAL & \[ COMPS & dtr\\
                                                        SPR & $-$\]\]\ \ NP\ \ NP
\end{avm}
\end{small}


\myslide{Solution: More Elaborate Valence Feature Values}
\begin{itemize}
\item The rules just say that heads combine with whatever
their lexical entries say they can (or must) combine
with.
\item The information about what a word can or must
combine with is encoded in \emp{list-valued} valence
features.
\begin{itemize}
\item The elements of the lists are themselves feature structures
\item The elements are \emp{``cancelled''} off the lists once heads
combine with their complements and specifiers.
\end{itemize}
\end{itemize}

\myslide{Complements}
Head-Complement Rule:

\begin{avm}\avmfont{\sc} 
\[ \asort{phrase}  val & \[ comps & \<  \> \] \]
              \ $\rightarrow$\ \HD \ 
             \[ \asort{word}
             val & \[ comps & \< \@1, \ldots, \@{n} \> \] \]\ \ \@1, \ldots, \@{n}
\end{avm}
\begin{itemize}
\item This allows for arbitrary numbers of complements, but only
applies when there is at least one.
\begin{itemize}
\item The possible complements are specified lexically
\item Heads in English probably never have more than 3 or 4
complements
\item This doesn’t apply where Head-Complement Rule 1 would. (Why?)
\end{itemize}
\item This can cover lots of cases not covered by the old Head-Complement Rules 1-3. (Examples?)
\end{itemize}

\myslide{The valence comes from the lexicon}

\begin{small}
  \begin{tabular}{ll}
    \begin{avm}\avmfont{\sc}
      \< \textnormal{devour}, \ \[ \asort{word}
      head & {\it verb} \\
      % \[ \asort{agr-cat} agr & \[ per & 3 \\ num & sg \] \]
      val & \[ comps & \< NP \> \]
      \]\>
    \end{avm} &
    \begin{avm}\avmfont{\sc}
      \< \textnormal{put}, \ \[ \asort{word}
      head & {\it verb} \\
      % \[ \asort{agr-cat} agr & \[ per & 3 \\ num & sg \] \]
      val & \[ comps & \< NP PP \> \]
      \]\>
    \end{avm} \\
    \begin{avm}\avmfont{\sc}
      \< \textnormal{eat}, \ \[ \asort{word}
      head & {\it verb} \\
      % \[ \asort{agr-cat} agr & \[ per & 3 \\ num & sg \] \]
      val & \[ comps & \< (NP) \> \]
      \]\>
    \end{avm} &
    \begin{avm}\avmfont{\sc}
      \< \textnormal{bet}, \ \[ \asort{word}
      head & {\it verb} \\
      % \[ \asort{agr-cat} agr & \[ per & 3 \\ num & sg \] \]
      val & \[ \< NP (NP) (S) \> \]
      \]\>
    \end{avm} \\
    \begin{avm}\avmfont{\sc}
      \< \textnormal{dine}, \ \[ \asort{word}
      head & {\it verb} \\
      % \[ \asort{agr-cat} agr & \[ per & 3 \\ num & sg \] \]
      val & \[ comps & \< \> \]
      \]\>
    \end{avm} &
    \begin{avm}\avmfont{\sc}
      \< \textnormal{fond}, \ \[ \asort{word}
      head & {\it adjective} \\
      % \[ \asort{agr-cat} agr & \[ per & 3 \\ num & sg \] \]
      val & \[ comps & \< PP\emp{:\textit{of}} \> \]
      \]\>
    \end{avm}
  \end{tabular}
\end{small}


\myslide{Question: What if English had postpositions?}

\begin{avm}\avmfont{\sc} 
\[ \asort{phrase}  val & \[ comps & \<  \> \] \]
              \ $\rightarrow$\ \HD \ 
             \[ \asort{word}  head & \val{verb \| adj \| noun} \\
             val & \[ comps & \< \@1, \ldots, \@{n} \> \] \]\ \ \@1, \ldots, \@{n}
\end{avm}

\begin{avm}\avmfont{\sc} 
\[ \asort{phrase}  val & \[ comps & \<  \> \] \]
              \ $\rightarrow$\ \ \ \@1, \ldots, \@{n}\ \ \HD \ 
             \[ \asort{word} head & \val{preposition} \\
             val & \[ comps & \< \@1, \ldots, \@{n} \> \] \]
\end{avm}




\myslide{Specifiers}

In English, nouns can agree with their specifiers.
\begin{itemize}
\item In Number:
  \begin{exe}
    \ix This dog barked.
    \ix \bad This dogs barked.
    \ix \bad These dog barked.
    \ix These dogs barked.
  \end{exe}
\item In Countability
  \begin{exe}
\ix Much furniture was broken.
\ix * A furniture was broken.
\ix * Much chair was broken.
\ix A chair was broken. 
\end{exe}
\end{itemize}

\myslide{Head-Specifier Rule (Version I)}

\begin{avm}\avmfont{\sc}
  \[{\it phrase}\\
    val & \[ comps & \el\\
      spr & \el\]\]\ \ \
  $\rightarrow$\ \ \ \ {\@2}\ \ \ \ \HD\[ %{\it phrase}\\
                                           val & \[ comps & \el\\
                                           spr & \q< {\@2} \q> \]\] 
\end{avm}
\begin{itemize}
\item Combines the rules expanding S and NP.
\item In principle also generalizes to other categories.
\item Question: Why is SPR list-valued?
\end{itemize}

\myslide{Question:}


Why are these right-branching? That is, what formal property of our
grammar forces the COMPS to be lower in the tree than the SPR?

\begin{center}
  \begin{tree}
    \br{S}{\br{NP}{} \br{VP}{\br{V}{}\br{NP}{}}}
  \end{tree}
  \begin{tree}
    \br{NP}{\br{D}{} \br{NOM}{\br{N}{}\br{PP}{}}}
  \end{tree}
\end{center}


\myslide{Another Question \ldots}
\begin{itemize}
\item What determines the VAL value of phrasal nodes?
\item \txx{The Valence Principle}
  \begin{quote}
    Unless the rule says otherwise, the mother’s
    values for the VAL features (SPR and
    COMPS) are identical to those of the head
    daughter.
  \end{quote}
\end{itemize}


\myslide{More on the Valence Principle}
\begin{itemize}
\item Intuitively, the VAL features list the contextual
requirements that haven’t yet been found.
\item This way of thinking about it (like talk of
“cancellation”) is bottom-up and procedural.
\item But formally, the Valence Principle (like the rest of
our grammar) is just a well-formedness constraint
on trees, without inherent directionality.

\end{itemize}

\myslide{So far, we have:}
\begin{itemize}
\item Replaced atomic-valued VAL features with list-valued
ones.
\item Generalized Head-Complement and Head-Specifier
rules, to say that heads combine with whatever their
lexical entries say they should combine with.
\item Introduced the Valence Principle to keep the information on the
  COMPS and SPR lists until it gets ``canceled'' by the
  Head-Complement and Head-Specifier rules.
\end{itemize}

\myslide{The Parallelism between S and NP}
\begin{itemize}
\item Motivation:
\begin{itemize}
  \item pairs like 
    \begin{exe}
      \ix Chris lectured about syntax
      \ix Chris’s lecture about syntax
    \end{exe}
  \item both S and NP exhibit agreement
 \begin{exe}
   \ix The bird sings/*sing
   \ix The birds sing/*sings
   \ix this/*these bird 
   \ix these/*this birds
\end{exe}
\end{itemize}
\item So we treat NP as the saturated category of type \val{noun}
and S as the saturated category of type \val{verb}.
\end{itemize}

\myslide{Any other reason to treat V as the head of S?}
\begin{itemize}
\item In standard English, sentences must have verbs.
\\ (How about non-standard English or other
languages?)
\item Verbs taking S complements can influence the form
of the verb in the complement:
\begin{exe}
  \ix I insist/*recall (that) you \ul{be} here on time.
\end{exe}
\item Making V the head of S helps us state such
restrictions formally
\end{itemize}

\myslide{A possible formalization of the restriction}


   \begin{avm}\avmfont{\sc}
      \< \textnormal{insist}, \ \[ \asort{word}
      head & \textit{verb}\\
      % \[ \asort{agr-cat} agr & \[ per & 3 \\ num & sg \] \]
      val & \[ spr & \< NP \> \\
               comps & \< S\[ head & \[  mood & subjunctive \] \] \> \]
      \]\>
    \end{avm} %\[ \asort{verb} mood & subjunctive \]

Note that this requires that the verb be the head of the
complement. We don’t have access to the features of the other
constituents of the complement.

\begin{center}\small
  \begin{avm}\avmfont{\sc}
    S =  \[ head & \textit{verb}\\
    val & \[ spr & \<  \> \\
    comps & \< \> \] \]
  \end{avm}
\end{center}

\myslide{Complements vs. Modifiers}
\MyLogo{Arguments vs Adjuncts}
\begin{itemize}
\item Intuitive idea: Complements introduce
essential participants in the situation denoted;
modifiers refine the description.
\item Generally accepted distinction, but disputes
over individual cases.
\item Linguists rely on heuristics to decide how to
analyze questionable cases (usually PPs).
\end{itemize}


\myslide{Heuristics for Complements vs. Modifiers}
\MyLogo{\ent = entail; \ngv = doesn't entail}
\begin{itemize}
\item Obligatory PPs are usually complements.
\item Temporal \& locative PPs are usually modifiers.
\item An entailment test:
\\ If \blu{\textit{X Ved (NP) PP} \ngv \textit{X did something PP}}, then the PP is a complement.
\begin{exe}
  \ix Pat relied on Chris    \ngv Pat did something on Chris
  \ix Pat put nuts in a cup  \ngv Pat did something in a cup
  \ix Pat slept until noon   \ent Pat did something until noon
  \ix Pat ate lunch at Bytes \ent Pat did something at Bytes
  \end{exe}
\end{itemize}

\myslide{Agreement}
\MyLogo{}
\begin{itemize}
\item Two kinds so far (namely?)
\item Both initially handled via stipulation in the
Head-Specifier Rule
\item But if we want to use this rule for categories
that don’t have the AGR feature (such as PPs
and APs, in English), we can’t build it into
the rule.
\end{itemize}

\myslide{The Specifier-Head Agreement Constraint (SHAC)}
Verbs and nouns must be specified as:
\begin{center}
  \begin{avm}\avmfont{\sc}
    \[ head & \[ agr \@1 \] \\
    val & \[ spr \< agr \@1 \> \] \]
  \end{avm}
\end{center}

\begin{itemize}
\item Why is this lexical?
\end{itemize}

\myslide{Type Hierarchy (new Agreement!)}

\begin{center}
  \begin{avmtree}%\renewcommand{\lf}[1]{\br{#1}{}}\avmfont{\sc}
\it
\br{feat-struc}{ \br{ expression  \[ \textsc{head, val} \] }{\lf{word}
                                  \lf{phrase}}
                 \br{agr-cat \[ \textsc{per,num} \] }{
                   \br{non-3sing}{
                     \br{non-1sing}{\lf{2sing} \lf{pl}} \lf{1sing}}
                   \br{3sing \[ gend \] }{}}
                 \br{val-cat \[ \textsc{spr, comps} \] }{ }
                 \br{pos}{\br{agr-pos \[ \textsc{agr} \] }{ 
                     \br{noun \[ \textsc{case} \]}{}
                     \br{verb \[ \textsc{aux} \]}{}
                     \br{det \[ \textsc{count} \]}{} }
                   \lf{prep}
                   \lf{adj}
                   \lf{conj}}}
\end{avmtree}
\end{center}



\myslide{An example}
\scalebox{0.6}{
\begin{avmtree}
\br{\begin{avm}
    \[ {\it phrase}\\
       HEAD & \@{0}\\ 
       VAL & \[ SPR & \el\\
                COMPS & \el \]\]
    \end{avm}}{\br{\begin{avm}
	       \@1\[{\it phrase}\\
		     HEAD & \@4\\
		     VAL & \[ SPR & \el\\
		              COMPS & \el \]\]
	          \end{avm}}{\psset{levelsep=1.4in}\br{\begin{avm}
			          \@2\[ {\it word}\\
					HEAD & \[ {\it det}\\
						  AGR & \@3\\
						  COUNT & + \]\\
					VAL & \[ SPR & \el\\
					         COMPS & \el \]\] 
				 \end{avm}}{\lf{The}}
			     \br{\begin{avm}
				  \[ {\it word}\\
				     HEAD & \@4\[ {\it noun}\\
						  AGR & \@3\[ {\it 3sing}\\
							      PER & 3rd\\
							      NUM & sg \] \]\\
				     VAL & \[ SPR & \q< {\@2} \q>\\
				              COMPS & \el \]\]
				 \end{avm}}{\lf{dog}}}
		\br{\begin{avm}
		    \[ {\it word}\\
		       HEAD & {\@0}\[{\it verb}\\
                                     AGR\ \ {\@3}\]\\
		       VAL & \[ SPR & \q< {\@1} \q>\\
		                COMPS & \el \]\]
		    \end{avm}}{\lf{walks}}}
\end{avmtree}}



\myslide{The Count/Mass Distinction}
\begin{itemize}
\item Partially semantically motivated
\begin{itemize}
  \item mass terms tend to refer to undifferentiated substances (\lex{air,
butter, courtesy, information})
\item count nouns tend to refer to individual entities (\lex{bird,
cookie, insult, fact})
\end{itemize}
\item But there are exceptions:
  \begin{itemize}
  \item \lex{succotash} (mass) denotes a mix of corn \& lima beans, so
it’s not undifferentiated.
  \item \lex{furniture, footwear, cutlery, \ldots} refer to individuatable
artifacts with mass terms
  \item \lex{cabbage} can be either count or mass, but many speakers
get \lex{lettuce} only as mass.
  \item borderline case: \lex{data}
  \end{itemize}
\end{itemize}


\myslide{Our Formalization of the Count/Mass Distinction}
\begin{itemize}
\item Determiners are:
\begin{itemize}
  \item $[\ft{count} -]$ (\lex{much} and, in some dialects, \lex{less}),
  \item $[\ft{count} +]$ (\lex{a, six, many, \ldots})
  \item lexically underspecified (\lex{the, all, some, no, \ldots})
  \end{itemize}
\item Nouns select appropriate determiners
  \begin{itemize}
  \item “count nouns” say $[\ft{spr} < [\ft{count} +] >]$
  \item “mass nouns” say $[\ft{spr} < [\ft{count} -] >]$
  \end{itemize}
\item Nouns themselves aren’t marked for the feature \ft{count}
\item So the SHAC plays no role in count/mass marking.
\end{itemize}

\myslide{Overview}
\begin{itemize}
\item A problem with the Chapter 3 grammar
\item Generalize COMPS and SPR
\item The Valence Principle
\item Agreement
\item The SHAC
\item (Work through problems 4.1, 4.5, 4.6)
\end{itemize}
\myslide{4.1 Valence Variations} Write lexical entries (including
HEAD, SPR, and COMPS values). You may use NP, VP, etc.\ as
abbreviations for the feature structures on COMPS lists.

As you do this problem, keep the following points in mind: (1) In 
chapter~4 COMPS became a list-valued feature,
and (2) heads select for their specifier and complements (if they have
any); the elements on the SPR and COMPS lists do not
simultaneously select for the head.

\noindent
[{\sl Hint: For the purposes of this problem, assume that adjectives
and prepositions all have empty SPR lists.}]

\begin{itemize}
% \item[A.] What does the grammaticality of sentences like {\it Kim put the
% book here/there} suggest about the COMPS and HEAD values of the 
% words {\it here} and {\it there}? 

\item[A.] Write lexical entries for the words {\it here} and {\it there} as
they are used in (i).

   \begin{exe}
   \exi{(i)}\it Kim put the book here/there.
   \end{exe}

\noindent
[{\sl Hint: Compare (i) to (7) on p97.}]
\newpage
% % \item[B.] What is the COMPS value for the adjective {\it fond}?

% \item[B.] Write a lexical entry for the \index{adjective}adjective {\it fond}.
% Your lexical entry should account for 
% the contrast between (\ref{fondex}e,g) and (\ref{fondex}h).
% %the data in (\ref{fondex}d--h).
% %Cite positive and negative examples supporting the COMPS value you
% %assign to this lexical item.

\item[C.] Assume that motion verbs like {\it jump}, {\it move}, etc.\
take an \index{optionality!optional PP complement}\index{prepositional phrase (PP)!complement!optional}optional PP complement:

\begin{quote}
\begin{avm}
\[ COMPS & \q< (PP) \q> \]
\end{avm}
\end{quote}
%
Write
the lexical entries for the
\index{preposition (P)}prepositions {\it out}, {\it from} and {\it of}:

   \begin{itemize}
   \item[(i)] Kim jumped out of the bushes.
   \item[(ii)] Bo jumped out from the bushes.
   \item[(iii)] Lee moved from under the bushes.   
   \item[(iv)] Leslie jumped out from under the bushes.
   \item[(v)] Dana jumped from the bushes.
   \item[(vi)] Chris ran out the door.
   \item[(vii)] \bad Kim jumped out of from the bushes.
   \item[(viii)] Kim jumped out.
   \item[(ix)] \bad Kim jumped from.
   \end{itemize}
\newpage
\item[D.] Based on the following data, 
% sketch 
write 
the \index{lexical entry} lexical entries for
the words {\it grew} (in the `become' sense, not the `cultivate'
sense), {\it seemed}, {\it happy}, and {\it close}. 

   \begin{itemize}
   \item[(i)] They seemed happy (to me).
   \item[(ii)] Lee seemed an excellent choice (to me).
   \item[(iii)] \bad They seemed (to me).
   \item[(iv)] They grew happy.
   \item[(v)] \bad They grew a monster (to me).
   \item[(vi)] \bad They grew happy to me.
   \item[(vii)] They grew close to me.
   \item[(viii)] They seemed close to me to Sandy.
   \end{itemize}

\noindent
[{\sl Note: \index{adjective phrase (AP)} APs have an internal structure 
%that is 
analogous to that of
VPs. Though no adjectives select NP complements (in English), there
are some adjectives that select PP complements (e.g.\ {\it to me}),
and some that do not.}]
\newpage
\item[E.] Using the lexical entries you wrote for part (D), draw
a tree (showing the values of HEAD, SPR, and COMPS at each node,
using tags as appropriate) for {\it They seemed close to me to Sandy}.

\end{itemize}



\myslide{4.5 Facts of English case}
\begin{itemize}
\item For each of the following positions,
determine which case the pronouns in that
position must have:
\begin{itemize}
\item Subject of a sentence
\item Direct object of a verb
\item Second object of a verb like \lex{give}
\item Object of a preposition
\end{itemize}
\item Give examples
\end{itemize}

\myslide{4.6 A lexicalist analysis}
\begin{itemize}
\item Section 4.8 hinted that case marking can be
limited in the same way that we handle
agreement, i.e., without any changes to the
grammar rules. Show how this can be done.
Your answer should include lexical entries
for they, us, likes, and with.
\item Hint: Assume that there is a feature CASE
with the values ‘acc’ and ‘nom’, and assume
that English pronouns have CASE features
specified in their lexical entries.
\end{itemize}

\myslide{Acknowledgments and References}

\begin{itemize}
\item Course design and slides borrow heavily from Emily Bender's course:
\textit{Linguistics 566: Introduction to Syntax for Computational Linguistics}
\\ \url{http://courses.washington.edu/ling566}
\end{itemize}

\end{document}

%%% Local Variables: 
%%% coding: utf-8
%%% mode: latex
%%% TeX-PDF-mode: t
%%% TeX-engine: xetex
%%% End: 
